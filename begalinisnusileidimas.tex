\subsection{Begalinis nusileidimas}

Begalinis nusileidimas yra Pierre de Fermat sugalvota (todėl kartais vadinama Ferma
nusileidimu) technika leidžianti kai kuriais atvėjais įrodyti, kad
diofantinė lygtis neturi sprendinių arba turi tik kai kuriuos išimtinius.
Ji remiasi idėja, kurią negriežtai galima suformuluoti taip:

\begin{center}
  \emph{
    Jei iš kiekvieno teigiamo diofantinės lygties sprendinio galima gauti
    mažesnį teigiamą sprendinį, tai tuomet diofantinė lygtis teigiamų
    sprendinių turėti negali.}
\end{center}

Sąvokos ``teigiamas'' ar ``mažesnis'' sprendinys bendru atveju, žinoma,
nėra apibrėžtos, todėl kas tiksliai jomis yra nusakoma priklauso nuo
konkretaus uždavinio.  

\begin{pav} Raskite lygties $x^2 = 2y^2$ sveikuosius sprendinius.
\end{pav}

\begin{sprendimas}
  Tarkime, kad ši lygtis turi sprendinį $(x_0,y_0)$, kur $x_0>0$ ir
  $y_0>0$. Kadangi dešinioji lygybės $x_0^2 = 2y_0^2$ pusė dalijasi iš $2$,
  tai ir kairioji turi dalintis iš $2$. Bet tada $x_0$ dalinsis iš $2$,
  $x_0^2$ dalinsis iš $4$ todėl ir $y_0$ turės dalintis iš $2$. Pažymėkime
  $x_0=2x_1$ ir $y_0=2y_1$. Įsistatę į lygtį ir suprastinę gausime $x_1^2 =
  2y_1^2$ - t.y. $(x_1, y_1)$ taip pat bus sprendinys. Taip tęsdami toliau
  gausime begalo daug sprendinių, kurių kiekviena iš koordinačių mažės, bet
  visuomet liks teigiama. Tačiau to būti negali, vadinasi lygtis negali
  turėti sprendinių su $x$ ir $y$ teigiamais. Kadangi abiejose pusėse
  kvadratai, tai negali turėti ir jokių kitų, išskyrus $(0,0)$, kuris
  vienintėlis ir tinka.

  Beveik visuomet samprotavimą apie begalinę seką galima pakeisti kiek
  kitokiu, tačiau ta pačia idėja iš esmės besiremiančiu argumentu. Šiuo
  atveju tai atrodytų taip: tegu duota lygtis turi sprendinį $(x_0, y_0)$.
  Jei $x_0$ ir $y_0$ turi bendrų daliklių, tai suprastinkime, ir gausime
  sprendinį $(x_1,y_1)$, kurio $x_1$ ir $y_1$ yra tarpusavyje pirminiai.
  Tačiau, kaip jau matėme, tuomet gausime, kad $x_1$ ir $y_1$ dalijasi iš
  $2$ - prieštara. Lieka vienintėlė sveikųjų skaičių pora, kurios neįmanoma
  suprastinti taip, kad skaičiai nebeturėtų bendrų daliklių - $(0,0)$.
\end{sprendimas}

\subsubsection{Uždaviniai}

\begin{enumerate}
  \item Raskite lygties $x^3 + 2y^3 + 4z^3 = 6xyz$ sveikuosius
    sprendinius.
  \item Raskite lygties $x^3 + 2y^3 + 4z^3 = 9w^3$ sveikuosius
    sprendinius.
    %Tarkime lygtis turi sprendinį (x,y,z,w). Kadangi $a^3\equiv \pm 1,
    %0\m{9}$, tai $3|x, 3|y, 3|z$, o tada ir $3|z$. Gauname naują sprendinį
    %($x/3, y/3, z/3, w/3$). Taikydami begalinio nusileidimo principą
    %gauname, kad sprendinys yra tik (0,0,0,0).
  \item Raskite lygties $5x^3 + 11y^3 + 13z^3 = 0$ sveikuosius sprendinius.
    %mod $13$
  \item Raskite lygties $( (px)^2 -1)^p + 1 = py^2$ natūraliuosius
    sprendinius.
  \item \text{[Austria 2002]} Raskite lygties $(19a + b)^{18} + (a+b)^{18} + (19b + a)^{18} =
    c^2$ sveikuosius sprendinius.
  \item Raskite sveikuosius lygčių sistemos sprendinius 
    $$\left\{ 
    \begin{array}{l} 3a^4 + 2b^3 = c^2 \\
      3a^6 + b^5 = d^2
    \end{array}\right.$$
  %Spręsdami antrąją lygtį moduliu $3$ gausime, kad $b\not \equiv 1
  %\m{3}$, o nagrinėdami pirmąją gausime, kad $b \not \equiv 2 \m{3}$.
  %Vadinasi $b$ dalijasi iš trijų. Lieka pritaikyti begalinį nusileidimą:
  %
  %
  %$$3|b \implies 3|c, 3|d \implies 3|a \implies 3^2|c, 3^2|d \implies 3^2|b
  %\implies 3^3|c, 3^4 |d \implies 3^2|a, \implies 3^3|c, 3^5|d.$$ Pažymėję
  %$a=3^2a_1, b=3^2b_1, c=3^3c_1, d=3^5d_1$, įstatę ir suprastinę gausime
  %sistemą $$\left\{ 
  %  \begin{array}{l} 3^3a_1^4 + 2b_1^3 = c_1^2 \\
  %    3^3a_1^6 + b_1^5 = d_1^2
  %  \end{array}\right.$$
  %Pakartoję samprotavimus vėl gausime tas pačias išvadas apie $a_1$,
  %$b_1$, $c_1$ ir $d_1$ dalumus iš $3$ ir taip galėsime tęsti kiek tik
  %norėsime, nes po suprastimų sistema visuomet atrodys panašiai, skirsis
  %nebent koeficientai prie $a$, bet išliks nelyginiai trejeto laipsniai.
  %Vadinasi pradinio sprendinio $a$, $b$, $c$ ir $d$ kiek norima kartų
  %daliasi iš trijų, todėl tegali būti lygus $(0,0,0,0)$.


    	
  \item Raskite visus sveikųjų skaičių ketvertus $(a,b,c,d)$, su kuriais
    $$a^2 + 5b^2 = 2c^2 + 2cd + 3d^2$$.
  %Padauginkime lygtį iš dviejų: $2a^2 + 10b^2 = (2c+d)^2 + 5d^2$.
  %Pažymėkime $2c+d = e$, parodysime, kad lygtis $2a^2 + 10b^2 = e^2 +
  %5d^2$ neturi sveikųjų sprendinių. Spręsdami lygtį moduliu $5$ gauname
  %$2a^2 \equiv e^2 \implies 5|a, 5e$. Pažymėję $a=5x$, $e=5y$, įstatę ir
  %suprastinę gauname $10a^2 + 2b^2 = 5e^2 + d^2$. Vėl nagrinėdami moduliu
  %$5$ gauname, kad ir $5|b, 5|d$, vadinasi lygtis turi tik nulinį
  %sprendinį $(0,0,0,0)$.
\end{enumerate}
