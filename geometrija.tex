\chapter{Geometrija}
%Asymptote direktorija
%\def\asydir{./iliustracijos/geometrija}

\thispagestyle{empty}
%%noparse
\section{Įžanga}
%%parse

Šiame skyriuje mokysimės spręsti geometrijos uždavinius. Geometrija reikalauja kiek
kitokio mąstymo nei algebra ar kombinatorika, ir dėl to dalis olimpiadininkų
geometrijos nelabai mėgsta/moka ir geometrijos uždaviniams spręsti renkasi
algebrinius metodais - kompleksinius skaičius ar trigonometriją.
Deja, nemažos dalies uždavinių šiais
metodais neįmanoma išspręsti, o bandant prarandama daug laiko. Todėl skyriaus 
tikslas yra išmokyti mąstyti geometriškai, lavinti pastabumą, surasti trumpą „sintetinį"
sprendimą, kurį greičiausiai uždavinio kūrėjai turi kaip oficialų. Žinoma, tai nereiškia, 
kad visada yra vienas geriausias sprendimas, o ir ne kiekvienas moksleivis turi gabumų 
ar patirties pastebėti gerokai neakivaizdžius dalykus. Tam
geometrijos skyrelis sukurtas taip, kad tiktų mokytis tiek jaunesniesiems
moksleiviams, kurie dar tik pradėjo mokytis geometrijos mokykloje, tiek
vyresniems, norintiems išmokti paprasčiausių ir efektyviausių gudrybių. 

Geometrijos
uždavinius lengva suskirstyti pagal temą, todėl uždaviniai yra surinkti
vos iš keletos olimpiadų - daugiausia Peterburgo miesto ir Miestų turnyro.
Jie yra panašaus sunkumo į Lietuvos Respublikinės olimpiados uždavinius,
ir todėl lengvesni nei kitų skyrių uždaviniai. Jie nebūtinai surikiuoti pagal
sunkumą, bet pirmieji dažniausiai lengvesni nei paskutiniai. Sunkiausieji
gali būti kietas riešutėlis net ir patyrusiems veteranams, bet tikrai yra 
išsprendžiami. Autorius siūlo tiesiog spręsti iš eilės, o užstrigus
prie uždavinio imti kitą.  Reikia paminėti, kad skyreliai yra išdėstyti eilės tvarka, 
t.y. prieš sprendžiant uždavinius reikėtų bent būti perskaičius, kas parašyta 
ankstesniuose skyreliuose. 
 
Kai kurie pastebės, kad kitaip nei daugumoje kitų geometrijos knygų,
„nemokyklinės" teorijos yra gana nedaug, o naujų teoremų tik keletas.
Taip yra dėl to, kad knyga skirta grynai ruoštis olimpiadoms ir palikti
tiktai praktinį pritaikymą turintys faktai.
O ir knyga dar bus papildoma ateityje.
Vietoje teoremų yra įdėtos „gerai žinomos lemos“, kurias derėtų išmokti mintinai.
Tai tiesiog naudingos gudrybės, kurias panaudojus olimpiados metu reikia arba įrodyti,
arba tiesiog įterpti į sprendimą. 



\subsubsection{Keletas patarimų, kaip brėžti brėžinius}

Geometrijos uždaviniai neatsiejami nuo brėžinių, todėl mokėti greitai ir
gražiai nubrėžti reikiamą brėžinį yra neįkanojamas ir nelengvai
išugdomas įgūdis; nors kiek toliau knygoje vietomis rasite pastabų, kaip
tai padaryti, tačiau bendri pastebėjimai yra surašyti čia: pirma, jei
duotas bet koks trikampis, keturkampis ar kitokia figūra, tai visomis
išgalėmis stenkitės kad brėžinyje tas trikampis nebūtų statusis, lygiašonis,
ar neduok Dieve, lygiakraštis, o keturkampis nebūtų rombas, lygiagretainis ar
trapecija. Sąlygai tai neprieštaraus, bet kartais trukdys spręsti, nes, 
pavyzdžiui, lygiašoniame trikampyje aukštinės ir pusiaukampinės pagrindai
brėžinyje sutaps arba bus labai arti ir maišysis, arba jei nubrėšite $ABCD$
lygiagretainį, tai keturkampyje $ABCD$ kraštinės $AB$ ir $CD$ kirsis kur nors
už jūsų popieriaus lapo ribų , ir todėl negalėsite pažymėti $AB$ ir $CD$
sankirtos taško. Nubrėžti smailą nelygiašonį trikampį yra nemenkas iššūkis,
 nes visada atsiras dvi kraštinės, nesiskiriančios per
daugiau nei 30 proc., ir šios kraštinės brėžinyje bus panašaus ilgio.
 Paprastą ganėtinai nelygiašonį smailų trikampį matote
paveikslėlyje žemiau.

%Nelygiašonis trikampis
\begin{center}
\begin{asy}
import olympiad;
size(200);
pair A, B, C;
A=(35,60); B=(100,0); C=(0,0);
draw(A--B--C--cycle);
label("$A$",A,N,blue);
label("$B$",B,E,blue);
label("$C$",C,W,blue);
\end{asy}
\end{center}

Taip pat brėždami brėžinius  nebijokite brėžti daug kartų: jei
nepavyko nubrėžti gražiai iš pirmo karto, brėžkite iš naujo, o ne bandykite
pataisyti. Taip pat nebrėžkite per mažo brėžinio, nes gali pradėti maišytis
raidės, kampai ir pan. Kiek įgudus galima išvis brėžiniuose nepalikti raidžių,
o jas pridėti tik išsprendus ir užrašant sprendimą.
 Ir svarbiausia, brėžkite tikslius brėžinius, t.y.
jei sąlyga rašo, kad $AB={CD}$, tai stenkitės, kad brėžinyje bent jau
panašiai būtų. Jei reikės, brėžkite kad ir 10 skirtingų brėžinių, kol
gausite tinkamą.  Viso šito reikia dėl 2 priežasčių: pirma, tai leidžia
pasitikrinti, ar gerai supratote sąlygą: jei reikia įrodyti kad
$AB\parallel{CD}$, o jūsų visi brėžiniai yra labai tikslūs ir visuose
brėžiniuose $AB\perp{CD}$, tai tikriausiai kažkur padarėte klaidą; ir
antra, tikslūs brėžiniai padeda išspręsti uždavinius, gali suteikti įžvagų,
pavyzdžiui, jei reikia įrodyti, kad keturkampis $ABCD$ yra įbrėžtinis, o
visuose jūsų brėžiniuose $ABCD$ atrodo labai panašus į kvadratą, tai galbūt
$ABCD$ iš tikrųjų yra kvadratas ir įrodyti, kad $ABCD$ yra kvadratas yra
lengviau. Taip pat
nepamirškite pasižymėti visų uždavinyje duotų sąlygų. Paskutinis
patarimas: jei jau pavyko nusibrėžti gerą brėžinį, bet nekyla jokių minčių kaip
išspręsti uždavinį, tai pabandykite persibrėžti brėžinį, tiktai apverstą ar
kitu kampu ( arba, žinoma, pasukti lapą).  Paprastai naujo tokio pat 
brėžinio nusibrėžimas daug naudos
neduoda, bet nusibrėžus pasuktą ar apverstą gali kilti naujų idėjų. 

\subsubsection{Būtini geometrijos pagrindai}

Jeigu dar nesimokote vienuoliktoje klasėje, tai mokykloje dar nesimokėte
viso mokyklinės geometrijos kurso. Šiaip šiame geometrijos skyriuje
tikimasi, kad žinote/mokate jį visą, tad jeigu dar kažko nežinote iš
mokyklos kurso, geriausia būtų nueiti į biblioteką ir pasiimti visus
matematikos vadovėlius iki 10 klasės ir išmokti visą geometriją -
geometrijos mokykloje yra labai nedaug ir ji gana lengva (stereometrijos
mokytis nebūtina). Tačiau jei esate aštuntokas ar devintokas ir bijote, kad
nesuprasite kosinusų teoremos, nenusiminkite - pirmuosiuose poskyriuose
turėtų pakakti tiek geometrijos, kiek yra iki 9 klasės, be to, skyreliuose
svarbiausia teorija bus duota.

\newpage
\section{Uždaviniai apšilimui} 
  
Šiame skyrelyje sudėti nesudėtingi uždaviniai. Beveik visi jie yra iš
septintokų, aštuntokų ar devintokų olimpiadų, ir todėl yra vieni lengviausių kokie
gali būti olimpiadoje. Beveik visi jie yra apie paprastus objektus - 
tieses, trikampius ir kampus. Priminsime keletą pagrindinių sąvokų ir
teiginių, kurie nevisuomet yra akcentuojami mokykloje, tačiau dažnai
sutinkami olimpiadose. 

\subsubsection{Dažnai pamirštamos savybės ir maišomos sąvokos}

\begin{api}
  Trikampio \emph{Pusiau\textbf{kraštinė}} eina iš trikampio viršūnės į
  prieš tą viršūnę esančią kraštinę ir dalija ją pusiau. Trikampio
  \emph{Pusiau\textbf{kampinė}} eina iš trikampio viršūnės į prieš tą
  viršūnę esančią kraštinę ir dalija prie tos viršūnės esantį kampą pusiau.
\end{api}

\begin{teig}[Trikampio pusiaukampinės savybė]
  Trikampio $ABC$ pusiaukampinė $AD$ dalija priešingą kraštinę į dvi
  atkarpas $BD$ ir $DC$, kurių ilgių santykis yra lygus kitų dviejų to
  trikampio kraštinių $AB$ ir $AC$ ilgių santykiui
  $$\frac{BD}{CD}=\frac{AB}{AC}.$$
\end{teig}

\begin{proof}[Įrodymas] 
  Nubrėžkime per tašką $C$ tiesę, lygiagrečią atkarpai $AB$. Tegu ši tiesė
  kerta pusiaukampinės $AD$ tęsinį taške $E$.  Tada $\angle ABD = \angle
  DCE, \angle BAD = \angle CED$, todėl trikampiai $ABD$ ir $CDE$ panašūs
  pagal 3 kampus, be to, trikampis $ACE$ yra lygiašonis, todėl
  $\frac{AB}{BD} = \frac{CE}{DC} = \frac{AC}{DC}$ ( prisiminkite, kad $\frac{a}{b} =
\frac{c}{d} \Longleftrightarrow \frac{a}{c} = \frac{b}{d}$), ką ir reikėjo įrodyti.
\end{proof}
  
%Pusiaukampinės savybė
\begin{center}
\begin{asy}
size(200);
pair A, B, C, X, D, E, Y;
A=(35,60); B=(100,0); C=(0,0);
X=bisectorpoint(C,A,B);
D=extension(A,X,B,C);
Y=B+C-A;
E=extension(A,D,B,Y);
add(anglem(C,A,D,400,green,1));
add(anglem(D,A,B,400,green,1));
add(anglem(B,E,D,400,green,1));
add(anglem(B,C,A,400,red,0));
add(anglem(D,B,E,400,red,0));
dot(D,blue);
dot(E,blue);
draw(A--E); draw(B--E);
draw(A--B--C--cycle);
label("$A$",A,N,blue);
label("$C$",B,SE,blue);
label("$B$",C,SW,blue);
label("$D$",D,SE,blue);
label("$E$",E,SE,blue);
add(pathticks(A--B,2,0.5,0,125,red)); 
add(pathticks(E--B,2,0.5,0,125,red));
\end{asy}
\end{center}

\begin{teig}[Trikampio priekampio savybė] 
  Trikampio $ABC$ kampai tenkina lygybę $\angle A + \angle B = 180^\circ -
  \angle C$. Geometriškai tai reiškia, kad kampų $A$ ir $B$ suma yra lygi
  išoriniam kampui $C$ (kitaip žinomam kaip kampo $C$ priekampiui).
  Paveikslėlyje žemiau dviejų žalių kampų suma yra lygi raudonam
  kampui (kampo $C$ priekampiui).
\end{teig}

%Priekampio savybė
\begin{center}
\begin{asy}
import olympiad;
size(200);
pair A, B, C;
A=(35,60); B=(100,0); C=(0,0);
add(anglem(A,C,(-30,0),300, red, 0));
add(anglem(C,A,B,300,green,0));
add(anglem(A,B,C,300,green,0));
draw(A--B--C--cycle);
draw((-30,0)--(0,0));
label("$A$",A,N,blue);
label("$B$",B,SE,blue);
label("$C$",C,SW,blue);
\end{asy}
\end{center}
  
\begin{teig}
  Paprasta, bet dažnai naudojama trikampių panašumo savybė: jeigu
  trikampiai $ABC$ ir $A'B'C'$ tenkina $\angle BAC = \angle B'A'C'$ ir
  $\frac{AB}{A'B'} = \frac{AC}{A'C'}$, tai tada trikampiai $ABC$ ir
  $A'B'C'$ yra panašūs.
\end{teig}

%Trikampių panašumo sąlyga 
\begin{center}
\begin{asy}
import olympiad;
size(200);
pair shift, A, B, C, AA, BB, CC;
shift = (-350,-75);
A=(35,60)+shift; B=(100,0)+shift; C=(0,0)+shift;
label("$A'$",A,N,blue);
label("$C'$",B,SE,blue);
label("$B'$",C,SW,blue);
draw(A--B--C--cycle);
AA = A*1.5;
BB = B*1.5;
CC = C*1.5;
add(anglem(C,A,B,600/1.5,blue,0));
add(anglem(CC,AA,BB,600,blue,0));
label("$A$",AA,N,blue);
label("$C$",BB,SE,blue);
label("$B$",CC,SW,blue);
draw(AA--BB--CC--cycle);
\end{asy}
\end{center}

Esminė panašiųjų trikampių savybė: panašiųjų trikampių atitinkamų
komponentų (turima omenyje tuos, kurie matuojami ilgio vienetais) santykis
yra lygus tų trikampių panašumo koeficientui, o atitinkami kampai tarp
atitinkamų tiesių ar atkarpų panašiuose trikampiuose yra lygūs. Pavyzdžiui,
tarkime, kad trikampiai $ABC$ ir $A'B'C'$ yra panašūs ir 
$$\frac{AB}{A'B'} = \frac{BC}{B'C'} = \frac{CA}{C'A'} = k.$$ 
Tada, paėmę taškus $D$ ir $D'$ ant atitinkamai $BC$ ir $B'C'$ taip, kad
$\frac{BD}{DC} = \frac{B'D'}{D'C'}$, gausime $\angle BAD = \angle B'A'D'$
bei $\frac{AD}{A'D'} = k$. Taip pat iš karto gauname, kad panašiųjų
trikampių atitinkamų aukštinių, pusiaukraštinių ir pusiaukampinių ilgių
santykis yra lygus pačių trikampių kraštinių ilgių santykiui (trikampių
panašumo koeficientui). 

%Trikampių panašumo savybė
\begin{center}
\begin{asy}
import olympiad;
size(200);
pair shift, A, B, C, D, AA, BB, CC, DD;
shift = (-350,-75);
A=(35,60)+shift;
B=(100,0)+shift;
C=(0,0)+shift;
D=(40,0)+shift;
AA = A*1.5;
BB = B*1.5;
CC = C*1.5;
DD = D*1.5;
add(anglem(C,A,D,600,green,1));
add(anglem(D,A,B,600,red,0));
add(anglem(CC,AA,DD,900,green,1));
add(anglem(DD,AA,BB,900,red,0));
dot(D,blue);
dot(DD,blue);
draw(AA--BB--CC--cycle);
draw(AA--DD);
draw(A--B--C--cycle);
draw(A--D);
label("$A$",AA,N,blue);
label("$C$",BB,SE,blue);
label("$B$",CC,SW,blue);
label("$D$",DD,SE,blue);
label("$A'$",A,N,blue);
label("$C'$",B,SE,blue);
label("$B'$",C,SW,blue);
label("$D'$",D,SE,blue);
\end{asy}
\end{center}

\begin{api}
  Sakome, kad taškas $A$ yra \textbf{simetriškas taškui $A'$ taško $O$ atžvilgiu},
   jeigu taškas $O$ yra atkarpos $AA'$ vidurio taškas.
\end{api}

%Taško simetrija
\begin{center}
\begin{asy}
import olympiad;
size(200);
pair A, AA, B, BB, O;
O=origin;
A=(-80,10);
B=(-50,-10);
AA=-A; BB=-B;
dot(A,blue);
dot(B,blue);
dot(AA,blue);
dot(BB,blue);
dot(O,blue);
draw(A--AA);
draw(B--BB);
label("$A$",A,NW,blue);
label("$B$",B,NW,blue);
label("$A'$",AA,NE,blue);
label("$B'$",BB,NE,blue);
label("$O$",O,N,blue);
add(pathticks(A--O,2,0.5,0,125,red));
add(pathticks(AA--O,2,0.5,0,125,red));
add(pathticks(B--O,2,0.5,30,125,red));
add(pathticks(BB--O,2,0.5,30,125,red));
\end{asy}
\end{center}

\begin{api}
  Taškas $A$ yra \textbf{simetriškas taškui $A'$ tiesės $l$ atžvilgiu}, jeigu tiesė
  $l$ eina per atkarpos $AA'$ vidurio tašką ir yra jai statmena. Tokiu
  atveju tiesė l vadinama simetrijos ašimi.
\end{api}

%Tiesės simetrija
\begin{center}
\begin{asy}
import olympiad;
size(200);
pair A, AA, B, BB, C, CC;
A=(75,14);
B=(45,10);
C=(25,18);
AA=conj(A);
BB=conj(B);
CC=conj(C);
label("$A$",A,NE,blue);
label("$B$",B,NE,blue);
label("$C$",C,NE,blue);
label("$A'$",AA,SE,blue);
label("$B'$",BB,SE,blue);
label("$C'$",CC,SE,blue);
label("$l$",(95,0),N,blue);
add(rightanglem(A,(A+AA)/2,(100,0),140));
add(rightanglem(B,(B+BB)/2,(100,0),140));
add(rightanglem(C,(C+CC)/2,(100,0),140));
draw((0,0)--(100,0));
draw(A--AA);
draw(B--BB);
draw(C--CC);
add(pathticks(A--(A+AA)/2,3,0.5,30,100,red));
add(pathticks(AA--(A+AA)/2,3,0.5,30,100,red));
add(pathticks(B--(B+BB)/2,2,0.5,30,100,red));
add(pathticks(BB--(B+BB)/2,2,0.5,30,100,red));
add(pathticks(C--(C+CC)/2,1,0.5,30,100,red));
add(pathticks(CC--(C+CC)/2,1,0.5,30,100,red));
dot(A,blue);
dot(B,blue);
dot(AA,blue);
dot(BB,blue);
dot(C,blue);
dot(CC,blue);
\end{asy}
\end{center}

\subsubsection{Pagrindiniai sprendimo būdai}

Jokių ypatingų triukų su lengvais uždavinias nėra;
pagrindinis sprendimo būdas yra turbūt „sprendimas kampais“
- sužymėti visus svarbius kampus kintamaisias, pažymėti
lygius kampus, tada ieškoti panašių, vienodų ar lygiašonių trikampių,
taikyti „mokyklines“ savybes ir panašiai. Nepamirškite
atidžiai perskaityti sąlygos  ir pirmiausia pažymėti tai,
kas duota sąlygoje.

Nerašyta uždavinių „Rask kampą" taisyklė: jeigu uždavinio
sąlygoje nėra nurodyta jokių specifinių detalių, t.y
nenurodyta jokie kampų dydžiai ar kraštinių ilgiai, o
sąlygoje prašo rasti kokio nors kampo dydį, tai tas kampas
greičiausiai bus $30^\circ$ kartotinis arba $45^\circ$. Kiek rečiau kampas būna
$15^\circ$ kartotinis, o itin retais atvejais pasitaiko, kad tas ieškomas kampas yra
$18^\circ$ kartotinis. Tad vos pamačius tokį uždavinį geriausia
būtų nusibrėžti kuo tikslesnį brėžinį ir pažiūrėti, ar
ieškomas kampas panašus į $30^\circ, 60^\circ$ arba
$90^\circ$, ir greičiausiai tai ir bus atsakymas; kitu atveju
reikia bandyti kitus $15^\circ$ kartotinius.

\begin{pav}
  Trikampyje $ABC$ pusiaukampinė ir pusiaukraštinė iš viršūnės
  $A$ sutampa. Įrodyti, kad $ABC$ lygiašonis.
\end{pav}

\begin{sprendimas}
  Tegu $M$ yra $BC$ vidurio taškas. Tada iš pusiaukampinės savybės
  $\frac{AB}{AC} = \frac{BM}{MC} = 1$, todėl $AB = AC$.
\end{sprendimas}

\begin{pav}
  Duota trapecija $ABCD$ su pagrindais $AD$ ir $BC$. Įrodyti, kad $AD$
  vidurio taškas, $BC$ vidurio taškas bei $AC$ ir $BD$ sankirtos taškas
  guli vienoje tiesėje. 
\end{pav}

\begin{sprendimas}
  Tegu $E$ yra $BC$ vidurio taškas, $F$ yra $AD$ vidurio taškas, o $G$ yra
  trapecijos įstrižainių sankirtos taškas.  Tada nesunkiai iš trijų kampų
  požymio trikampiai $BGC$ ir $AGD$ yra panašūs, todėl mes galime taikyti
  savybę, paminėtą aukščiau: abiejuose šiuose trikampiuose mes nubrėžiame
  atitinkamas pusiaukraštines $GE$ ir $GF$, ir, kadangi atitinkami kampai
  tarp atitinkamų atkarpų yra lygūs (šiuo atveju kampai tarp kraštinės ir
  pusiaukraštinės), mes gauname, kad $\angle DGF = \angle BGE$ ir todėl taškai
  $F, G, E$ guli vienoje tiesėje. 
\end{sprendimas}

%Trapecijos įstrižainių susikirtimo taško ir pagrindų vidurių koliniarumas
\begin{center}
\begin{asy}
import olympiad;
size(200);
pair A, B, C, D, G, E, F;
A=(0,0); B=(30,40); C=(90,40); D=(100,0);
G=extension(A,C,B,D);
E=midpoint(B--C);
F=midpoint(A--D);
label("$A$",A,SW,blue);
label("$B$",B,NW,blue);
label("$C$",C,NE,blue);
label("$D$",D,SE,blue);
label("$E$",E,N,blue);
label("$F$",F,S,blue);
label("$G$",G,1.5*right,blue);
add(anglem(E,G,B,400,red,0));
add(anglem(F,G,D,400,red,0));
add(anglem(E,C,G,400,blue,2));
add(anglem(F,A,G,400,blue,2));
add(anglem(G,B,E,400,green,1));
add(anglem(G,D,F,400,green,1));
draw(A--B--C--D--cycle);
draw(A--C); draw(B--D); draw(E--F);
add(pathticks(B--E,2,0.5,30,125,red));
add(pathticks(E--C,2,0.5,30,125,red));
add(pathticks(A--F,2,0.5,0,125,red));
add(pathticks(F--D,2,0.5,0,125,red));
dot(G,blue);
dot(E,blue);
dot(F,blue);
\end{asy}
\end{center}
\begin{pav}
  Duotas iškilasis keturkampis $ABCD$. Įstrižainių $BD$ ir
  $AC$ vidurio statmenys kerta kraštinę $AD$ atitinkamai
  taškuose $X$ ir $Y$ taip, kad $X$ yra tarp $A$ ir $Y$
  Pasirodė, kad $BX\parallel{CY}$. Įrodyti, kad $BD\perp{AC}$.
\end{pav} 

%Iškilo keturkampio vidurio statmenys kerta kraštinę..
\begin{center}
\begin{asy}
import olympiad;
size(220);
pair shift, A, B, C, D, X, Y, E, F, I, H, EE, FF;
A=(0,0); B=(50,70); C=(75,60);
D=extension(A,(100,0),B,foot(B,A,C));
dot((0,-10), invisible);
label("$A$",A,NW,blue);
label("$B$",B,NW,blue);
label("$C$",C,NE,blue);
label("$D$",D,NE,blue);
draw(A--B--C--D--cycle);
draw(A--C); draw(B--D);
F=midpoint(A--C);
E=midpoint(B--D);
FF=bisectorpoint(A,C);
EE=bisectorpoint(B,D);
drawline(E,EE);
drawline(F,FF);
H=extension(A,C,B,D);
I=extension(E,EE,F,FF);
dot(E,blue);
dot(F,blue);
dot(I,blue);
dot(H,blue);
label("$E$",E,N,blue);
label("$F$",F,N,blue);
label("$I$",I,N,blue);
label("$H$",H,N,blue);
add(rightanglem(H,E,I,180));
add(rightanglem(I,F,H,180));
X=extension(I,E,A,D);
Y=extension(I,F,A,D);
draw(B--X); draw(C--Y);
dot(X,blue);
dot(Y,blue);
label("$X$",X,NW,blue);
label("$Y$",Y,NE,blue);
add(pathticks(B--E,2,0.55,0,125,red));
add(pathticks(E--D,2,0.5,0,125,red));
add(pathticks(A--F,2,0.5,30,125,red));
add(pathticks(F--C,2,0.5,30,125,red));
\end{asy}
\end{center}

\begin{sprendimas}
  Tegu taškai $H, E, I, F$ yra atitinkamai įstrižainių sankirtos taškas,
  įstrižainės $BD$ vidurio taškas, tų vidurio statmenų sankirtos taškas bei
  įstrižainės $AC$ vidurio taškas (žr. paveikslėlį viršuje). Mums reikia
  įrodyti, kad $BD \perp AC$. Tai būtų tas pats, kaip ir įrodyti, kad $HEIF$
  yra stačiakampis, o tai yra ekvivalentu $\angle XIY = 90^\circ$.
  Pastebėkime, kad trikampiai $XBD$ ir $ACY$ yra lygiašoniai. Tada
  trikampyje $XIY$ $\angle IXY + \angle IYX = \angle EXD  + \angle FYA  =
  \frac{\angle BXD }{2} + \frac{\angle CYA }{2} = \frac{\angle CYD }{2} +
  \frac{\angle CYA }{2} = 90^\circ$, todėl $\angle XIY = 90^\circ$, ką ir
  reikėjo įrodyti.
\end{sprendimas}

Pastebėkime, kad spręsdami mes ne puolėme tiesiai įrodinėti,
kad $BD\perp{AC}$, o suradome ekvivalentų teiginį kurį
įrodyti buvo lengviau. Taip mokėti pastebėti tokius
ekvivalenčius faktus yra svarbu ne tik geometrijoje, bet ir
kitose matematikos šakose, mat atsiranda pasirinkimo laisvė -
galima išsirinkti, ką bandyti įrodyti. Nemaža dalis olimpiadinių
geometrijos uždavinių yra dirbtinai pasunkinami tokiu principu.


\begin{pav}
  Ant stačiojo trikampio $ABC$ įžambinės $AB$ paimti taškai
  $M$ ir $N$ tokie, kad $BC=BM$ ir $AC=AN$. Įrodyti, kad $
  \angle MCN=45^\circ$ 
\begin{center}
\begin{asy}
import olympiad;
size(200);
pair A, AA, B, C, N, M;
AA=(20,60); B=(100,0); C=(0,0);
A=foot(C,B,AA);
dot(A);
M=(arclength(C--A),0);
N=B-(arclength(A--B),0);
add(anglem(N,A,M,300,blue,0));
dot(M);
dot(N);
draw(A--B--C--cycle);
label("$C$",A,NW,blue);
label("$A$",B,E,blue);
label("$B$",C,W,blue);
label("$M$",M,NE,blue);
label("$N$",N,NE,blue);
draw(A--N); draw(A--M);
\end{asy}
\end{center}
\end{pav}

\begin{sprendimas}
  Pastebėkime, kad trikampiai $BCM$ ir $NCA$ yra lygiašoniai.
  Tada $\angle MCN = (\angle BCN + \angle NCM) + (\angle NCM +
  \angle MCA) - (\angle BCN + \angle NCM + \angle MCA) =
  \angle BCM + \angle NCA - 90^\circ = \frac{180^\circ-\angle
  CBM }{2} + \frac{ 180^\circ-\angle CAN }{2} -90^\circ =
  45^\circ$, ko ir reikėjo.  
\end{sprendimas}

Mes šiame pavyzdyje pasinaudojome viena gerai žinoma lema ir įrodėme ją:
jeigu iš taško $O$ išeina atkarpos $OA, OB, OC, OD$ taip, kaip parodyta
pavekslėlyje žemiau (tokia pačia tvarka), tai tada $\angle AOC + \angle
BOD = \angle AOD + \angle BOC$. Tokiu atveju, jei žinome 3 iš 4 šios
tapatybės kampų, tai galime nesunkiai rasti ir ketvirtąjį.

%keturi spinduliai iš vieno taško
\begin{center}
\begin{asy}
import olympiad;
size(200);
pair O, A, B, C, D;
O=origin;
A=(85,20); B=(90,5); C=(90,-10); D=(85,-25);
dot(A,blue);
dot(O,blue);
dot(B,blue);
dot(C,blue);
dot(D,blue);
label("$A$",A,NE,blue);
label("$B$",B,NE,blue);
label("$C$",C,NE,blue);
label("$D$",D,NE,blue);
label("$O$",O,N,blue);
draw(A--O);
draw(B--O);
draw(C--O);
draw(D--O);
\end{asy}
\end{center}

\subsubsection{Uždaviniai}

\begin{enumerate} 
\item Įrodykite tapatybę viršuje.
  %Tegu $\angle AOB = a, \angle BOC = b, \angle COD = c$.  Tada $\angle AOC
  %+\angle BOD = (a+b)+(b+c)=(a+b+c)+b= \angle AOD + \angle BOC$.
\item Ant trikampio $ABC$ kraštinių $AB, BC, CA$ atitinkamai paimti taškai
  $C_1, A_1, B_1$. Ar atkarpų $AA_1, BB_1, CC_1$ vidurio taškai gali būti
  vienoje tiesėje? 
  %Ne, negali. Tarkime priešingai. $AA_1, BB_1, CC_1$ vidurio taškai yra
  %ant trikampio $ABC$ vidurio linijų, lygiagrečių atitinkamai kraštinėms
  %$BC, AC, AB$. Tačiau šios vidurio linijos sudaro trikampį, o 
  %linija kerta  trikampį daugiausia dviejuose taškuose, o mums
  %reikia trijų. Prieštara.
\item  Ant lygiagretainio $ABCD$ kraštinės $AB$ (arba ant jos tęsinio)
  paimtas taškas $M$ toks, kad $\angle MAD = \angle AMO$, kur $O$ -
  lygiagretainio įstrižainių sankirtos taškas. Įrodyti, kad $MD=MC$. 
  %Tegu $MO$ kerta CD taške $N$. Tada $MNDA$ arba $MDNA$ yra lygiašonė
  %trapecija, priklausomai nuo to, ar $M$ yra ant $AB$, ar ant jos tęsinio.
  %Tada ji yra tokia pati kaip ir trapecija $MBCN$ arba, kitu atveju,
  %trapecija $BMCN$.  Tada šių trapecijų visos įstrižainės yra lygios, ir
  %todėl $MD = MC$.
\item Ant trikampio $ABC$ kraštinių $AB, BC, CA$ atitinkamai paimti taškai
  $C', A', B'$. Žinoma, kad $\angle AC'B' = \angle B'A'C$, $\angle CB'A' =
  \angle A'C'B$, $\angle BA'C' = \angle C'B'A$. Įrodyti, kad $A', B', C'$ -
  kraštinių vidurio taškai.
  %Tegu $\angle AC'B' = b$, $\angle CB'A' =a$, $\angle BA'C' = c$. Sudėję
  %trikampių $AB'C'$, $A'BC'$, $A'B'C$ kampus, mes gauname $540^\circ=
  %180^\circ+180^\circ +180^\circ = (\angle A +b+c) + (c+a+\angle B ) +
  %(\angle C +a+b) = 2(a+b+c)+180^\circ$, todėl $a+b+c = 180^\circ$. Taigi
  %$ \angle A = a ,\angle B = b, \angle C =c $. Todėl $AB\parallel{A'B'},
  %BC\parallel{B'C'}, AC\parallel{A'C'}$.  Tada iš Talio teoremos
  %$\frac{CA'}{A'B} = \frac{AC'}{C'B} = \frac{B'A}{B'C} = \frac{A'B}{CA'}$,
  %taigi $A'$ yra $BC$ vidurio taškas.  Panašiai su $B'$ ir $C'$.
\item Duotas trikampis, jo pusiaukampinių sankirtos taškas sujungtas su
  viršūnėmis, ir taip gauti 3 mažesni trikampiai. Vienas jų panašus į
  pradinį.  Rasti trikampio kampus.
  %Tegu pradinio trikampio kampai būna lygūs $2a, 2b, 2c$, $a \geq b \geq
  %c$. Pastebėkime, kad visi 3 gautieji trikampiai yra bukieji (įrodykite
  %tai!), o 2 iš jų turi kampą, lygų $c$. Tačiau pradinio trikampio visi
  %kampai didesni už $c$, taigi trikampis su kampais $a, b, 180^\circ-a-b$
  %yra panašus į pradinį. Todėl $180^\circ-a-b = 2a $, $a = 2b$, $b = 2c$.
  %Išsprendę gauname, kad trikampio kampai yra lygūs $\frac{180^\circ}{7},
  %\frac{360^\circ}{7}, \frac{720^\circ}{7}$. 
\item Duotas lygiagretainis $ABCD$, $M-DC$ vidurio taškas, $H$ - taško $B$
  projekcija į $AM$. Įrodyti, kad $BCH$ yra lygiašonis.
  %Tegu $AM$ kerta tiesę $CB$ taške $N$. Tada trikampiai $ADM$ ir $CMN$ yra
  %vienodi, todėl $BC = AD = CN$. Tada $HC$ yra stačiojo trikampio $BHN$
  %pusiaukraštinė iš stačiojo kampo, ir todėl $HC = CB$. 
\item Trikampyje $ABC$ $BE$ ir $CF$ yra aukštinės, o $D$ yra $BC$ vidurio
  taškas. Jei $DEF$ yra lygiakraštis, tai ar būtinai $ABC$ taip pat yra
  lygiakraštis?
  %Ne, $ABC$ nebūtinai yra lygiakraštis. Pavyzdžiui, jeigu $ABC$ yra
  %smailusis $\angle A = 60^\circ$, tai tada pagal stačiojo trikampio
  %pusiaukraštinės savybę $ED = \frac{CB}{2} = DF$, $\angle EDF = \angle
  %EDB + \angle FDC - 180^\circ = 2(\angle C + \angle B) - 180^\circ =
  %60^\circ$ iš priekampio savybės ir pirmojo uždavinio tapatybės. Todėl
  %$ABC$ nebūtinai lygiakraštis. ($ABC$ nebūtinai turi būti smailusis.
  %Galite papildomai įrodyti, kad jei $DEF$ yra lygiakraštis, tai visada 
  %$\angle A = 60^\circ$).
\item Duotas trikampis $ABC$ su $\angle A = 60^\circ$. $N$ yra $AC$ vidurio
  statmens ir $AB$ sankirta, o $M$ yra $AB$ vidurio statmens ir $AC$
  sankirta. Įrodyti, kad $MN = BC$.
  %Pastebėkime, kad $ABM$ yra lygiašonis. Bet $\angle A = 60^\circ$, todėl
  %$ABM$ lygiakraštis. Panašiai ir $ACN$ lygiakraštis. Todėl $CMBN$  yra
  %lygiašonė trapecija ir todėl $MN = BC$.
\item $M$ ir $N$ yra atitinkamai kvadrato $ABCD$ kraštinių $BC$ ir $AD$
  vidurio taškai. $K$ yra bet koks taškas ant spindulio $CA$ už taško $A$.
  $KM$ ir $AB$ kertasi taške $L$. Įrodyti, kad $\angle KNA = \angle LNA$.
  %Tegu tiesės $AB$ ir $KN$ kertasi taške $E$, o $AC$ ir $NM$ taške $F$.
  %Akivaizdžiai $NF=FM$, $LE \parallel{NM}$, todėl iš trikampių $KNM$ ir 
  %$KEL$ panašumo $LA = AE$. Tada $AN$ yra
  %$LE$ vidurio statmuo ir taigi $\angle KNA = \angle LNA$.
\item Duotas trikampis $ABC$ toks, kad kampo $A$ pusiaukampinė, kraštinės
  $AB$ vidurio statmuo ir aukštinė iš taško $B$ kertasi viename taške.
  Įrodyti, kad kampo $A$ pusiaukampinė, kraštinės $AC$ vidurio statmuo ir
  aukštinė iš taško $C$ kertasi viename taške.
  %Tegu kampo $A$ pusiaukampinė ir kraštinės $AB$ vidurio statmuo kertasi
  %taške $E$, o $BH$ ir $CF$ yra aukštinės.  Tada $AEB$ yra lygiašonis.
  %Taigi $\angle ABE = \angle BAE = \angle EAC$, ir iš čia $\angle A =
  %60^\circ$. Tegu kampo $A$ pusiaukampinė ir $CF$ kertasi taške $K$. Tada
  %$\angle KAC = \angle KCA = 30^\circ$, taigi $AC$ vidurio statmuo eina
  %per tašką $K$.
\item Duotas iškilasis keturkampis $ABCD$ toks, kad  jo įstrižaines
  statmenos ir kertasi taške $O$, $BC = AO$.  Taškas $F$ paimtas toks, kad
  $CF \perp{CD}$ ir $CF = BO$.  Įrodyti, kad $ADF$ yra lygiašonis. 
  %Iš Pitagoro teoremos $AD^2 = AO^2 + OD^2 = BC^2 + OD^2 = BO^2 + OC^2 +
  %OD^2 = CF^2 + CD^2 = DF^2$, taigi $AD = DF$.
\item Duotas trikampis $ABC$ su $\angle ACB = 70^\circ$, o jo viduje yra
  taškas $M$ toks, kad $\angle BAM = \angle ABC$, $\angle AMB = 100^\circ$.
  Įrodyti, kad $BM<AC$.
  %Paimkime ant kraštinės $BC$ tašką $K$ tokį, kad $\angle AKC = 80^\circ$.
  %Tada trikampiai $AMB$ ir $AKB$  yra vienodi pagal du kampus ir kraštinę.
  %Taigi $BM = AK$, bet $AK < AC$ iš trikampio nelygybės pritaikytos
  %trikampiui $AKC$, ko ir reikėjo. 
\item Įrodyti, kad keturkampio kraštinių vidurio taškai yra lygiagretainio
  viršūnės.
  %Keturkampio, sudaryto iš keturių kito keturkampio kraštinių vidurio
  %taškų, priešingos kraštinės lygiagrečios viena kitai iš trikampio
  %vidurio linijos savybės. 
\item Ant tiesės $a$ rasti tašką $G$ tokį, kad $AG+BG$ butų mažiausias, kur
  $A$ ir $B$ yra taškai toje pačioje tiesės pusėje.
  %Tegu taškas $B'$ yra simetriškas taškui $B$ tiesės $a$ atžvilgiu. Tegu
  %$AB'$ kerta $a$ taške $G$. Šis taškas ir yra reikiamas taškas: jeigu $H$
  %yra koks nors kitas taškas ant $a$, tai tada $AH + BH = AH + B'H \geq
  %AB' = AG + GB' = AG + GB$ iš trikampio nelygybės.
  %\begin{center}
  %\begin{asy}
  %import olympiad; size(200); pair A, B, BB, F, G, H;
  %draw((0,0)--(100,0));
  %label("$a$",(5,0),N,blue);
  %A=(15,25); B=(90,10); H=(45,0);
  %BB=conj(B);
  %G=extension((0,0),(100,0),A,BB);
  %F=extension((0,0),(100,0),B,BB);
  %dot(A,blue);
  %dot(B,blue);
  %dot(BB,blue);
  %dot(G,blue);
  %dot(H,blue);
  %dot(F,blue);
  %label("$A$",A,NE,blue);
  %label("$B$",B,NE,blue);
  %label("$B'$",BB,NE,blue);
  %label("$H$",H,N,blue);
  %label("$G$",G,N,blue);
  %label("$F$",F,NE,blue);
  %draw(A--BB); 
  %draw(A--H);
  %draw(H--B);
  %draw(H--BB);
  %draw(G--B);
  %draw(G--BB);
  %draw(F--B);
  %draw(F--BB);
  %add(pathticks(H--B,2,0.5,30,100,red));
  %add(pathticks(H--BB,2,0.5,30,100,red));
  %add(pathticks(F--B,2,0.5,0,100,red));
  %add(pathticks(F--BB,2,0.5,0,100,red));
%\end{asy}
  %\end{center}
\item Duotas kvadratas $ABCD$, jo viduje taškas $M$.  Įrodykite, kad
  trikampių $ABM$, $BCM$, $CDM$ ir $DAM$ pusiaukraštinių susikirtimo taškai
  taip pat yra kvadrato viršūnes.
  %Tegu kvadratas būna $ABCD$, o kraštinių vidurio taškai $E, F, G, H$, bei
  %trikampių $ABM$, $BCM$, $CDM$, $DAM$ pusiaukraštinių susikirtimo taškai
  %$P, O, N, L$ - taip, kaip parodyta paveikslėlyje žemiau. Tada $EFGH$ yra
  %taip pat kvadratas. Iš pusiaukraštinių sankirtos taško savybės,
  %$\frac{MP}{MH} = \frac{MO}{MG} = \frac{2}{3}$, todėl iš Talio teoremos
  %arba panašiųjų trikampių, $PO \parallel{HG}$ ir $\frac{PO}{HG} =
  %\frac{2}{3}$. Panašiai su kitomis kraštinėmis. Todėl $PONL$ turi
  %gretimas kraštines statmenas, o taip pat visos kraštinės lygios. Taigi
  %$PONL$ yra kvadratas. 
  %\begin{center}
  %\begin{asy}
  %import olympiad;
  %size(200);
  %pair A, B, C, D, E, F, G, H, L, N, O, P, M;
  %A=(0,100);
  %B=(0,0);
  %C=(100,0);
  %D=(100,100);
  %M=(45,65);
  %E=(A+D)/2;
  %F=(C+D)/2;
  %G=(B+C)/2;
  %H=(A+B)/2;
  %P=M+(H-M)*2/3;
  %L=M+(E-M)*2/3;
  %N=M+(F-M)*2/3;
  %O=M+(G-M)*2/3;
  %dot(A,blue);
  %dot(B,blue);
  %dot(C,blue);
  %dot(D,blue);
  %dot(E,blue);
  %dot(G,blue);
  %dot(H,blue);
  %dot(F,blue);
  %dot(L,blue);
  %dot(M,blue);
  %dot(N,blue);
  %dot(O,blue);
  %dot(P,blue);
  %label("$A$",A,NW,blue);
  %label("$B$",B,SW,blue);
  %label("$C$",C,SE,blue);
  %label("$D$",D,NE,blue);
  %label("$E$",E,NE,blue);
  %label("$H$",H,W,blue);
  %label("$G$",G,S,blue);
  %label("$F$",F,NE,blue);
  %label("$L$",L,NE,blue);
  %label("$N$",N,NE,blue);
  %label("$O$",O,SE,blue);
  %label("$P$",P,NW,blue);
  %draw(A--B--C--D--cycle);
  %draw(L--N--O--P--cycle);
  %draw(H--E--F--G--cycle);
  %draw(H--M);
  %draw(E--M);
  %draw(F--M);
  %draw(G--M);
  %draw(A--M);
  %draw(B--M);
  %draw(C--M);
  %draw(D--M);
  %label("$M$",M,NE,blue);
%\end{asy}
  %\end{center}
\item Duotas trikampis $ABC$. Per jo viršūnes $A$ ir $B$ išvestos dvi
  tiesės, kurios padalina šitą trikampį į 4 figūras: 3 trikampius ir vieną
  keturkampi. Žinoma, kad trijų iš šių figurų plotai vienodi. Įrodykite,
  kad tarp tų trijų yra keturkampis.
  %Tegu ta tiesė per tašką $A$ kerta $BC$ taške $A'$, tiesė per $B$ kerta
  %$AC$ taške $B'$, o $AA'$ ir $BB'$ kertasi taške $M$. Tarkime, kad tarp
  %trijų figūrų nėra keturkampio. Tada visų trijų trikampių plotai yra
  %lygūs.  Tada trikampiai $ABM$ ir $AMB'$ turi bendrą aukštinę iš taško
  %$A$ ir vienodą plotą. Todėl $MB$ = $MB'$. Panašiai ir $AM$ = $MA'$.
  %Todėl $ABA'B'$ yra lygiagretainis. Tai neįmanoma, nes tada $AC
  %\parallel{CB}$.
\item Trikampyje dvi aukštinės yra ne trumpesnės nei kraštinės į kurias jos
  remiasi. Rasti trikampio kampus. 
  %Bet kokiame stačiajame trikampyje statinis trumpesnis už įžambinę. Jei
  %mūsų trikampis yra $ABC$, su aukštinėmis $AA'$ ir $BB'$, bei $AA'
  %\geq BC$ ir $BB' \geq AC$, tai tada $AA' \geq BC \geq BB' \geq AC \geq
  %AA'$, su lygybėmis tada ir tik tada jei $ABC$ lygiašonis statusis.
  %Todėl kampai yra lygūs $90^\circ ,45^\circ ,45^\circ$.
\item Duotas trikampis $ABC$ su $\angle A = 60^\circ$, pusiaukraštinė $CM$
  ir aukštinė $BN$ kertasi taške $K$, $CK=6$, $KM=1$. Rasti trikampio $ABC$
  kampus.
  %Tegu $P$ yra atkarpos $BN$ vidurio taškas. Tada $MP \parallel{AN}
  %\parallel{NC}$, taigi $MPCN$ yra trapecija.  Todėl trikampiai $PKM$ ir
  %$CKN$ yra panašūs.  Tada $6=\frac{CK}{KM}=\frac{NC}{PM}$, taigi
  %$\frac{AC}{AB} = \frac{AN+NC}{AB} = \frac{AN}{AB} + \frac{NC}{AB} =
  %\frac{1}{2} + \frac{NC}{MP}\frac{MP}{AN}\frac{AN}{AB} =\frac{1}{2} +
  %6\cdot\frac{1}{2}\cdot\frac{1}{2} = 2$. Todėl trikampio $ABC$ kampai yra lygūs
  % $30^\circ, 60^\circ, 90^\circ$. 
\item Ant trikampio $ABC$ kraštinių $AB$ ir $BC$ atitinkamai paimti taškai
  $D$ ir $E$ tokie, kad $\frac{AD}{DB} =\frac{BE}{EC}=2$ ir $\angle ACB
  =2\angle DEB$. Įrodyti, kad $ABC$ lygiašonis.
  %Tegu $AB$ ilgis būna $6x$, o $M$ kraštinės $AB$ vidurio taškas. Tada $BM
  %= 3x$, $DB = 2x$, taigi $\frac{MB}{DB} = \frac{3}{2} = \frac{CB}{EB}$.
  %Todėl trikampiai $CMB$ ir $EDB$ yra panašūs ir dėl to $\angle BCM =
  %\angle BED = \frac{\angle ACB}{2}$, todėl $CM$ yra trikampio $ABC$
  %pusiaukampinė ir pusiaukraštinė tuo pat metu, taigi $ABC$
  %lygiašonis($AC$ = $CB$).
\item Duotas trikampis $ABC$. $D$-kraštinės $AC$ vidurio taškas. Ant $BC$
  paimtas taškas $E$ toks, kad $\angle BEA =\angle CED$. Rasti
  $\frac{AE}{DE}$.
  %Tegu tiesė, lygiagreti $DE$ eina per viršūnę $A$ ir kerta kraštinę $BC$
  %taške $M$. Tada $ED$ yra trikampio $AMC$ vidurio linija bei $\angle AMC
  %= \angle DEC = \angle AEB$, todėl $AME$ yra lygiašonis. Taigi
  %$\frac{AE}{DE} = \frac{AM}{DE} = 2$.
\end{enumerate}

\newpage
\section{Panašieji trikampiai ir brėžinio papildymai}

Spręsdami praeito skyrelio uždavinius ar skaitydami jų sprendimus
greičiausiai pastebėjote, kad daugumoje jų reikėjo kažką papildomai
pažymėti - tašką, tiesę ar atkarpą, ir tik tada sprendimas tapdavo poros
 eilučių ilgio (pavyzdyje žemiau matote retą išimtį, kai užtenka originalaus 
 brėžinio). Sugebėjimas pastebėti, ką pribrėžti yra turbūt svarbiausias 
 raktas sėkmingam olimpiadinių uždavinių sprendimui, nes beveik visų 
 uždavinių sprendimai įtraukia kitų objektų (taškų ar tiesių), nei duota sąlygoje.
Šiame skyrelyje bus duota keletas paprastų patarimų, kurie kartais suteikia
idėjų, ką ir kur pribrėžti. 

\begin{pav}
  Duota trapecija $ABCD$ su pagrindais $AD$ ir $CB$. Įstrižainės kertasi
  taške $E$. Per $E$ išvesta tiesė, lygiagreti pagrindams, kerta $AB$ taške
  $F$ ir $CD$ taške $G$. Įrodyti, kad $GE = FE$.
\end{pav}

%Trapecija ir tiesė per įstrižainių susikirtimo tašką lygiagreti pagrindams
\begin{center}
\begin{asy}
import olympiad;
size(200);
pair A, B, C, D, E, F, G, X;
A=(0,0);
B=(20,40);
C=(60,40);
D=(100,0);
E=extension(A,C,B,D);
X=E+(10,0);
F=extension(A,B,E,X);
G=extension(C,D,E,X);
add(anglem(D,A,E,400,blue,1));
add(anglem(G,E,C,400,blue,1));
add(anglem(E,D,A,400,red,0));
add(anglem(B,E,F,400,red,0));
dot(A,blue);
dot(B,blue);
dot(C,blue);
dot(D,blue);
dot(E,blue);
dot(G,blue);
dot(F,blue);
label("$A$",A,SW,blue);
label("$B$",B,NW,blue);
label("$C$",C,NE,blue);
label("$D$",D,SE,blue);
label("$E$",E,N,blue);
label("$F$",F,NW,blue);
label("$G$",G,NE,blue);
draw(A--B--C--D--cycle);
draw(A--C);
draw(B--D);
draw(F--G);
\end{asy}
\end{center}

\begin{sprendimas}
  Trikampiai $BEF$ ir $BAD$, $BEC$ ir $AED$, $CEG$ ir $CAD$ yra panašūs.
  Taigi, $\frac{AD}{FE} = \frac{BD}{BE} = \frac{BE+ED}{BE} = 1 +
  \frac{ED}{BE} = 1 + \frac{EA}{EC} = \frac{AC}{EC} = \frac{AD}{EG}$, taigi
  $EG = FE$.
\end{sprendimas}

\subsubsection{Panašieji ir vienodieji trikampiai}

Geometrijos uždavinių brėžiniuose pagal sąlygą visada reikia susižymėti 
lygius kampus ir lygias atkarpas. Taip yra dėl to, kad tada nesunkiai galima
pastebėti panašiuosius trikampius, o kaip vėliau pamatysime, ir įbrėžtinius 
keturkampius bei kitokias figūras. Panašieji trikampiai yra vienas svarbiausių
sprendimo būdų olimpiadinėje geometrijoje, todėl juos pastebėti yra labai svarbu.
Vis dėlto, jų kartais brėžinyje nebūna ir pasimato tik papildžius brėžinį.
Todėl dažnai brėžinį papildyti reikia taip, kad atsirastų panašiųjų
trikampių. Tai gali atrodyti per daug abstraktus patarimas, bet yra keletas 
idėjų, kurios dažnai pasiteisina:
\begin{itemize}
\item Kad atsirastų panašus trikampis, dažniausiai tereikia nubrėžti tik
viena atkarpą. Jei turime trikampį $S$, ir atrodo, kad būtų naudinga turėti
kitą trikampį, panašų į $S$, tai brėžinyje verta ieškoti kampo, kuris lygus
vienam iš $S$ kampų (tam, žinoma, reikia būti susižymėjus lygius kampus brėžinyje).
 Tada jį galime panaudoti kaip pagrindą panašiojo trikampio 
statybai. 
\begin{pav}
Smailiajame trikampyje $ABC$ ant $AC$ ir $AB$ atitinkamai paimti taškai $K$
ir $L$ taip, kad $KL \parallel BC$ ir $KL = KC$. Ant kraštinės $BC$ paimtas
taškas $M$ taip, kad $\angle KMB = \angle BAC$. Įrodyti, kad $KM = AL$. 
\end{pav}

%Smailiajame trikampyje ant kraštinių paimti taškai..
\begin{center}
\begin{asy}
import olympiad;
size(200);
pair A, B, C, K, L, M, N;
A=(65,60); K=(0,0); L=(100,0);
draw(A--K--L--cycle);
C=rotate(degrees(A)-180,K)*L;
B=extension(C,C+L-K,A,L);
draw(A--B--C--cycle);
N=C+2*(foot(K,C,B)-C);
M=extension(C,B,K,rotate(degrees(L-A),K)*N);
add(anglem(K,A,L,600,green,0));
add(anglem(N,M,K,600,green,0));
add(anglem(B,C,A,600,blue,1));
add(anglem(L,K,A,600,blue,1));
add(anglem(K,N,C,600,blue,1));
draw(K--N);
draw(K--M);
dot(A,blue);
dot(B,blue);
dot(C,blue);
dot(K,blue);
dot(L,blue);
dot(M,blue);
dot(N,blue);
label("$A$",A,NE,blue);
label("$B$",B,SE,blue);
label("$C$",C,SW,blue);
label("$K$",K,NW,blue);
label("$L$",L,NE,blue);
label("$M$",M,SW,blue);
label("$N$",N,SE,blue);
add(pathticks(K--L,2,0.5,0,200,red));
add(pathticks(K--N,2,0.5,0,200,red));
add(pathticks(K--C,2,0.5,0,200,red));
\end{asy}
\end{center}

\begin{sprendimas}
  Paimkime tašką $N$ (kitą negu $C$) ant $BC$ taip, kad $KL = KC = KN$.
  Tada $KCN$ yra lygiašonis, taigi $\angle LKA = \angle BCA = \angle KNM$.
  Pagal sąlygą $\angle KMN = \angle LAK$, taigi trikampiai $KNM$ ir $LKA$
  yra vienodi pagal kraštinę ir 3 kampus, ir todėl $LA = KM$. 
\end{sprendimas}
\item Vienodus trikampius galime pribrėžti susiradę ne tik vienodus kampus, bet ir 
vienodas atkarpas, t.y. jei trikampis $S$ turi kraštinę, lygią $a$, o brėžinyje
 yra kita kraštinė, lygi $a$, tai ją galima pabandyti panaudoti trikampio,
 tokiam pačiam kaip ir  $S$, pagrindui ( Galima sakyti, padarome trikampio kopiją).
  Dažnai uždavinio sąlyga sufleruoja, kur tai daryti. 
 \begin{pav}
  Kvadrate $ABCD$ ant kraštinių $BC$ ir $CD$ atitinkamai yra
  paimti taškai $K$ ir $M$ taip, kad $AM$ yra kampo $\angle
  KAD$ pusiaukampinė. Įrodyti, kad $AK = DM + BK$.
\end{pav}       

\begin{proof}[Sprendimas]
Geriausia būtų kaip nors panaudoti tai ko prašo, t.y $AK =
DM + BK$. Trikampio $KAB$ kraštinė $AB$
yra tokio pat ilgio, kaip ir kitos kvadrato kraštinės, tai
prie vienos jų galima perkelti trikampį $KAB$. Mes
pasirenkame kraštinę $AD$ - taigi pastatome trikampį, tokį
patį, kaip ir $KAB$ prie kraštinės $AD$. Tada $DM + BK = DM +
ED = EM$ - tai jau nemažas pasiekimas, nes radome atkarpą,
kurios ilgis yra $DM + BK$. Taigi reikia įrodyti, kad $EM =
AK$, arba, kad $EM = AE$. Tam tereikia įrodyti, kad $AEM$ yra
lygiašonis, kas beveik akivaizdu: $\angle AMD = \angle MAB =
\angle MAE$, ko ir reikėjo.
%Trikampių perkėlimas kvadrate
\begin{center}
\begin{asy}
import olympiad;
size(200);
pair A, B, C, D, E, K, M;
D=(0,0);
A=(0,60);
B=(60,60);
C=(60,0);
M=waypoint(D--C,0.7);
K=extension(A, rotate(degrees((M-A)/(D-A)),A)*M, B, C);
E=(K-A)*(0,-1) + A;
add(anglem(B,K,A,200,red,0));
add(anglem(D,E,A,200,red,0));
add(anglem(D,A,M,200,blue,2));
add(anglem(M,A,K,200,blue,2));
add(anglem(E,A,D,200,green,1));
add(anglem(K,A,B,200,green,1));
draw(A--B--C--D--cycle);
draw(A--E);
draw(A--M);
draw(A--K);
draw(E--D);
dot(A,blue);
dot(B,blue);
dot(C,blue);
dot(D,blue);
dot(K,blue);
dot(M,blue);
dot(E,blue);
label("$A$",A,NW,blue);
label("$B$",B,NE,blue);
label("$C$",C,SE,blue);
label("$D$",D,SW,blue);
label("$K$",K,right,blue);
label("$M$",M,SW,blue);
label("$E$",E,SW,blue);
add(pathticks(A--B,2,0.5,0,100,red));
add(pathticks(A--D,2,0.5,0,100,red));
add(pathticks(A--K,2,0.5,30,100,red));
add(pathticks(A--E,2,0.5,30,100,red));
\end{asy}
\end{center}
\end{proof}
\end{itemize}
Jeigu neturite jokių idėjų, ką reiktų pribrėžti, tai tada bandykite išvesti
daugybę statmenų ir tada ieškoti panašių trikampių ir sudarinėti „lygybių
grandinėles“, ieškoti panašių stačiųjų trikampių. Tokiu atveju sprendimą
užrašyti būna sunkiau ir jis būna ilgesnis, tačiau nereikia spėlioti, ką
pribrėžinėti ir kaip. 

\subsubsection{$X + Y = Z$} 

Kartais pasitaiko, kad uždavinio sąlygoje duota, kad kažkurių dviejų
atkarpų ilgių, tarkime $a$ ir $b$, suma yra lygi kažkokios trečios atkarpos
ilgiui $c$. Panašiai pasitaiko, kad kažką tokio reikia įrodyti, kaip
pavyzdyje aukščiau.
 Norint panaudoti tokią iš pažiūros keistoką sąlygą dažniausiai
tenka veikti taip: arba ant atkarpos, kurios ilgis $c$, pažymėti vieną iš
taškų, dalinančių ją į atkarpas ilgių $a$ ir $b$, ir tada bandyti panaudoti
tą tašką, arba pratęsti vieną iš trumpesniųjų atkarpų tiek, kad gautume
atkarpą, kurios ilgis lygus $c$. Tuomet pratęsimo ilgis bus lygus $B$.
Taip brėžinyje atsiras nauja pora lygių atkarpų. Kartais vien to neužtenka: atkarpų,
kurių ilgis yra $a$, $b$ arba $c$ gali būti daugiau nei viena ir
dažniausiai jos būna „pasislėpę“ ir jas iš pradžių reikia surasti, ir tik
tada pritaikyti šitą fokusą. Be to, jis ne visada suveikia (nors dažnai vos
pabandžius iš karto matosi ar suveiks, ar ne).

\begin{pav}[LitMO 2010] 
  Duota trapecija $ABCD$ su $AB \parallel CD$ ir $AB + CD = BC$. Įrodyti,
  kad kampų $B$ ir $C$ pusiaukampinės kertasi ant $AD$.  
\end{pav}

%Litmo ????
\begin{center}
\begin{asy}
import olympiad;
size(130);
pair A, B, C, D, E;
B=origin; E=(15,60);
A=rotate(-degrees(E))*E;
C=E*1.3;
D=rotate(180-degrees(E),C)*E;
dot(A,blue);
dot(B,blue);
dot(C,blue);
dot(D,blue);
dot(E,blue);
draw(A--B--C--D--cycle);
draw(A--E);
draw(D--E);
label("$A$",A,right,blue);
label("$B$",B,W,blue);
label("$C$",C,W,blue);
label("$D$",D,right,blue);
label("$E$",E,NW,blue);
add(rightanglem(A,E,D,180));
drawline(B,bisectorpoint(A,B,E));
drawline(C,bisectorpoint(B,C,D));
add(pathticks(C--D,2,0.5,30,100,red));
add(pathticks(C--E,2,0.5,30,100,red));
add(pathticks(E--B,2,0.5,0,100,red));
add(pathticks(A--B,2,0.5,0,100,red));
\end{asy}
\end{center}

\begin{sprendimas}
  Kaip ir sako patarimas, pažymėkime ant $BC$ tašką $E$ tokį, kad $BE = AB$
  ir $CE = CD$. Tada trikampiai $ABE$ ir $CED$ yra lygiašoniai. Be to,
  $\angle AED = 180^\circ - \angle AEB - \angle DEC = 180^\circ -
  \frac{180^\circ - \angle ABE}{2} - \frac{180^\circ - \angle DCE}{2} =
  \frac{\angle ABE + \angle DCE}{2} = 90^\circ$. Kampų $B$ ir $C$
  pusiaukampinės yra stačiojo trikampio $ABE$ kraštinių $AE$ ir $DE$
  vidurio statmenys, kurie akivaizdžiai kertasi ant įžambinės vidurio
  taško, ko ir reikėjo.
\end{sprendimas}  

Panašiai galima elgtis ir su uždaviniais su sąlyga „$\angle A + \angle B =
\angle C$. Pirmiausia reiktų išreikšti visus svarbiausius kampus brėžinyje
per keletą kintamųjų, ir tada bandyti geometriškai interpretuoti tą sąlygą.

\subsubsection{Nuo kurio taško pradėti?}
Retais atvejais pasitaiko, kad sąlyga liepia brėžtis figūrą,
pvz. trikampį, kuris neturi jokių ypatingų bruožų, bet iš sąlygos paaiškėja,
kad nusibrėžę mes gauname netikslų brėžinį. Pabandykite nubrėžti
brėžinį šiam uždaviniui:

\begin{pav}
Duotas trikampis $ABC$, ant kampo $B$ pusiaukampinės paimtas taškas $M$
taip, kad $AC=AM,\angle BCM=30^\circ$. Rasti $\angle AMB$.
\end{pav}

Nesunku matyti, kad tikrai ne bet kuriam trikampiui $ABC$ tai pavyktų padaryti:
$\angle MCB$ nėra lygus $30^\circ$ visiems trikampiams $ABC$. Kadangi nežinome kokių
sąlygų reikia, kad $\angle MCB=30^\circ$, tai negalėsime nusibrėžti absoliučiai
tikslaus brėžinio, net ir su matlankiu bei liniuote. Galite pagalvoti, kad tai menkas
nuostolis - apytikslis brėžinys yra taip pat puikus. Tačiau yra brėžimo būdas, kuris
ne tik padeda nusibrėžti tokius brėžinius tiksliai, bet ir kartais suteikia naujų idėjų
sprendimui. Tai yra \textit{BRĖŽIMAS IŠ KITO GALO}.

Kaip pavadinimas sako, reikia brėžti iš kito galo. Tam pirmiausiai brėžiame ne
taškus $A,B,C$, o kampą $B$ ir jo pusiaukampinę. Tada paimame bet kokį tašką $M$
ant pusiaukampinės. Tada imame bet tašką $C$ ant kampo kraštinės taip, kad 
$\angle MCB=30^\circ$. Tada galiausiai $CM$ vidurio statmens ir kitos kampo kraštinės
sankirta pažymima $A$. Galite įsitikinti kad dabar brėžinys tenkina visas sąlygas,
nors mes pakeitėme tik taškų brėžimo tvarką.

\subsubsection{Uždaviniai}

\begin{enumerate}
\item Lygiagretainyje $ABCD$ $AB + CD = AC$. Ant kraštinės $BC$ yra taškas
  $K$ toks, kad $\angle ADB = \angle BDK$.  Raskite $\frac{BK}{KC}$.
  %Tegu $E$ yra lygiagretainio įstrižainių sankirtos taškas, o $F$ yra $ED$
  %vidurio taškas. Tada $\angle BDK = \angle BDA = \angle DBK$, tad
  %trikampis $BDK$ lygiašonis ir todėl $KE \perp{BD}$. Be to, $CD =
  %\frac{CA}{2} = CE$, tagi $DCE$ taip pat lygiašonis, todėl $CF
  %\perp{BD}$. Iš Talio teoremos, $\frac{BK}{KC} = \frac{BE}{EF} = 2$.  
\item Trapecijos $ABCD$ ($AD, BC$ - pagrindai) įstrižainė $AC = AD + CD$, o
  kampas tarp įstrižainių yra lygus $60^\circ$. Įrodyti, kad trapecija yra
  lygiašonė.
  %Pažymėkime ant $AC$ tašką $E$ taip, kad $AE = AD$ ir $CE = CB$. Tegu
  %trapecijos įstrižainės kertasi taške $F$.  Tada $ADF$ ir $CBF$ yra
  %panašūs trikampiai, todėl $\frac{AF}{CF} = \frac{AD}{BC} =
  %\frac{AE}{CE}$. Taigi taškai $E$ ir $F$ sutampa, ir todėl $60^\circ =
  %\angle AFD = \angle AED$. Dėl to $AED$ yra lygiašonis su kampu prie
  %pagrindo lygiu $60^\circ$, todėl yra lygiakraštis.  Panašiai ir $CEB$
  %lygiakraštis. Tada iš simetrijos trapecija yra lygiašonė.  
\item Duotas lygiašonis trikampis $ABC$, $AB = BC$. Ant kraštinių $AB$ ir
  $BC$ atitinkamai paimti taškai $K$ ir $L$ taip, kad $AK + LC = KL$.
  Linija, lygiagreti $BC$, nubrėžta per tašką $M$, kuris yra atkarpos $KL$
  vidurio taškas. Ši linija kerta kraštinę $AC$ taške $N$. Rasti kampą
  $\angle KNL$.
  %Paimkime tašką $D$ ant $AC$ taip, kad $KD\parallel{BC}$.  Tada $DKBC$ ir
  %$DKLC$ yra trapecijos, o $AKD$ lygiašonis.  Iš trapecijos vidurio
  %linijos formulės $MN = \frac{KD+LC}{2} = \frac{KA+LC}{2} = \frac{KL}{2}
  %= MK = ML$. Taigi $\angle LNK = 90^\circ$. 
\item Duota trapecija $ABCD$, $AD\parallel{BC}$. $K$ yra bet koks taškas
  ant $AB$. Nubrėžta linija per $A$, lygiagreti $KC$, ir linija per $B$,
  lygiagreti $DK$. Įrodyti, kad šios linijos kertasi ant $CD$.
  %Tegu $CK$ ir $AD$ kertasi taške $E$, $KD$ ir $BC$ taške $F$. Tegu
  %linija, lygiagreti $KC$ ir einanti per tašką $A$ kerta $CD$ taške $P$, o
  %linija, lygiagreti $DK$ ir einanti per tašką $B$ kerta $CD$ taške $Q$.
  %Reikia įrodyti, kad $P = Q$. Tai visai nesunku: $\frac{DP}{PC} =
  %\frac{DA}{AE} = \frac{FB}{BC} = \frac{DQ}{QC}$, taigi $P = Q$.
  %($\frac{DA}{AE} = \frac{FB}{BC}$, nes $\frac{DA}{FB} = \frac{AK}{BK} =
  %\frac{EA}{BC}$).  
\item Duotas trikampis $ABC$, $M$ - $AC$ vidurio taskas.  Taškas $D$ ant
  kraštinės $BC$ toks, kad $\angle BMA = \angle DMC$. Jei $CD + DM = BM$,
  įrodyti, kad $\angle ACB +\angle ABM=\angle BAC$.
  %Paimkime tašką $K$ ant $BM$ tokį, kad $BK = CD$ ir $KM = DM$.  Tada
  %$DMC$ ir $AMK$ vienodi pagal dvi kraštines ir kampą, taigi $KA = DC =
  %BK$. Todėl $BKA$ lygiašonis, ir dėl to $\angle BAC = \angle KAM + \angle
  %KAB = \angle DCM + \angle KBA = \angle MBA + \angle BCA$.
\item Duotas trikampis $ABC$, ant kraštinės $AC$ paimti taškai $K$ ir $L$
  taip, kad $L$ yra $AK$ vidurio taškas, o $BK$ - kampo $LBC$
  pusiaukampinė. Jei $BC = 2BL$, įrodyti, kad $KC = AB$. 
  %Tegu $M$ yra $BC$ vidurio taškas. Tada $BKL$ ir $BKM$ yra vienodi pagal
  %dvi kraštines ir kampą. Taigi, $\angle BLA = 180^\circ - \angle BLK =
  %180^\circ - \angle BMK = \angle CMK$. Bet $CM = BL$ ir $AL = KM$, taigi
  %$KMC$ ir $BLA$ yra vienodi. Taigi $KC = BA$. 
\item Duotas trikampis $ABC$. Linija, lygiagreti $AC$, kerta $AB$ ir $BC$
  atitinkamai taškuose $K$ ir $M$. $AM$ ir $KC$ kertasi taške $O$. Jei $KM
  = MC$ ir $AO = AK$, tai įrodykite, kad $AM = BK$. 
  %Trikampiai $KMC$ ir $KOA$ yra lygiašoniai. Todėl $\angle CMA = 180^\circ
  %- \angle MOC - \angle MCO = 180^\circ - \angle AKO - \angle MKC = \angle
  %MKB$. Taigi trikampiai $MCA$ ir $MKB$ yra vienodi pagal 2 kampus ir
  %kraštinę, todėl $AM = BK$.
\item Duotas iškilasis keturkampis $ABCD$ toks, kad $AC = BD$, be to,
  $\angle BAC = \angle ADB$, $\angle CAD + \angle ADC = \angle ABD$. Rasti
  kampą $\angle BAD$.
  %Tegu $AB$ ir $CD$ keratsi taške $E$. Iš trikampio priekampio savybės,
  %$\angle ABD = \angle ADC + \angle CAD = \angle ACE$. Tada trikampiai
  %$ACE$ ir $ADB$ yra vienodi pagal du kampus ir kraštinę, taigi $AE = AD$.
  %Bet $\angle EAD =\angle DEA$, taigi $AD = DE$. Todėl $ADE$ yra
  %lygiašonis ir $\angle BAD = 60^\circ$.
\item Duotas trikampis $ABC$, $AF$ pusiaukraštinė, $D$ yra $AF$ vidurio
  taškas, $E$ - $CD$ ir $AB$ sankirtos taškas.  Jei $BD = BF = CF$,
  įrodyti, kad $AE = DE$.
  %Trikampiai $ADB$ ir $DFC$ yra vienodi pagal 2 kraštines ir kampą, nes
  %$\angle DFC=180^\circ-\angle BFD=180^\circ-\angle BDF=\angle BDA$, taigi
  %$\angle DAE = \angle CDF = \angle EDA$, todėl $ADE$ yra lygiašonis. 
\item $ABCD$ yra iškilasis keturkampis su $\angle CBD = \angle CAB$,
  $\angle ACD = \angle BDA$. Įrodyti, kad $\angle ABC = \angle ADC$.
  %Tegu keturkampio įstrižainės kertasi taške $E$. Tada trikampiai $ADE$ ir
  %$ADC$ yra panašūs pagal du kampus.  Taigi $\angle ADC = \angle AED$. Bet
  %trikampiai $CEB$ ir $CAB$ taip pat panašūs pagal du kampus, taigi
  %$\angle AED = \angle CEB = \angle ABC$. 
\item Duotas trikampis $ABC$. $A_1, B_1, C_1$ yra atitinkamai $BC, CA, AC$
  vidurio taškai. Tada ant $C_1 B_1$ pratęsimo į $B_1$ pusę paimtas taškas
  $K$ toks, kad $B_1 K = \frac{BC}{4}$. Duota, kad  $AA_1 = BC$. Įrodyti,
  kad $AB = BK$.
  %Tegu $E$ yra $B_1C_1$ vidurio taškas. Iš panašiųjų trikampių, $AA_1$
  %eina per $E$. Be to, $KE = KB_1 + B_1E = \frac{BC}{4} + \frac{BC}{4} =
  %\frac{BC}{2} = CA_1$, taigi $CKEA_1$ yra lygiagretainis ir tada $CK =
  %A_1E = BA_1$. Bet $\angle AA_1B = \angle KCB$, taigi trikampiai $KCB$ ir
  %$AA_1B$ yra vienodi pagal 2 kraštines ir kampą.  Tad $AB = BK$.
\item Duotas kvadratas $ABCD$, ant $BC$ ir $CD$ atitinkamai paimti taškai 
  $E$ ir $F$ taip, kad $\angle EAF=45^\circ$. $BD$ kerta $AE$ taške $G$, o
  $FA$ taške $H$. Įrodyti, kad $GH^2=HD^2+BG^2$.
  %Tereikia įrodyti, kad iš atkarpų, kurių ilgiai yra $GH,HD,BG$ galima 
  %suformuoti statujį trikampį. Tam imame tašką $X$ ant spindulio $EB$ už $B$
  %taip, kad $XB=FD$. Tada trikampiai $AFD$ ir $ABX$ yra vienodi. Imame tašką
  %$P$ ant $AX$ taip, kad $\frac{XP}{PA}=\frac{FH}{HA}$. Tada $\angle ADH=\angle ABP$, 
  % $PB=DH, PA=HA$, $\angle GAH=45^\circ=PAG$, $\angle GBP=45^\circ+45^\circ=
  %90^\circ$. Trikampiai $GAH$ ir $GAP$ yra vienodi pagal  du kampus ir dvi 
  %kraštines, taigi $BGP$ yra mūsų ieškomas statusis trikampis.
\item Smailiajame trikampyje $ABC$ išvesta aukštinė $CH$.  Pasirodė, kad
  $AH = BC$. Įrodyti, kad kampo $B$ pusiaukampinė, aukštinė $AF$ iš kampo
  $A$ ir tiesė, einanti per $H$ ir lygiagreti $BC$, kertasi viename taške.
  %Tegu kampo $B$ pusiaukampinė kerta $AF$ taške $P$, o tiesė, lygiagreti
  %$BC$ ir einanti per tašką $H$, kerta $AF$ taške $Q$. Reikia įrodyti, kad
  %$P = Q$. Pastebėkime, kad $CHB$ ir $AFB$ yra panašūs. Tada iš
  %pusiaukampinės savybės ir Talio teoremos $\frac{FP}{PA} = \frac{FB}{BA}
  %= \frac{HB}{CB} = \frac{HB}{AH} = \frac{FQ}{QA}$. Taigi $P=Q$.
\item Trikampyje $ABC$ nubrėžtos pusiaukampinės $AA_1, BB_1, CC_1$. Jeigu
  $C_1A_1$ yra $\angle BC_1C$ pusiaukampinė, tai įrodykite, kad $B_1C_1$
  yra kampo $\angle AC_1C$ pusiaukampinė.
  %Iš pusiaukampinės savybės $\frac{C_1A}{C_1C} =
  %\frac{C_1A}{BC_1}\frac{BC_1}{C_1C} = \frac{AC}{BC} \frac{BA}{AC} =
  %\frac{BA}{BC} = \frac{AB_1}{B_1C}$, taigi $B_1C_1$ yra kampo $\angle
  %AC_1C$ pusiaukampinė. 
\item Duotas iškilasis keturkampis $ABCD$ toks, kad $\angle B = \angle C$
  ir $CD = 2AB$. Ant tiesės $BC$ parinktas taškas $X$ toks, kad $\angle BAX
  = \angle CDA$. Įrodyti, kad $AD = AX$.
  %Tegu tiesė per tašką $A$, lygiagreti $BC$, kerta $CD$ taške $E$. Tada
  %$ABCE$ yra lygiašonė trapecija, taigi $CE = AB = \frac{CD}{2} = ED$. Be
  %to, $\angle AED = \angle BCD = \angle ABC$, taigi trikampiai $ABX$ ir
  %$AED$ yra vienodi pagal kraštinę ir du kampus. Taigi $AD = AX$.
\item Duotas lygiakraštis trikampis $ABC$. Ant $AB, AC, BC$ atitinkamai
  parinkti $X, Y, Z$ taip, kad $BZ = 2AY$, $\angle XYZ = 90^\circ$.
  Įrodykite, kad $AX + CZ = XZ$.
  %Tegu tiesė, lygiagreti $AB$ ir einanti per tašką $Z$, kerta $AC$ taške
  %$V$, o $N$ tebūnie $XZ$ vidurio taškas.  Tada $ZVC$ yra lygiakraštis, o
  %$XZVA$ trapecija su vidurio linija $NY$. Be to, $NY$ taip pat yra
  %stačiojo trikampio $XZY$ pusiaukraštinė iš stačiojo kampo. Taigi, $AX +
  %ZC = AX + ZV = 2NY = XZ$.
\item Iškilajame penkiakampyje $ABCDE$ $AE = AD,AB = AC$, $\angle CAD =
  \angle AEB + \angle ABE$. Įrodyti, kad $CD$ dvigubai ilgesnė už
  trikampio $ABE$ pusiaukraštinę $AM$.
  %Tegu taškas $F$ yra simetriškas taškui $E$ taško $A$ atžvilgiu. Tada iš
  %trikampio priekampio savybės, $\angle CAD = \angle AEB + \angle ABE =
  %\angle BAF$ bei $EA = AE = AD$.  Tada trikampiai $BAF$ ir $CAD$ yra
  %vienodi pagal dvi kraštines ir kampą, taigi $CD = BF$. Bet iš vidurio
  %linijos savybės $BF$ yra dvigubai ilgesnė už trikampio $ABE$
  %pusiaukraštinę $AM$.
\item Duota trapecija $ABCD$ su pagrindais $AD$, $BC$.  $P,Q$ - $AD$ ir
  $BC$ vidurio taškai. Pasirodė, kad $AB = BC$, ir be to, $P$ guli ant
  kampo $B$ pusiaukampines.  Įrodyti, kad $BD = 2PQ$.
  %Pastebėkime, kad $PA = PC$ (nes $P$ guli ant kampo $B$ pusiaukampinės),
  %taigi $APC$ yra lygiašonis. Bet $\angle BCA = \angle CAP$, taigi $ABC$
  %ir $ACP$ yra vienodi, ir todėl $BCPA$ yra rombas. Jeigu $M$ yra $BA$
  %vidurio taškas, tai tada iš simetrijos $QP = MP$. Bet $MP =
  %\frac{BD}{2}$ iš trikampio vidurio linijos savybės.
\item Duotas keturkampis $ABCD$ toks, kad $\angle CBD = \angle CAB$ ir
  $\angle ACD = \angle ADB$. Įrodyti, kad iš atkarpų $BC, AC, AD$ galima
  sudėti statujį trikampį.
  %Tegu keturkampio įstrižainės kertasi taške $E$. Tada trikampiai $ECB$ ir
  %$CBA$ yra panašūs pagal tris kampus, taigi $\frac{EC}{CB} =
  %\frac{CB}{AC}$.  Panašiai $\frac{AE}{AD} = \frac{AD}{AC}$. Iš šių
  %lygybių $CB^2 = AC\cdot EC$ ir $AD^2 = AE\cdot AC$. Sudėję abi lygybes,
  %gauname $CB^2 + AD^2 = AC \cdot (EC + AE) = AC^2$, ko ir reikėjo. 
\item Duotas trikampis $ABC$, $AL$-pusiaukampinė. Pasirode, kad $AL = LB$.
  Ant spindulio $AL$ pasirinktas taškas $K$ toks, kad $CL = AK$. Įrodyti,
  kad $AK = CK$.
  %Iš pusiaukampinės savybės $\frac{AC}{AK} = \frac{AC}{CL} = \frac{AB}{BL}
  %= \frac{AB}{AL}$. Bet $\angle CAK = \angle LAB$, taigi trikampiai $ACK$
  %ir $LAB$ yra panašūs.  Kadangi $BAL$ lygiašonis, tai $ACK$ taip pat
  %lygiašonis, todėl $AK = CK$.
\item Trapecijoje $ABCD$ su pagrindais $AD$ ir $BC$ paimtas taškas $E$ ant
  kraštinės $AB$ taip, kad $\frac{AE}{BE} = \frac{AD}{BC}$. Taško
  projekcija $D$ ant tiesės $CE$ yra taškas $H$. Įrodyti, kad $AH = AD$.
  %Tegu $CE$ kerta tiesę $AD$ taške $F$. Tada trikampiai $EAF$ ir $EBC$ yra
  %panašūs, ir todėl $\frac{AF}{BC} = \frac{AE}{BE} = \frac{AD}{BC}$, taigi
  %$AF = AD$.  Bet trikampis $DFH$ yra status, o $AH$ yra pusiaukraštinė iš
  %stačiojo kampo, todėl $AH = AD$. 
\item Duotas statusis trikampis su stačiu kampu $A$ ($AC>AB$), aukštine $AD$. Ant
  kraštinės $BC$ paimtas taškas $E$ toks, kad $ED = DA$, o ant kraštinės $AC$
  paimtas taškas $F$ toks, kad $FE \perp{ED}$. Rasti kampą $\angle ABF$.
  %Trikampiai $ADB$ ir $ADC$ yra panašūs. Iš Talio teoremos $\frac{AF}{DE}
  %= \frac{AC}{CD} = \frac{AB}{AD} = \frac{AB}{DE}$, taigi $AF = AB$.
  %Gauname, kad $BAF$ yra statusis lygiašonis, todėl $\angle ABF =
  %90^\circ$.
\item Ant trikampio $ABC$ kraštinių $AB$  ir $BC$ atitinkamai paimti taškai
  $X$ ir $Y$ tokie, kad $AX=BY$ ir $\angle XYB = \angle BAC$.
  $BB_1$-trikampio $ABC$ pusiaukampinė iš taško $B$. Įrodyti, kad $XB_1
  \parallel{BC}$.
  %Trikampiai $XYB$ ir $BAC$ yra panašūs pagal du kampus.  Taigi,
  %$\frac{AB_1}{B_1C} = \frac{AB}{BC} = \frac{YB}{BX} = \frac{AX}{BX}$,
  %todėl trikampiai $AB_1X$ ir $ACB$ yra panašūs, taigi
  %$B_1X\parallel{BC}$.
\end{enumerate}

\newpage 
\section{Apskritimai}

Šiame skyriuje pradėsime spręsti uždavinius su apskritimais; gerai
išmanyti tokius uždavinius yra labai svarbu, nes daugybė
geometrijos uždavinių olimpiadose yra vienaip ar kitaip su jais susiję.
Daugiausia dėmesio skirsime įbrėžtiniams keturkampiams - apibrėžtines
figūras nagrinėsime kituose skyriuose.  

\subsubsection{Tai, kas svarbiausia}
Čia pateiksiu svarbiausius ir naudingiausius faktus, susijusius su
apskritimais. Kai kurių jų dar šiame skyrelyje nereikės, bet galbūt
prireiks vėliau.

\begin{teig} 
  Kampas, besiremiantis į apskritimo lanką, yra dvigubai mažesnis nei
  išcentrinis to lanko kampas. 
\end{teig}

%Įbrėžtiniai kampai
\begin{center}
\begin{asy}
import olympiad;
size(200);
pair A, B, C, D, E, F, G, H;
path a1, a2;
A=origin; B=(100,0);
a1=circle(A,40);
a2=circle(B,40);
draw(a1);
draw(a2);
dot(A,blue);
dot(B,blue);
C=waypoint(a1,-0.10);
D=waypoint(a1,-0.40);
E=waypoint(a1,0.20);
F=waypoint(a2,0.1);
G=waypoint(a2,0.2);
H=waypoint(a2,0.44);
add(anglem(D,A,C,300,green,0));
add(anglem(D,E,C,300,green,0));
add(anglem(H,G,F,300,blue,0));
add(anglem(H,B,F,300,blue,0));
label("$\beta$",E,1.5*(-0.4,-2),deepgreen);
label("$2\beta$",A,1.5*(0,-2),deepgreen);
label("$\alpha$",G,1.5*(0,-2),deepblue);
label("$2\alpha$",B,1.5*(0,-2),deepblue);
draw(A--C--E--D--cycle);
draw(C--D);
draw(B--F--G--H--cycle);
draw(F--H);
dot(A,blue);
dot(B,blue);
dot(C,blue);
dot(D,blue);
dot(E,blue);
dot(F,blue);
dot(G,blue);
dot(H,blue);
\end{asy}
\end{center}

\begin{teig}
  Jeigu iškilasis keturkampis $ABCD$ yra įbrėžtinis ir F yra įstrižainių
  sankirtos taškas, o $E$ yra $AB$ ir $CD$ sankirtos taškas, tai tada
  trikampiai $ABF$ ir $CDF$ yra panašūs. Be to, trikampiai $AFD$ ir $CFB$
  taip pat yra panašūs. Ir galiausiai trikampiai $ADE$ ir $CBE$ taip pat
  yra panašūs. Visos šios savybės yra akivaizdžios iš įbrėžtinių kampų ir
  trikampių panašumo pagal 2 kampus savybių. Tuomet, iš panašiųjų trikampių
  kraštinių santykio savybių mes gauname $BF\cdot FD = AF\cdot CF$ bei
  $EA\cdot EB = ED\cdot EC$. Šias savybes reikia mokėti kaip penkis
  pirštus, nes jų labai dažnai reikia.
\end{teig}

%Susikertančių stygų savybė
\begin{center}
\begin{asy}
import olympiad;
size(200);
pair O, A, B, C, D, E, F;
path a1;
O=origin;
a1=circle(0,40);
draw(a1);
A=waypoint(a1,0.09);
B=waypoint(a1,0.40);
C=waypoint(a1,0.74);
D=waypoint(a1,0.94);
F=extension(B,D,A,C);
E=extension(B,A,D,C);
add(anglem(C,B,D,400,green,1));
add(anglem(C,A,D,400,green,1));
add(anglem(B,D,C,400,blue,2));
add(anglem(B,A,C,400,blue,2));
add(anglem(A,D,B,400,yellow,3));
add(anglem(A,C,B,400,yellow,3));
add(anglem(D,B,A,400,red,0));
add(anglem(D,C,A,400,red,0));
label("$A$",A,NE,blue);
label("$B$",B,NW,blue);
label("$C$",C,down,blue);
label("$D$",D,SE,blue);
label("$O$",O,left,blue);
label("$E$",E,NE,blue);
label("$F$",F,(1.1,0.4),blue);
draw(A--B--C--D--cycle);
draw(A--C);
draw(B--D);
draw(D--E);
draw(A--E);
dot(A,blue);
dot(B,blue);
dot(C,blue);
dot(D,blue);
dot(E,blue);
dot(F,blue);
dot(O,blue);
\end{asy}
\end{center}

\begin{teig}
Kampo tarp stygos ir liestinės savybė: kampas tarp apskritimo stygos ir
liestinės, išvestos apskritimui viename iš stygos galų, yra lygus
įbrėžtiniam kampui, besiremenčiam į tą stygą iš kitos jos pusės. 
\end{teig}

%Kampo tarp stygos ir liestinės savybė
\begin{center}
\begin{asy}
import olympiad;
size(180);
pair O, A, B, C, D, E;
path a1;
O=origin;
a1=circle(0,40);
draw((-50,50)--(50,-50),invisible);
A=waypoint(a1,0.09);
B=waypoint(a1,0.29);
E=waypoint(a1,0.54);
C=waypoint(a1,0.70);
D=waypoint(a1,0.94);
add(anglem(B,A,D,300,green,2));
add(anglem(B-(O-B)*(0,1),B,D,300,green,2));
add(anglem(D,B,B+(O-B)*(0,1),300,red,0));
add(anglem(D,C,B,300,red,0));
add(anglem(D,E,B,300,red,0));
draw(a1);
drawline(B,B+(O-B)*(0,1));
draw(B--A--D--C--cycle);
draw(E--B--D--cycle);
dot(A,blue);
dot(B,blue);
dot(C,blue);
dot(D,blue);
dot(E,blue);
dot(O,blue);
\end{asy}
\end{center}

Taip pat teisinga yra ir atvirkščia savybė: jeigu $\angle ACB = \angle ABD$
ir $D$ yra ant atkarpos $AC$, tai $AB$ liečia apie $CBD$ apibrėžtą
apkritimą taške $B$.

%Atvirkštinė stygos ir liestinės savybė
\begin{center}
\begin{asy}
import olympiad;
size(200);
pair E, B, C, D;
B=(0,0); C=(100,-9); D=(65,30);
E=waypoint(B--C,0.7);
add(anglem(E,B,D,300,green,0));
add(anglem(E,D,C,300,green,0));
label("$C$",B,SW,blue);
label("$A$",C,SE,blue);
label("$B$",D,up,blue);
label("$D$",E,down,blue);
draw(D--C--B--cycle);
draw(E--D);
draw(circumcircle (B,E,D));
dot(B,blue);
dot(C,blue);
dot(D,blue);
dot(E,blue);
\end{asy}
\end{center} 

\begin{teig}
  Panašiai kaip ir dviejuose prieš tai buvusiuose teiginiuose, jeigu iš
  taško $A$ išvesime apskritimui dvi liestines, tai tada tos liestinės bus
  vienodo ilgio. Be to, jei tiesė per $A$ kerta apskritimą taškuose $C$ ir
  $D$, tai trikampiai $ABD$ ir $ABC$ bus panašūs (kaip ir trikampiai $ADE$
  ir $ACE$). Iš jų panašumo gauname, kad $AC \cdot AD = AE^2 = AB^2$.
\end{teig}


%Liestinės ilgio kvadratas lygus ..
\begin{center}
\begin{asy}
import olympiad;
size(200);
pair O, A, B, C, D, E;
path a1;
O=origin;
a1=circle(0,40);
draw(a1);
C=waypoint(a1,0.65);
D=waypoint(a1,0.96);
A=D+(D-C)*1.5;
B=tangent(A,O,40,1);
E=tangent(A,O,40,2);
add(anglem(E,C,D,500,green,1));
add(anglem(A,E,D,500,green,1));
add(anglem(D,C,B,500,red,0));
add(anglem(D,B,A,500,red,0));
label("$A$",A,NE,blue);
label("$B$",B,up,blue);
label("$C$",C,SW,blue);
label("$D$",D,SE,blue);
label("$O$",O,NE,blue);
label("$E$",E,SE,blue);
draw(E--D--B--C--cycle);
draw(A--C);
draw(A--B);
draw(A--E);
dot(A,blue);
dot(B,blue);
dot(C,blue);
dot(D,blue);
dot(E,blue);
dot(O,blue);
\end{asy}
\end{center}

\begin{pav}
  Plokštumoje yra du apskritimai taip, kad vienas nėra 
  kito viduje. Jiems nubrėžtos dvi bendros išorinės liestinės:
  pirmoji liečia pirmą apskritimą taške $A$, o antrąjį 
  taške $B$. Antroji liečia pirmą apskritimą taške $C$,
  o antrą taške $D$. $AD$ kerta pirmą apskritimą taške
  $E$, o antrą taške $F$. Įrodyti, kad $AE=FD$.
\end{pav}

\begin{sprendimas}
  $AF\cdot AD=AB^2=CD^2=DE\cdot DA$, taigi $AF=DE$.
  Iš čia $AE=FD$.
  %Nelygiašonis trikampis
\begin{center}
\begin{asy}
import olympiad;
size(200);
pair A, B, C, D, E, X, Y, Z, Q, S, T;
path a2, a1;
pair[] V, P;
A=(0,0);
C=(80,0);
B=(25,40);
D=incenter(A,B,C);
X=bisectorpoint(B,A,-C);
Y=bisectorpoint(B+B-C,B,A);
Z=foot(D,A,C);
Q=foot(D,B,C);
E=extension(A,X,B,Y);
S=foot(E,-C,C);
T=foot(E,B+B-C,B);
a1=circle(E,arclength(E--foot(E,B,A)));
a2=incircle(A,B,C);
V=intersectionpoints(a1,(T--Z),-1);
P=intersectionpoints(a2,(T--Z),-1);
dot(D,blue);
dot(E,blue);
dot(T,blue);
dot(S,blue);
dot(Q,blue);
dot(Z,blue);
dot(V[0],blue);
dot(P[0],blue);
draw (a1);
draw (incircle(A,B,C));
draw(S--Z);
draw(T--Q);
draw(T--Z);
label("$A$",Z,down,blue);
label("$C$",Q,up,blue);
label("$B$",S,down,blue);
label("$D$",T,up,blue);
label("$F$",V[0],left,blue);
label("$E$",P[0],right,blue);
\end{asy}
\end{center}
\end{sprendimas}
\begin{teig}
  Lygūs kampai apskritime remiasi į lygius lankus. Dėl to, pavyzdžiui,
  trikampio pusiaukampinė dalija apie tą trikampį apibrėžto apskritimo lanką,
  kurį atkerta nuo apskritimo priešinga kraštinė, į dvi lygias dalis.
\end{teig}

%Lygus kampai lygios stygos
\begin{center}
\begin{asy}
import olympiad;
size(200);
pair O, B, C, D, E;
path a1;
O=origin;
a1=circle(0,40);
draw(a1);
B=waypoint(a1,0.21);
C=waypoint(a1,0.54);
E=waypoint(a1,0.74);
D=waypoint(a1,0.95);
add(anglem(C,B,E,300,green,0));
add(anglem(E,B,D,300,green,0));
label("$B$",B,up,blue);
label("$C$",C,left,blue);
label("$D$",D,right,blue);
label("$O$",O,NE,blue);
label("$E$",E,down,blue);
draw(E--D--B--C--cycle);
draw(B--E);
draw(C--D);
dot(B,blue);
dot(C,blue);
dot(D,blue);
dot(E,blue);
dot(O,blue);
label("$\beta$",B,(2,-8)/1.5,deepgreen);
label("$\beta$",B,(-4,-8)/1.5,deepgreen);
add(pathticks(C--E,2,0.5,0,125,red));
add(pathticks(D--E,2,0.5,0,125,red));
\end{asy}
\end{center}

Pavyzdžiui, paveikslėlyje viršuje lankai $CE$ ir $ED$ yra vienodi, taigi
$CE = ED$ ir todėl  $CED$ yra lygiašonis.

\begin{teig}
  Įbrėžtinio keturkampio kampo tarp įstrižainių savybė: jeigu iškilojo
  įbrėžtinio keturkampio $ABCD$ įstrižainės kertasi taške $E$, tai kampas
  $\angle AED$ yra lygus sumai kampų, besiremenčių į lankus $AD$ ir $BC$, t.y
  $\angle AED = \angle ABD +\angle BAC$.
\end{teig}

%Įbrėžtinio keturkampio kampas tarp įstrižainių
\begin{center}
\begin{asy}
import olympiad;
size(200);
pair O, B, C, D, A, E;
path a1;
O=origin;
a1=circle(O,40);
draw(a1);
C=waypoint(a1,0.26);
B=waypoint(a1,0.44);
A=waypoint(a1,0.84);
D=waypoint(a1,0.12);
E=extension(A,C,B,D);
add(anglem(A,B,E,200,green,0));
add(anglem(E,A,B,200,green,0));
add(anglem(A,E,D,200,red,0));
label("$B$",B,left,blue);
label("$C$",C,up,blue);
label("$D$",D,NE,blue);
label("$A$",A,SE,blue);
label("$E$",E,NE,blue);
draw(A--B--C--D--cycle);
draw(A--C);
draw(B--D);
dot(B,blue);
dot(C,blue);
dot(D,blue);
dot(E,blue);
dot(A,blue);
\end{asy}
\end{center}
Tai akivaizdu iš paveikslėlio  viršuje ir priekampio savybės. 

\begin{teig}
  Jeigu du trikampiai turi tokio pat dydžio kampą ir tokio
  pat ilgio kraštinę prieš tą kampą, tai apie tuos
  trikampius apibrėžtų apskritimų spinduliai yra vienodi.
  Taip pat jei du trikampiai turi vienodo ilgio kraštinę ir
  viename trikampyje kampas prieš tą kraštinę yra lygus $a$,
  o kitame $180^\circ - a$, tai abie tuos trikampius
  apibrėžtų apskritimų spinduliai taip pat vienodi.  
\end{teig}

\begin{teig}
  Kampas, besiremiantis į apskritimo skersmenį, yra status.
\end{teig}

\subsubsection{Kaip įrodyti, kad keturkampis yra įbrėžtinis}

Dažnas uždavinys olimpiadose yra įrodyti,
kad keturkampis yra įbrėžtinis (arba keturi taškai guli ant
vieno apskritimo). Tarkime, kad tos keturkampio viršūnės yra
$A,B,C,D$. Tada pagrindiniai būdai tai padaryti yra šie:

\begin{itemize}
\item Įrodyti, kad keturkampio priešingų kampų suma yra lygi
  $180^\circ$. Šį požymį galima suformuluoti ir taip: jeigu
  iškilojo keturkampio kampas yra lygus priešingo kampo
  priekampiui, tai keturkampis yra įbrėžtinis.
%Kampas lygus priekampiui
\begin{center}
\begin{asy}
import olympiad;
size(200);
pair O, B, C, D, A, E;
path a1;
O=origin;
a1=circle(O,40);
C=waypoint(a1,0.24);
B=waypoint(a1,0.54);
A=waypoint(a1,0.94);
D=waypoint(a1,0.12);
add(anglem(C,D,A,200,red,0));
add(anglem(C,B,B+(B-A)*2,200,red,0));
/*label("$B$",B,left,blue);
label("$C$",C,up,blue);
label("$D$",D,NE,blue);
label("$A$",A,SE,blue);
label("$E$",E,NE,blue);*/
draw(a1);
draw(A--B--C--D--cycle);
drawline(B,A);
dot(B,blue);
dot(C,blue);
dot(D,blue);
dot(A,blue);
draw((-50,0)--(50,0),invisible);
\end{asy}
\end{center}
\item Įrodyti, kad $\angle ABD=\angle ACD$ (jei $B$ ir $C$
  yra toje pačioje $AD$ pusėje).
  \begin{pav}
  Trikampyje $ABC$ nubrėžtas statmuo $AD$, o $M,K,L$ yra 
  $BC,CA,AB$ vidurio taškai. Įrodyti, kad $MKLD$ įbrėžtinis. 
\end{pav}  
\begin{sprendimas} 
  $\angle KDL=\angle KAL$, nes $D$ ir $A$ simetriški $KL$ atžvilgiu.
  $\angle KML=\angle KAL$, nes $KALM$ lygiagretainis. Taigi $\angle 
  KDL=\angle KML$, ir todėl $MKDL$ įbrėžtinis.
\end{sprendimas}
\item Jei $AC$ ir $BD$ (nebūtinai įstrižainės) kertasi taške
  $E$, tai $A,B,C,D$ yra ant vieno apskritimo tada ir tik tada, jei
  $AE\cdot EC=BE\cdot DE$.
\begin{pav}
  Trikampyje $ABC$ nubrėžtas statmuo $AD$, o iš $D$ nuleisti 
  statmenys $DE$ ir $DF$ į $AB$ ir $AC$ atitinkamai. Įrodyti,
  kad $BEFC$ įbrėžtinis. 
\end{pav}  
\begin{sprendimas} 
  Kadangi $\angle ADE=\angle ABD$, tai $AD^2=AE\cdot AB$. Panašiai
  $ AD^2=AC\cdot AF$. Todėl $AE\cdot AB=AC\cdot AF$, taigi $BEFC$
  įbrėžtinis.
\begin{center}
\begin{asy}
import olympiad;
size(200);
pair A, B, C, D, E, F;
path a1;
B=origin;
A=(40,70);
C=(100,0);
D=foot(A,B,C);
E=foot(D,A,B);
F=foot(D,A,C);
dot(A,blue);
dot(B,blue);
dot(C,blue);
dot(D,blue);
dot(E,blue);
dot(F,blue);
draw(A--B--C--cycle);
draw(A--D);
draw(E--D);
draw(F--D);
label("$A$",A,up,blue);
label("$B$",B,left,blue);
label("$C$",C,right,blue);
label("$D$",D,SE,blue);
label("$E$",E,left,blue);
label("$F$",F,right,blue);
add(rightanglem(A,D,D,170));
add(rightanglem(D,E,A,170));
add(rightanglem(D,F,A,170));
\end{asy}
\end{center}
\end{sprendimas}
\item (Retai naudojamas) Kažkurių trijų atkarpų iš aibės ${AB,BC,CD,DA,AC,BD}$
  vidurio statmenys kertasi viename taške (tos trys atkarpos
  turi nesudaryti trikampio).
\begin{proof}[Įrodymas] Tegu tas sankirtos taškas yra $O$.
  Jis yra vienodai nutolęs nuo kiekvienos atkarpos, ant kurios vidurio 
  statmens jis yra, galų. Todėl $O$ nutolęs vienodai nuo trijų porų
  taškų, ir mes darome išvadą, kad visi atstumai $OA,OB,OC,OD$ vienodi.
  Tada apskritimas su centru $O$ ir spinduliu $OA$ eina per visus keturis
  taškus.
\end{proof}
\item (Retai naudojamas) $ABCD$ yra įbrėžtinis jeigu yra taškas $O$, toks kad
$OA=OB=OC=OD$.
\end{itemize} 

Yra keletas kitų būdų, bet jie naudojami rečiau ir
sunkesniuose uždaviniuse; artimiausiuose skyriuose pilnai
pakaks ir šių. 

\subsubsection{Pavyzdžiai}
\begin{pav}[2006 Lietuvos atranka į Baltijos kelio olimpiadą]
  Duotas trikampis $ABC$, $F,D,E$ yra atitinkamai kraštinių 
  $AB,BC,CA$ vidurio taškai. Įrodyti, kad $\angle DAC=\angle 
  ABE$ tada ir tik tada, jei $\angle AFC=\angle ADB$.
\end{pav}

\begin{sprendimas}
  Tegu $M$ yra  pusiaukraštinių susikirtimo taškas. Tada 
 $\angle DAC=\angle ABE \Leftrightarrow \angle MDF=\angle FBM
 \Leftrightarrow FBDM$ įbrėžtinis $\Leftrightarrow 
 \angle BDA=\angle AFC $.
\end{sprendimas}

\begin{pav}[„Gerai žinoma lema“] Duotas iškilasis
  keturkampis $ABCD$ toks, kad $\angle DAC = \angle CAB$ ir
  $DC = CB$.  Įrodyti, kad tas keturkampis arba yra
  deltoidas, arba įbrėžtinis.
\end{pav}

\begin{sprendimas}
Tegu $DC = CB = a$. Išveskime statmenis $CF$ ir $CE$ iš $C$
į atitinkamai $AB$ ir $AD$. Pasižymime ant $AE$ ir $AF$ po du 
taškus $D_1,D_2,B_1,B_2$, nutolusius nuo $C$ per $a$. 
Kadangi $C$ yra ant kampo $A$
pusiaukampinės, tai $CF = CE$. Tada trikampiai $CFB$ ir
$CDE$ yra vienodi pagal 2 kraštines ir kampą. Tokiu atveju
mes turime keturis skirtingus atvejus: 
\begin{itemize}
  \item Brėžinyje $B=B_1$ ir $D=D_1$. Tokiu atveju
    $AD=AE-DE=AF-BF=AB$. Taigi keturkampis yra deltoidas.
  \item Brėžinyje $B=B_1$ ir $D=D_2$. Tada $\angle
    ABC=180^\circ-\angle CBF=180^\circ-\angle ADC$, taigi
    trikampis yra įbrėžtinis.
  \item Brėžinyje $B=B_2$ ir $D=D_1$. Čia taip pat įbrėžtinis.
  \item Brėžinyje $B=B_2$ ir $D=D_2$. Dabar keturkampis ne
    iškilasis („išsigimęs“ deltoidas).
\end{itemize}
%Deltoidas arba įbrėžtinis
\begin{center}
\begin{asy}
import olympiad;
size(200);
pair A, B, BB, C, D, DD, E, F, X, FF, EE;
A=(0,0); FF=(100,10);
C=rotate(-15)*FF;
EE=rotate(-15)*C;
F=foot(C,A,FF);
E=foot(C,A,EE);
B=waypoint(A--F,0.83);
D=waypoint(A--E,0.83);
BB=2*F-B;
DD=2*E-D;
add(anglem(C,A,F,500,green,0));
add(anglem(E,A,C,500,green,0));
add(anglem(C,DD,E,400,red,0));
add(anglem(C,B,F,400,red,0));
label("$A$",A,SW,blue);
label("$B_1$",B,up,blue);
label("$C$",C,(1.3,1),blue);
label("$D_1$",D,SW,blue);
label("$E$",E,SW,blue);
label("$F$",F,up,blue);
label("$B_2$",BB,up,blue);
label("$D_2$",DD,SW,blue);
label("$\alpha$",A,(8,-0.2),deepgreen);
label("$\alpha$",A,(7.8,-2.5),deepgreen);
add(rightanglem(C,F,BB,200));
add(rightanglem(C,E,D,200));
draw((-10,0)--(110,0),invisible);
draw(A--BB*1.2);
draw(A--DD*1.2);
draw(A--C*1.2);
draw(C--B);
draw(C--BB);
draw(C--F);
draw(C--D);
draw(C--DD);
draw(C--E);
dot(A,blue);
dot(B,blue);
dot(C,blue);
dot(D,blue);
dot(BB,blue);
dot(DD,blue);
dot(E,blue);
dot(F,blue);
add(pathticks(C--B,2,0.5,0,125,red));
add(pathticks(C--BB,2,0.5,0,125,red));
add(pathticks(C--D,2,0.5,0,125,red));
add(pathticks(C--DD,2,0.5,0,125,red));
\end{asy}
\end{center}
\end{sprendimas}

\begin{pav}[Lietuvos TST 2010] 
  Rombo $ABCD$ įstrižainėje $AC$ ir kraštinėje $BC$
  atitinkamai parinkti taškai $M$ ir $N$ tokie, kad $DM=MN$
  ($N$ nesutampa su $B$). $AC$ ir $DN$ kertasi taške $P$, o
  tiesės $AB$ ir $DM$ taške $R$. Įrodyti, kad $PR=DP$.
\end{pav}

\begin{sprendimas}
  Pastebėkime, kad keturkampis $CDMN$ tenkina prieš tai
  buvusios lemos sąlygas. Kadangi $CD\neq CN$, tai keturkampis
  nėra deltoidas ar išsigimęs deltoidas, taigi yra įbrėžinis.
  Tada $\angle RAP=\angle BAC=\angle DCA=\angle ACB=\angle
  MCN=\angle MDN=\angle RDP=\angle DAP$. Taigi $ADPR$ yra
  įbrėžtinis su $DP=PR$ (iš vieno aukščiau buvusių teiginių).
\end{sprendimas}

\begin{pav}[Lietuvos TST 2006?]
  Duotas trikampis $ABC$, kampas $A$ status. $M$ yra $BC$ vidurio 
  taškas. Paimkime $D$ ant $AC$ taip, kad $AD=AM$. Tegu apie $AMC$
  ir $BDC$ apibrėžti apskritimai kertasi taške $P$. Įrodyti, kad
  $CP$ yra kampo $ACB$ pusiaukampinė.
\end{pav}  

\begin{sprendimas}
   Paimsime tašką 
  $P'$ kuris tenkina tas savybes, kurias reikia įrodyti taškui $P$, 
  tada įrodysime, kad jis tenkina tas pačias savybes, kaip ir taškas 
  $P$, ir galiausiai parodysime, kad jie sutampa (Naudojamės schema 
  „Įrodome ne kad (jei X tiesa, tai ir Y tiesa), o (jei Y tiesa, tai
  X irgi tiesa )“). Taigi tegu $P'$ yra $AMC$ apibrėžtinio apskritimo
  ir kampo $C$ pusiaukampinės sankirta. Tada $AP'=P'M$. Trikampiai 
  $AP'D$ ir $MP'B$ vienodi pagal dvi kraštines ir kampą. Taigi $BP'
  =P'D$. Iš lemos $BCDP'$ yra arba įbrėžtinis, arba $BC=CD$. Antras
  atvejis yra neįmanomas, taigi $P'$ guli ant abiejų apskritimų, ir 
  $P'=P$.
  %Nelygiašonis trikampis
\begin{center}
\begin{asy}
import olympiad;
size(200);
pair A, B, C, D, M, X, Y, Z, V;
A=(0,0);
B=(0,50);
C=(100,0);
M=C/2+B/2;
D=rotate(-degrees(M),A)*M;
X=bisectorpoint(B,C,A);
Y=bisectorpoint(A,M);
Z=A/2+M/2;

V=extension(Z,Y,C,X);
dot(A,blue);
dot(B,blue);
dot(C,blue);
dot(D,blue);
dot(M,blue);
dot(V,blue);
draw(A--B--C--cycle);
draw(A--M);
draw(A--V);
draw(M--V);
draw(V--B);
draw(V--D);

label("$A$",A,left,blue);
label("$B$",B,left,blue);
label("$C$",C,right,blue);
label("$P'$",V,NE,blue);
label("$M$",M,up,blue);
label("$D$",D,up,blue);
add(anglem(B,M,V,200,red,0));
add(anglem(D,A,V,200,red,0));
add(pathticks(A--M,2,0.5,0,150,red));
add(pathticks(A--D,2,0.5,0,150,red));
add(pathticks(C--M,2,0.5,0,150,red));
add(pathticks(B--M,2,0.5,0,150,red));
add(pathticks(A--V,2,0.5,30,150,red));
add(pathticks(V--M,2,0.5,30,150,red));
\end{asy}
\end{center}
\end{sprendimas}
\subsubsection{Uždaviniai}

\begin{enumerate}
\item $ABC$ yra trikampis. Ant $AB$, $BC$, $CA$ atitinkamai
  paimti taškai $K,L,M$ taip, kad $\angle BLK=\angle
  CLM=\angle BAC$. $BM$ ir $CK$ kertasi taške $P$. Įrodyti,
  kad keturkampis $AKPM$ yra įbrėžtinis.
  %Kadangi $\angle BAC + \angle KLC = \angle BLK + \angle
  %KLC = 180^\circ$, tai keturkampis $AKLC$ yra įbrėžtinis.
  %Panašiai ir $AMLB$ yra įbrėžtinis. Tada iš įbrėžtinių
  %kampų savybės, $\angle AMP + \angle AKP = \angle AMB +
  %\angle AKC = \angle ALB + \angle ALC = 180^\circ$, taigi
  %$AKPM$ yra įbrėžtinis.   
\item Duotas trikampis $ABC$, $M$-$BC$ vidurio taškas,
  $AA',BB',CC'$ yra aukštinės, $AB$ ir $A'B'$ kertasi taške
  $X$, o $MC'$ ir $AC$ taške $Y$. Įrodyti, kad
  $XY\parallel{BC}$.
  %Tegu $H$ būna trikampio aukštinių susiskirtimo taškas.
  %Pastebėkime, kad trikampis $MBC'$ yra lygiašonis, o
  %keturkampiai $CBC'B'$, $ABA'B'$ ir $AB'HC'$ yra
  %įbrėžtiniai (nes $\angle AA'B = \angle AB'B = \angle CC'B
  %= \angle CB'B$).  Taigi, $\angle XC'Y = \angle MC'B =
  %\angle MBC' = \angle CB'A' = \angle XB'Y$, todėl
  %$XB'C'Y$ yra įbrėžtinis. Tada $\angle XYC =180^\circ-\angle XC'B'
  %= \angle BCA$, todėl $XY\parallel{BC}$.
\item Duotas statusis trikampis $ABC$ su stačiu kampu $A$.
  $M$ yra $BC$ vidurio taškas, o $AH$ yra aukštinė. Linija
  per $M$, statmena $AC$, kerta apie $AMC$ apibrėžtą
  apskritimą taške $P$. Įrodyti, kad $BP$ dalija $AH$
  pusiau.
  %Tegu $G$ yra $CP$ ir $AB$ sankirtos taškas. Kadangi $AMC$
  %yra lygiašonis, tai $PM$ yra ne tik $AMC$ aukštinė, bet
  %ir pusiaukampinė. Tada $APCM$ yra įbrėžtinis deltoidas,
  %taigi $\angle PAM = \angle PCM = 90^\circ$. Be to, $P$
  %yra $GC$ vidurio taškas, nes $P$ yra stataus trikampio
  %$ACG$ įžambinės ir statinio $AC$ vidurio statmens
  %susikirtimo taškas. Taigi $PB$ dalija $GC$ pusiau. Bet
  %trikampiai $BGC$ ir $BAH$ yra panašūs, taigi $BP$ taip
  %pat dalina $AH$ pusiau. 
\item Ant stačiojo trikampio $ABC$ įžambinės $AB$ išorėje
  nubrėžtas kvadratas $ABDE$. Stataus kampo $C$
  pusiaukampinė kerta $DE$ taške $F$. Rasti $\frac{EF}{FD}$,
  jeigu žinoma, kad $AC=1$ ir $BC=3$. 
  %Tegu $H$ būna kvadrato $ABDE$ centras, o $CF$ kerta $AB$
  %taške $K$. Tada $\angle AHB = \angle ACB = 90^\circ$,
  %taigi $CBHA$ yra įbrėžtinis. Be to, $HA = HB$, todėl
  %$\angle ACH = \angle HCB$, t.y kampo $C$ pusiaukampinė
  %eina per tašką $H$. Tada iš simetrijos $BK = EF$ ir todėl
  %$\frac{EF}{FD} = \frac{BK}{KA} = \frac{BC}{CA} = 3$.
\item Į kampą įbrėžti du apskritimai su centrais $A$ ir $B$
  taip, kad jie liečia kampo kraštines ir vienas kitą.
  Įrodyti, kad apskritimas, kurio skersmuo yra $AB$, taip
  pat liečia kampo kraštines.
  %Tegu apskritimai liečia kampo kraštines taškuose $D$ ir
  %$I$, o patys liečiasi taške $G$. Tegu bendra vidinė
  %abiejų apskritimų liestinė kerta kampo kraštinę taške
  %$H$. Tada $DHGA$ ir $HIBG$ yra deltoidai, ir be to,
  %$\angle DHA = \frac{\angle DHG}{2} =
  %\frac{180^\circ-\angle IHG}{2} = 90^\circ - \angle BHG =
  %\angle HBG$ bei $\angle AHB = \angle AHG + \angle GHB =
  %\frac{\angle DHG}{2} + \frac{\angle GHI}{2} =
  %\frac{180^\circ}{2} = 90^\circ$. Taigi apie $AHB$
  %apibrėžtas apskritimas turi skersmenį $AB$ ($ABH$ status)
  %bei liečia $CH$ (iš vienos minėtųjų savybių).
  %\begin{center}
  %\begin{asy}
  %import olympiad;
  %size(200);
  %pair A, B, C, D, E, F, G, H, I;
  %path a1, a2;
  %real r;
  %r=15;
  %C=(0,0); A=(50,0); 
  %a1=circle(A,r);
  %D=tangent(C,A,r,2);
  %G=A+(r,0);
  %H=extension(C,D,G,G+(G-A)*(0,1));
  %I=rotate(180-degrees((G-H)/(D-H)),H)*G;
  %B=extension(C,A,H,(I+G)/2);
  %a2=circle(B,arclength(B--I));
  %add(anglem(H,B,G,300,blue,0));
  %add(anglem(D,H,A,300,blue,0));
  %add(anglem(G,H,B,300,red,0));
  %add(anglem(G,A,H,300,red,0));
  %add(rightanglem(A,D,H,150));
  %add(rightanglem(H,G,A,150));
  %add(rightanglem(H,I,B,150));
  %draw(a1);
  %draw(a2);
  %drawline(C,D);
  %drawline(C,A);
  %drawline(C,tangent(C,A,r,1));
  %drawline(G,H);
  %draw(B--I);
  %draw(B--H);
  %draw(A--D);
  %draw(A--H);
  %draw((-10,-40)--(120,40),invisible);
  %label("$A$",A,SW,blue);
  %label("$B$",B,SW,blue);
  %label("$C$",C,up,blue);
  %label("$D$",D,up,blue);
  %label("$I$",I,up,blue);
  %label("$G$",G,SW,blue);
  %label("$H$",H,up,blue);
%\end{asy}
  %\end{center}
\item Trikampyje $ABC$ $AB=BC$. $BH$-aukštinė, $M$ yra $AB$ 
  vidurio taškas, $K$ yra $BH$ ir apie $MBC$ apibrėžto apskritimo
  sankirtos taškas. Įrodyti, kad $BK=\frac{3R}{2}$, kur $R$ yra 
  apie $ABC$ apibrėžto apskritimo spindulys.
  % Tegu $X$ yra $BC$ vidurio taškas. Tada iš simetrijos ir lygių
  %įbrėžtinių kampų $KC=KM=KX$. Tegu $O$ yra apie $ABC$ apibrėžto
  %apskritimo centras, o $Q$ yra $XC$ vidurio taškas.
  % Tada $OX\parallel KQ$ (abi linijos statmenos $BC$), ir iš 
  %Talio teoremos $\frac{BK}{BO}=\frac{BQ}{BX}=\frac{3}{2}$.
\item Per apskritimo $e$ centrą nubrėžtas apskritimas $f$.
  $A$ ir $B$ - šių apskritimų sankirtos taškai. Liestinė
  apskritimui $f$ taške $B$  kerta apskritimą $e$ taške $C$.
  Įrodyti, kad $AB=BC$.   
  %Tegu apskritimo $e$ centras yra taškas $O$. Tada iš
  %kampo tarp stygos ir liestinės savybės, $\angle OBA =
  %\angle OAB = \angle OBC$, taigi $A$ ir $C$ yra simetriški
  %$OB$ atžvilgiu, taigi $BA = BC$. 
\item Du apskritimai kertasi taškuose $A$ ir $B$. Taške $A$
  abiems apskritimams išvestos liestinės, kertančios
  apskritimus taškuose $M$ ir $N$. Tiesės $BM$ ir $BN$
  atitinkamai dar syki kerta apskritimus taškuose $P$ ir
  $Q$. Irodyti, kad $MP$ = $NQ$.  
  %Vėl iš kampo tarp stygos ir liestinės savybės, $\angle
  %NAB = \angle AMB$ bei $\angle MAB = \angle ANB$. Tada iš
  %paveikslėlio žemiau matyti, kad $\angle MQA = \angle MQB
  %+ \angle BQA = \angle APB + \angle BPN = \angle APN$ bei
  %$\angle QAM = \angle NAP$. Todėl $ANP$ ir $QMA$ yra
  %panašūs, taigi $\angle AQM = \angle ABP = \angle ANP =
  %\angle QMA$, taigi $QA = MA$. Panašiai $AN=AP$. Tada trikampiai $AQN$ ir
  %$AMP$ yra vienodi pagal kraštinę ir 2 kampus, todėl $MP =
  %NQ$.  
  %\begin{center}
  %\begin{asy}
  %import olympiad;
  %size(200);
  %pair A, B, M, N, P, Q;
  %path a1, a2;
  %a1=circle((0,0),50);
  %a2=circle((70,0),30);
  %A=intersectionpoints(a1,a2)[0];
  %B=intersectionpoints(a1,a2)[1];
  %M=intersectionpoints(a1,A--A+((70,0)-A)*(0,-10))[1];
  %N=intersectionpoints(a2,A--A+((0,0)-A)*(0,10))[0];
  %Q=intersectionpoints(a1,N--B+(B-N)*10)[0];
  %P=intersectionpoints(a2,M--B+(B-M)*10)[0];
  %add(anglem(B,Q,A,300,blue,0));
  %add(anglem(B,M,A,300,blue,0));
  %add(anglem(B,A,N,300,blue,0));
  %add(anglem(B,P,N,300,blue,0));
  %add(anglem(Q,B,M,300,red,0));
  %add(anglem(Q,A,M,300,red,0));
  %add(anglem(N,B,P,300,red,0));
  %add(anglem(N,A,P,300,red,0));
  %add(anglem(M,Q,B,300,yellow,0));
  %add(anglem(M,A,B,300,yellow,0));
  %add(anglem(A,N,B,300,yellow,0));
  %add(anglem(A,P,B,300,yellow,0));
  %draw(a1);
  %draw(a2);
  %dot(A,blue);
  %dot(B,blue);
  %dot(M,blue);
  %dot(N,blue);
  %dot(P,blue);
  %dot(Q,blue);
  %draw(Q--A--B--M--cycle);
  %draw(P--N--B--A--cycle);
  %draw(Q--B);
  %draw(M--A);
  %draw(N--A);
  %draw(P--B);
  %label("$A$",A,1.4*up,blue);
  %label("$B$",B,1.4*down,blue);
  %label("$M$",M,SW,blue);
  %label("$N$",N,SE,blue);
  %label("$P$",P,right,blue);
  %label("$Q$",Q,left,blue);
%\end{asy}
  %\end{center}
\item Kokiu kampu is stačiojo trikampio stataus kampo matoma
  į tą trikampį įbrėžto apskritimo projekcija į įžambinę?
  %Tegu tas statusis trikampis būna $ACH$ su stačiu kampu
  %$A$. Įbrėžto apskritimo centras tebūnie $D$. Visi likę
  %taškai pažymėti paveikslėlyje. Reikia rasti kampą $\angle
  %KAL$. Pastebėkime, kad $AFDE$, $DPKG$, $DJLG$ yra visi
  %vienodi kvadratai. Todėl $DK = DL = DA$, t.y $D$ yra apie
  %$AKL$ apibrėžto apskritimo centras. Tačiau $\angle KDL =
  %90^\circ$, taigi $\angle KAL = \frac{\angle KDL}{2} =
  %45^\circ$.
  %\begin{center}
  %\begin{asy}
  %import olympiad;
  %size(200);
  %pair A, H, C, K, G, L, J, D, P, F, E;
  %path a1;
  %H=(0,0); A=(30,50);
  %C=extension(A,A+(H-A)*(0,1),H, (100,0));
  %D=incenter(A,C,H);
  %a1=incircle(A,C,H);
  %E=foot(D,A,C);
  %G=foot(D,H,C);
  %F=foot(D,A,H);
  %P=shift(-inradius(A,C,H),0)*D;
  %J=2*D-P;
  %L=J+G-D;
  %K=P+G-D;
  %draw(a1);
  %dot(A,blue);
  %dot(H,blue);
  %dot(C,blue);
  %dot(K,blue);
  %dot(G,blue);
  %dot(L,blue);
  %dot(J,blue);
  %dot(D,blue);
  %dot(P,blue);
  %dot(F,blue);
  %dot(E,blue);
  %draw(A--H--C--cycle);
  %draw(A--K);
  %draw(A--L);
  %draw(A--D);
  %draw(P--D);
  %draw(J--D);
  %draw(G--D);
  %draw(E--D);
  %draw(F--D);
  %draw(K--D);
  %draw(L--D);
  %draw(J--L);
  %draw(P--K);
  %label("$A$",A,up,blue);
  %label("$H$",H,SW,blue);
  %label("$C$",C,SE,blue);
  %label("$E$",E,NE,blue);
  %label("$F$",F,NW,blue);
  %label("$J$",J,right,blue);
  %label("$P$",P,left,blue);
  %label("$D$",D,NE,blue);
  %label("$K$",K,down,blue);
  %label("$G$",G,down,blue);
  %label("$L$",L,down,blue);
  %add(pathticks(F--D,2,0.5,0,100,red));
  %add(pathticks(E--D,2,0.5,0,100,red));
  %add(pathticks(J--D,2,0.5,0,100,red));
  %add(pathticks(G--D,2,0.5,0,100,red));
  %add(pathticks(P--D,2,0.5,0,100,red));
  %add(pathticks(F--D,2,0.5,0,100,red));
  %add(pathticks(A--D,2,0.5,30,100,red));
  %add(pathticks(K--D,2,0.5,30,100,red));
  %add(pathticks(L--D,2,0.5,30,100,red));
  %add(rightanglem(P,K,G,180));
  %add(rightanglem(H,A,C,180));
  %add(rightanglem(L,G,D,180));
  %add(rightanglem(C,L,J,180));
%\end{asy}
  %\end{center}
\item $AK$ - smailiojo trikampio $ABC$ pusiaukampinė, $P$ ir
  $Q$ - taškai ant pusiaukampinių (ar jų tęsinių) $BB'$ ir
  $CC'$ tokie, kad $PA=PK$, $QA=QK$. Įrodykite, kad $\angle
  PAQ= 90^\circ-\frac{\angle BAC}{2}$. 
  %Pritaikę vieną iš minėtųjų savybių, gauname, kad
  %keturkampiai $CAPK$ ir $QABK$ yra arba įbrėžtiniai, arba
  %deltoidai. Tačiau deltoidais nė vienas jų būti negali,
  %nes kitaip kažkurios dvi trikampio kraštinės bus
  %lygiagrečios.  Todėl jie abu yra įbrėžtiniai, ir todėl
  %$\angle PAQ = \angle PAK + \angle KAQ = \angle PBK +
  %\angle QCK = \frac{\angle BCA}{2} + \frac{\angle ABC}{2}
  %= 90^\circ-\frac{\angle BAC}{2}$. 
\item Du apskritimai kertasi taškuose $A$ ir $B$. Nubrėžta
  jiems bendra išorinė liestinė liečia pirmą apskritimą
  taške $C$, o antrą  taške $D$. Tarkime, kad taškas $B$ yra
  arčiau $CD$ negu taškas $A$. $CB$ kerta antrąjį apskritimą
  antrą kartą taške $E$. Įrodyti, kad $AD$ yra kampo $\angle
  CAE$ pusiaukampinė.  
  %Pasinaudoję kampo tarp stygos ir liestinės savybe bei
  %priekampio savybe, mes gauname, kad $\angle EAD = \angle
  %EBD = \angle BDC + \angle BCD = \angle BAD + \angle BAC =
  %\angle DAC$, ko ir reikėjo.
  
\item Duotas įbrėžtinis keturkampis $ABCD$ kurio įstrižainės 
  statmenos, o apibrėžtinio apskritimo centras yra $O$. 
  Įrodyti, kad statmens iš $O$ į $AD$ ilgis dvigubai trumpesnis
  nei kraštinė $BC$.
  %Tegu $H$ ir $G$ yra pagrindai statmenų, nuleistų iš $O$ į
  % atitinkamai $BC$ ir $AD$. Tada $\angle GOA=\frac{\angle 
  % DOA}{2}=\angle DBA=90^\circ-\angle BAC=90^\circ-\angle BOH=
  %\angle OBH$. Taigi trikampiai $BOH$ ir $AOG$ vienodi pagal
  %kraštinę ir tris kampus. Tad $OG=\frac{BC}{2}$. 
\item Ant apskritimo $K$ su centru taške $O$ stygos $AB$
  paimtas taškas $C$. Apie $AOC$ apibrėžtas apskritimas kerta 
  $K$ taške  $D$. Įrodyti, kad $CD=CB$.
  %Kadangi $AO=DO$, tai $\angle OCD=\angle OAD=\angle ADO=\angle 
  % OCB$. Taigi taškai $D$ ir $B$ yra simetriški $OC$ atžvilgiu,
  %ko ir reikėjo.
\item Lygiagretainio $ABCD$ įstrižainės kertasi taške $O$.
  Įrodykite, kad jei apie $ABO$ apibrėžtas apskritimas
  liečia $BC$, tai apie $BCO$ apibrėžtas apskritimas liečia
  $CD$.
  %Iš kampo tarp stygos ir liestinės mes turime $\angle ACD
  %= \angle CAB = \angle OBC$, ir pritaikę vieną iš minėtų
  %naudingų faktų turime, kad apie $BCO$ apibrėžtas
  %apskritimas liečia $CD$.
\item Duotas trikampis $ABC$, apie jį nubrėžtas apskritimas.
  Dvi tiesės eina per tašką $A$ ir kerta atkarpą $BC$
  taškuose $K$ ir $M$, o lanką $BC$ (tą, kuris neturi taško
  $A$) tiesės $AK$ ir $AM$  kerta atitinkamai taškuose $L$
  ir $N$. Jei $KLMN$ yra įbrėžtinis, tai įrodykite, kad
  $ABC$ lygiašonis.
  %$\angle AKB = \angle CKL = \angle 180^\circ - \angle ANL
  %= \angle ABL$. Tada iš
  %aukščiau minėtų naudingų faktų, $AB^2 = AK \cdot AL$.
  %Panašiai ir $AC^2 = AM \cdot AN$. Bet iš tų pačių faktų,
  %$AM \cdot AN = AK \cdot AL$. Taigi $AC^2 = AB^2$, ko ir
  %reikėjo. 
\item Duotas apskritimas su centru $O$, jį atkarpa $AB$
  liečia taške $A$. $AB$ yra pasukta aplink $O$ ir taip
  gauta $A'B'$. Įrodyti, kad $AA'$ eina per atkarpos $BB'$
  vidurio tašką.
\begin{center}
\begin{asy}
import olympiad;
size(200);
pair O, A, AA, B, BB;
path a1;
O=origin;
a1=circle(O,40);
A=waypoint(a1,0.29);
B=A+(O-A)*(0,2);
AA=rotate(-40)*A;
BB=rotate(-40)*B;
draw(a1);
dot(A,blue);
dot(AA,blue);
dot(B,blue);
dot(BB,blue);
dot(O,blue);
draw(A--B--BB--AA--O--cycle);
drawline(A,AA);
label("$A$",A,up,deepred);
label("$A'$",AA,up,deepred);
label("$B$",B,NE,deepgreen);
label("$B'$",BB,NE,deepgreen);
label("$O$",O,right,blue);
add(pathticks(A--B,2,0.5,0,100,red));
add(pathticks(AA--BB,2,0.5,0,100,red));
\end{asy}
\end{center}
  %Tegu $M$ yra $BB'$ vidurio taškas. Trikampiai $OAB$ ir
  %$OA'B'$ yra vienodi, todėl $OB = OB'$, ir todėl $\angle
  %OMB' = 90^\circ = \angle OAB = \angle OA'B'$. Gavome, kad
  %keturkampiai $MOAB$ ir $MOA'B'$ yra įbrėžtiniai, ir iš
  %čia $\angle AMO = \angle ABO = \angle A'B'O = \angle
  %A'MO$. Todėl taškai $M, A, A'$ yra vienoje tiesėje.
\item Duotas įbrėžtinis keturkampis $ABCD$. $K, L, M, N$ yra
  atitinkamai kraštinių $AB$, $BC$, $CD$, $DA$ vidurio
  taškai. $P$ yra įstrižainių susikirtimo taškas. Įrodyti,
  kad apie trikampius $PKL$, $PLM$, $PMN$, $PNK$ apibrėžtų
  apskritimų spinduliai yra vienodi.
  %Pastebėkime, kad trikampiai $PAB$ ir $PCD$ yra panašūs, o
  %$PK$ ir $PM$ yra jų abiejų pusiaukraštinės iš atitinkamo
  %kampo. Tada iš vieno anksčiau minėtų naudingų faktų
  %(„Uždavinių apšilimui“ skyrius), $\angle KPB = \angle
  %CPM$. Taigi, $\angle NMP = \angle MPC = \angle KPB =
  %\angle PKN$. Apie trikampius $PNM$ ir $PNK$
  %apibrėžtų apskritimų spinduliai vienodi, nes juose prieš
  %lygius kampus yra lygios kraštinės ( tie apskritimai simetriški
  % $NP$ atžvilgiu). Panašiai ir su
  %kitomis trikampių poromis.
\item Apskritimas $S_1$ su centru $O_1$ eina per kito
  apskritimo $S_2$ centrą $O_2$. Ant $S_1$ taip pat paimtas
  bet koks taškas $C$, ir iš to taško $C$ išvestos liestinės
  apskritimui $S_2$ kerta apskritimą $S_1$ taškuose $A$ ir
  $B$. Įrodyti, kad $AB\perp{O_1 O_2}$. 
  %Apskritimas $S_2$ liečia kampo $\angle ACB$ kraštines,
  %taigi $CO_2$ yra kampo $\angle ACB$ pusiaukampinė.
  %Pritaikę vieną iš šio skyriaus naudingų faktų, turime,
  %kad $AO_2 = BO_2$, t.y $O_1AO_2B$ yra deltoidas. Taigi
  %$AB\perp{O_1 O_2}$.
\item Duotas rombas $ABCD$ su $A=120^\circ$. $M$ ir $N$ yra
  taškai atitinkamai ant kraštinių $BC$ ir $CD$ tokie, kad
  $\angle NAM=60^\circ$. Įrodyti, kad apie trikampį $NAM$
  apibrėžto apskritimo centras guli ant rombo įstrižainės.
  %$\angle NAM + \angle NCM = 60^\circ + 120^\circ =
  %180^\circ$, taigi apie $NAM$ apibrėžtas apskritimas eina
  %per tašką $C$. Taigi jo centras guli $AC$ vidurio
  %statmens, t.y ant įstrižainės $BD$.
\item Duotas trikampis $ABC$. Ant kraštinių $AB$ ir $BC$
  atitinkamai paimti taškai $X$ ir $Y$. $AY$ ir $CX$ kertasi
  taške $Z$. Pasirodė, kad $AY = YC$ ir $AB = ZC$.
  Įrodyti, kad $B, Z, X, Y$ guli ant vieno apskritimo.
  %Paimkime tašką $D$ ant $CY$ tokį, kad $YZ = YD$. Tada iš
  %simetrijos $AD = CZ = AB$. Taigi, $\angle XBY = \angle
  %ABD = \angle ADY = \angle CZY = 180^\circ - \angle XZY$,
  %taigi $XBYZ$ yra įbrėžtinis.
\item Duotas rombas $ABCD$. Ant linijos $CD$ paimtas taškas
  $K$ kuris nesutampa $C$ ar $D$ taip, kad $AD = BK$. $P$
  yra tiesės $BD$ ir atkarpos $BC$ vidurio statmens
  sankirtos taškas. Įrodyti, kad taškai $A$, $P$, $K$ guli
  ant vienos tieses.
  %Kadangi yra įmanomos kelios skirtingos konfigūracijos ir
  %uždavinys labai lengvas, tai pateiksime tik nepilną
  %sprendimą: abiem atvejais keturkampiai $BKCP$ ir $DKAB$
  %yra įbrėžtiniai; vienu atveju tada nesunkiai įrodome, kad
  %$\angle PKB + \angle AKB = 180^\circ$, kitu atveju
  %$\angle APC + \angle KPC = 180^\circ$, iš ko ir seka
  %rezultatas.
\item Trikampyje $ABC$ nubrėžtos aukštinės $BE$ ir $AD$
  kertasi taške $H$. $X$ ir $Y$ - atitinkamai $CH$ ir $AB$
  vidurio taškai. Įrodyti, kad $XY\perp{DE}$.
  %Keturkampis $CEHD$ yra įbrėžtinis, nes $\angle HEC +
  %\angle HDC = 180^\circ$. Be to, $X$ yra apie šį
  %keturkampį apibrėžto apskritimo centras. Panašiai ir
  %$BDEA$ yra įbrėžtinis, o apie jį apibrėžto apskritimo
  %centras yra $Y$. Šie du apskritimai kertasi taškuose $D$
  %ir $E$, taigi $DXEY$ yra deltoidas, ir todėl
  %$XY\perp{DE}$.
\item Duotas trikampis $ABC$ ir taškas $P$ jo viduje toks,
  kad $\angle ABP = \angle ACP$ ir $\angle CBP = \angle
  CAP$.  Įrodyti, kad $P$ yra trikampio $ABC$ aukštinių
  susikirtimo taškas.
  %Tegu $AP, BP, CP$ kerta trikampio kraštines taškuose $A',
  %B', C'$. Tada keturkampiai $ABA'B'$ ir $BCB'C'$ yra
  %įbrėžtiniai, nes $\angle ABP = \angle ACP$ ir $\angle CBP
  %= \angle CAP$. Bet tada $\angle B'C'C = \angle B'BC =
  %\angle A'AC$, todėl keturkampis $AB'PC'$ yra įbrėžtinis.
  %Taigi, $\angle AB'P = \angle BC'P = \angle BB'C =
  %180^\circ - \angle AB'P$, taigi $\angle AB'P = 90^\circ =
  %\angle AC'P$, t.y $CC'$ yra aukštinė. Panašiai ir $AA'$
  %yra aukštinė, todėl $H$ yra aukštinių sankirtos taškas.
\item Duotas įbrėžtinis keturkampis $ABCD$, ant spindulio
  $AD$ už taško $D$ paimtas taškas $E$ taip, kad $AC=CE$,
  $\angle BDC=\angle DEC$. Įrodyti, kad $AB=DE$. 
  %Trikampis $ACE$ yra lygiašonis, taigi $\angle EAC =
  %\angle DAC = \angle DEC = \angle BDC = \angle BAC$, taigi
  %$BC = CD$. Be to, $\angle DEC = \angle BAC$ ir $\angle
  %EDC = \angle ABC$, taigi trikampiai $ABC$ ir $EDC$ yra
  %vienodi pagal kraštinę ir 2 kampus. Todėl $AB = DE$.
\item Duotas lygiašonis trikampis $ABC$ su $AB = BC$. Ant
  $AB, BC, CA$ atitinkamai paimti taškai $C_1, A_1, B_1$
  tokie, kad $\angle BC_1A_1 = \angle CA_1B_1 = \angle A$,
  $P$ - atkarpų $BB_1$ ir $CC_1$ sankirtos taškas. Įrodyti,
  kad keturkampis $AB_1PC_1$ yra įbrėžtinis.
  %Iš duotų kampų mes turime $C_1A_1\parallel{AC}$, taigi
  %$CAC_1A_1$ yra lygiašonė trapecija. Be to, keturkampis
  %$ABA_1B_1$ yra įbrėžtinis, taigi $\angle AB_1B = \angle
  %AA_1B = \angle CC_1B = 180^\circ - \angle AC_1C$, taigi
  %keturkampis $AB_1PC_1$ yra įbrėžtinis.
\item Duotas statusis trikampis su stačiu kampu $B$. Per
  tašką $B$ išvesta pusiaukraštinė $BM$. Į trikampį $ABM$
  įbrėžtas apskritimas liečia $AM$ ir $AB$ taškuose $K$ ir
  $L$. Jei $LK\parallel{BM}$, raskite kampą $\angle ACB$.
  %Jei $LK\parallel{BM}$, tai trikampiai $ALK$ ir $ABM$ yra
  %panašūs. Tačiau $ALK$ yra lygiašonis su $AL = AK$, o
  %$ABM$ lygiašonis su $MB = MA$, taigi $AB = AM = MA$.
  %Todėl $ABM$ yra lygiakraštis, ir iš čia $\angle BCA =
  %30^\circ$.
\item Duotas trikampis $ABC$ su aukštinėmis $AA_1$, $BB_1$.
  Kampo $C$ pusiaukampinė kerta $AA_1$ ir $BB_1$ atitinkamai
  taškuose $F$ ir $L$. Įrodyti, kad atkarpos $FL$ vidurio
  taškas vienodai nutolęs nuo taškų $B_1$ ir $A_1$.
  %Tegu $H$ yra aukštinių susikirtimo taškas, o $M$ yra $FL$
  %vidurio taškas. Tada trikampiai $CA_1F$ ir $CB_1L$ yra
  %panašūs pagal du kampus, todėl $\angle CFA_1 = \angle
  %CLB_1$, todėl $HFL$ yra lygiašonis pagal du kampus. Tada
  %$\angle HMC = 90^\circ = \angle HA_1C = \angle HB_1C$,
  %tad penkiakampis $CA_1HMB_1$ yra įbrėžtinis. Kadangi $CM$ 
  %yra kampo $\angle A_1HC_1$ pusiaukampinė, mes turime $MB_1
  % = MA_1$.  
\item Duotas deltoidas $ABCD$, $AB=BC$, $AD=DC$. Ant
  įstrižainės $AC$ paimtas toks taškas $K$, kad $BK = KA$.
  Jei keturkampis $CDKB$ yra  įbrėžtinis, tai įrodykite, kad
  $CD = BD$.
  %Pastebėkime, kad trikampiai $AKB$ ir $ABC$ yra panašūs.
  %Tada iš įbrėžtinių kampų ir priekampio savybės
  %$180^\circ - \angle DBC - \angle BCD = \angle BDC =
  %\angle BKC = \angle BAK + \angle ABK = 2 \cdot \angle BCA
  %= 2 \cdot (90^\circ - \angle DBC) = 180^\circ - \angle
  %DBC - \angle DBC$, todėl $\angle DBC = \angle BCD$, tad
  %$BD = DC$. 
\item Duotas kvadratas $ABCD$. Ant $AB$ paimtas taškas $K$,
  ant $CD$ - $L$, o ant $KL$ - $M$. Įrodykite, kad apie
  $AKM$ ir $MLC$ apibrėžtų apskritimų sankirtos taškas
  (kitas negu $M$) yra ant įstrižainės $AC$.
  %Tegu apie trikampį $MLC$ apibrėžtas apskritimas kerta
  %$AC$ taške $F$. Tada $\angle MFC = \angle MLC = \angle
  %AKM$, todėl keturkampis $AKMF$ yra įbrėžtinis, ko ir
  %reikėjo.
  %\begin{center}
  %\begin{asy}
  %import olympiad;
  %size(200);
  %pair A, B, C, D, K, L, M, F;
  %path a1;
  %D=(0,0);C=(100,0);A=(0,100);B=(100,100);
  %K=waypoint(A--B,0.8);
  %L=waypoint(D--C,0.5);
  %M=waypoint(L--K,0.8);
  %a1=circumcircle(L,C,M);
  %F=intersectionpoints(A--C,a1)[0];
  %add(anglem(C,L,M,300,red,0));
  %add(anglem(A,K,M,300,red,0));
  %add(anglem(C,F,M,300,red,0));
  %draw(a1);
  %dot(A,blue);
  %dot(B,blue);
  %dot(C,blue);
  %dot(D,blue);
  %dot(K,blue);
  %dot(L,blue);
  %dot(M,blue);
  %dot(F,blue);
  %draw(A--B--C--D--cycle);
  %draw(A--C);
  %draw(K--L);
  %draw(M--F);
  %label("$A$",A,NW,blue);
  %label("$B$",B,NE,blue);
  %label("$C$",C,SE,blue);
  %label("$D$",D,SW,blue);
  %label("$K$",K,up,blue);
  %label("$L$",L,down,blue);
  %label("$M$",M,NE,blue);
  %label("$F$",F,left,blue);
%\end{asy}
  %\end{center}
\item Duotas trikampis $ABC$, jame išvestos aukštinė $AH$ ir
  pusiaukampinė $BE$. Žinoma, kad $\angle BEA = 45^\circ$.
  Įrodyti, kad $\angle EHC = 45^\circ$.
  %Tegu taškas $Q$ yra simetriškas taškui $A$ atkarpos $BE$
  %atžvilgiu. Šis taškas yra ant atkarpos $BC$ bei $\angle
  %QEA = 90^\circ = \angle QHA$ todėl keturkampis $AHQE$ yra
  %įbrėžtinis. Tad $\angle EHC = \angle EHQ = \angle EAQ =
  %45^\circ$ (nes $AQE$ yra status lygiašonis).
\item Duotas trikampis $ABC$ su pusiaukampinėmis $AL$ ir
  $BM$. Jei trikampių $ACL$ ir $BCM$ apibrėžtiniai
  apskritimai kertasi ant atkarpos $AB$, tai įrodykite, kad
  $\angle ACB=60^\circ$.
  %Tegu trikampių $ACL$ ir $BCM$ apibrėžtiniai apskritimai
  %kertasi ant atkarpos $AB$, taške $X$. Tada $\angle BCA =
  %\angle BCX + \angle ACX = \angle LCX + \angle MCX =
  %\angle LAX + \angle MBX = \frac{\angle BAC}{2} +
  %\frac{\angle CBA}{2} = \frac{\angle BAC+\angle CBA}{2} =
  %\frac{180^\circ-\angle BCA}{2} = 90^\circ - \frac{\angle
  %BCA}{2}$, taigi $\angle BCA = 60^\circ$.
\item Trikampio $ABC$ pusiaukampinės $BD$ ir $CE$ kertasi
  taške $O$. Įrodykite, kad jeigu $OD = OE$, tai arba
  trikampis $ABC$ lygiašonis, arba kampas $\angle
  A=60^\circ$.
  %Pritaikome vieną iš minėtųjų naudingų faktų keturkampiui
  %$ADOE$, ir gauname, kad $ADOE$ yra arba įbrėžtinis, arba
  %deltoidas. Jeigu jis deltoidas, tai iš simetrijos $ABC$
  %yra lygiašonis. Jeigu jis įbrėžtinis, tai tada paimkime
  %tašką $H$ ant $BC$ tokį, kad $\angle ODA = \angle OHC =
  %\angle OEB$.  Tada keturkampiai $ABHD$ ir $AEHC$ yra
  %įbrėžtiniai ir toliau mes galime tęsti kaip prieš tai
  %buvusiame uždavinyje.
\item Duotas trikampis $ABC$ su $\angle ABC = 60^\circ$.
  $I$ - įbrėžto apskritimo centas, $CL$ - pusiaukampinė.
  Apskritimas, apibrėžtas apie trikampį $ALI$, kerta $AC$
  antrą kartą taške $D$. Įrodyti, kad $BCDL$ - įbrėžtinis.
  %Iš duotos sąlygos $\frac{\angle A+\angle C}{2} =
  %60^\circ$.  Tada iš priekampio savybės $\angle LDA =
  %\angle LIA = \frac{A}{2} + \frac{\angle C}{2} = 60^\circ
  %= \angle LBC$, taigi keturkampis $BCDL$ įbrėžtinis.
\item Duotas iškilasis šešiakampis $ABCDEF$. Žinoma, kad $AD
  = BE = CF$. $AD$ ir $CF$ sankirtos taškas yra $P$, $BE$ ir
  $CF$ sankirtos taškas yra $R$, o $AD$ ir $BE$ - taškas
  $Q$.  Jeigu $AP = PF$, $BR = CR$, $DQ = EQ$ tai įrodyti,
  kad šešiakampis yra įbrėžtinis.
  %Iš duotų sąlygų keturkampiai $CDFA$, $DEAB$ ir $EFBC$ yra
  %lygiašonės trapecijos. Tada kraštinių $CD$ ir $AF$ vidurio
  %statmenys sutampa, ir jie taip pat sutampa su kampo $\angle
  %QPR$ pusiaukampine. Panašiai ir su kitomis kraštinėmis ir
  %jų vidurio statmenimis. Tada visų kraštinių $AB$, $BC$,
  %$CD$, $DE$, $EF$, $FA$ vidurio statmenys kertasi viename
  %taške, nes trikampio $PQR$ pusiaukampinės kertasi viename
  %taške. Todėl visos šešiakampio viršūnės vienodai
  %nuotolusios nuo šio taško, tad šešiakampis yra įbrėžtinis. 
\item $AC$ ir $BD$ yra du statmeni kažkokio apskritimo
  skersmenys. Taškas $K$ ant apskritimo nesutampa su taškais
  $A, B, C, D$. $AK$ ir $BD$ kertasi taške $M$, o $DK$ ir
  $CB$ - taške $N$. Įrodyti, kad $AC\parallel{MN}$.
  %Tarkime, kad taškas $K$ yra ant mažesniojo lanko $AB$
  %(Kiti atvejai sprendžiami panašiai). Tada $\angle KNB =
  %180^\circ - \angle NKB - \angle NBK = 180^\circ -
  %90^\circ - \angle KAC = \angle AMD = \angle KMB$, taigi
  %$KNMB$ įbrėžtinis. Tada $\angle ACN = \angle BKM = \angle
  %BNM$, taigi $AC\parallel{NM}$. 
\item Duotas smailusis trikampis $ABC$. Jame nubrėžtos
  aukštinės $AA_1, BB_1, CC_1$. Įrodyti, kad $C_1$
  projekcijos į tieses $AC$, $BC$, $BB_1$, $CC_1$ yra
  vienoje tieseje.
  %Tegu visi taškai yra tokie, kokie pažymėti brėžinyje
  %apačioje. Tada keturkampiai $KPC_1B$ ir $ FPC_1L$ yra
  %įbrėžtiniai. Taigi, $\angle KPB = \angle KC_1B =
  %180^\circ - 90^\circ - \angle B = \angle BCC_1 = \angle
  %CC_1L = \angle FPL$, taigi $KP\parallel{PL}$ ir todėl
  %taškai $K, P, L$ yra vienoje tiesėje. Panašiai ir taškai
  %$P, L, H$ guli vienoje tiesėje. Todėl taškai $K, P, L, H$
  %guli vienoje tiesėje.
  %\begin{center}
  %\begin{asy}
  %import olympiad;
  %size(200);
  %pair A, B, C, AA, BB, CC, K, P, L, H, F;
  %C=(35,60); A=(80,0); B=(0,0); 
  %AA=foot(A,B,C);
  %BB=foot(B,A,C);
  %CC=foot(C,B,A);
  %K=foot(CC,B,C);
  %P=foot(CC,B,BB);
  %L=foot(CC,A,AA);
  %H=foot(CC,A,C);
  %F=extension(A,AA,B,BB);
  %add(anglem(K,P,B,200,red,0));
  %add(anglem(K,CC,B,200,red,0));
  %draw(A--B--C--cycle);
  %draw(A--AA);
  %draw(B--BB);
  %draw(C--CC);
  %draw(CC--K);
  %draw(CC--P);
  %draw(CC--L);
  %draw(CC--H);
  %draw(K--H);
  %dot(A,blue);
  %dot(B,blue);
  %dot(C,blue);
  %dot(AA,blue);
  %dot(BB,blue);
  %dot(CC,blue);
  %dot(F,blue);
  %dot(K,blue);
  %dot(P,blue);
  %dot(L,blue);
  %dot(H,blue);
  %label("$A$",A,SE,blue);
  %label("$B$",B,SW,blue);
  %label("$C$",C,up,blue);
  %label("$F$",F,(0.8,1.5),blue);
  %label("$A_1$",AA,NW,blue);
  %label("$B_1$",BB,NE,blue);
  %label("$C_1$",CC,down,blue);
  %label("$L$",L,up,blue);
  %label("$K$",K,NW,blue);
  %label("$P$",P,up,blue);
  %label("$H$",H,NE,blue);
  %add(rightanglem(A,AA,B,120));
  %add(rightanglem(B,BB,A,120));
  %add(rightanglem(CC,K,B,120));
  %add(rightanglem(CC,P,B,120));
  %add(rightanglem(CC,L,A,120));
  %add(rightanglem(CC,H,A,120));
  %add(rightanglem(A,CC,C,120));
%\end{asy}
  %\end{center} 
  \item Smailajame trikampyje $ABC$ nubrėžti statmenys $BD$ ir $CE$.
Apskritimas su skersmeniu $AC$ kerta spindulį $DB$ taške $F$, o 
apskritimas su skersmeniu $AB$ kerta spindulį $EC$ taške $G$ ir
spindulį $CE$ taške $H\neq G$. sĮrodyti, kad $\angle FHG+\angle
 FGA=90^\circ$.
 % Kadangi $\angle ADF=\angle AFC$, tai $AF^2=AD\cdot AC$.
 %Panašiai ir $AG^2=AE\cdot AB$. Bet $BEDC$ įbrėžtinis, taigi
 %$AE\cdot AB=AD\cdot AF$. Taigi $AG=AF$. Panašiai gauname, kad 
 %$CF^2=CD\cdot CA=CG\cdot CH$. Tada $\angle FHG+\angle FGA=
 %\angle CFG+\angle AFG=90^\circ$
\end{enumerate}



\newpage

\section{Plotai}

Iki šiol beveik visi uždaviniai buvo apie kampus ir
kraštines. Tačiau geometrija nėra vien tik kampai ir
kraštinės - nedažnai, tačiau pasitaiko uždavinių apie
plotus, perimetrus, geometrines nelygybes, tapatybes ir
panašiai. Šio skyrelio tema yra plotų uždaviniai. 
\textbf{Trikampio  $ABC$ plotą, jei nepasakyta kitaip,
 žymėsime $S_{ABC}$}.

\subsubsection{Tai, kas svarbiausia}

\begin{teig}
Trikampio plotas yra lygus trikampio kraštinei, padaugintai
iš aukštinės, nuleistos į  tą kraštinę ir padalinus iš
dviejų. Bene svarbiausia išvada iš to yra ta, kad
trikampiai, turintys tą pačią ar tokią pačią kraštinę, ir
aukštinę, nuleistą į tą kraštinę, turi vienodus plotus, pvz
paveikslėlyje apačioje visi trikampiai su kraštine  $AB$
turi vienodus plotus tada ir tik tada, jei tiesė $b$ yra
 lygiagreti tiesei $AB$:
%Vienodi plotai
\begin{center}
\begin{asy}
import olympiad;
size(200);
pair A, B, C, D, E, F, G, H;
A=(30,0); B=(70,0);
C=(0,30); D=(30,30);
E=(60,30); F=(90,30);
G=(120,30);
draw(A--B);
draw(C--A);
draw(C--B);
draw(D--A);
draw(D--B);
draw(E--A);
draw(E--B);
draw(F--A);
draw(F--B);
draw(G--A);
draw(G--B);
drawline(C,D);
draw((-10,0)--(130,0),invisible);
dot(A,blue);
dot(B,blue);
dot(C,blue);
dot(D,blue);
dot(E,blue);
dot(F,blue);
dot(G,blue);
label("$A$",A,down,blue);
label("$B$",B,down,blue);
label("$b$",C,NW+3*W+N/2,blue);
\end{asy}
\end{center}

Tai yra labai naudingas faktas interpretuojant sąlygą
geometriškai ar bandant įrodyti kokią nors tapatybę su
plotais.
\end{teig}

\begin{pav}
Duotas trikampis $ABC$, ant $BC$ paimtas bet koks 
taškas $E$, o ant $CA$ paimtas bet koks taškas $F$. $M$ ir $N$
yra atitinkamai $AE$ ir $BF$ vidurio taškai. Įrodyti, kad $S_{CFM}=
S_{CEN}$.
\end{pav}
\begin{sprendimas}
Tegu $L$ yra $FE$ vidurio taškas. Tada $LN\parallel CB$, taigi 
$S_{CEN}=S_{CLE}=\frac{S_{CFE}}{2}$. Panašiai ir $S_{CFM}=
\frac{S_{CFE}}{2}$.
\begin{center}
\begin{asy}
import olympiad;
size(200);
pair A, B, C, E, F, M, N, L;
A=(0,0);
B=(40,60);
C=(100,0);
E=2*C/3+B/3;
F=C/4+3*A/4;
M=A/2+E/2;
N=B/2+F/2;
L=E/2+F/2;
dot(A,blue);
dot(B,blue);
dot(C,blue);
dot(M,blue);
dot(E,blue);
dot(F,blue);
dot(L,blue);
dot(N,blue);
fill(C--E--N--cycle,cyan);
fill(C--F--M--cycle,cyan);
draw(A--B--C--cycle);
draw(A--E);
draw(B--F);
draw(E--F);
draw(L--M);
draw(L--N);
label("$A$",A,left,blue);
label("$B$",B,left,blue);
label("$C$",C,right,blue);
label("$E$",E,right,blue);
label("$F$",F,down,blue);
label("$L$",L,down,red);
label("$M$",M,NW,blue);
label("$N$",N,left,blue);
add(pathticks(B--N,2,0.5,0,125,red)); 
add(pathticks(F--N,2,0.5,0,125,red)); 
add(pathticks(A--M,2,0.5,30,125,red)); 
add(pathticks(E--M,2,0.5,30,125,red)); 
\end{asy}
\end{center}
\end{sprendimas}

\begin{teig}
Trikampio plotas yra lygus dviejų jo kraštinių sandaugai
padaugintai iš kampo tarp jų sinusui ir padalinus iš dviejų;
todėl jeigu turime du trikampius, vieną su kraštinėmis $a$,
$b$ ir kampu $\alpha$ tarp jų, ir kitą su kraštinėmis $a$,
$b$ ir kampu $180^\circ-\alpha$ tarp jų, tai tų trikampių
plotai lygūs.
\end{teig}

\begin{teig}
Iškilojo keturkampio plotas yra lygus įstrižainių sandaugai,
padaugintai iš kampo tarp įstrižainių sinuso ir padalintai
iš dviejų. Dėl to, pavyzdžiui, jeigu keturkampio įstrižainės
statmenos, tai jo plotas lygus įstrižainių sandaugos pusei.  
\end{teig}

\begin{pav}
Keturkampis įstrižainėmis padalintas į keturis trikampius. 
Įrodyti, kad priešingų trikampių plotų sandaugos lygios.
\end{pav}

\begin{sprendimas}
Tegu įstrižainės dalija viena kitą į keturias atkarpas, kurių
ilgiai yra $a,b,c,d$, o kampas tarp įstrižainių yra $\alpha$.
Tada ieškomos plotų sandaugos bus lygios $abcd (\sin \alpha)^2 = 
abcd(\sin (180^\circ-\alpha))^2 $
\end{sprendimas}

Dar vienas naudingas triukas sprendžiant įvairius uždavinius
(ne tik geometrijos) kuriuose reikia įrodyti kokią nors lygybę
yra pridėti ar atimti tą patį dydį prie abiejų lygybės pusių.
Geometrijoje kartais to pakanka išspręsti uždaviniui.

\begin{pav}
Duotas lygiagretainis $ABCD$, ant $BC$ paimtas bet koks 
taškas $E$, $AE$ ir $BD$ kertasi taške $F$. Įrodyti, kad
$S_{BFE}+S_{ECD}=S_{AFD}.$ 
\end{pav}

\begin{sprendimas}
Pridedame prie abiejų pusių po $S_{FED}$ ir viskas pasidaro 
akivaizdu.
\begin{center}
\begin{asy}
import olympiad;
size(200);
pair A, B, C, D, E, F;
A=(0,0);
D=(80,0);
B=(20,50);
C=B+D;
E=B+D*2/7;
F=extension(A,E,B,D);
fill(B--F--E--cycle,cyan);
fill(C--E--D--cycle,cyan);
fill(A--F--D--cycle,green);
dot(A,blue);
dot(B,blue);
dot(D,blue);
dot(C,blue);
dot(E,blue);
dot(F,blue);
draw(A--B--C--D--cycle);
draw(A--E);
draw(B--D);
draw(E--D);
label("$A$",A,left,blue);
label("$B$",B,left,blue);
label("$D$",D,right,blue);
label("$C$",C,right,blue);
label("$E$",E,up,blue);
label("$F$",F,left,blue);
\end{asy}
\end{center}
\end{sprendimas}

\subsubsection{Uždaviniai}

\begin{enumerate}
\item Įrodykite, kad iš visų keturkampių, įbrėžtų į fiksuoto
  spindulio apskritimą, didžiausią plotą turi kvadratas.
  %Tegu $R$ - apskritimo spindulys, $a,b$ - keturkampio
  %įstrižainių ilgiai, $\alpha$ - kampas tarp įstrižainių,
  %$S$ - keturkampio plotas. Tada $S = \frac{a\cdot
  %b\cdot\sin \alpha}{2} \leq \frac{2R \cdot 2R \cdot 1}{2}
  %= 2R^2$, kas yra kvadrato, įbrėžto į apskritimą su
  %spinduliu $R$, plotas.

\item Duotas įbrėžtas keturkampis $ABCD$, kurio įstrižainės
  yra statmenos. Įrodyti, kad laužtė $AOC$ dalija keturkampį
  į dvi lygiaplotes dalis.
  %Tegu $P$ yra $O$ projekcija į $BD$. Tarkime, kad $B$ ir
  %$O$ yra skirtingose $AC$ pusėse. Tada $ABCO$ plotas lygus
  %$ABCP$ plotui, kuris yra $\frac{BP\cdot AC}{2} = \frac{BD
  %\cdot AC}{4}$, t.y pusė $ABCD$ ploto. 

\item Iškilajame šešiakampyje $ABCDEF$ $AB\parallel{CF}$,
  $CD\parallel{BE}$, $EF\parallel{AD}$. Įrodyti, kad
  trikampių $ACE$ ir $BDF$ plotai lygūs.
  %Paveikslėliuose žemiau abu šešiakampiai išskaidyti į 7
  %dalis, ir lygiaplotės nuspalvintos vienoda
  %spalva. (Naudojamės tuo, kad jei dviejų trikampių
  %pagrindai ir aukštinės lygios, tai ir plotai lygūs, pavyzdžiui,
  % abiejų mėlynų dalių plotai yra lygūs trikampio $AFX$ plotui).
  %\begin{center}
  %\begin{asy}
  %import olympiad;
  %size(200);
  %pair A, B, C, D, E, F, X, Y, Z;
  %A=(18,40);
  %B=(80,30);
  %C=(90,0);
  %F=C+(A-B)*1.4;
  %D=(36,-40);
  %E=extension(B,B+D-C,F,F+D-A);
  %X=extension(A,D,F,C);
  %Y=extension(E,B,F,C);
  %Z=extension(E,B,A,D);
  %fill(B--X--F--cycle,cyan);
  %fill(B--X--Y--cycle,orange);
  %fill(B--Y--D--cycle,magenta);
  %fill(D--Y--Z--cycle,yellow);
  %fill(D--Z--F--cycle,brown);
  %fill(Z--X--F--cycle,green);
  %fill(B--X--F--cycle,cyan);
  %fill(X--Y--Z--cycle,red);
  %draw(A--B--C--D--E--F--cycle);
  %draw(D--A);
  %draw(C--F);
  %draw(E--B);
  %draw(F--B);
  %draw(X--B);
  %draw(Y--B);
  %draw(D--B);
  %draw(D--Y);
  %draw(D--Z);
  %draw(D--F);
  %draw(F--Z);
  %draw(F--X);
  %dot(A,blue);
  %dot(B,blue);
  %dot(C,blue);
  %dot(D,blue);
  %dot(E,blue);
  %dot(F,blue);
  %label("$A$",A,up,blue);
  %label("$B$",B,NE,blue);
  %label("$C$",C,right,blue);
  %label("$D$",D,down,blue);
  %label("$E$",E,SW,blue);
  %label("$F$",F,left,blue);
  %label("$X$",X,right,blue);
  %label("$Y$",Y,right,blue);
  %label("$Z$",Z,right,blue);
%\end{asy}
  %\begin{asy}
  %import olympiad;
  %size(200);
  %pair A, B, C, D, E, F, X, Y, Z;
  %A=(18,40);
  %B=(80,30);
  %C=(90,0);
  %F=C+(A-B)*1.4;
  %D=(36,-40);
  %E=extension(B,B+D-C,F,F+D-A);
  %X=extension(A,D,F,C);
  %Y=extension(E,B,F,C);
  %Z=extension(E,B,A,D);
  %fill(A--X--E--cycle,cyan);
  %fill(A--X--Y--cycle,orange);
  %fill(A--Y--C--cycle,magenta);
  %fill(C--Y--Z--cycle,yellow);
  %fill(C--Z--E--cycle,brown);
  %fill(Z--X--E--cycle,green);
  %fill(A--X--E--cycle,cyan);
  %fill(X--Y--Z--cycle,red);
  %draw(A--B--C--D--E--F--cycle);
  %draw(D--A);
  %draw(C--F);
  %draw(E--B);
  %draw(E--A);
  %draw(X--A);
  %draw(Y--A);
  %draw(C--A);
  %draw(C--Y);
  %draw(C--Z);
  %draw(C--E);
  %draw(E--Z);
  %draw(E--X);
  %dot(A,blue);
  %dot(B,blue);
  %dot(C,blue);
  %dot(D,blue);
  %dot(E,blue);
  %dot(F,blue);
  %label("$A$",A,up,blue);
  %label("$B$",B,NE,blue);
  %label("$C$",C,right,blue);
  %label("$D$",D,down,blue);
  %label("$E$",E,SW,blue);
  %label("$F$",F,left,blue);
  %label("$X$",X,right,blue);
  %label("$Y$",Y,right,blue);
  %label("$Z$",Z,right,blue);
%\end{asy}
  %\end{center}
\item Duotas kvadratas $ABCD$ su kurio kraštinės ilgis 1.
  Ant $AB$, $BC$, $CD$, $DA$ atitinkamai paimti taškai $K$,
  $L$, $M$, $N$ taip, kad $KM\parallel{BC}$ ir
  $NL\parallel{AB}$. Jei $BKL$ perimetras yra lygus 1, rasti
  trikampio $DMN$ plotą.
  %Tegu $KB = a$, $LB = b$. Tada $1 = a + b + \sqrt{a^2+b^2}
  %\implies \sqrt{a^2+b^2} = 1 - a - b \implies a^2 + b^2 =
  %1 + a^2 + b^2 - 2a - 2b + 2ab \implies 1 - 2a - 2b + 2ab
  %= 0 \implies $$2(1 - a)(1 - b) = 1 \implies \frac{1}{4} =
  %\frac{(1-a)(1-b)}{2}$. Taigi $DMN$ plotas yra
  %$\frac{1}{4}$. 
 
\item Iškilojo keturkampio įstrižainės dalija jį į keturis
  mažus trikampius. Pasirodė, kad dviejų priešingų trikampių
  plotų suma yra lygi kitų dviejų trikampių plotų sumai.
  Įrodyti, kad viena iš įstrižainių dalija kitą pusiau.
  %Tarkime priešingai. Tegu kampas tarp įstrižainių yra
  %$\alpha$, o įstrižainės dalija viena kitą į atkarpas
  %ilgio $a_1, a_2, b_1, b_2$. Tarkime, kad $a_1 > a_2, b_1
  %> b_2$.  Tada iš sąlygos $\frac{a_1b_1\sin\alpha}{2} +
  %\frac{a_2b_2\sin\alpha}{2} = \frac{a_1b_2\sin\alpha}{2} +
  %\frac{a_2b_1\sin\alpha}{2}$.  (Prisiminkite, kad $\sin
  %\alpha = \sin(180^\circ-\alpha)$).  Bet tai ekvivalentu
  %$(a_1 - a_2)(b_1 - b_2)\sin \alpha = 0$, kas yra
  %neįmanoma, nes kairė lygybės pusė yra griežtai teigiama.
\item Iškilajame šešiakampyje $AC'BA'CB'$  $AB'=AC'$,
  $BC'=BA'$, $CA'=CB'$ ir $\angle A+\angle B+\angle C=\angle
  A'+\angle B'+\angle C'$. Įrodykite, kad $ABC$ plotas yra
  lygus pusei šešiakampio ploto. 
  %Pastebėkime, kad iš sąlygos $\angle A' + \angle B' +
  %\angle C' = 360^\circ$. Nuspalvinkime $AB'$ ir $AC'$
  %mėlynai, $BC'$ ir $BA'$ žaliai, $CA'$ ir $CB'$ raudonai.
  %Tada iškirpkime $B'AC, C'AB, A'BC$ iš popieriaus ir iš jų
  %sudėkime trikampį (dedame taip, kad vienodos spalvos
  %briaunos sutaptų). Tada sudėtas trikampis yra toks pat
  %kaip ir $ABC$, nes jų kraštinės vienodo ilgio. Iš čia ir
  %seka rezultatas.

\item Duota trapecija $ABCD$ su pagrindais $AB$ ir $CD$.
  $M$-$AD$ vidurio taškas. Įrodyti, kad $BCM$ plotas yra lygus
  pusei trapecijos ploto.
  % $S_{ABCD}=S_{ABC}+S_{ADC}=S_{ABD}+S_{ADC}=2(S_{ABM}+S_{DCM})=
  % 2(S_{ABCD}-S_{BCM})$, iš ko gauname rezultatą. 
  
\item Duotas iškilasis keturkampis $ABCD$. Paimti taškai $E,F$
  ant $BC$ tokie, kad $BE=EF=FC$. Paimti taškai $H,G$ ant $AD$ tokie, 
  kad $AH=HG=GD$. Įrodyti, kad $S_{EFGH}=\frac{S_{ABCD}}{3}$.
  %$S_{EFGH}=S_{EFG}+S_{EGH}=S_{BEG}+S_{GDE}=S_{BEDG}=S_{BGD}+S_{BED}=
  %\frac{S_{ABD}}{3}+\frac{S_{CDB}}{3}=\frac{S_{ABCD}}{3}$.
   
\item Įbrėžtiniame keturkampyje $ABCD$ $BC=CD$. Įrodykite,
  kad jo plotas lygus $\frac{AC^2\sin A}{2}$. 
  %Vėl naudosime „trikampio pribrėžimą“: pratesiame $AD$ iki 
  %taško $E$ taip, kad $DE=AB$. Trikampiai $ABC$ ir $EDC$ tada
  %yra vienodi pagal dvi kraštines ir kampą. Taigi $ABCD$
  %plotas yra lygus $ACE$ plotui, o $ACE$ plotas yra lygus
  %$\frac{AC\cdot AE\sin \angle ACE}{2} = \frac{AC\cdot AC
  %\sin\angle BCD}{2} = \frac{AC^2 \sin A}{2}$ (Galite palyginti
  % su priešpaskutiniu pavyzdžiu iš „geometrinių nelygybių“ skyrelio).

\item Duotas įbrėžtinis šešiakampis $ABCDEF$. Pasirodė, kad
  $AB=BC$, $CD=DE$, $EF=FA$. Įrodyti, kad trikampio $BDF$
  plotas lygus pusei šešiakampio ploto.
  %Užtenka įrodyti, kad šešiakampio kampų $B, D, F$ suma yra
  %$360^\circ$ ir tęsti kaip 6 uždavinyje. O tai yra beveik
  %akivaizdu: $\angle B + \angle D + \angle F = (180^\circ -
  %\angle AEC) + (180^\circ - \angle CAE) + (180^\circ -
  %\angle ACE) = 540^\circ - 180^\circ = 360^\circ$.

\item Duotas smailusis trikampis $ABC$, $O$-apibrėžto
  apskritimo centras, $BO$ kerta apibrėžtinį apskritimą
  antrą kartą taške $D$, o aukštinės iš viršūnės $A$ tęsinys
  kerta apskritimą taške $E$. Įrodyti, kad trikampio $ABC$
  plotas lygus keturkampio $BECD$ plotui.
  %Tegu trikampyje $ABC$ yra aukštinė $AH$, o $P$ yra $D$
  %projekcija į $AH$. Tada $ABC$ plotas yra $\frac{BC\cdot
  %AH}{2}$, o $BDCE$ plotas yra $\frac{BC\cdot DC+BC\cdot
  %HE}{2}$. Taigi užtenka įrodyti, kad $DC + HE = AH$. Tai
  %akivaizdu, nes $CD = HP$ ir $HE = AP$ (AECD yra lygiašonė
  %trapecija).

\item  Ant lygiagretainio kraštinių paimta po vieną tašką. 
  Keturkampio, kurio viršūnės yra tuose taškuose, plotas yra
  dvigubai mažesnis nei lygiagretainio. Įrodyti, kad bent viena 
  keturkampio įstrižainė yra lygiagreti lygiagretainio kraštinei.
  %Tegu $ABCD$ yra lygiagretainis, o taškai $K,L,M,N$ yra 
  %atitinkamai ant kraštinių $AB,BC,CD,DA$. Tarkime, kad $KM$ 
  %ir $LN$ nėra lygiagrečios lygiagretainio kraštinėms. Paimkime
  %tašką $Z$ ant $AB$ tokį, kad $MZ\parallel BC$. Tada abiejų
  %keturkampių $KLMN$ ir $ZLMN$ yra lygūs pusei lygiagretainio 
  %ploto, ir todėl trikampių $ZLN$ ir $KLN$ plotai lygūs. Bet tada
  %$ZK\parallel LN$, prieštara. 

\item Tegu $ABCDE$ yra iškilasis penkiakampis toks, kad
  $AB=AE=CD=1$, $\angle ABC=\angle DEA=90^\circ$ ir
  $BC+DE=1$. Rasti penkiakampio plotą.  
  %Paimkime tašką $M$ ant spindulio $DE$ taip, kad $DM=1$.
  %Trikampiai $ABC$ ir $AEM$ yra vienodi pagal 2 kraštines ir kampą.
  % Tada $ABCDE$
  %plotas lygus $ACDM$ plotui. Bet $MD =
  %1 = CD$ ir $AM = AC$, taigi $ACDM$ yra deltoidas. Bet
  %$AMD$ plotas yra $\frac{MD\cdot AE}{2} = 0.5$. Todėl
  %penkiakampio plotas yra 1.

\end{enumerate}


\newpage
\section{Apibrėžtinės figūros}

Apskritimų skyriuje daugiausia dėmesio buvo skiriama
įbrėžtinėms figūroms, t.y. toms, kurios buvo apskritimų
viduje. Šio skyrelio tema yra apibrėžtinės figūros, todėl
čia svarbiausia bus tai, kas yra apskritimo išorėje. 

\subsubsection{Keletas svarbiausių savybių}

Pati svarbiausia šio skyrelio savybė yra ši:

\begin{teig}
  Dvi liestinės apskritimui iš taško yra vienodo ilgio.
\end{teig}

ir jai panaši
\begin{teig}
  Bendros išorinės ar vidinės liestinės dviems apskritimams yra vienodo ilgio.
\end{teig}

Kita labai svarbi savybė yra ši:

\begin{teig}
  Jei apskritimui su centru $O$ taške $A$ išvesta liestinė,
  tai ta liestinė statmena $AO$.
\end{teig}

Vien su šiais teiginiais galima išsspręsti nemažai
uždavinių.

\begin{pav}
  Įrodykite, kad jei iškilasis keturkampis yra apibrėžtinis,
  tai priešingų kraštinių sumos lygios. 
\end{pav}

\begin{proof}[Sprendimas]
  Tegu keturkampis $ABCD$ yra apibrėžtinis, o įbrėžtas
  apskritimas liečia $AB,BC,CD,DA$ atitinkamai taškuose
  $A',B',C',D'$. Tada $AB+CD = AA'+A'B+CC'+C'D =
  AD'+BB'+CB'+DD' = AD+BC$.
\end{proof}

Teisingas ir priešingas faktas: 

\begin{teig}
  Jeigu iškilojo keturkampio priešingų kraštinių sumos lygios,
  tai tas keturkampis yra apibrėžtinis.
\end{teig}

Taip pat yra teisingas kiek kitoks, bet taip pat svarbus
faktas:

\begin{teig}
  Iškilasis n-kampis yra apibrėžtinis tada ir tik tada jei jo
  kažkurių $n-1$ kampų pusiaukampinės kertasi viename taške.
\end{teig}

Todėl, pavyzdžiui, visi trikampiai yra apibrėžtiniai, nes jų
dviejų kampų pusiaukampinės kertasi viename taške.
Nepamirškite ir šio fakto:

\begin{teig}
  Jeigu liestinės iš dviejų skirtingų taškų tam pačiam
  apskritimui yra vienodo ilgio, tai atstumai nuo tų taškų iki
  apskritimo centro yra vienodi. Be to, keturi taškai,
  kuriuose keturios liestinės iš tų dviejų taškų liečia
  apskritimą yra lygiašonės trapecijos viršūnės. 
\end{teig}

\subsubsection{Išorinės pusiaukampinės}

Iki šiol, minėdami figūros kampo pusiaukampinę, turėdavome
omeny tiesę, kuri dalina figūros vidinį kampą pusiau ir eina
iš figūros išorės į figūros vidų. Bet yra ir išorinės
pusiaukampinės, kurios dalija figūros kampo priekampį
pusiau. Jos yra tiesės, kurios visos yra figūros išorėje.
Pavyzdžiui, tiesė $d$  paveikslėlyje žemiau yra trikampio
$ABC$ kampo $\angle C$ išorinė pusiaukampinė, o tiesė $a$
yra kampo $\angle C$ pusiaukampinė (Jeigu nepasakyta, kad
pusiaukampinė yra išorinė, tai ji yra paprasta):
%Išorinė pusiaukampinė
\begin{center}
\begin{asy}
import olympiad;
size(200);
pair A, B, C, F, X;
A=(0,0);
B=(100,0);
C=(35,60);
F=extension(A,B,C,bisectorpoint(A,C,B));
X=(rotate(180-degrees((B-C)/(A-C)),C)*B + B)/2;
add(anglem(A,C,F,300,red,0));
add(anglem(F,C,B,300,red,0));
add(anglem(B,C,X,300,blue,0));
add(anglem(X,C,2*C,300,blue,0));
draw(A--B);
drawline(A,C);
drawline(B,C);
drawline(F,C);
drawline(X,C);
draw((-10,-10)--(110,70),invisible);
dot(A,blue);
dot(B,blue);
dot(C,blue);
dot(F,blue);
label("$A$",A,NW,blue);
label("$B$",B,NE,blue);
label("$C$",C,NW+2*left,blue);
label("$F$",F,NE,blue);
label("$a$",C/2+F/2,NW,blue);
label("$d$",(0,55),NE,blue);
\end{asy}
\end{center}

\begin{teig}
  Kampo pusiaukampinė ir išorinė pusiaukampinė yra statmenos
  viena kitai.
\end{teig}

Išorinės pusiaukampinės turi savybę, labai panašią į
paprastų pusiaukampinių:

\begin{teig}
  Tegu trikampio $ABC$ ($BA>BC$) kampo $\angle B$ išorinė
  pusiaukampinė kerta spindulį $AC$ taške $D$. Tada
  $\frac{CD}{AD}=\frac{CB}{AB}$
\end{teig}

Išorinės pusiaukampinės susijusios su pribrėžtinias
apskritimais:

\subsubsection{Pribrėžtiniai apskritimai}

Mes žinome, kad į kiekvieną apskritimą galima įbrėžti
apskritimą, kuris yra to trikampio viduje ir liečia visas
tris trikampio kraštines. Tačiau yra trys apskritimai, kurių
kiekvienas liečia vieną trikampio kraštinę ir kitų dviejų
kraštinių tęsinius, kaip paveikslėlyje apačioje:
%Pribrėžtiniai apskritimai
\begin{center}
\begin{asy}
import olympiad;
size(200);
pair A, B, C, AA, BB, CC, C1, C2, C3;
A=(0,0);
B=(100,0);
C=(35,60);
AA=bisectorpoint(B,A,-C);
BB=bisectorpoint(C,B,2*B);
CC=bisectorpoint(B,C,2*C);
C1=extension(A,AA,B,BB);
C2=extension(A,AA,C,CC);
C3=extension(C,CC,B,BB);
draw(circle(C1, arclength(C1--foot(C1,A,B))),red);
draw(circle(C2, arclength(C2--foot(C2,A,C))),red);
draw(circle(C3, arclength(C3--foot(C3,C,B))),red);
drawline(A,C);
drawline(B,C);
drawline(A,B);
draw((-10,-10)--(110,70),invisible);
dot(A,blue);
dot(B,blue);
dot(C,blue);
\end{asy}
\end{center}

Paprastai jei nepasakyta kitaip, pribrėžtinio apskritimo,
kuris liečia kraštinę $BC$ ( ne jos tęsinį), centras žymimas
$I_A$. Panašiai kiti centrai žymimi $I_B$ ir $I_C$. Jų
spinduliai atitinkamai žymimi $r_A,r_B,r_C$.

\begin{teig}
  Pribrėžtinio apskritimo priešais kampą $\angle A$ centras
  guli ant kampo $\angle A$ pusiaukampinės ir ant kampų
  $\angle B$ ir $\angle C$ išorinių pusiaukampinių. 
\end{teig}

\begin{proof}[Įrodymas]
  Kadangi tas apskritimas liečia $AB$ ir $AC$, tai jo centras
  yra ant kampo $\angle A$ pusiaukampinės. Taip pat kadangi
  jis liečia tieses $AB$ ir $BC$, tai jo centras yra ant kampo
  $\angle B$ išorinės pusiaukampinės. Panašiai jis yra ir ant
  kampo $\angle C$ išorinės pusiaukampinės.  
\end{proof}

Iš čia seka tokia išvada:

\begin{teig}
  Trikampio kampo pusiaukampinė ir kitų dviejų kampų išorinės
  pusiaukampinės kertasi viename taške ( tai galima įrodyti ir
  be pribrėžtinių apskritimų: Kadangi dviejų iš šių trijų
  tiesių sankirtos taškas vienodai nutolęs nuo visų trikampio
  kraštinių, tai ir trečia tiesė eina per šį tašką).
\end{teig}

\subsubsection{Uždaviniai}

\begin{enumerate}
\item \emph{Čia svarbus uždavinys - įsiminkite šiuos
  rezultatus.} Tegu į trikampį $ABC$ įbrėžtas apskritimas su
  centru $I$ liečia kraštines $AB,BC,CA$ atitinkamai
  taškuose $E,F,G$, o pribrėžtinis apskritimas prieš viršūnę
  $A$ liečia tas pačias kraštines taškuose $L,M,N$ (kaip
  paveikslėlyje viršuje). Jei $AB=c$, $AC=b$ ir $BC=a$ , tai
  tada 
  %%noparse
  \begin{itemize} 
    \item Įrodykite, kad $AL=AN=s$ kur
      $s=\frac{a+b+c}{2}$ - pusperimetris.  
    \item Įrodykite,
      kad $GC=BM=s-c$. Panašiai įrodykite, kad $NC=BE=s-b$
      ir $AE=AG=s-a$.  
    \item Įrodykite, kad $AI\cdot
      AI_A=AB\cdot AC$.  
    \item Įrodykite, kad
      $S=r_A\cdot(s-a)$, kur $S$ yra $ABC$ plotas 
    \item
      Įrodykite Herono formulę: $S=\sqrt{s(s-a)(s-b)(s-c)}$ 
  \end{itemize} 
%Pribrėžtinio apskritimo savybės
\begin{center}
\begin{asy}
import olympiad;
size(300);
pair A, B, C, E, F, G, L, M, N, I, Ia, CC;
path a1;
C=(0,0);
A=(60,20);
B=(-33,50);
CC=bisectorpoint(B,C,-A);
I=incenter(A,B,C);
Ia=extension(C, CC, A, I);
a1=circle(Ia, arclength(Ia--foot(Ia,C,B)) );
E=tangent(A, I, inradius(A,B,C),1);
G=tangent(A, I, inradius(A,B,C),2);
F=tangent(C, I, inradius(A,B,C),2);
L=tangent(B, Ia, arclength(Ia--foot(Ia,C,B)) ,1);
M=tangent(B, Ia, arclength(Ia--foot(Ia,C,B)) ,2);
N=tangent(C, Ia, arclength(Ia--foot(Ia,C,B)) ,2);
add(anglem(B,A,I,300,red,0));
add(anglem(I,A,C,300,red,0));
add(anglem(B,C,Ia,300,blue,1));
add(anglem(Ia,C,N,300,blue,1));
add(anglem(L,B,Ia,300,green,2));
add(anglem(Ia,B,C,300,green,2));
dot(A,blue);
dot(B,blue);
dot(C,blue);
dot(E,blue);
dot(F,blue);
dot(G,blue);
dot(L,blue);
dot(M,blue);
dot(N,blue);
dot(I,blue);
dot(Ia,blue);
draw(a1);
draw(incircle(A,B,C));
draw(A--B--C--cycle);
draw(Ia--C);
draw(Ia--B);
draw(C--N);
draw(B--L);
drawline(A,I);
draw((-100,0)--(70,0),invisible);
label("$A$",A,down,blue);
label("$B$",B,NE,blue);
label("$C$",C,down,blue);
label("$E$",E,NE,blue);
label("$F$",F,right,blue);
label("$G$",G,down,blue);
label("$L$",L,NE,blue);
label("$M$",M,right,blue);
label("$N$",N,down,blue);
label("$I$",I,up,blue);
label("$Ia$",Ia,up,blue);
\end{asy}
\end{center}
  %%parse
  %Tegu $IG=r$. $AL=AN$ iš liestinių savybių. Tada $AL =
  %\frac{AL+AN}{2} = \frac{LB+BA+AC+CN}{2} =
  %\frac{BM+BA+AC+CM}{2} = \frac{a+b+c}{2} = s$.  $GC =
  %\frac{FC+GC}{2} = \frac{FC+GC+BF-BE+AG-AE}{2} =
  %\frac{CB+AC-AB}{2} = \frac{a+b-c}{2} = \frac{a+b+c}{2} -
  %c = s - c$.  $BM = BL = AL - AB = s - c$, taigi $GC =
  %BM$. Panašiai gauname ir kad $NC = BE = s - b$ ir $AE = s
  %- a$. Trečiajai daliai pastebėkime, kad $\angle I_ACN =
  %\frac{\angle BCN}{2} = \frac{\angle CAB + \angle CBA}{2}
  %= \frac{\angle CBA}{2} + \frac{\angle CAB}{2} = \angle
  %BII_A$, taigi $\angle BIA = \angle I_ACA$ ir todėl
  %trikampiai $I_ACA$ ir $BIA$ yra panašūs pagal 2 kampus.
  %Tada $\frac{AB}{AI} = \frac{AI_A}{AC}$, iš kur gauname
  %$AI \cdot AI_A = AB \cdot AC$. Ketvirtajai daliai
  %pastebėkime, kad $AI_AN$ ir $AIG$ yra panašūs ir iš čia $S
  %= s*r = AN \cdot IG = AG \cdot I_AN = (s - a) \cdot r_A$.
  %Penktajai daliai reikia įrodyti, kad $S^2 = (s - c)(s -
  %a)(s - b)s$. Bet mes žinome, kad $S = (s - a)r_A$ ir $S =
  %rs$, taigi $S^2 = sr(s - a)r_A$. Užtenka įrodyti, kad
  %$I_AM \cdot IG = rr_A = (s - b)(s - c) = MC\cdot CG$,
  %arba kad $\frac{IA_M}{MC} = \frac{CG}{IG}$. Tai akivaizdu
  %iš trikampių $I_AMC$ ir $IGC$ panašumo: jie abu statūs ir
  %$\angle MI_AC = 90^\circ - \angle MCI_A = 90^\circ -
  %\frac{\angle BAC + \angle CBA}{2} = \frac{\angle BCA}{2}
  %= \angle ICG$.
\item Trikampio $ABC$ pribrėžtinių apskritimų spinduliai yra
  $I_A, I_B, I_C$, o įbrėžtinio apskritimo centras yra $I$.
  Įrodykite, kad trikampio $I_AI_BI_C$ aukštinių susikirtimo
  taškas yra $I$.
  %Mes žinome iš minėtųjų faktų, kad $I,A,I_A$ yra vienoje
  %tiesėje ir $IA\perp I_BI_C$ (nes $I_CI_B$ yra išorinė
  %kampo $\angle A$ pusiaukampinė). Taigi $I_AA$ yra
  %trikampio $I_AI_BI_C$ aukštinė. Panašiai su $BI_B$ ir
  %$CI_C$. Kadangi $CI_C, BI_B, AI_A$ kertasi taške $I$, tai
  %tas taškas ir yra trikampio $I_AI_BI_C$ aukštinių
  %susikirtimo taškas.
  \item Iškilojo keturkampio priešingų kraštinių sumos lygios.
   Įrodyti, kad trikampių, gautų nubrėžus vieną įstrižainę, 
   įbrėžtiniai apskritimai liečiasi.
   %Tarkime, kad keturkampis yra $ABCD$, nubrėžta įstrižainė 
   %$AC$, į $ABC$ įbrėžtas apskritimas liečia $AC$ taške $K$,
   %o į $ADC$ įbrėžtas apskritimas liečia $AC$ taške $L$.
   %Pastebėkime, kad iš sąlygos $AB-BC=AD-DC$. Tada $KC=
   %\frac{AC+BC-AB}{2}=\frac{AC+DC-AD}{2}=CL$, ir iš čia $K=L$.
\item $ABCD$ yra apibrėžtinis keturkampis, kurio priešingų
  kraštinių sandaugos lygios. Kampas tarp vienos iš
  kraštinių ir įstrižainės yra $25^\circ$. Rasti kampą tarp
  tos kraštinės ir kitos įstrižainės.
  %Kadangi keturkampis apibrėžtinis, tai jo priešingų
  %kraštinių sumos lygios. Bet šiuo atveju sandaugos taip
  %pat lygios. Iš paprastos algebros, keturkampis yra
  %deltoidas (įrodykite tai). Tada nesunkiai ieškomas
  %kampas yra lygus $65^\circ$.
\item Duotas kvadratas $ABCD$. Ant kraštinės $BC$ yra taškas
  $M$, o ant kraštinės $DC$ - taškas $K$ taip, kad trikampio
  $CMK$ perimetras yra dvigubai ilgesnis už kvadrato
  kraštinę. Rasti kampą $\angle MAK$.
  %Tegu trikampio $MCK$ pribrėžtinis apskritimas prieš kampą
  %$\angle C$ liečia spindulį $CM$ taške $B'$. Iš pirmojo
  %uždavinio, $AB' = \frac{CM+MK+KC}{2} = AB$, taigi $B' =
  %B$.  Todėl tas pribrėžtinis apskritimas eina per $B$ ir
  %$D$, o jo centras yra taškas $A$. Tada $\angle MAK =
  %180^\circ - \angle KMA - \angle MKA = 180^\circ -
  %\frac{\angle KMB}{2} - \frac{\angle MKD}{2} =
  %\frac{360^\circ-(180^\circ-\angle MKC)-(180^\circ-\angle
  %KMC)}{2} = 45^\circ$. 
\item Duotas apibrėžtinis keturkampis $ABCD$. Kraštinės $AB,
  BC, CD, DA$ liečia tą apskritimą atitinkamai taškuose $K,
  L, M, N$. $KM$ ir $LN$ kertasi taške $S$. Jeigu $SKBL$ yra
  įbrėžtinis, tai įrodykite, kad $SNDM$ taip pat įbrėžtinis.
  %Iš kampo tarp stygos ir liestinės savybės, $\angle KSB =
  %\angle KLB = \angle KML$ ir todėl $ML\parallel{BS}$.
  %Panašiai ir $NK\parallel{BS}$. Taigi $KLMN$ yra lygiašonė
  %trapecija, ir todėl iš simetrijos $NSDM$ yra įbrėžtinis.
\item Duotas trikampis $ABC$. Išvestos tiesės, simetriškos
  tiesei $AC$ tiesių $BC$ ir $BA$ atžvilgiu, ir jos kertasi
  taške $K$. Įrodyti, kad apie trikampį $ABC$ apibrėžto
  apskritimo centras guli ant tieses $BK$.
  %Tarkime, kad taškai $K$ ir $B$ yra skirtingose $AC$
  %pusėse (atvejis, kai jie vienoje pusėje sprendžiamas
  %labai panašiai). Tiesės $BA$ ir $BC$ yra išorinės
  %trikampio $ACK$ pusiaukampinės, taigi $B$ yra trikampio
  %$ACK$ pribrėžtinio apskritimo centras. Todėl $B$ yra ant
  %kampo $\angle AKC$ pusiaukampinės. Tegu $I$ yra trikampio
  %$ACK$ įbrėžto apskritimo centras. Tada $BA\perp AI$ ir
  %$BC\perp CI$, ir todėl $BAIC$ yra įbrėžtinis su centru
  %ant tiesės $IB$, kuri sutampa su tiese $BK$.  
\item Trikampyje $ABC$ nubrėžtos pusiaukampinės $AD, BE$ ir
  $CF$. Jei $\angle BAC=120^\circ$, tai įrodykite, kad
  $ED\perp FD$.
  %Kadangi $\angle DAC = 60^\circ = \frac{180^\circ - \angle
  %BAD}{2}$, tai $AC$ yra išorinė kampo $\angle BAD$
  %pusiaukampinė. Tada $E$ yra trikampio $BAD$ pribrėžto
  %apskritimo centras (nes jis yra sankirta pusiaukampinės
  %ir išorinės pusiaukampinės), ir todėl $DE$ yra kampo
  %$\angle ADC$ pusiaukampinė. Panašiai ir $FD$ yra kampo
  %$\angle BDA$ pusiaukampinė. Tada $\angle FDE = \angle
  %FDA + \angle ADE = \frac{\angle BDA}{2} + \frac{\angle
  %CDA}{2} = 90^\circ$. 
\item Per smailiojo trikampio $ABC$ ($AC>AB$) viršūnę $A$
  nubrėžė pusiaukampinę $AM$ ir išorinę pusiaukampinę $AN$
  bei liestinę $AK$ apskritimui, apibrėžtam apie $ABC$
  (taškai $M,K,N$ yra ant spindulio $CB$). Įrodykite, kad
  $MK=KN$. 
  %$\angle AMK = \angle ACM + \angle CAM = \angle KAB +
  %\angle MAB = \angle MAK$. Taigi $AMK$ lygiašonis. Bet
  %$\angle NAM$ yra status, ir todėl $K$ yra $MN$ ir $AM$
  %vidurio statmens sankirta. Todėl $MK = KN$.
\item Į trapeciją galima įbrėžti apskritimą. Įrodyti, kad
  apskritimai, kurių skersmenys yra trapecijos šoninės
  kraštinės, liečia vienas kitą.
  %Tegu trapecija būna $ABCD$, $AD\parallel{CB}$. Du
  %apskritimai liečia vienas kitą jeigu atstumas tarp jų
  %centrų yra lygus jų spindulių sumai. Tegu $K$ yra $AB$, o
  %$L$ yra $CD$ vidurio taškai. Tada $KL = \frac{AD+BC}{2} =
  %\frac{AB+CD}{2} = KA + LD$, ko ir reikėjo. 
\item Taškas $O$ yra pribrėžtinio apskritimo, liečiančio
  trikampio $ABC$ kraštinę $AC$ ir kraštinių $BA$ ir $BC$
  tęsinius, centras. $D$ - apskritimo, einančio per taškus
  $B, A, O$, centras. Įrodykite, kad taškai $A, B, C$ ir $D$
  yra ant vieno apskritimo. 
  %$\angle DOB = \frac{180^\circ-\angle BDO}{2} =
  %\frac{180^\circ-(2(180^\circ-\angle BAO))}{2} = \angle
  %BAO - 90^\circ = \angle BAC + \frac{\angle CBA+\angle
  %BCA}{2} - 90^\circ = \frac{\angle BAC}{2} = \angle COB$
  %(žr. 1 uždavinį). Taigi $C, O, D$ yra vienoje tiesėje.
  %Tada $\angle BDC + \angle BAC = \angle BDO + \angle BAC =
  %2(180^\circ - \angle BAO) + \angle BAC = (180^\circ -
  %\angle BAC) + \angle BAC = 180^\circ$. Taigi $B, A, C, D$
  %yra ant vieno apskritimo.
\item Penkiakampis $ABCDE$ apibrėžtas aplink apskritimą $S$.
  $AB=BC=CD$, $BC$ liečia $S$ taške $K$. Įrodyti, kad
  $EK\perp BC$.
  %Tegu kraštinės $AB, BC, CD, DE, EA$ liečia $S$
  %atitinkamai taškuose $H, K, L, M, N$. Tada $KC = BC - BK
  %= AB - BH = AH = AN$ ir $BK = BC - KC = CD - CL = LD =
  %MD$. Taigi $KLMH$ ir $HKLN$ yra lygiašonės trapecijos, ir
  %todėl $KN = HL = KM$. Tada $KMEN$ yra deltoidas ir todėl
  %$KE$ eina per apskritimo centrą.  Taigi $EK \perp BC$
\item Trikampyje $ABC$ nubrėžė pusiaukampines $AD$ ir $BE$.
  Jei $DE$ yra kampo $\angle ADC$ pusiaukampinė, tai raskite
  kampą $\angle BAC$. 
  %Atsakymas yra $120^\circ$. Sprendimas beveik toks pats
  %kaip ir 8 uždavinio.
\item Duotas trikampis $ABC$, ant spindulio $CB$ už taško
  $B$ paimtas taškas $D$ toks, kad $BD=AB$. Kampų $\angle A$
  ir $\angle B$ išorinės pusiaukampinės kertasi taške $M$.
  Įrodyti, kad taškai $M,A,C,D$ guli ant vieno apskritimo.
  %$BM$ yra kampo $\angle DBA$ pusiaukampinė, taigi taip pat
  %yra $DA$ vidurio statmuo. Todėl $\angle DMB = \angle BMA$.
  %Tada $\angle DMA + \angle DCA = 2\angle BMA + \angle BCA
  %= 2(180^\circ - \angle MBA - \angle MAB) + \angle BCA =
  %360^\circ -\angle DBA - (180^\circ - \angle BAC) + \angle
  %BCA = 180^\circ -(\angle DBA - \angle BCA - \angle BAC) =
  %\angle 180^\circ$, taigi $MDCA$ įbrėžtinis.
\item Apibrėžtiniame penkiakampyje $ABCDE$ įstrižainės $AD$
  ir $CE$ kertasi taške $O$, kuris yra įbrėžto apskritimo
  centras. Įrodyti, kad $BO \perp{DE}$.
  %Tegu kraštinės $AB, BC, CD, DE, EA$ liečia apskritimą
  %atitinkamai taškuose $H$, $K$, $L$, $M$, $N$. Kadangi $AD$ eina
  %per apskritimo centrą, tai $HN \perp AD$. Panašiai ir $LM
  %\perp AD$. Taigi $LM \parallel{HN}$ ir todėl $HNML$ yra
  %lygiašonė trapecija. Panašiai ir $KLMN$ yra lygiašonė
  %trapecija. Galime užbaigti kaip ir 12 uždavinyje.
\item Duotas smailusis  trikampis $KEL$, į jį įbrėžtas
  apskritimas su spinduliu $R$. Šiam apskritimui išvestos 3
  liestinės taip, kad $KEL$ yra padalintas į 3 stačius
  trikampius ir viena šešiakampį, kurio perimetras yra $Q$.
  Rasti į tris stačiuosius trikampius įbrėžtų apskritimų
  spindulių sumą. 
  %Tegu visi taškai būna tokie, kokie yra paveiklėlyje
  %apačioje. Tada $GZOA_1$ yra kvadratas, nes jo visi kampai
  %statūs ir $GZ = GA_1$. Lygiai taip pat kvadratai yra
  %$TMB_1N$, $UWPV$, $AFOC$, $ACIP$, $ABNH$. Be to, didysis
  %apskritimas yra trikampių $EQO$, $RLP$, $KNS$
  %pribrėžtinis apskritimas. Tada iš pirmo uždavinio $ZO =
  %QF$, $NB_1 = HS$, $RI = WP$. Tada $GZ + UW + MT = ZO +
  %NB_1 + WP = QF + HS + RI = \frac{QF+QB+HS+DS+DR+RI}{2} =
  %\frac{Q-6R}{2}$.  
  %\begin{center}
  %\begin{asy}
  %import olympiad;
  %size(200);
  %pair K, T, M, N, B1, H, S, D, R, I, W, P, U, V, L, C, A,
  % B, Q, F, Z, O, G, A1, E;
  %path a1;
  %real r;
  %E=(0,0);
  %L=(100,0);
  %K=(45,70);
  %A=incenter(K,L,E);
  %a1=incircle(K,L,E);
  %r=inradius(K,L,E);
  %B=foot(A,K,E);
  %C=foot(A,L,E);
  %D=foot(A,K,L);
  %I=A+dir(E--L)*r;
  %F=A+dir(L--E)*r;
  %H=A+dir(E--K)*r;
  %P=I+dir(A--C)*r;
  %O=F+dir(A--C)*r;
  %N=H+dir(A--B)*r;
  %Q=extension(O,F,E,K);
  %R=extension(P,I,L,K);
  %S=extension(N,H,L,K);
  %U=incenter(R,P,L);
  %G=incenter(E,Q,O);
  %M=incenter(S,N,K);
  %Z=foot(G,F,O);
  %A1=foot(G,E,O);
  %W=foot(U,I,P);
  %V=foot(U,P,L);
  %T=foot(M,N,K);
  %B1=foot(M,N,H);
  %dot(A,blue);
  %dot(B,blue);
  %dot(C,blue);
  %dot(D,blue);
  %dot(E,blue);
  %dot(F,blue);
  %dot(G,blue);
  %dot(H,blue);
  %dot(I,blue);
  %dot(K,blue);
  %dot(L,blue);
  %dot(M,blue);
  %dot(N,blue);
  %dot(O,blue);
  %dot(P,blue);
  %dot(Q,blue);
  %dot(R,blue);
  %dot(S,blue);
  %dot(T,blue);
  %dot(U,blue);
  %dot(V,blue);
  %dot(W,blue);
  %dot(Z,blue);
  %dot(A1,blue);
  %dot(B1,blue);
  %draw(a1);
  %draw(incircle(E,Q,O));
  %draw(incircle(R,L,P));
  %draw(incircle(S,N,K));
  %draw(E--K--L--cycle);
  %draw(A--C);
  %draw(A--F);
  %draw(A--B);
  %draw(A--H);
  %draw(A--D);
  %draw(A--I);
  %draw(G--Z);
  %draw(G--A1);
  %draw(Q--O);
  %draw(R--P);
  %draw(S--N);
  %label("$A$",A,SW,blue);
  %label("$B$",B,NW,blue);
  %label("$C$",C,down,blue);
  %label("$D$",D,NE,blue);
  %label("$E$",E,SW,blue);
  %label("$F$",F,SW,blue);
  %label("$G$",G,SW,blue);
  %label("$H$",H,left+down/2,blue);
  %label("$I$",I,SW,blue);
  %label("$K$",K,up,blue);
  %label("$L$",L,SE,blue);
  %label("$M$",M,down/5,blue);
  %label("$N$",N,left,blue);
  %label("$O$",O,SE,blue);
  %label("$P$",P,SW,blue);
  %label("$R$",R,NE,blue);
  %label("$S$",S,NE,blue);
  %label("$T$",T,left,blue);
  %label("$U$",U,SE,blue);
  %label("$V$",V,SE,blue);
  %label("$W$",W,SW,blue);
  %label("$Z$",Z,SE,blue);
  %label("$Q$",Q,NW,blue);
  %label("$A_1$",A1,SW,blue);
  %label("$B_1$",B1,down,blue);
  %add(rightanglem(B,N,S,140));
  %add(rightanglem(C,O,F,140));
  %add(rightanglem(R,P,C,140));
%\end{asy}
  %\end{center}
\item Duotas apibrėžtinis keturkampis $ABCD$. Į jį įbrėžtas
  apskritimas liečia kraštines $AB,BC,CD,DA$ atitinkamai
  taškuose $E,F,G,H$. Įrodykite, kad linija, jungianti
  trikampių $HAE$ ir $FCG$ įbrėžtinių apskritimų centrus yra
  statmena linijai, jungiančiai į trikampius $EBF$ ir $GDH$
  įbrėžtų apskritimų centrus.
  %Tegu $I_A$ yra apskritimo, įbrėžto į $AEH$, centras (o
  %kiti centrai yra $I_B, I_C, I_D$). Kadangi $AEH$
  %lygiašonis, tai tada $\angle HI_AE = 180^\circ - \angle
  %I_AEH - \angle I_AHE = 180^\circ - \angle AEH = 180^\circ
  %- \angle HFE$, todėl $I_A$ yra ant $ABCD$ įbrėžto
  %apskritimo.  Panašiai visi kitų trikampių įbrėžtinių
  %apskritimų cenrai irgi yra ant to paties apskritimo.
  %Pastebėkime, kad $I_A, I_B, I_C, I_D$ atitinkamai dalina
  %lankus $HE, EF, FG, GH$ pusiau. Tada kampas tarp $I_AI_C$
  %ir $I_BI_D$ yra lygus $\angle I_AI_CI_B + \angle
  %I_CI_BI_D = \frac{\angle HGF}{2} + \frac{\angle HEF}{2} =
  %90^\circ$, ko ir reikėjo.
\item Į kampą įbrėžtas apskritimas su centru $O$. Per tašką
  $A$, simetrišką taškui $O$ vienos iš kampo kraštinių
  atžvilgiu, nubrėžė apskritimui dvi liestinės, kurios kerta
  labiau nuo taško $A$ nutolusią kampo kraštinę taškuose $B$
  ir $C$. Įrodyti, kad apie $ABC$ apibrėžto apskritimo
  centras yra ant duotojo kampo pusiaukampinės.
  %Tegu į kampą įbrėžtas apskritimas liečia $AB$ ir $AC$
  %atitinkamai taškuose $P$ ir $J$, ir liečia kampo
  %kraštines taškuose $E$ ir $D$ ($D$ yra $AO$ vidurio
  %taškas), taip kaip paveikslėlyje apačioje. Tada $\angle
  %OAJ = 30^\circ$, nes stačiajame trikampyje $AOJ$ įžambinė
  %lygi pusei statinio. Tegu taškas $F$ yra simetriškas
  %taškui $O$ taško $E$ atžvilgiu. Tada $\angle BFC = \angle
  %BOC = 180^\circ - \angle OBC - \angle OCB = 180^\circ -
  %\frac{\angle ABC + \angle ACB}{2} = 180^\circ -
  %\frac{120^\circ}{2} = 120^\circ = 180^\circ - \angle
  %BAC$, taigi $F$ yra ant apie $ABC$ apibrėžto apskritimo.
  %Tada $AF$ vidurio statmuo sutampa su $DE$ vidurio
  %statmeniu, kuris yra $\angle DQE$ pusiaukampinė (nes
  %$QDOE$ yra deltoidas). Bet $AF$ vidurio statmuo eina ir
  %per apie $ABC$ apibrėžto apskritimo centrą, ko ir
  %reikėjo.
  %\begin{center}
  %\begin{asy}
  %import olympiad;
  %size(200);
  %pair A, B, C, D, E, F, J, K, L, Q, O, P; 
  %path a1, a2;
  %Q=(0,0);
  %O=(50,5);
  %a1=circle(O,10);
  %D=tangent(Q,O,10,2);
  %E=tangent(Q,O,10,1);
  %A=2*D-O;
  %F=2*E-O;
  %P=tangent(A,O,10,1);
  %J=tangent(A,O,10,2);
  %B=extension(A,P,Q,E);
  %C=extension(A,J,Q,E);
  %a2=circumcircle(A,B,C);
  %K=intersectionpoint(Q--D,a2);
  %L=intersectionpoint(D--2*D,a2);
  %dot(A,blue);
  %dot(B,blue);
  %dot(C,blue);
  %dot(D,blue);
  %dot(E,blue);
  %dot(F,blue);
  %dot(J,blue);
  %dot(K,blue);
  %dot(L,blue);
  %dot(O,blue);
  %dot(Q,blue);
  %dot(P,blue);
  %draw(a1);
  %draw(a2);
  %draw(A--B--C--cycle);
  %draw(Q--L);
  %draw(Q--C);
  %draw(O--A);
  %draw(O--B);
  %draw(O--C);
  %draw(O--J);
  %draw(O--F);
  %label("$A$",A,NW,blue);
  %label("$B$",B,SW,blue);
  %label("$C$",C,SE,blue);
  %label("$D$",D,NE+W/2,blue);
  %label("$E$",E,SW,blue);
  %label("$F$",F,SW,blue);
  %label("$J$",J,NE,blue);
  %label("$K$",K,NW,blue);
  %label("$L$",L,NE,blue);
  %label("$O$",O,left,blue);
  %label("$P$",P,left,blue);
  %label("$Q$",Q,NW,blue);
  %add(rightanglem(Q,D,O,100));
  %add(rightanglem(O,E,B,100));
  %add(rightanglem(O,J,C,100));
  %add(pathticks(A--D,2,0.5,0,80,red));
  %add(pathticks(D--O,2,0.5,0,80,red));
  %add(pathticks(O--J,2,0.5,0,80,red));
  %add(pathticks(O--E,2,0.5,0,80,red));
  %add(pathticks(E--F,2,0.5,0,80,red));
%\end{asy}
  %\end{center}
\item Ant trikampio $ABC$ kraštinės $BC$ paimtas taškas $D$.
  Į trikampius $ABD$ ir $ACD$ įbrėžti apskritimai, ir
  nubrėžta bendra jiems liestinė (kita nei $BC$), kertanti
  $AD$ taške $K$. Įrodykite, kad atkarpos $AK$ ilgis
  nepriklauso nuo taško $D$ pasirinkimo.
  %Tegu taškai būna tokie, kokie pažymėti paveikslėlyje.
  %Tegu $KI$ ir $DC$ kertasi taške $U$ . Tada apskritimas su
  %centru $F$ yra trikampio $KDU$ įbrėžtinis apskritimas, o
  %apskritimas su centru $E $- pribrėžtinis. Tada iš pirmojo
  %uždavinio rezultatų, $KL = MD = DG$ ir $KM = LD = DH$.
  %Taip pat iš pirmojo uždavinio rezultatų, $MD =
  %\frac{AD+BD-AB}{2}$ ir $MK = DH = \frac{DC+DA-CA}{2}$.
  %Taigi, $AK = AD - DM - MK = AD - \frac{AD+BD-AB}{2} -
  %\frac{DC+DA-CA}{2} = \frac{AB+CA-BC}{2}$, kas tikrai
  %nepriklauso nuo $D$. 
  %\begin{center}
  %\begin{asy}
  %import olympiad;
  %size(200);
  %pair A, B, C, D, E, F, G, H, I, K, L, M, N, O, P, U;
  %path a1, a2;
  %A=(68,60);
  %B=(0,0);
  %C=(100,0);
  %D=waypoint(B--C,0.72);
  %a1=incircle(A,B,D);
  %a2=incircle(A,D,C);
  %E=incenter(A,B,D);
  %F=incenter(A,D,C);
  %G=foot(E,C,B);
  %N=foot(E,A,B);
  %M=foot(E,A,D);
  %H=foot(F,C,B);
  %O=foot(F,C,A);
  %L=foot(F,A,D);
  %P=G+2*(foot(G,E,F)-G);
  %I=H+2*(foot(H,E,F)-H);
  %U=extension(P,I,B,C);
  %K=extension(A,D,P,I);
  %dot(A,blue);
  %dot(B,blue);
  %dot(C,blue);
  %dot(D,blue);
  %dot(E,blue);
  %dot(F,blue);
  %dot(G,blue);
  %dot(H,blue);
  %dot(I,blue);
  %dot(K,blue);
  %dot(L,blue);
  %dot(M,blue);
  %dot(N,blue);
  %dot(O,blue);
  %dot(P,blue);
  %dot(U,blue);
  %draw(A--B--C--cycle);
  %drawline(P,I);
  %draw(A--D);
  %draw(C--U);
  %draw(a1);
  %draw(a2);
  %label("$A$",A,up,blue);
  %label("$B$",B,SW,blue);
  %label("$C$",C,SE,blue);
  %label("$D$",D,down,blue);
  %label("$E$",E,right,blue);
  %label("$F$",F,right,blue);
  %label("$G$",G,down,blue);
  %label("$H$",H,down,blue);
  %label("$I$",I,right+up/2,blue);
  %label("$L$",L,right,blue);
  %label("$K$",K,right+up/2,blue);
  %label("$M$",M,right+up/2,blue);
  %label("$N$",N,up,blue);
  %label("$P$",P,NE,blue);
  %label("$O$",O,right/1.5+up/2,blue);
  %label("$U$",U,1.1*down,blue);
%\end{asy}
%\end{center}
\end{enumerate}

\newpage
\section{Vienareikšmiški uždaviniai}


Olimpiadose retkarčiais pasitaiko „vienareikšmiškų“ uždavinių.
Tai tokie
uždaviniai, kur žinomi visi kampai tarp visų tiesių, arba kitais
žodžiais tariant, visi kampai vienareikšmiški. Dėl
šios priežasties juos patogu spręsti trigonometriniais
metodais. Kita vertus, juos dažnai trumpiau ir gražiau
galima išspręsti geometriniais metodais. Tačiau bandydami
surasti tokį sprendimą galite prarasti daug laiko, kai tuo
tarpu trigonometrinis sprendimas greičiausiai duos
vaisių. Tuo šie uždaviniai primena galvosūkius arba kai kuriuos
kombinatorikos uždavinius - išspręsti galima tik gudriai pastebėjus
sprendimą, arba darant ilgai ir nuobodžiai. 
Olimpiadose sutikus tokį uždavinį reikėtų tikėtis, kad yra gana
paprastas geometrinis sprendimas, nes uždavinys,
kuris yra išsprendžiamas tik trigonometriniais metodais, yra
prarandęs savo „olimpiadiškumą“. ( Tai vienas didžiųjų skirtumų
tarp realaus pasaulio uždaviniai nuo olimpiadinių -
olimpiadiniai uždaviniai visada turi pakankamai trumpą sprendimą ).
Dėl šių priežasčių teorijos šiame skyriuje yra nedaug.

\begin{pav}
  Duotas trikampis $ABC$ su $AB=AC$ ir $\angle
  BAC=100^\circ$.  $BD$ yra pusiaukampinė. Įrodyti, kad
  $BD+DA=BC$.
\end{pav}

\begin{sprendimas}
  Tegu $DC$ vidurio statmuo kerta $BC$ taške $F$. Tada $DCF$
  yra lygiašonis, ir todėl $\angle DFC=100^\circ$. Tada
  $ABFD$ yra įbrėžtinis, o $BFD$ lygiašonis. Kadangi $BD$
  yra kampo $\angle B$ pusiaukampinė, tai $AD=DF=FC$. Taigi
  $BD+DA=BF+FC=BC$.  
\end{sprendimas}

\begin{center}
\begin{asy}
import olympiad;
size(200);
pair A, B, C, D, F;
A=(0,0);
C=(70,0);
B=extension(A,rotate(100)*C, C, rotate(-40,C)*A);
D=extension(A,C,B,bisectorpoint(A,B,C));
F=extension(B,C,bisectorpoint(D,C),bisectorpoint(C,D));
dot(A,blue);
dot(B,blue);
dot(C,blue);
dot(D,blue);
dot(F,blue);
dot(D/2+C/2,blue);
draw(A--B--C--cycle);
draw(B--D);
draw(D--F);
draw(D/2+C/2--F);
label("$A$",A,SW,blue);
label("$B$",B,up,blue);
label("$C$",C,SE,blue);
label("$D$",D,SE,blue);
label("$F$",F,NE,blue);
add(pathticks(A--D,2,0.5,0,120,red));
add(pathticks(D--F,2,0.5,0,120,red));
add(pathticks(F--C,2,0.5,0,120,red));
label("$80^\circ$",F,5*left,deepred);
label("$80^\circ$",D,5.6*up+right,deepred);
label("$40^\circ$",C,5*W+2*N,deepblue);
label("$100^\circ$",A,5*NE,deepgreen);
add(anglem(F,D,B,200,red,0));
add(anglem(B,F,D,200,red,0));
add(anglem(F,C,D,200,blue,0));
add(anglem(C,A,B,200,green,0));
add(anglem(D,B,C,200,yellow,0));
label("$20^\circ$",B,7*SE+down,olive);
\end{asy}
\end{center}


\begin{pav}
Duotas keturkampis $ABCD$ toks, kad $AD=CD$, $\angle A=75^\circ,
\angle D=60^\circ, \angle C=135^\circ$. $E$ yra $CD$ vidurio taškas.
 Rasti $\frac{BE}{ED}$
\end{pav}

\begin{sprendimas}
Tegu $F$ yra $CA$ vidurio taškas. Mes nesunkiai suskaičiuojame, kad
 $\angle B=90^\circ$, $\angle BFE=\angle BFC+\angle CFE=30^\circ+
 60^\circ=90^\circ$,  taigi $FE=FC=FB$, ir iš 
Pitagoro teoremos randame $\frac{BE}{ED}=\frac{BE}{EF}=\sqrt{2}$.

\begin{center}
\begin{asy}
import olympiad;
size(200);
pair A, B, C, D, E, F, X, Y;
A=(0,0);
D=(60,0);
C=rotate(60,A)*D;
X=rotate(75,A)*D;
Y=rotate(-135,C)*D;
B=extension(A,X,C,Y);
E=D/2+C/2;
F=C/2+A/2;
dot(A,blue);
dot(B,blue);
dot(C,blue);
dot(D,blue);
dot(E,blue);
dot(F,blue);
draw(A--B--C--D--cycle);
draw(A--C);
draw(F--E);
draw(F--B);
label("$A$",A,left,blue);
label("$B$",B,left,blue);
label("$C$",C,right,blue);
label("$E$",E,right,blue);
label("$F$",F,down,blue);
label("$D$",D,down,blue);
add(pathticks(A--D,2,0.5,0,125,red)); 
add(pathticks(C--D,2,0.4,0,125,red)); 
add(pathticks(A--F,2,0.4,30,125,red)); 
add(pathticks(C--F,2,0.4,30,125,red)); 
add(pathticks(B--F,2,0.4,30,125,red)); 
add(pathticks(F--E,2,0.4,30,125,red)); 
add(anglem(C,D,A,130,red,0));
add(anglem(D,A,B,130,green,0));
add(anglem(B,C,D,130,blue,0));
label("$60^\circ$",D,4*NW,red);
label("$75^\circ$",A,4*NE,green);
label("$135^\circ$",C,4*S,blue);
\end{asy}
\end{center}
\end{sprendimas}
\begin{pav}[Turkijos TST 1995]
Iškilajame keturkampyje $ABCD, \angle{CAB}= 40^{\circ},
\angle{CAD}= 30^{\circ},\angle{DBA}= 75^{\circ},\angle DBC=25^\circ$.
Rasti $\angle BDC$.
\end{pav}

\begin{sprendimas}
Nesunkiai suskaičiuojame, kad $ABC$ lygiašonis su $AB=BC$.
Imame tašką $E$ ant $AD$ tokį, kad $\angle AEB=70^\circ$.
Tada vėl nesunkiai suskaičiuojame, kad $AEB$ lygiašonis,
$EBC$ lygiakraštis (pagal kampą ir dvi lygias kraštines),
 o $EBD$ lygiašonis su $ED=EB=EC$. Taigi $E$ yra apie $CDB$ apibrėžto
apskritimo centras, ir iš čia $\angle CDB=\frac{\angle CEB}{2}=30^\circ$.

\begin{center}
\begin{asy}
import olympiad;
size(200);
pair A, B, C, D, E, X, Y, Z, W, Q;
A=(0,0);
D=(80,0);
X=rotate(-70,A)*D;
Y=rotate(35,D)*A;
Z=rotate(-30,A)*D;
B=extension(A,X,D,Y);
W=rotate(-25,B)*D;
C=extension(B,W,Z,A);
Q=rotate(35,B)*D;
E=extension(B,Q,A,D);
dot(A,blue);
dot(B,blue);
dot(D,blue);
dot(C,blue);
dot(E,blue);
draw(A--B--C--D--cycle);
draw(A--C);
draw(B--D);
draw(B--E);
draw(E--C);
label("$A$",A,left,blue);
label("$B$",B,left,blue);
label("$D$",D,right,blue);
label("$C$",C,right,blue);
label("$E$",E,up,blue);
add(anglem(B,A,C,200,red,0));
add(anglem(C,A,D,200,green,0));
add(anglem(C,B,D,200,blue,0));
add(anglem(D,B,E,200,olive,0));
add(anglem(E,B,A,200,red,0));
add(anglem(A,D,B,200,olive,0));
add(anglem(A,C,B,200,red,0));
add(pathticks(C--E,2,0.5,0,125,red)); 
add(pathticks(B--E,2,0.5,0,125,red)); 
add(pathticks(A--B,2,0.5,0,125,red)); 
add(pathticks(C--B,2,0.5,0,125,red)); 
add(pathticks(E--D,2,0.5,0,125,red)); 
label("$30^\circ$",A,5*right+down,green);
label("$40^\circ$",A,3*right+4*down,red);
label("$40^\circ$",B,5*up,red);
label("$35^\circ$",B,4*up+3*right,olive);
label("$25^\circ$",B,5*right+2*up,blue);
label("$25^\circ$",B,5*right+2*up,blue);
label("$35^\circ$",D,5*left+down,olive);
label("$40^\circ$",C,5*left+up,red);
\end{asy}
\end{center}
\end{sprendimas}

Toliau pateikta savybė yra naudinga sprendžiant įvairius geometrijos 
uždavinius, bet ypač naudinga sprendžiant vienareikšmiškus:

\begin{teig}[Žinotina lema]
  Tarkime, kad turime iškilajį keturkampį $ABCD$ tokį, kad
  $BC=AB$ ir $\angle ADC+\frac{\angle ABC}{2}=180^\circ$. Tada
  $BC=BD=BA$. 
\end{teig}

\begin{proof}[Įrodymas]
Paimkime apskritimą su centru $B$ ir spinduliu $AB$. Tada
šis apskritimas eina per taškus $A$ ir $C$. Tegu $AB$ antrą
kartą kerta tą apskritimą taške $E$. Tada $\angle
AEC+\angle ADC=\frac{\angle ABC}{2}+\angle ADC=180^\circ$.
Taigi $AECD$ įbrėžtinis ir todėl $BA=BC=BD$.
\begin{center}
\begin{asy}
import olympiad;
size(200);
pair A, B, C, D, E;
path a1;
B=origin;
a1=circle(B,50);
E=waypoint(a1,0.23);
A=waypoint(a1,0.73);
D=waypoint(a1,0.82);
C=waypoint(a1,0.95);
dot(A,blue);
dot(B,blue);
dot(C,blue);
dot(D,blue);
dot(E,blue);
draw(a1);
draw(A--B--C--D--cycle);
draw(B--D);
draw(A--C);
draw(B--E);
draw(C--E);
label("$A$",A,down,blue);
label("$B$",B,left,blue);
label("$C$",C,right,blue);
label("$D$",D,SE,blue);
label("$E$",E,up,blue);
add(rightanglem(E,C,A,170));
add(pathticks(B--E,2,0.5,0,150,red));
add(pathticks(B--C,2,0.5,0,150,red));
add(pathticks(B--A,2,0.5,0,150,red));
\end{asy}
\end{center}
\end{proof}

\begin{pav}
Keturkampyje $ABCD$ $AB=BC=1$. Kampas $B=100^\circ$,
$\angle D=130$. Rasti $BD$ \end{pav}
\begin{sprendimas}
Keturkampis $ABCD$ tenkina visas sąlygas, minėtas viršuje:
 $\frac{\angle ABC}{2}+\angle ADC=50^\circ+130^\circ=180^\circ$
  ir $AB=BC$. Taigi $BD=BC=BA=1$.
\end{sprendimas}
\subsubsection{Uždaviniai}
\begin{enumerate}
\item Duotas kvadratas $ABCD$. Jo viduje paimtas taškas $M$
  toks, kad $\angle MAC=\angle MCD=u$. Rasti $\angle MBA$. 
  %Aiškiai $M$ yra trikampio $ACD$ viduje. $\angle AMC =
  %180^\circ - \angle MAC - \angle MCA = 180^\circ - \angle
  %MAC - (45^\circ - \angle MCD) = 135^\circ$. Todėl $MABC$
  %tenkina pirmąją minėtąją savybę, taigi $MB = BC = BA$.
  %Tada $\angle MBA = 180^\circ - 2\angle MBA = 180^\circ -
  %2(45^\circ + u) = 90^\circ - 2u$. 
\item Duotas statusis lygiašonis trikampis $ABC$ su $AB=AC$
  ir $\angle BAC=90^\circ$. Nubrėžta pusiaukraštinė $BM$, o
  jai per tašką $A$ išvestas statmuo. Įrodykite, kad šis
  statmuo dalina $BC$ santykiu 2:1.
  %Tegu taškas $N$ yra simetriškas taškui $A$ taško $C$
  %atžvilgiu, o $P$ yra $A$ projekcija į $BM$. Tada
  %$\frac{NA}{AB} = 2 = \frac{AB}{AM}$, taigi trikampiai
  %$ANB$ ir $ABM$ panašūs. Kadangi $\angle PAN = \angle MBA
  %= \angle ANB$, tai tiesė $PA$ yra trikampio $BAN$
  %pusiaukraštinė.  Bet $BC$ taip pat yra pusiaukraštinė, o
  %pusiaukraštinės dalija viena kitą santykiu 2:1. 
\item (Langlėjaus uždavinys) Duotas trikampis $ABC$ su
  $\angle B=20^\circ$, $\angle A=\angle C=80^\circ$. Ant
  $AB$ ir $BC$ paimti taškai $E$ ir $D$ atitinkamai taip,
  kad $\angle CAD=60^\circ$ ir $\angle ACE=50^\circ$. Rasti
  kampą $\angle ADE$.
  %Trikampis $ACE$ yra lygiašonis, nes turi du kampus po
  %$50^\circ$. Be to, $\angle ADC = 40^\circ$. Paimkime
  %tašką $F$ ant $BC$ tokį, kad $\angle AFC = 80^\circ$.
  %Tada $AF = AC$ bei $\angle EAF = 60^\circ$, taigi $EAF$
  %lygiakraštis. Be to, $DAF$ yra lygiašonis, nes jo kampai
  %lygūs $100^\circ, 40^\circ, 40^\circ$. Tada $DF = FA =
  %FE$, taigi $F$ yra apie $DAE$ apibrėžto apskritimo centras.
  % Iš čia nesunkiai gauname,
  %kad ieškomas kampas lygus $30^\circ$.   
  %\begin{center}
  %\begin{asy}
  %import olympiad;
  %size(200);
  %pair A, B, C, D, E, F, G, H, I, K, T;
  %A=(0,0);
  %C=(100,0);
  %K=rotate(80,A)*C;
  %I=rotate(-80,C)*A;
  %B=extension(A,K,C,I);
  %H=rotate(-50,C)*A;
  %E=extension(A,B,C,H);
  %G=rotate(60,A)*C;
  %D=extension(C,B,A,G);
  %T=rotate(20,A)*C;
  %F=extension(A,T,B,C);
  %dot(D,blue);
  %dot(E,blue);
  %dot(F,blue);
  %dot(A,blue);
  %dot(B,blue);
  %dot(C,blue);
  %draw(A--B--C--cycle);
  %draw(A--F);
  %draw(A--D);
  %draw(E--D);
  %draw(E--C);
  %draw(E--F);
  %label("$A$",A,left,blue);
  %label("$B$",B,up,blue);
  %label("$E$",E,left,blue);
  %label("$D$",D,right,blue);
  %label("$F$",F,right,blue);
  %label("$C$",C,SE,blue);
%\end{asy}
  %\end{center}
\item Duotas trikampis $ABC$ su kampais $\angle B=90^\circ$,
  $\angle A=50^\circ$, $\angle C=40^\circ$. Paimti taškai
  $K$ ir $L$ ant $BC$ taip, kad $\angle KAC=\angle
  LAB=10^\circ$. Rasti $\frac{KC}{BL}$.
  %Tegu linija, lygiagreti $AB$ ir einanti per tašką $K$,
  %kerta liniją, lygiagrečią $AL$ ir einančią per tašką $C$
  %taške $P$. Tada kadangi $\angle BKA = 50^\circ$, tai
  %$\angle AKP = 40^\circ$. Panašiai $\angle BCP =
  %80^\circ$, tai $\angle ACP = 40^\circ$. Taigi $AKCP$
  %įbrėžtinis. Kadangi $CA$ yra $\angle KCP$ pusiaukampinė, tai $AK = AP$.
  %Tada $KP = 2BA$ ir iš trikampių $CKP$ ir $BLA$ panašumo
  %$\frac{KC}{BL} = 2$.  
\item Duotas deltoidas $ABCD$, $AB=BC$, $AD=DC$, $\angle
  ADC=3\angle ACB$, $AE$-trikampio $ABC$ pusiaukampinė, $DE$
  ir $AC$ kertasi taške $F$. Įrodyti, kad $CEF$ lygiašonis.
  %Tegu $\angle ACB = x$. Pastebėkime, kad keturkampis
  %$ADCE$ tenkina sąlygą, minėtą pirmame naudingame fakte,
  %nes $DA = DC$ bei $\angle AEC + \frac{\angle ADC}{2} =
  %(180^\circ - x - \frac{x}{2}) + 1.5x = 180^\circ$. Taigi
  %$DE = DC = DA$. Kadangi $D$ yra apie $AEC$ apibrėžto
  %apskritimo centras, $\angle EDC = 2\angle EAC = x =
  %\angle ECF$. Tada $EDC$ ir $EFC$ panašūs pagal du kampus,
  %taigi $EFC$ lygiašonis.
\item Duotas keturkampis $ABCD$ toks, kad $AB=BD$, $\angle 
  BCA=65^\circ, \angle ACD=50^\circ$. Rasti $\angle ABD$.
  %Tegu taškas $F$ yra simetriškas taškui $D$ tiesės $BC$
  %atžvilgiu.Kadangi $\angle FCB+\angle BCA=180^\circ$, tai $A,C,F$
  %yra vienoje tiesėje. Tada $\angle BAC=\angle BFC=\angle BDC$,
  %taigi $ABCD$ yra įbrėžtinis ir iš čia $\angle ABD=50^\circ$.
  
\item Lygiakraščiam trikampiui $ABC$ ant kraštinės $BC$
  trikampio išorėje nubrėžtas pusapskritimis. Per tašką $A$
  išvestos tiesės dalina tą pusapskritimio lanką į tris
  lygias dalis. Įrodykite, kad tos tiesės taip pat dalina
  $BC$ į tris lygias dalis.
  %Tegu taškai $B',C'$ dalina pusapskritimį į tris lygias
  %dalis, o $AB'$ ir $AC'$ atitinkamai kerta $BC$ taškuose
  %$F$ ir $E$. Tegu $D$ yra $BC$ vidurio taškas. Tada $DB
  %\parallel{AB}$, taigi $\frac{BF}{FD} = \frac{AB}{DB'} =
  %\frac{AB}{DB} = 2$. Bet $\frac{EF}{FD} = 2$, taigi $BF =
  %EF$. Panašiai ir $EF = CE$.  
  %\begin{center}
  %\begin{asy}
  %import olympiad;
  %size(200);
  %pair A, B, C, D, E, F, BB, CC;
  %path a1;
  %D=(50,0);
  %a1=arc(D,20,-90,90);
  %B=waypoint(a1,0);
  %BB=waypoint(a1,1/3);
  %CC=waypoint(a1,2/3);
  %C=waypoint(a1,1);
  %A=rotate(60,B)*C;
  %E=extension(B,C,A,CC);
  %F=extension(B,C,A,BB);
  %D=midpoint(C--B);
  %draw(a1);
  %dot(A,blue);
  %dot(B,blue);
  %dot(C,blue);
  %dot(D,blue);
  %dot(E,blue);
  %dot(F,blue);
  %dot(BB,blue);
  %dot(CC,blue);
  %draw(A--B--C--cycle);
  %draw(A--CC);
  %draw(A--BB);
  %draw(D--BB);
  %label("$A$",A,SW,blue);
  %label("$B$",B,SE,blue);
  %label("$C$",C,NE,blue);
  %label("$D$",D,NE,blue);
  %label("$E$",E,NE,blue);
  %label("$F$",F,NE,blue);
  %label("$B'$",BB,SE,blue);
  %label("$C'$",CC,NE,blue);
%\end{asy}
  %\end{center}
\item Trikampyje $ABC$ $\angle A=20^\circ$, $AB=AC$. Ant
  kraštinės $AB$ pažymėta atkarpa $AD$, lygi $BC$. Rasti
  $\angle BCD$. 
  %Paimame tašką $F$ trikampio viduje taip, kad $FBC$ būtų
  %lygiakraštis. Tada trikampiai $FBA$ ir $ADC$ vienodi pagal
  % dvi kraštines ir kampą. 
  %Taigi $\angle BCD = 80^\circ - 10^\circ = 70^\circ$.
\item Kvadrato $ABCD$ viduje paimtas taškas $M$ taip, kad
  $\angle MCD=\angle MDC=15^\circ$. Rasti kampą $\angle AMB$
  %Paimkime kvadrato viduje tašką $P$ taip, kad $\angle PBC
  %= \angle PCB = 15^\circ$. Tada $\angle PCM = 60^\circ$,
  %$PC = MC$, taigi $PCM$ lygiakraštis. Tada $MPB$
  %lygiašonis, iš kur $\angle MBP = 15^\circ$. Tada $\angle
  %MBC = 30^\circ$, taigi $ABM$ lygiakraštis.  Todėl $\angle
  %AMB = 60^\circ$.
\item Duotas keturkampis $ABCD$ toks, kad $\angle DAC=\angle DBA=45^\circ$,
  $AB=BC=CA$. Rasti $\angle ADC$.
  %Nesunkiai randame $\angle BDA=30^\circ$. Tada apie $ABD$ 
  %apibrėžto apskritimo centras $O$ tenkina $\angle BOA=60^\circ$,
  %t.y. $O=C$. Tada $CA=CD, \Longrightarrow \angle ADC=\angle DAC=
  %45^\circ$.
\item Trikampyje $ABC$, $\angle A=30^\circ, \angle C=45^\circ$.
  Ant $AC$ paimtas taškas $D$ toks, kad $CD=BA$. Rasti $\angle ABD$.
  %Tegu $BN$ yra aukštinė. Tada $BN=\frac{BA}{2}=\frac{DC}{2}$, taigi
  %$BN=NC=ND$. Iš čia $\angle BDC=45^\circ, \angle ABD=
  %105^\circ-90^\circ=15^\circ$.  
  
\item Duotas keturkampis $ABCD$ toks, kad $\angle BCA=21^\circ,
  \angle CDA=78^\circ, \angle CAD=39^\circ, BC=CD$. Rasti $\angle 
  BAC$.
 % Imame tašką $E$ ant $AD$ tokį, kad $CE=CD$ ($E\neq D$).
  %Paskaičiavę kampus, gauname, kad $CEA$ lygiašonis, o 
  %$CEB$ lygiakraštis. Tada $EA=EB=EC$, taigi $E$ yra apie $ABC$
  %apibrėžto apskritimo centras. Iš čia $\angle BAC=\frac{\angle 
  %BEC}{2}=30^\circ$.
  
\item Iškilajame keturkampyje $ABCD$, kuris nėra trapecija,
  kampai tarp įstrižainės $AC$ ir kraštinių yra $55^\circ,
  55^\circ, 16^\circ, 19^\circ$ kažkokia tvarka.  Rasti
  visus įmanomus smailius kampus tarp $AC$ ir $BD$. 
  %Atidžiai išnagrinėje visus variantus, gauname, kad yra
  %tik du skirtingi atvejai, pavaizduoti apačioje:
  %\begin{center}
  %\begin{asy}
  %import olympiad;
  %size(200);
  %pair A, B, C, D, AA, BB, CC, DD;
  %A=(0,0);
  %C=(40,0);
  %AA=(50,10);
  %CC=(100,10);
  %D=extension(A,rotate(55,A)*C,C,rotate(-55,C)*A);
  %B=extension(A,rotate(-16,A)*C,C,rotate(19,C)*A);
  %DD=extension(AA,rotate(-55,AA)*CC,CC,rotate(16,CC)*AA);
  %BB=extension(AA,rotate(55,AA)*CC,CC,rotate(-19,CC)*AA);
  %label("$55^\circ$",A,4*NE+down,deepred);
  %label("$55^\circ$",C,4*NW+S,deepred);
  %label("$55^\circ$",AA,3*NE+right,deepred);
  %label("$55^\circ$",AA,3*SE+right,deepred);
  %label("$16^\circ$",A,4*SE+N,deepblue);
  %label("$16^\circ$",CC,4*SW+N,deepblue);
  %label("$19^\circ$",C,4*SW+N,deepgreen);
  %label("$19^\circ$",CC,4*NW+S,deepgreen);
  %add(anglem(C,A,D,200,red,0));
  %add(anglem(D,C,A,200,red,0));
  %add(anglem(DD,AA,CC,200,red,0));
  %add(anglem(CC,AA,BB,200,red,0));
  %add(anglem(B,A,C,200,blue,0));
  %add(anglem(AA,CC,DD,200,blue,0));
  %add(anglem(A,C,B,200,green,0));
  %add(anglem(BB,CC,AA,200,green,0));
  %dot(A,blue);
  %dot(B,blue);
  %dot(C,blue);
  %dot(D,blue);
  %dot(AA,blue);
  %dot(BB,blue);
  %dot(CC,blue);
  %dot(DD,blue);
  %draw(A--B--C--D--cycle);
  %draw(AA--BB--CC--DD--cycle);
  %draw(A--C);
  %draw(B--D);
  %draw(AA--CC);
  %draw(DD--BB);
  %label("$A$",A,SW,blue);
  %label("$B$",B,down,blue);
  %label("$C$",C,SE,blue);
  %label("$D$",D,up,blue);
  %label("$A'$",AA,SW,blue);
  %label("$B'$",BB,up,blue);
  %label("$C'$",CC,NE,blue);
  %label("$D'$",DD,down,blue);
%\end{asy}
  %\end{center}
  %Atvejis kairėje tenkina šio skyrelio naudingąją savybę, nes $\angle
  %ABC + \frac{\angle ADC}{2} = 145^\circ + 35^\circ =
  %180^\circ$ ir $AD = DC$. Taigi $D$ yra apie apskritimą
  %$ABC$ apibrėžto apskritimo centras ir todėl $\angle BDC =
  %2\angle BAC = 34^\circ$, ir kampas tarp keturkampio
  %įstrižainių yra $55^\circ + 32^\circ = 87^\circ$. O
  %atvejui dešinėje reikia atskiro sprendimo - mes
  %įrodysime, kad kampas tarp įstrižainių lieko toks pats.
  %Paimkime keturkampį $ABCD$ kaip paveikslėlyje viršuje
  %kairėje. Tegu įstrižainės kertasi taške $E$, linija,
  %lygiagreti $DC$ ir einanti per tašką $A$, kerta $BD$ taške
  %$F$, o linija per $D$, lygiagreti $AB$, kerta $AC$ taške
  %$G$. Tada trikampiai $DEG$ ir $AEB$ yra panašūs, taip pat
  %kaip ir trikampiai $DEC$ ir $AEF$. Tada $\frac{EB}{EC} =
  %\frac{EB}{DE}\cdot\frac{DE}{EC} =
  %\frac{AE}{EG}\cdot\frac{EF}{AE} = \frac{EF}{EG}$, taigi
  %$EBC$ ir $EFG$ panašūs ir todėl $\angle EGF = \angle ECB
  %= 19^\circ$. Belieka pastebėti, kad keturkampis $ADGF$
  %yra būtent tas kurio mums reikia, nes kampai tarp
  %įstrižainių ir kraštinių yra $55^\circ, 55^\circ,
  %16^\circ, 19^\circ$.
  %\begin{center}
  %\begin{asy}
  %import olympiad;
  %size(200);
  %pair A, B, C, D, E, F, G;
  %A=(0,0);
  %C=(40,0);
  %D=extension(A,rotate(55,A)*C,C,rotate(-55,C)*A);
  %B=extension(A,rotate(-16,A)*C,C,rotate(19,C)*A);
  %E=extension(A,C,B,D);
  %G=extension(A,C,D,rotate(109,D)*A);
  %F=extension(D,B,A,A+C-D);
  %label("$55^\circ$",A,4*NE+down,deepred);
  %label("$55^\circ$",C,4*NW+S,deepred);
  %label("$16^\circ$",A,4*SE+N,deepblue);
  %label("$16^\circ$",G,(-5,1),deepblue);
  %label("$19^\circ$",C,4*SW+N,deepgreen);
  %add(anglem(C,A,D,200,red,0));
  %add(anglem(D,C,A,200,red,0));
  %add(anglem(B,A,C,200,blue,0));
  %add(anglem(D,G,C,200,blue,0));
  %add(anglem(A,C,B,200,green,0));
  %dot(A,blue);
  %dot(B,blue);
  %dot(C,blue);
  %dot(D,blue);
  %dot(F);
  %dot(E);
  %dot(G);
  %draw(A--B--C--D--cycle);
  %draw(A--C);
  %draw(B--D);
  %draw(D--G);
  %draw(B--F);
  %draw(A--F);
  %draw(F--G);
  %draw(C--G);
  %label("$A$",A,SW,blue);
  %label("$B$",B,SW,blue);
  %label("$C$",C,SE,blue);
  %label("$D$",D,up,blue);
  %label("$F$",F,SW,blue);
  %label("$E$",E,NW,blue);
  %label("$G$",G,SE,blue);
%\end{asy}
%\end{center}
\item $P$ - vidinis trikampio $ABC$ taškas ($AB = BC$).
  $\angle ABC = 80^\circ$, $\angle PAC = 40^\circ$, $\angle
  ACP = 30^\circ$. Rasti $\angle BPC$. 
  % Pirmas būdas: Tegu $AP$ kerta $BC$ taške $D$, o $CP$ kerta $AB$ taške
  %$E$. Tada nesunkiai paskaičiuodami kampus gauname, kad
  %$BDA$ yra statusis, o $BCE$ lygiašonis. Todėl $BE = 2BD$.
  %Paimkime ant tiesės $AD$ tašką $F$ tokį, kad $BE = BF$
  %($F$ nėra $ABC$ viduje). Tada $BFD$ statusis, ir $BF =
  %2BD$, taigi $\angle FBD = 60^\circ$. Tada keturkampyje
  %$BEPF$ $BE = BF$ ir $\angle EPF + \frac{\angle EBF}{2} =
  %110^\circ + \frac{60^\circ + 80^\circ}{2} = 180^\circ$,
  %taigi $BEPF$ tenkina minėtąją savybę, ir todėl $BP = BE$.
  %Tada nesunkiai $\angle BPD = 30^\circ$, taigi $\angle BPC
  %= 100^\circ$.  
  %\begin{center}
  %\begin{asy}
  %import olympiad;
  %size(200);
  %pair A, B, C, D, E, F, P;
  %A=(0,0);
  %C=(100,0);
  %B=extension(A,rotate(50,A)*C,C,rotate(-50,C)*A);
  %P=extension(A,rotate(40,A)*C,C,rotate(-30,C)*A);
  %D=extension(A,P,B,C);
  %E=extension(C,P,B,A);
  %F=extension(A,D,B,rotate(60,B)*D);
  %add(anglem(C,A,B,200,green,0));
  %add(anglem(B,C,A,200,green,0));
  %add(rightanglem(B,D,P,150));
  %dot(A,blue);
  %dot(B,blue);
  %dot(C,blue);
  %dot(D,blue);
  %dot(F);
  %dot(E);
  %dot(P);
  %draw(A--B--C--cycle);
  %draw(A--F);
  %draw(E--C);
  %draw(B--P);
  %draw(B--F);
  %label("$A$",A,SW,blue);
  %label("$B$",B,up,blue);
  %label("$C$",C,SE,blue);
  %label("$D$",D,down,blue);
  %label("$F$",F,NE,blue);
  %label("$E$",E,NW,blue);
  %label("$P$",P,down,blue);
  %add(pathticks(E--B,2,0.5,0,125,red));
  %add(pathticks(F--B,2,0.5,0,125,red));
%\end{asy}
  %\end{center}
  %
  %Antras būdas: Tegu $CP$ kerta kampo $B$ pusiaukampinę taške $M$.
  %Nesunkiai skaičiuodami kampus gauname, kad $P$ guli ant kampų $BMA$ ir 
  %$BAM$ pusiaukampinių, ir todėl yra trikampio $ABM$ įbrėžto apskritimo centras.
  %Todėl $BP$ yra $\angle ABM$ pusiaukampinė ir vėl  paskaičiavę kampus gauname tą  
  % patį atsakymą. 
\item  ( IMO 1975 motyvais) Duotas bet koks trikampis $ABC$. Jo išorėje 
  sukonstruoti trikampiai $ABR,BCP,ACQ$, taip, kad $\angle BCP=\angle 
   ACQ=30^\circ$, $\angle CBP=\angle CAQ=60^\circ-x$, $\angle RBA=\angle 
   RAB=x$. Įrodyti, kad $PR=QR$.
   %Sukonstruojame lygiakraštį trikampį $ABT$ trikampio $ABC$ išorėje.
   %Tada trikampiai $ART$ ir $AQC$ panašūs pagal 3 kampus (abu turi kampus
   %lygius $30^\circ$, $60^\circ-x$, $90^\circ+x$). Tada $\frac{AT}{AC}=
   %\frac{AR}{AQ}$, $\angle TAC=\angle RAQ$. Taigi trikampiai $TAC$ ir $RAQ$
   %panašūs pagal 2 kraštines ir kampą, o panašumo koeficientas yra 
   %$\frac{AT}{AR}=\frac{AB}{AR}$. Panašiai $RBP$ ir $CBT$ irgi panašūs
   %su tuo pačiu panašumo koeficientu $\frac{AB}{BR}$. Taigi $QR=
   %\frac{CT\cdot AR}{AB}=\frac{CT\cdot RB}{AB}=PR$.
     %\begin{center}
  %\begin{asy}
  %import olympiad;
  %size(200);
  %pair A, B, C, D, T, Q, R, P;
  %A=(0,0);
  %C=(100,0);
  %B=(70,55);
  %R=extension(A,rotate(38,A)*B,B,rotate(-38,B)*A);
  %T=extension(A,rotate(60,A)*B,B,rotate(-60,B)*A);
  %Q=extension(A,rotate(-22,A)*C,C,rotate(30,C)*A);
  %P=extension(B,rotate(22,B)*C,C,rotate(-30,C)*B);
  %dot(A,blue);
  %dot(B,blue);
  %dot(C,blue);
  %dot(T,blue);
  %dot(Q,blue);
  %dot(P,blue);  
  %dot(R,blue);
  %draw(A--B--C--cycle);
  %draw(A--R);
  %draw(B--R);
  %draw(B--T);
  %draw(A--T);
  %draw(A--Q);
  %draw(C--Q);
  %draw(C--P);
  %draw(B--P);
  %draw(R--Q);
  %draw(P--R);
  %draw(T--R);
  %label("$A$",A,SW,blue);
  %label("$B$",B,up,blue);
  %label("$C$",C,SE,blue);
  %label("$Q$",Q,SE,blue);
  %label("$T$",T,up,blue);
  %label("$R$",R,up,blue);
  %label("$P$",P,up,blue);
  %add(anglem(P,C,B,200,green,0));
  %add(anglem(A,C,Q,200,green,0));
  %add(anglem(C,B,P,200,red,0));
  %add(anglem(Q,A,C,200,red,0));
  %add(anglem(R,A,T,200,red,0));
  %add(anglem(T,B,R,200,red,0));
  %label("$30^\circ$",C,4*up+left,deepgreen);
  %label("$30^\circ$",C,4*W+S,deepgreen);
%\end{asy}
  %\end{center} 
\item Duotas keturkampis $ABCD$ toks, kad $AB=AD$, $\angle CBD=30^\circ$,
  $\angle BAC=48^\circ$, $\angle DAC=16^\circ$. Rasti $\angle ACD$.
  %Paimkime tašką $E$ tokį, kad $EAD$ būtų lygiakraštis, o $B$ ir $E$
  %būtų skirtingose $AD$ pusėse. Tegu $BE$ kerta $AC$ 
  %taške $F$. Suskaičiavę kampus nesunkiai gauname, kad $EAF$ yra lygiašonis,
  %ir tada $EFD$ taip pat lygiašonis. Toliau suskaičiavę kampus gauname, 
  %kad $\angle BDF=44^\circ=\angle BCF$, taigi $BCDF$ įbrėžtinis.
  %Iš čia nesunkiai gauname, kad $\angle DCF=\angle DBF=30^\circ$. 
  %\begin{center}
  %\begin{asy}
  %import olympiad;
  %size(200);
  %pair A, B, C, D, E, F;
  %A=(0,0);
  %D=(100,0);
  %B=rotate(64,A)*D;
  %C=extension(A,rotate(16,A)*D,B,rotate(30,B)*D);
  %E=rotate(60,D)*A;
  %F=extension(B,E,C,A);
  %dot(A,blue);
  %dot(B,blue);
  %dot(C,blue);
  %dot(E,blue);
  %dot(F,blue);  
  %dot(D,blue);
  %draw(A--B--C--D--cycle);
  %draw(A--C);
  %draw(A--E);
  %draw(B--D);
  %draw(B--E);
  %draw(D--E);
  %label("$A$",A,SW,blue);
  %label("$B$",B,up,blue);
  %label("$C$",C,SE,blue);
  %label("$D$",D,SE,blue);
  %label("$F$",F,SW,blue);
  %label("$E$",E,down,blue);
  %add(pathticks(E--A,2,0.5,0,125,red));  
  %add(pathticks(E--D,2,0.5,0,125,red)); 
  %add(pathticks(A--B,2,0.5,0,125,red)); 
  %add(pathticks(A--D,2,0.5,0,125,red)); 
  %add(pathticks(E--F,2,0.5,0,125,red)); 
  %add(anglem(F,A,B,300,red,0));
  %add(anglem(D,A,F,300,deepgreen,0));
  %add(anglem(F,B,D,300,blue,0));
  %add(anglem(F,C,D,300,blue,0));
  %add(anglem(D,B,C,300,olive,0));
  %label("$16^\circ$",A,4*right+up,deepgreen);
  %label("$48^\circ$",A,3*up+3*right,red);
  %label("$30^\circ$",B,3*right+3*down,olive);
%\end{asy}
  %\end{center} 
\item Duotas keturkampis $ABCD$ toks, kad jo įstrižainės statmenos,
ir $\angle BAC=20^\circ,\angle DAC=10^\circ, \angle BCA=50^\circ$.
Rasti $\angle BDC$.
  %(Pirmas būdas) Trigonometrinis būdas: Nubrėžiame kuo tikslesnį
  % brėžinį ir spėjame, kad atsakymas yra $60^\circ$. Tegu įstrižainės 
  %kertasi taške $O$. Tada $\tan \angle BDC= \frac{CE}{DE}=
  %\frac{CE \cdot BE \cdot AE}{BE\cdot AE \cdot DE}=\tan 20^\circ
  %\tan 40^\circ \tan 80^\circ$. Belieka parodyti, kad $\tan 20^\circ
  %\tan 40^\circ \tan 80^\circ=\tan 60^\circ$, kas yra vidutinio
  %sunkumo trigonometrijos uždavinys, paliekamas skaitytojui.
  %
  %(Antras būdas) Geometrinis būdas: Tegu įstrižainės vėl kertasi taške 
  %$O$. Imame tašką $H$ ant $OC$ tokį, kad $\angle DHA=40^\circ$ ( 11 uždavinio idėja).
  %Tada iš trikampių $HDO$ ir $OCB$ panašumo $\frac{HO}{OB}=\frac{OD}{OC}$,
  %taigi trikampiai $COD$ ir $HOB$ panašūs ir mums tada tereikia rasti
  %kampą $\angle BHO$. Imame tašką $K$ ant spindulio $HD$ taip, kad $\angle DKA=
  %80^\circ$. Tegu $W$ yra tiesių $DH$ ir $AB$ sankirta, o taškas 
  %$Z$ simetriškas taškui $D$ kampo $\angle AWK$ pusiaukampinės atžvilgiu
  % (Tada $AWK$ yra lygiašonis su kampais $80^\circ$ prie pagrindo). Taikome trečiojo 
  % uždavinio sprendimą trikampiui $AWK$ ir gauname, kad $\angle ZHD=70^\circ=\angle DBZ$,
   %tai yra $ABHK$ yra lygiašonė trapecija. Iš čia nesunkiai ieškomas kampas
   %yra $60^\circ$. 
          %\begin{center}
  %\begin{asy}
  %import olympiad;
  %size(200);
  %pair A, B, C, D, W, Z, K, O, H, J, L;
  %O=(0,0);
  %A=(100,0);
  %L=(0,10);
  %B=extension(O,L,A,rotate(-20,A)*O);
  %D=extension(O,L,A,rotate(10,A)*O);
  %C=extension(O,A,D,rotate(60,D)*B);
  %H=extension(O,A,D,rotate(50,D)*B);
  %W=extension(D,H,A,B);
  %K=rotate(-20,W)*A;
  %Z=rotate(20,W)*D;
  %dot(A,blue);
  %dot(B,blue);
  %dot(C,blue);
  %dot(Z,blue);
  %dot(W,blue);  
  %dot(O,blue);
  %dot(H,blue);
  %dot(K,blue);
  %dot(D,blue);
  %draw(A--B--C--D--cycle);
  %draw(B--W);
  %draw(W--K);
  %draw(B--D);
  %draw(C--A);
  %draw(A--K);
  %draw(B--H);
  %draw(Z--H);
  %label("$A$",A,right,blue);
  %label("$B$",B,up,blue);
  %label("$C$",C,left,blue);
  %label("$D$",D,SW,blue);
  %label("$H$",H,down,blue);
  %label("$Z$",Z,NE,blue);
  %label("$O$",O,NE,blue);
  %label("$K$",K,SE,blue);
  %label("$W$",W,left,blue);
%\end{asy}
  %\end{center} 
%   
%   Išvada: kartais trigonometrinis sprendimas yra geriau.
 \item Duotas statusis trikampis $ABC$ su $\angle A=50^\circ$, $\angle 
 C=40^\circ$. Paimti taškai $D$ ir $E$ ant atitinkamai $BC$ ir $BA$ tokie, 
 kad $\angle BAD=20^\circ$, $\angle BCE=10^\circ$. Rasti $\angle EDA$.
 %Mes pirmiau ieškosime $\angle BDE$. Tam imame tašką $F$, simetrišką taškui
 %$D$ taško $B$ atžvilgiu, ir tašką $G$, simetrišką taškui $E$ taško $B$
  %atžvilgiu. Tereikia rasti $\angle BFG$. Dabar pastebime, kad uždavinys
  %pasidarė neįtikėtinai panašus į prieš tai buvusį. Tiesą sakant, sprendimas 
  %nuo šios vietos irgi yra kone identiškas -  jį paliekame skaitytojui.
  %(Atsakymas ytra $40^\circ$). Abiejų uždavinių gražumas slypi tame, kad
  %egzistuoja šeši keturkampiai, kurių įstrižainės dalina juos į keturis stačiuosius
  %trikampius su kampais $ (30^\circ,60^\circ),(20^\circ,70^\circ),(40^\circ,50^\circ),
  %(10^\circ,80^\circ)$, o keturkampiai vienas su kitu susiję 11 uždavinio 
  %konstrukcijomis. Kaip matėme prieš tai buvusiame uždavinyje, su jais susidūrus
  %geriau naudotis trigonometrija (taip pat ir šiuo atveju). 
\end{enumerate}

\newpage
\section{Geometrinės nelygybės}



Geometrijos ir nelygybių temos susikerta geometrinių nelygybių
uždaviniuose. Juos galima išskaidyti į dvi pagrindines kategorijas:
algebrinės nelygybės, kurių kintamieji yra trikampio komponentai (šios
nelygybės dažniausiai būna ciklinės kampų atžvilgiu), ir nelygybės,
lyginančios ilgius bei plotus. Šiame skyiuje nagrinėsime antrojo 
tipo nelygybes. Tokie uždaviniai olimpiadose pasitaiko ne itin dažnai, bet jie būna
įvairiausio sunkumo. Be to, tai vieni tų uždavinių, kuriuos
dažnai galima paversti į algebros uždavinį ir bandyti spręsti 
algebriniais metodais. Tačiau šiame skyriuje nagrinėsime geometrinius
 jų sprendimo būdus, kurie nors reikalauja šiokio tokio išmoningumo,
yra trumpesni ( Nors tikrai ne visas nelygybes įmanoma 
taip išspręsti - kai kurios daromos tik algebrinias metodais).

\begin{teig}
Keletas gerai žinomų nelygybių:
\begin{itemize}

  \item Trikampio nelygybė: Jeigu trikampio kraštinių ilgiai
    yra $a,b,c$, tai tada $a+b>c$, $b+c>a$, $a+c>b$.
   \item Jeigu $ABC$ yra trikampis, $R$ - apie tą trikampį
    apibrėžto apskritimo spindulys, $r$ - įbrėžto apskritimo
    spindulys, tai tada $R\geq2r$. Lygybės atvejis tada ir tik
    tada, kai trikampis yra lygiakraštis. Įrodymas duotas žemiau. 
  \item Jeigu ant trikampio $ABC$ kraštinės $AB$ paimtas
    taškas $D$ ( nesutampantis su $A$ ar $B$), tai tada arba
    $AC>CD$, arba $BC>CD$, arba $AC>CD$ ir $BC>CD$. Bet kokiu
    atveju, $AC+BC>CD$.
  \item Apskritimo styga visada trumpesnė už skersmenį.
  \item Kampo kosinusas ir sinusas visada yra intervale
    $[-1,1]$.
\end{itemize}
\end{teig}

\begin{pav}
Duotas trikampis $ABC$. Įrodykite, kad $R\geq 2r$.
\end{pav}
\begin{sprendimas}
Tegu $K,L,M$ yra trikampio kraštinio vidurio taškai. Tada $KLM$ 
yra dvigubai mažesnis nei $ABC$, taigi $R_{KLM}=\frac{R}{2}$. Tegu
$\omega$  yra apie $KLM$ apibrėžtas apskritimas. Nubrėžkime tris 
liestines apskritimui $\omega$, lygiagrečias $AB,BC,CD$ taip, kad jų
sankirtos yra trikampio, kurio viduje yra trikampis $ABC$, viršūnės.
Tegu šis trikampis yra $QPR$, ir jis akivaizdžiai panašus į $ABC$ ir 
už jį nemažesnis. Taigi iš atitinkamų elementų panašumo $\frac{R}{2}\
=R_{KLM}\geq r$.
\end{sprendimas}

\begin{pav}
Duotas taisyklingasis šešiakampis. Įrodykite, kad suma atstumų nuo jo 
 viršūnių iki centro yra mažesnė nei tokia suma iki bet kurio kito 
 taško.
\end{pav}
\begin{sprendimas}
Tegu $ABCDEF$ yra tas šešiakampis, $O$-bet koks taškas. Tada $(OA+OD)+
(OB+OE)+(OC+OF)\geq AD+BE+CF$.
\end{sprendimas}

Dalis nelygybių gali būti išsprendžiamos vien prisibrėžiant ir pritaikant
trikampio nelygybę.
\begin{pav}[LitMO 2011 rajono etapas]
  Duotas lygiašonis trikampis $ABC$, $AB=AC$. Ant spindulio $BC$ už 
  taško $C$ paimtas taškas $E$, o ant atkarpos $BC$ paimtas taškas 
  $F$ taip, kad $BF=CE$. Įrodyti, kad $AB+AC<AE+AF$. 
\end{pav}

\begin{sprendimas}
  Tereikia nubrėžti trikampį kuriam galėtume taikyti trikampio nelygybę.
  Tam imame tašką $G$ ant spindulio $AB$ už taško $B$ taip, kad $BG=AB$.
  Tada trikampiai $ACE$ ir $BFG$ vienodi pagal dvi kraštines ir kampą,
  taigi $AB+AC=AB+BG=AG<AF+FG=AF+AE$, ko ir reikėjo. 
  %Nelygiašonis trikampis
\begin{center}
\begin{asy}
import olympiad;
size(200);
pair A, B, C, F, E, G;
B=(0,0);
C=(80,0);
A=(40,50);
F=(25,0);
E=(105,0);
G=(-40,-50);
dot(A,blue);
dot(B,blue);
dot(C,blue);
dot(E,blue);
dot(F,blue);
dot(G,blue);
draw(A--B--C--cycle);
draw(A--E--F--cycle);
draw(B--G--F--cycle);
label("$A$",A,up,blue);
label("$B$",B,left,blue);
label("$C$",C,down,blue);
label("$E$",E,right,blue);
label("$G$",G,left,blue);
label("$F$",F,down,blue);
add(pathticks(A--B,2,0.5,0,150,red)); 
add(pathticks(A--C,2,0.5,0,150,red)); 
add(pathticks(B--G,2,0.5,0,150,red)); 
add(pathticks(A--E,2,0.5,45,150,red)); 
add(pathticks(F--G,2,0.5,45,150,red)); 
add(pathticks(F--B,3,0.5,45,150,green)); 
add(pathticks(E--C,3,0.5,45,150,green)); 
add(anglem(E,C,A,200,red,0));
add(anglem(G,B,F,200,red,0));
\end{asy}
\end{center}
\end{sprendimas}

\begin{pav}
Duotas trikampis $ABC$. Kampo $B$ pusiaukampinė pratęsta iki
 susikirtimo su apibrėžtiniu apskritimu taške $D$. Įrodyti, kad $2BD>
 AB+BC$
\end{pav}

\begin{sprendimas}
%Nelygiašonis trikampis
\begin{center}
\begin{asy}
import olympiad;
size(200);
pair A, B, C, D, E, X, Y;
A=(0,0);
C=(80,0);
B=(30,50);
X=rotate(-degrees((B-C)/((B-A)))/2,A)*C;
Y=rotate(-degrees((B-C)/((B-A)))/2,B)*C;
D=extension(A,X,B,Y);
E=rotate(degrees((A-D)/(C-D)),D)*B;
dot(A,blue);
dot(B,blue);
dot(D,blue);
dot(C,blue);
dot(E,blue);
draw(A--B--C--D--cycle);
draw(A--C);
draw(B--D);
draw(B--E);
draw(E--D);
draw(circumcircle(A,B,C));
label("$A$",A,left,blue);
label("$B$",B,up,blue);
label("$D$",D,down,blue);
label("$C$",C,right,blue);
label("$E$",E,up,blue);
add(pathticks(E--A,2,0.5,0,125,red)); 
add(pathticks(B--C,2,0.5,0,125,red)); 
add(pathticks(A--D,2,0.5,30,125,red)); 
add(pathticks(C--D,2,0.5,30,125,red)); 
add(pathticks(E--D,3,0.5,30,125,red)); 
add(pathticks(B--D,3,0.5,30,125,red)); 
add(anglem(B,C,D,200,red,0));
add(anglem(E,A,D,200,red,0));
\end{asy}
\end{center}
Šio uždavinio sunkumas tame, kad kitaip nei trikampio nelygybėje, sumą
turime kitoje lygybės pusėje, todėl gali pasirodyti, kad su trikampio
nelygybe nieko nepavyks. Tačiau mes galime parašyti $2BD=BD+BD$, ir 
pabandyti surasti trikampį su dvejomis kraštinėmis, lygiomis $BD$,
ir trečia kraštine, lygia $AB+BC$. Tam mes pažymime tašką $E$ ant $BA$ 
tęsinio taip, kad $AE=CB$. Tada $ADE$ ir $CBD$ vienodi pagal 2 
kraštines ir kampą. $BDE$ ir yra ieškomas trikampis.




\end{sprendimas}

\begin{pav}
$a,b,c$ yra kažkokio trikampio kraštinės. Įrodyti, kad $\frac{1}{a+b},
\frac{1}{a+c},\frac{1}{c+b}$ taip pat yra kažkokio trikampio kraštinės.
\end{pav}

\begin{sprendimas}
Užtenka parodyti, kad galioja $\frac{1}{a+b}+\frac{1}{b+c}>\frac{1}{a+b}$
ir ciklinės perstatos. Išprastinę vardiklius gauname $(b-a)(b-c)<
(a+c)(a+c)$, kas yra akivaizdu, nes $b-a<a+c$ ir $b-c<a+c$.
\end{sprendimas}

\subsubsection{Uždaviniai}

\begin{enumerate}
\item Duotas trikampis $ABC$, taškas $O$ jo viduje.
  Įrodykite, kad $AB+BC+CA>AO+BO+CO> \frac{AB+BC+CA}{2}$.
  %Tegu tiesė, lygiagreti $BC$ ir einanti per $O$, kerta
  %kraštines $AB$ ir $AC$ atitinkamai taškuose $X$ ir $Y$.
  %Tada $AB + BC + CA > AB + CA + XY = AX + XB + CY + YA +
  %XO + OY = (AX+AY) + (XO+OB) + (YO+OC) > AO + OB + OC$ iš
  %trikampio nelygybės. Kitai nelygybės pusei vėl taikome
  %trikampio nelygybę: $AO + BO + CO = (\frac{AO+BO}{2}) +
  %(\frac{BO+CO}{2}) + (\frac{CO+AO}{2})$$ > (\frac{AB}{2})
  %+ (\frac{BC}{2}) + (\frac{AC}{2})$. 
\item Duotas trikampis $ABC$, $AM$ - pusiaukraštinė. Įrodyti,
  kad $AB + AC \geq 2AM$.
  %Tegu taškas $P$ yra simetriškas taškui $A$ taško $M$
  %atžvilgiu. Tada $ABPC$ yra lygiagretainis, ir iš
  %trikampio nelygybės $AB + AC = AB + BP > AP = 2AM$.
\item Įrodyti, kad trikampio pusiaukraštinių ilgių suma yra
  mažesnė už trikampio perimetrą, bet didesnė už
  $\frac{3}{4}$ perimetro.
  %Pirma dalis seka iš 2 uždavinio, pritaikyto visoms
  %pusiaukraštinėms iš eilės ir sudėjus tris gautas
  %nelygybes. Kitai daliai galime pritaikyti 2 uždavinio
  %nelygybę pusiaukraštinių susikirtimo taškui ir padauginti
  %rezultatą iš $\frac{3}{2}$.
\item Duotas trikampis, o į jį įbrėžtas kvadratas taip, kad 
  jo dvi viršūnės yra ant vienos kraštinės, o ant kitų kraštinių
  po vieną viršūnę. Tegu $a$ yra kvadrato kraštinės ilgis, o
  $r$ įbrėžtinio apskritimo spindulys. Įrodyti, kad
  $\sqrt{2}r<a<2r$. 
  %Akivaizdu, kad į kvadratą įbrėžto apskritimo spindulys
  %mažesnis nei į trikampį įbrėžto apskritimo spindulys,
  % o apibrėžto didesnis. Iš čia viskas akivaizdžiai seka 
  %(palyginti galite su įrodymu kad $R>2r$ iš pavyzdžių).
\item Trikampis $ABC$ lygiakraštis su $AB=1$. Taškas $O$ yra
  trikampio viduje. Įrodykite, kad $2\geq OA+OB+OC$.
  %Sprendimo idėja tokia pati, kaip ir 2 uždavinio pirmosios
  %dalies - brėžiame per tašką $O$ tiesę, lygiagrečią $AB$,
  %kuri kerta $CA$ ir $CB$ atitinkamai taškuose $X$ ir $Y$.
  %Tada $2=CA+CB=(AX+XO)+(OY+YB)+CY>AO+BO+CO$.

\item Duotas keturkampis, o jo viduje taškas. Įrodyti, kad
  atstumų nuo to taško iki keturkampio viršūnių suma
  neviršija $D_1 + D_2 + P$, kur $P$ - keturkampio
  perimetras, $D1, D2$ - įstrižainių ilgiai.
  %Tarkime, kad keturkampis yra $ABCD$, o įstrižainės
  %kertasi taške $O$, taškas viduje yra $P$. Neprarasdami
  %bendrumo, galime teigti, kad $P$ yra trikampyje $AOB$.
  %Tada iš antro uždavinio $AB + BC + CA > PA + PB + PC$ ir
  %$AD + DB > PD$. Sudėję nelygybes ir pridėję $DC$ prie
  %kairės pusės, gauname tai, ko reikia. (Gali pasiroyti, kad
  %ši nelygybė visai negriežta, bet taip nėra, pavyzdžiui,
  %jei taškai $A, B, C$ beveik sutampa, o $D$ yra labai toli
  %nuo jų ir $P$ yra arti $D$, tai gauname visai gerą
  %įvertį).

\item Duotas keturkampis su kraštinėmis $a,b,c,d$ (tokia tvarka).
  Įrodyti, kad $S<\frac{(a+b)(c+d)}{4}$, kur $S$ yra keturkampio
  plotas.
  %Padalinę keturkampį įstrižaine į du trikampius kurių dvi 
  %kraštinės yra $a$ ir $d$, $b$ ir $c$ gauname $2S<ad+bc$.
  %Tada imame keturkampį kurio kraštinės yra $a,b,d,c$ (tokia 
  %tvarka), kuris gaunamas pradinį keturkampį perkirpus pusiau
  %kita įstrižaine į du trikampius, viena jų apvertus ir 
  %trikampius suklijavus atgal ta pačia įstrižaine. Šio keturkapio
  %plotas irgi yra $S$, ir kaip anksčiau gauname $2S<ac+bd$. 
  %Sudėję dvi nelygybes, gauname ką reikia. 
  
\item Ant kvadrato, kurio kraštinės ilgis 1, kiekvienos
  kraštinės yra pastatytas statusis trikampis, kurių
  įžambinė yra to kvadrato kraštinė. Tų 4 stačiųjų trikampių
  statieji kampai yra $A, B, C, D$, o į stačiuosius trikampius
  įbrėžtų apskritimų centrai yra $O_1$, $O_2$, $O_3$, $O_4$.
  Įrodyti, kad $ABCD$ plotas neviršija 2, o $O_1O_2O_3O_4$
  plotas neviršija 1.
  %Tegu $M$ yra kvadrato kraštinės prie viršūnės $A$ vidurio
  %taškas, o $O$-kvadrato centras. Tada $AO \leq OM + MA =
  %\frac{1}{2} + \frac{1}{2}$, taigi $ABCD$ telpa į
  %apskritmą su centru $O$ ir spinduliu 1. Mes žinome iš
  %pirmojo uždavinio, kad jei keturkampis telpa į apskritimą
  %spindulio $R$, tai jo plotas neviršija $2R^2$. Iš čia ir
  %seka rezultatas. Antrajai daliai pastebėkime, kad taškai
  %$O_1$, $O_2$, $O_3$, $O_4$ guli ant apskritimo, apibrėžto
  %apie kvadratą (suskaičiuokite kampus). Vėl pritaikome tą
  %patį faktą: šį sykį apskritimo spindulys yra
  %$\frac{1}{\sqrt{2}}$, o tai ir yra tai, ko reikia.  
.

\item Duotas įbrėžtinis keturkampis $ABCD$ toks, kad
  $BC=CD$. $E$ - kraštinės $AC$ vidurio taškas. Įrodyti, kad
  $BE+DE\geq AC$.
  %$AC$ yra kampo $\angle BAD$ pusiaukampinė, nes kampai
  %$\angle BAD$ ir $\angle BAC$ remiasi į lygius lankus.
  %Tegu taškas $F$ yra simetriškas taškui $B$ $AC$
  %atžvilgiu. Tada taškai $A, D, F$ yra vienoje tiesėje, ir
  %trikampis $CDF$ yra lygiašonis, nes $CD = BC = FC$. Tegu
  %$M$ yra $DF$ vidurio taškas. Tada $EM = \frac{CA}{2}$,
  %nes $CMA$ statusis.  Pritaikę 3 uždavinio nelygybę
  %trikampiui $EFD$, mes gauname $ED + EF\geq 2EM$, kas
  %ekvivalentu $BE + DE\geq AC$.
\item Į taisyklingąjį septynkampį įbrėžtas apskritimas ir
  aplink taip pat apibrėžtas apskritimas. Tada pats
  septynkampis yra žiede, suformuotame iš dviejų apskritimų.
  Tas pats padaryta su taisyklinguoju 17-kampiu. Taip jau
  nutiko, kad abiejų žiedų plotai lygūs. Įrodyti, kad
  septynkampio ir septyniolikakampio kraštinės yra vienodo
  ilgio.
  %Tegu įbrėžto į septynkampį apskritimo skersmuo yra $r$,
  %apibrėžto - $R$, o septynkampio kraštinė $a$. Tada žiedo
  %plotas yra $\pi R^2 - \pi r^2 = \pi (R^2-r^2)$. Bet $(R^2
  %- r^2) = \frac{a^2}{4}$ iš Pitagoro teoremos, taigi žiedo
  %plotas yra $\frac{\pi a^2}{4}$. Taigi žiedo plotas
  %tiesiogiai priklauso nuo kraštinės ilgio ir nepriklauso
  %nuo kraštinių skaičiaus, todėl septynkampio ir
  %septyniolikakampio kraštinės vienodo ilgio. 

\item Stačiakampį $ABCD$ kurio plotas 1, sulenkė taip, kad
  taškas $C$ sutapo su tašku $A$. Įrodyti, kad gauto
  penkiakampio plotas yra mažesnis negu $\frac{3}{4}$.
  %Užtenka įrodyti, kad persiklojančios dalies (lygiašonio
  %trikampio) plotas yra daugiau negu $\frac{1}{4}$. Tai yra
  %beveik akivaizdu: šoninė jo kraštinė yra daugiau nei pusė
  %pradinio stačiakampio kraštinės, o aukštinės, nuleistos į
  %tą kraštinę, ilgis sutampa su kitos stačiakampio
  %kraštinės ilgiu.

\item Trikampyje $ABC$ paimta pusiaukraštinė $AM$. Ar galejo
  taip nutikti, kad trikampių $AMC$ ir $AMB$ įbrėžtinių
  apskritimų spinduliai skiriasi lygiai du kartus?
  %Tegu $AB = c, AC = b, BC = a, AM = m$, į $AMB$ įbrėžto
  %apskritimo skersmuo $r$, o į $AMC$ $2r$. Tada $AMB$
  %plotas yra $\frac{r(c+\frac{a}{2}+m)}{2}$, o $AMC$ plotas
  %$r(\frac{a}{2} + m + b)$. Bet šie plotai lygūs, taigi $c
  %+ m + \frac{a}{2} = a + 2m + 2b$, arba $c = \frac{a}{2} +
  %m + b$, kas prieštarauja trikampio nelygybei pritaikytai
  %trikampiui $AMB$.

\item Duotas trikampis $ABC$ su aukštinėmis $AA_1$, $BB_1$,
  $CC_1$ ir pusiaukraštinėmis $AA_2$, $BB_2$,$CC_2$.
  Įrodyti, kad iš atkarpų $A_1B_2$, $B_1C_2$, $C_1A_2$
  galima sudėti trikampį.
  %Kadangi $CC_1B$ yra statusis, tai $A_2C_1 =
  %\frac{CB}{2}$.  Panašiai ir $A_1B_2 = \frac{AC}{2}$ ir
  %$B_1C_2 = \frac{AB}{2}$. Taigi iš šių atkarpų galima
  %sudėti trikampį, dvigubai mažesnį už $ABC$.
\item Trikampis $A_1A_2A_3$ įbrėžtas į apskritimą su
  spinduliu 2. Įrodykite, kad ant lankų $A_1A_2$, $A_2A_3$,
  $A_3A_1$ galima atitinkamai paimti taškus $B_1$, $B_2$,
  $B_3$ taip, kad šešiakampio $A_1B_1A_2B_2A_3B_3$ ploto
  skaitinė vertė būtų lygi trikampio $A_1A_2A_3$ perimetro
  skaitinei vertei.
  %Tegu apskritimo centras yra $O$. Jeigu pažymėsime lanko
  %$A_1A_2$ vidurio tašką raide $B_1$, kitus taškus
  %panašiai, tai šešiakampio $A_1B_1A_2B_2A_3B_3$ ploto
  %skaitinė vertė bus $\frac {R\cdot A_1A_2}{2} + \frac
  %{R\cdot A_1A_3}{2} + \frac {R\cdot A_3A_2}{2} = A_1A_2 +
  %A_2A_3 + A_3A_1$. (Išskaidžius į tris keturkampius
  %$A_1B_1A_2O$, $A_2B_2A_3O$, $A_3B_3A_1O$).
\item Rasti trikampį su kraštinių ilgiais $a, b, c$ ir
  apibrėžto apskritimo spinduliu $R$, kuris tenkintų $R(b+c)
  = a\sqrt{bc}$.
  %Kadangi styga ne ilgesnė už skersmenį, tai $2R \geq a$.
  %Be to, iš AM-GM nelygybės $b + c \geq 2\sqrt{bc}$. Taigi
  %$\frac{R}{a}\geq \frac{1}{2} \geq \frac{\sqrt{bc}}{b+c}$. Pagal
  %sąlygą, abi nelygybės yra lygybės. Taigi $a = 2R$ (todėl
  %trikampis statusis) bei $b = c$ (trikampis lygiašonis).
  %Taigi kraštinių ilgiai yra $2R,\sqrt{2}R, \sqrt{2}R$.
  
\item Trikampio $ABC$ viduje yra du apskritimai, kurių vienas 
  liečia $AB$ ir $BC$, o kitas $AC$ ir $BC$, o abu irgi liečiasi 
  išoriškai. Įrodyti, kad jų spindulių suma didesnė nei
   įbrėžto apskritimo spindulys.
   %Tegu liestinė pirmajam apskritimui, lygiagreti $AC$ ir
   %esanti arčiau jos, kerta $BC$ taške $E$, o liestinė 
   %antrajam, lygiagreti $BA$ ir esanti arčiau jos, kerta
   %$BC$ taške $F$. Tada atkarpos $BE$ ir $CF$ turi 
   %turėti bendrų taškų, arba kitaip du apskritimai negalėtų
   %liestis.Todėl $\frac{r_1}{r_{ABC}}+\frac{r_2}{r_{ABC}}=
   %\frac{BE}{BC}+\frac{CF}{BC}>\frac{BC}{BC}=1$, ko ir reikėjo.

\item Duotas trikampis $ABC$, $AB>BC$. Nubėžtos pusiaukampinės
  $AK$ ir $CM$. Įrodyti, kad $AM>MK>KC$.
  %Kadangi $\angle MCA<\angle MCB+\angle MBC=\angle AMC$, tai 
  %$AM<AC$ ir panašiai $KC<AC$. Tegu $MC$ kerta apie $AKC$ 
  %apibrėžtą apskritimą taške $X$. Įrodysime, kad $AM>MK$. 
  %Tam parodysime, kad $M$ ir $C$ yra toje pačioje $AK$ vidurio
  %statmens pusėje.Kadangi $X$ yra ant $AK$ vidurio statmens
  %vidurio statmens, tai pakanka parodyti, kad $M$ yra ant 
  %atkarpos $XC$(ne ant tęsinio), arba kad $\angle KAM<\angle
   %KAX$. Tas akivaizdu, nes $\angle KAM=\frac{\angle BAC}{2}<
   %\frac{\angle BCA}{2}=\angle KAX$. Panašiai įrodome kitą 
   %nelygybę.

\item Duotas trikampis $ABC$. Tiesė kerta jo 
  kraštines $AB$ ir $BC$ atitinkamai taškuose $M$ ir $K$,
  ir dalina $ABC$ plotą pusiau. Įrodyti, kad 
  $\frac{MB+BK}{AM+CA+CK}>\frac{1}{3}$.
  %Pastebėkime, kad į $MBK$ įbrėžto apskritimo spindulys
  %mažesnis nei į $ABC$ įbrėžto apskritimo spindulys.
  %Taigi $2P_{MBK}=\frac{4S_{MBK}}{r_{MBK}}>\frac{2S_{A
  %BC}}{r_{ABC}}=P_{ABC}$. Todėl $4(MB+BK)>2(MB+BK)+2MK=
  %2P_{MBK}>P_{ABC}=(MB+BK)+(MA+AC+CK)$, iš ko seka rezultatas.

  \end{enumerate}
