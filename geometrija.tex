\section{Kompleksinė geometrija}
\subsection{Teoremos}

Taškus kompleksinėje plokštumoje žymėsime mažosiomis raidėmis.Pavyzdžiui, 
Euklido geometrijos taškus, tokius kaip $A$, $P$ ir $X$ atitiks kompleksniai skaičiai $a$, $p$ ir $x$.

Toliau yra pateiktos dažniausiai naudojamos teoremos, su kurių pagalba
geometriniai brėžiniai bus paverčiami į algebrinius pertvarkius.

\begin{thmnr}\ 
\begin{enumerate}
\item Tiesė ab yra lygiagreti tiesei cd tada ir tik tada, kai $\frac{a - b}{\overline{a} -\overline{b}}=\frac{c - d}{\overline{c} - \overline{d}}$.
\item Tiesė ab yra statmena tiesei cd tada ir tik tada, kai $\frac{a - b}{\overline{a} -\overline{b}}=-\frac{c - d}{\overline{c} - \overline{d}}$.
\item Taškai a, b ir c priklauso vienai tiesei tada ir tik tada, kai $\frac{a - b}{\overline{a} -\overline{b}}=\frac{a-c}{\overline{a} - \overline{c}}$.
\end{enumerate}
\end{thmnr}

\begin{thmnr} Visiems $\bigtriangleup$abc galioja:\
\begin{enumerate}
\item Jei t yra trikampio pusiaukraštinių susikirtimo taškas, tai $t=\frac{a+b+c}{3}$.
\item Jei o yra apibrėžtinio apskritimo centras, o h yra aukštinių susikirtimo taškas, tai $h+2o=a+b+c$.
\item Jei S lygus $\bigtriangleup$ABC plotui, tai 
$S=\tfrac{i}{4}\Big(a\overline{b} +b\overline{c}+c\overline{a} - b\overline{a} -c\overline{b}-a\overline{c}\Big)$
\end{enumerate}
\end{thmnr}

\begin{thmnr}  Jei a, b, c ir d priklauso vienetiniam apskritimui, tada:\
\begin{enumerate}
\item $\frac{a - b}{\overline{a} -\overline{b}}=-ab$, analogiškai su kitomis poromis.
\item Jei x priklauso tiesei ab, tai $\overline{x}=\frac{a + b - x}{ab}$.
\item Jei x yra tiesių ab ir cd susikirtimo taškas, tai $x=\frac{ab(c+d) - cd(a+b)}{ab - cd}$.
\end{enumerate}
\end{thmnr}

\begin{thmnr} Jei a, b ir c priklauso vienetiniam apskritimui, tada:\
\begin{enumerate}
\item Egzistuoja u, v, w tokie kad $u^2$, $v^2$, $w^2$ lygūs a, b, c atitinkamai, o $-uv$, $-vw$, $-wu$ yra $\smile$ab, $\smile$bc, $\smile$ca vidurio taškai, kuriems nepriklauso taškai c, a, b atitinkamai.
\item Jei i yra $\bigtriangleup$abc įbrėžtinio apskritimo centras, tai $i= - (ab + bc + ca)$.
\end{enumerate}
\end{thmnr}

\begin{thmnr} a, b, c ir d priklauso vienam apskritimui, tada ir tik tada, kai: 
$$\frac{a - c}{b - c}:\frac{a-d}{b-d}\in\mathbb{R}.$$
\end{thmnr}

\begin{thmnr} Vienetinis apskritimas yra trikampio abc įbrėžtinis apskritimas, kuris jo kraštines ab, bc, ca liečia taškuose p, q, r, tada:\
\begin{enumerate}
\item Galioja $a=\frac{2qr}{q + r}$, $b=\frac{2pr}{p + r}$, $a=\frac{2qp}{q + p}$.
\item Jei h yra $\bigtriangleup$abc aukštinių susikirtimo taškas, tai
\begin{equation*}
h=\frac{2(p^2q^2 + q^2r^2 + r^2p^2 + pqr(p + q+ r))}{(p+q)(q+r)(r+p)}.
\end{equation*}
\item Jei o yra apibrėžtinio apie  $\bigtriangleup$abc apskritimo centras, tai
\[o=\frac{2pqr(p+q+r)}{(p+q)(q+r)(r+p)}.\]
\end{enumerate}
\end{thmnr}

\subsection{Teoremų įrodymai}
Čia yra pateiktos tik pačios svarbiausios teoremos, 
o likusios pirmojo skyrelio teoremos nesunkiai įsirodo
naudojantis šiomis, pradinėmis.
  
\subsection*{Teoremos 1.1 įrodymas}

Tarkime, kad atkarpa AB su realiųjų skaičių ašimi Re sudaro kampą $\angle X$, tuomet
turime, kad:
$$ e^{\angle x \pi } = \frac {a-b} { Ia-bI }.$$
Iš kompleksinių skaičių algebrinių savybių nesunkiai plaukia, kad:
$$ e^{2 \angle x \pi} = \frac { a-b} { \bar a - \bar b }.$$
Kad tiesė CD būtų lygiagreti tiesei AB, ji su realiųjų skaičių ašimi RE turi sudaryti
arba kampą $\angle X$ arba kampą $ \pi + \angle X$. Abiem atvejais, gauname, kad
$$ \frac{c-d}{\bar c - \bar d } = e^{2 \angle x \pi}.$$
\\
Dar vienas būdas kaip galima lengviau įsivaizduoti kompleksinių skaičių formules
yra suprasti, kad kiekvienas kompleksinis skaičius $z$ kompleksinėje plokštumoje 
yra visiškai apibrėžtas dviem parametrais - savo ilgiu nuo koordinačių centro $IzI$ ir
kampu, kurį sudaro tiesė $Oz$ su realiaja ašimi.
\\
\\
\\
Kompleksiniams skaičiams yra būdinga ši savybė:
dauginant (dalinant) du kompleksinius skaičius jų ilgiai yra sudauginami (padalijami)
kaip realieji skaičiai, o kampai, kuriuos kompleksiniai skaičiai
sudaro su realiaja ašimi, yra sudedami(atimami).
\\
Kai matome išraišką
$$ \frac{a-b}{\bar a - \bar b} $$
suprantame, kad jos rezultatas yra kompleksinis skaičius.
Jo ilgis yra lygus vienetui, nes skaičių $a-b$ ir $\bar a - \bar b$
ilgiai yra vienodi, o dalijant du kompleksinius skaičius jų ilgiai pasidalija.
Taigi, ši kompleksinių skaičių trupmena parodo tik vieną kintamą
dalyką - kampą, kuriuo yra pasvirusi tiesė $ab$, tačiau nieko nesako nei 
apie tai, kur tiesė $ab$ yra, nei kokio ilgio atkarpa $ab$ yra.
\subsection*{Teoremos 1.2 įrodymas}
Ši teorema nesunkiai mintinai išplaukia iš Teoremos 1.1 - jeigu mes turime tris taškus $a, b$ ir $c$ tai tiesės $ac$ ir $ab$ turi būti lygiagriačios (o dar tiksliau, sutampančios, nes abiem tiesėms priklauso taškas $a$).
\subsection*{Teoremos 1.3 įrodymas}

Kaip ir įrodydami teoremą 1.1, tarkime, kad atkarpa AB su realiųjų skaičių ašimi Re sudaro kampą $\angle X$, tuomet turime, kad:
$$ e^{\angle x \pi } = \frac {a-b} { Ia-bI }.$$

Tiesė CD yra statmena statmena tiesei AB tada ir tik
tada, kada tiesė CD su Re ašimi sudaro arba kampą $x+ \frac{\pi}{2}$
arba kampą $x- \frac{\pi}{2}$.

Tuomet 

$$ \frac{c-d}{I c - dI } = e^{(x +- \frac{\pi}{2})i}.$$ 

Arba pakėlus kvadratu turime:

$$ \frac{c-d}{\bar c - \bar d } = e^{(2x +-\pi)i}=e^{2xi} e^{+-\pi i}= - e^{2xi} = - \frac{a-b}{\bar a - \bar b}.$$ 


\subsection*{Teorema 2.3 - trikampio plotui apsirašyti}

Uždaviniuose kartais prireikia surasti, palyginti trikampių plotus.
Formulė, kurią dabar įsirodysime, padės aprašyti pasirinkto trikampio
plotą:

Trikampio $ABC$ plotas yra lygus 
$$ \frac{i}{4}(a\bar{b}+ b \bar{c} + c \bar{a} - b\bar{a} - c \bar{b} - a \bar{c})$$

Įrodymas:

Tegu trikampio $ABC$ aukštinė yra $BD$. Tuomet yra aišku, jog trikampio plotas yra lygus 
$ \frac {1}{2} BD \cdot AC $. Visgi, kompleksinėje plokštumoje yra kiek painiau, nes mes
negalime taip dauginti atkarpų $\frac {1}{2} (a-c)(b-d) $ kadangi rezultatas bus 
kompleksinis skaičius, o plotas turėtų priklausyti realiųjų skaičių aibei. Tad reikia
nagrinėti ne atkarpas, kurios yra kompleksiniai skaičiai, o tų atkarpų modulius. Visgi, 
tokiu atveju reiškiniai pasidaro itin griozdiški, todėl mes panaudosime gudrystę, kuri 
gerokai palengvins mūsų sprendimą:

Žinome, kad $\angle BDC = \frac {\pi}{2}$, todėl, padauginę atkarpą $BD$ iš $i$, pasuksime atkarpą $BD$ taip, kad ji būtų lygiagreti su atkarpa $AC$. Galiausiai, prisiminkime, kad jei norime sudauginti dvi lygiagrečias atkarpas, tai pavertę vieną atkarpą jos jungtine, gausime atkarpų modulių sandaugą. Trupmpiau tariant, turime, kad plotas yra lygus:

$$ S_{ABC} = \frac{1}{2}( a-c)[(b-d)i] $$

Naudodamiesi tuo, kad taškas $D$ priklauso atkarpai $AC$, bei tuo, kad $BD$ yra statmena $AC$, lengvai išsireiškiame tašką $D$:

$$ d =\frac {1}{2} ( a+b + (\bar{b}-\bar{a}) \frac{a-c}{ \bar{a}-\bar{c}}).$$

Įsistatę gautąją $d$ israišką į reiškinį $ S_{ABC} = \frac{1}{2}( a-c)[(b-d)i] $ bei atlikę 
nesudėtingus prastinimo veiksmus, gauname, kad
$$ \frac {i}{4}(a\bar{b}+ b \bar{c} + c \bar{a} - b\bar{a} - c \bar{b} - a \bar{c}), $$ 
o tą mums ir reikėjo įrodyti.  



\subsection{Kompleksinė geometrija and vienetinio apskritimo}

 $~$ Šis uždavinių tipas yra, turbūt, pats paprasčiausias. Taip yra dėl to, jog gaunamos 
išraiškos yra sąlyginai paprastos, nes mes patogiai išnaudojame koordinačių centro parinkimą - visų taškų $x$ ant vienetinio apskritimo jungtiniai yra lygūs $\frac {1}{x}$. 
  
  
  Kokie uždaviniai priklauso šiam skyriui? Ogi tie, kuriuose visi (arba beveik visi) svarbūs taškai priklauso apie pradinį trikampį apibrėžtam apskritimui arba yra to apskritimo stygų bei liestinių susikirtimo taškai. 
  
  Spręsdami kompleksinių skaičių metodu mes tarsime, kad visi geometriniai taškai yra tam tikri kompleksiniai skaičiai koordinačių plokštumoje su dviem ašimis - realiąja ir numanomąja. Toliau, mes surasime pradinius taškus (dažniausiai tai bus pradinis trikampis $\Delta abc$, tačiau kartais pradiniais laikysime ir daugiau taškų) ir stengsimės kiekvieną naują sąlygoje figūruojantį tašką išreikšti per pradinius kompleksinius taškus, naudodamiesi pirmame skyrelyje pateiktomis formulėmis.
  
  Kad pasidarytų aiškiau, keliaukime prie uždavinių ir pažiūrėkime, kaip šis metodas veikia praktikoje.
\\
\\
PS: Sripresniems matematikams yra rekomenduojama skirti ypatingą dėmesį šviežiems
pasaulinės olimpiados uždaviniams, kurie yra sprendžiami skyrelių pabaigose. Šie
uždaviniai tik parodo, koks naudingas yra kompleksinių skaičių geometrijoje metodas
ir kad šiuo metodu galima išspresti didžiąją dalį IMO ir kitų garsių olimpiadų uždavinių.

\begin{pavnr}

Apie trikampį ABC apibrežto apskritimo skersmuo tegu bus AD, H - aukštinių susikirtimo
taškas (dar žinomas kaip ortocentras), o M - kraštines BC vidurio taškas. Mūsų yra prašoma įrodyti, jog taškas M yra atkarpos HD vidurys.
\end{pavnr}
\begin{sprendimas}
\\ $ \phantom{a}$ Tegu vienetinis apskritimas būna apibrėžtas apie $ABC$ apskritimas. 
\\ $ \phantom{a}$ Pagal teoremą 2.2 mes turime, kad
 $$ h = a+b+c.$$
 \\ $~$Kadangi M yra BC vidurio taškas, tai; $$m= \frac{b+c}{2}.$$
$ ~$   Kadangi skersmens vidurio taškas yra koordinačių pradžios taškas ir jis yra lygus nuliui, tai taškas $d$  apsirašo taip: $$d= -a.$$
$ ~$   Galiausiai mūsų laukia visai paprastutis aritmetinis veiksmas - surasti $DH$ vidurio tašką, dar žinomą $ \frac{d+h}{2} =$(įsistatome)$= \frac {(-a)+a+b+c}{2}= \frac{b+c}{2}$. O, juk tai yra mūsų jau surastas taškas $M$, būtent tai, ką mums ir reikėjo įrodyti!
\end{sprendimas}


\begin{pavnr}
USAMO'2010 Sąlyga
\\
   \\Duotas iškilas apibrėžtinis penkiakampis AXYZB toks, kad AB yra apibrėžtojo apskritimo
   skersmuo. Pažymėkime P, Q, R, S statmenimis iš taško Y tiesėms  AX, BX, AZ, BZ atitinkamai.
   Kad įveiktumėte šį uždavinį, Jūs turite įrodyti, kad smailusis kampas tarp tiesių PQ ir RS
   yra dvigubai mažesnis nei kampas $ \angle XOZ$.
\end{pavnr}
\begin{sprendimas}
  \\ $\phantom{a}$Žinoma, tarsime, kad apie penkiakampį apibrėžtasis apskritimas yra vienetinis.
      Tuomet $a = -b$, nes $ab$ yra skersmuo.
  \\ $\phantom{a}$
Suraskime taškus $p, q, r, s$ naudodamiesi tuo, kad visi šie taškai yra statmenys iš taškų ant vienetinio apskritimo
ant to apskritimo stygų (teorema 3.2):
     $$p = \frac {1}{2}( a+x+y- \frac {ax}{y}),$$
     $$q = \frac {1}{2}( a+z+y- \frac {az}{y}),$$ 
     $$r = \frac {1}{2}( -a+x+y- \frac {ax}{y}),$$
     $$s = \frac {1}{2}( -a+z+y- \frac {az}{y}).$$
  \\ $\phantom{a}$ Galiausiai lieka apdoroti sąlygą, ką mums reikia įrodyti. Iš tiesų yra
      ganėtinai sudėtinga tvarkyti kompleksiniais skaičiais kampų lygybes, 
      kai kampai yra ne lygūs, o keletą kartų didesni vienas už kitą.
      Šiuo atveju, nesunkus ir labai visus skaičiavimus pagražinantis būdas
      pasilengvinti prastinimą yra Euklidinės geometrijos panaudojimas: pagal
      įbrėžtinius kampus turime, kad $ \angle XOZ = 2 \angle XAZ$, todėl,
      mums lieka įrodyti, jog  $ \angle XAZ $ yra lygus kampui tarp tiesių $PQ$ ir
      $RS$. Šis kampų sulyginimas yra ekvivalentus kompleksiniuose skaičiuose šiai lygybei, kurią reikia įrodyti: 
$$\frac {x-a} {\bar x - \bar a} : \frac{z-a}{\bar z - \bar a} = \frac{p-r}{\bar p - \bar r} : \frac{q-s}{\bar q - \bar s}$$
  \\ $\phantom{a}$Įsistačius vietoje $p, q, r, s$ žinomas išraiškas per $a, x, y, z,$ gauname, kad 
   lygybė teisinga, todėl uždavinys išspęstas.
     \end{sprendimas}
     
     \begin{pavnr}
Sąlyga (Simediana)
\\ Duotas apie smailųjį trikampį ABC apibrėžtas apskritimas, kurio
liestinės iš taškų A ir B susikerta taške X. Jei M yra kraštinės 
AB vidurio taškas, įrodykite, kad $\angle ACX = \angle BCM$.
\end{pavnr}
\begin{sprendimas}
\\ $\phantom{a}$ Pagal teoremą 3.3 turime, kad $ x= \frac {2ab}{a+b}$.
\\$\phantom{a}$ Toliau mes išsireikšime kampus:
\\
\\ $$ e^{ 2\angle ACX i} = \frac { a-c}{ \bar a - \bar c} : \frac {c-x}{\bar c - \bar x}=
- \frac { 2ab -ac-bc}{2bc - ab- b^2}.$$
\\
\\$\phantom{a}$ Analogiškai apibrėžiame kampą $ \angle BCM $, kur $m = \frac {a+b}{2}$:
\\
\\  $$ e^{2 \angle BCM i} = \frac { m-c}{ \bar m - \bar c} : \frac {c-b}{\bar c - \bar b}=
- \frac { 2ab -ac-bc}{2bc - ab- b^2}.$$
\\
\\$\phantom{a}$ Taigi, mes gauname, kad  $ e^{ 2 \angle ACX i} = e^{2 \angle BCM i}$, iš ko seka, kad arba $ \angle ACX = \pi + \angle BMC$, kas yra neįmanoma, nes $\Delta abc$ yra smaiusis trikampis, arba $\angle ACX = \angle BCM$, ką mums ir reikėjo įrodyti.
\end{sprendimas}

\begin{pavnr}
IMO 2009

Tegul O yra apie trikampį ABC apibrėžto apskritimo
centras. Taškai P ir Q atitinkamai yra atkarpų CA ir AB vidiniai
taškai. Tegul $\Gamma$ yra apskritimas, einantis per atkarpų BP, CQ ir
PQ vidurio taškus K, L ir M, o tiesė PQ yra apskritimo $\Gamma$ liestinė.
Įrodykite, kad OP = OQ.
\end{pavnr}
\begin{sprendimas}

Tegu apie $\bigtriangleup$abc apibrėžtas apskritimas yra vienetinis. Kadangi p ir q priklauso ca ir ab, atitinkamai, tai
\begin{equation*}
\overline{p}=\tfrac{a+c-p}{ac},
\end{equation*}
\begin{equation*}
\overline{q}=\tfrac{a+b-q}{ab}.
\end{equation*}
Mums reikia įrodyti:
\begin{equation*}
(p-o)(\overline{p-o})=(q-o)(\overline{q-o}) \Leftrightarrow p\overline{p}=q\overline{q} \Leftrightarrow \tfrac{p(a+c-p)}{ac}=\tfrac{q(a+b-q)}{ab}.
\end{equation*}

Liko neišnaudota viena sąlyga - tiesė pq liečia apskritimą $\Gamma$. Galima bandyti susirasti apskritimo centrą ir įrodyti kad statmuo iš m tiesei pq eina per $\Gamma$ centrą, arba naudotis tuom, kad $\angle qmk=\angle mlk$ tada ir tik tada, kai tiesė pq yra $\Gamma$ liestinė taške m. Pasižymime $\omega=\angle qmk=\angle mlk$, tada:
\begin{equation*}
\tfrac{m-q}{|m-q|}\omega=\tfrac{m-k}{|m-k|} \Rightarrow \tfrac{m-q}{\overline{m}-\overline{q}}\omega^2=\tfrac{m-k}{\overline{m}-\overline{k}} \Leftrightarrow \omega^2=\tfrac{m-k}{\overline{m}-\overline{k}} \tfrac{\overline{m}-\overline{q}}{m-q}.
\end{equation*}
Analogiškai ir su m, l, k:
\begin{equation*}
\tfrac{l-m}{|l-m|}\omega=\tfrac{l-k}{|l-k|} \Rightarrow  \omega^2=\tfrac{l-m}{\overline{l}-\overline{m}}\tfrac{\overline{l}-\overline{k}}{l-k}
\end{equation*}
Sulyginame reiškinius, įsistatome m, k ir l reikšmes, tada p ir q jungtinius:

\begin{equation*}
 \tfrac{m-k}{\overline{m}-\overline{k}} \tfrac{\overline{m}-\overline{q}}{m-q}=\tfrac{l-m}{\overline{l}-\overline{m}}\tfrac{\overline{l}-\overline{k}}{l-k} \Leftrightarrow 
\end{equation*}
\begin{equation*}
 \Leftrightarrow (q-b)(\overline{p}-\overline{q})(c-p)(\overline{c}+\overline{q}-\overline{b}-\overline{p})=(p-q)(\overline{q}-\overline{b})(c+q-b-p)(\overline{c}-\overline{p}) \Leftrightarrow
\end{equation*}
\begin{equation*}
\Leftrightarrow (q-b)(\tfrac{ab-ac+qc-pb}{abc})(c-p)(\tfrac{pb-qc}{abc})=(p-q)(\tfrac{b-q}{ab})(\tfrac{p-c}{ac})(c+q-b-p),
\end{equation*}
gautą lygybę galima nesunkiai suprastinti ir pertvarkyti į reiškinį, kurį mums reikėjo įrodyti.
\end{sprendimas}

\subsection{Komplikuotesni uždaviniai su apskritimais}

Iš tiesų, šio skyriaus uždaviniai mažai skiriasi nuo praeitojo, nes mes vėl turime vieną apskritimą, aplink kurį sukasi visas uždavinio veiksmas. Visgi, formulės, kurias čia naudosime yra kiek komplikuotesnės. Vienas iš variantų, kas pasikeičia, gali būti tai,  
jog vienetinis apskritimas tampa į pradinį trikampį įbrėžtu apskritimu, arba mes naudojamės
teoremos 4 savybėmis ir pradiniais taškais paimame $u^2$,$v^2$ ir $w^2$, kas leidžia
pakankamai gražiai surasti vienetinio apskritimo vidurio lankus ir pusiaukampinių susikirtimo tašką. Vėlgi, visą šią abstrakčią litaniją geriausiai atspindi uždavinių 
sprendimo pavyzdžiai.

\begin{pavnr}
(Litmo 2005)Apie trikampį ABC apibrėžtas apskritimas. Taškas M yra lanko AC (kuriam nepriklauso 
viršūnė B) vidurio taškas, o N yra lanko AB (kuriam nepriklauso viršūnė C) vidurio taškas. 
Atkarpos MN ir AB kertasi taške K. Įbrėžto į trikampį ABC apskritimo centras yra O. 
Įrodykite, kad KO yra lygiagreti kraštinei AC. 
\end{pavnr}
\begin{sprendimas}
  \\Tarsime, kad apie ABC apibrėžtasis apskritimas yra vienetinis. Šiame uždavinyje mums
reikia gudriai apsirašyti taškus M, N ir O, todėl mes taikysime teoremą 4:
\\
   $$a = x^2,$$
   $$b=y^2,$$
   $$c = z^2,$$
   $$ o = - (xy+xz+yz),$$
   $$m = - xz,$$
   $$ n = - xy.$$
    \\
  \\ Kadangi k yra stygų ab ir mn susikirtimo taškas, tai tašką k surasime pagal teoremą 3.3:
          $$k = \frac{ x^2yz(x^2+y^2)+ x^2 y^2 (xy+xz)}{ x^2 yz - x^2 y^2},$$
  \\ $\phantom{a}$ Galiausiai lieka įrodyti, kad ko yra statmena ac pagal teoremą 1.2:
     $$\frac{k-o}{\bar k - \bar o}  =  x^2 z^2 = - \frac{ x^2 - z^2}{\bar x^2 - \bar z^2},$$
\\ Taigi, gavome būtent tai, ką ir reikėjo įrodyti!
\end{sprendimas}



\begin{pavnr}
  \ $\phantom{a}$  Į trikampį ABC įbrėžtas apskritimas su centru I liečia kraštines BC, AC, AB taškuose D, E ir F atitinkamai.
    Tegu M ir N yra atkarpų AB ir BC vidurio taškai, o X yra taškas, kur susikerta tiesės NM ir DF.
   Reikia įroryti, kad  (a) tiesės IC, NM, FD eina per vieną tašką, o didesni adrenalino fanatikai galės pabandyti dar ir
    įrodyti, jog galioja lygybė (b) $\angle AXC = \frac{\pi}{2}$. 
\end{pavnr}

\begin{sprendimas}
\\$\phantom{a}$Pradėsime nuo (a) dalies. Vienetinis apskritimas bus įbrėžtinis į $\Delta ABC$ apskritimas, todėl mes stengsimės visus taškus išreikšti per $d$, $e$ ir $f$.
\\  $\phantom{a}$Tarkime, kad $X$ yra taškas, kur kertasi $DF$ ir $IC$. Tuomet užduotis prašo mūsų įrodyti, kad $X$ priklauso tiesei $NM$.
        Aprašysime duotas sąlygas:
            \\$\phantom{a}$Trikampio kraštinės yra įbrėžto trikampio liestinės, todėl turime, jog:
                    $$ a =\frac{2ef}{e+f},$$
                     $$ b=\frac{2df}{d+f},$$
                    $$ c=\frac{2ed}{e+d}.$$
            \\ $\phantom{a}$Svarbiausias ir patogiausias dalykas, kurį mums duoda vienetinio apskritimo parinkimas, yra tai, jog
            galioja lygybės: $\bar{d}= \frac{1}{d}$,$\bar{e}= \frac{1}{e}$,$\bar{f}= \frac{1}{f}$. Šių 
            lygybių teisingumas plaukia iš paties kompleksinių skaičių jungtinių apibrėžimo bei fakto, kad visų kompleksinių 
            skaičių ant vienetinio apskritimo ilgiai yra lygūs $1$.\\
            \\ $\phantom{a}$Iš to seka, kad 
                  $$ \bar a =\frac{2}{e+f},$$
                  $$ \bar b=\frac{2}{d+f},$$
                  $$ \bar c=\frac{2}{e+d}.$$
           \\$\phantom{a}$Naudojamės vektorių savybėmis kompleksiniams skaičiams ir gauname, jog
                  $$ m=\frac{a+b}{2}, \phantom{a} n=\frac{c+b}{2}.$$
            \\$\phantom{a}$Taškas $X$ priklauso tiesei $IC$:
                     $$ \frac{c-0}{\bar{c}-0}=\frac{x-0}{\bar{x} - 0} {\phantom{a}}\leftrightarrow \phantom{a} \bar{x}= \frac{\bar{c}}{c}\cdot  x $$ 
           \\ $\phantom{a}$Taškas $X$ taip pat priklauso tiesei $FD$, todėl turime, kad:
                  \\ $$\frac{f-d}{\bar{f}-\bar{d}}=\frac{x-f}{\bar{x} - \bar{f}}  {\phantom{a}}\leftrightarrow \phantom{a} \bar{x}= \frac{ \bar{f} - \bar{d} }{f-d}\cdot  (x-f)+\bar{f} $$
      \\$\phantom{a}$Sulyginame gautas dvejas $\bar{x}$ išraiškas ir gauname, kad 
                   $$ x = e \cdot \frac{d+e}{f+e}$$
      \\Lieka parodyti, jog gautas taškas $X$ priklauso atkarpai $NM$, o tai yra ekvivalentu šiai lygybei:
          $$ \frac{m-n}{\bar{m}-\bar{n}}=\frac{x-n}{\bar{x} - \bar{n}}$$
\\$\phantom{a}$ O tai mes gauname įsistatę jau turimas taškų išraiškas $m$, $n$ ir $x$ per taškus $d, e$ ir $f$.
\\
\\$\phantom{a}$ Keliaujame prie (b) dalies: ši dalis yra sunki tik tuo atveju, jei spręstume uždavinį
besinaudodami klasikine Euklidine geometrija. Sprendžiant kompleksinių skaičių metodu, (b) dalis pasidaro
juokingai paprasta, ypač turint iš (a) dalies visų reikalingų taškų nesudėtingas išraiškas. Tereikia įrodyti, jog
galioja lygybė: 
 $\frac{a-x}{\bar{a}-\bar{x}}= - \frac{x-c}{\bar{x} - \bar{c}}$ , kas taip pat tiesiogiai plaukia įsistačius turimas išraiškas ir šiek tiek praprastinus gautas trupmenas.
\\ Štai ir įveiktas šis uždavinys!
\end{sprendimas}

\begin{pavnr}
IMO 2010
\\
Tegul I yra trikampio ABC pusiaukampiniu˛ susikirtimo taškas, o $\Gamma$ – apie trikampi˛
ABC apibrėžtas apskritimas. Tiesė AI kerta $\Gamma$ taške D, kur $D\ne A$. Taškas E priklauso apskritimo
$\Gamma$ lankui $\smile$BDC, o taškas F – atkarpai BC. Be to,
$\angle BAF = \angle CAE < \tfrac{1}{2}\angle BAC$.
Taškas G yra atkarpos IF vidurio taškas. Įrodykite, kad tiesių DG ir EI susikirtimo taškas priklauso
apskritimui $\Gamma$.
\end{pavnr}

\begin{sprendimas}

Tegu $\Gamma$ yra vienetinis apskritimas kompleksinėje plokštumoje. Naudojamės 4.2 teorema: tegu $a=u^2$,$b=v^2$,$c=w^2$, tada 
\begin{equation*}
i = -(uv+vw+wu).
\end{equation*}

Tegu $af$ kerta $\Gamma$ taške $h\ne a$. $\angle bah =\angle eac =\alpha$, pažymime $\phi=e^{i\alpha}$. Tada $\frac{h-a}{|h-a|}\phi=\frac{b-a}{|b-a|}$ Pakėlę kvadratu ir pasinaudoję 3.1 teorema gauname $h=\frac{v^2}{\phi^2}$. Analogiškai randame 
\begin{equation*}
e=\phi^2w^2.
\end{equation*}

Tegu $ei$ kerta $\Gamma$ taške $x \ne e$. Iš 1.1 teoremos gauname $\frac{e-x}{\overline{e}-\overline{x}}=\frac{e-i}{\overline{e}-\overline{i}}$ ir kadangi $x\overline{x}=1$, tai 
\begin{equation*}
x=-uvw\tfrac{w^2\phi ^2 +uv+vw+wu}{uvw +w^2\phi ^2(u+v+w)}.
\end{equation*}

Prisiminę teoremą 3.3, randame 
\begin{equation*}
f=\tfrac{v^2w^2(u^2 + \frac{v^2}{\phi ^2}) - \frac{u^2v^2}{\phi ^2}(v^2+w^2)}{v^2w^2 - \frac{u^2v^2}{\phi ^2}}=\tfrac{w^2(u^2\phi^2 + v^2) - u^2(v^2+w^2)}{w^2\phi^2 - u^2}.
\end{equation*}

Taškas $d=-vw$ pagal 4.1 teoremą, nes ai yra trikampio abc pusiaukampinė.

Galiausiai $g=\frac{1}{2}(f+i)$ ir pagal 1.1 teoremą įrodome, kad $ d, g$ ir $x$ priklauso vienai tiesei.
\end{sprendimas}
\begin{pavnr}
IMO 2009

Tegul O yra apie trikampį ABC apibrėžto apskritimo
centras. Taškai P ir Q atitinkamai yra atkarpų CA ir AB vidiniai
taškai. Tegul $\Gamma$ yra apskritimas, einantis per atkarpų BP, CQ ir
PQ vidurio taškus K, L ir M, o tiesė PQ yra apskritimo $\Gamma$ liestinė.
Įrodykite, kad OP = OQ.
\end{pavnr}
\begin{sprendimas}

Tegu apie $\bigtriangleup$abc apibrėžtas apskritimas yra vienetinis. Kadangi p ir q priklauso ca ir ab, atitinkamai, tai
\begin{equation*}
\overline{p}=\tfrac{a+c-p}{ac},
\end{equation*}
\begin{equation*}
\overline{q}=\tfrac{a+b-q}{ab}.
\end{equation*}
Mums reikia įrodyti:
\begin{equation*}
(p-o)(\overline{p-o})=(q-o)(\overline{q-o}) \Leftrightarrow p\overline{p}=q\overline{q} \Leftrightarrow \tfrac{p(a+c-p)}{ac}=\tfrac{q(a+b-q)}{ab}.
\end{equation*}

Liko neišnaudota viena sąlyga - tiesė pq liečia apskritimą $\Gamma$. Galima bandyti susirasti apskritimo centrą ir įrodyti kad statmuo iš m tiesei pq eina per $\Gamma$ centrą, arba naudotis tuom, kad $\angle qmk=\angle mlk$ tada ir tik tada, kai tiesė pq yra $\Gamma$ liestinė taške m. Pasižymime $\omega=\angle qmk=\angle mlk$, tada:
\begin{equation*}
\tfrac{m-q}{|m-q|}\omega=\tfrac{m-k}{|m-k|} \Rightarrow \tfrac{m-q}{\overline{m}-\overline{q}}\omega^2=\tfrac{m-k}{\overline{m}-\overline{k}} \Leftrightarrow \omega^2=\tfrac{m-k}{\overline{m}-\overline{k}} \tfrac{\overline{m}-\overline{q}}{m-q}.
\end{equation*}
Analogiškai ir su m, l, k:
\begin{equation*}
\tfrac{l-m}{|l-m|}\omega=\tfrac{l-k}{|l-k|} \Rightarrow  \omega^2=\tfrac{l-m}{\overline{l}-\overline{m}}\tfrac{\overline{l}-\overline{k}}{l-k}
\end{equation*}
Sulyginame reiškinius, įsistatome m, k ir l reikšmes, tada p ir q jungtinius:

\begin{equation*}
 \tfrac{m-k}{\overline{m}-\overline{k}} \tfrac{\overline{m}-\overline{q}}{m-q}=\tfrac{l-m}{\overline{l}-\overline{m}}\tfrac{\overline{l}-\overline{k}}{l-k} \Leftrightarrow 
\end{equation*}
\begin{equation*}
 \Leftrightarrow (q-b)(\overline{p}-\overline{q})(c-p)(\overline{c}+\overline{q}-\overline{b}-\overline{p})=(p-q)(\overline{q}-\overline{b})(c+q-b-p)(\overline{c}-\overline{p}) \Leftrightarrow
\end{equation*}
\begin{equation*}
\Leftrightarrow (q-b)(\tfrac{ab-ac+qc-pb}{abc})(c-p)(\tfrac{pb-qc}{abc})=(p-q)(\tfrac{b-q}{ab})(\tfrac{p-c}{ac})(c+q-b-p),
\end{equation*}
gautą lygybę galima nesunkiai suprastinti ir pertvarkyti į reiškinį, kurį mums reikėjo įrodyti.
\end{sprendimas}

\subsection{Uždaviniai, kuriuose yra ieškoma ploto}

Šiame skyriuje atkreipiame Jūsų dėmesį į teoremą 2.3. Neturint šios teoremos
mums reikėtų gerokai paprakaituoti, sunkant galvą, kaip rasti kompleksinėje 
plokštumoje realiaisiais vienetais matuojamą plotą. Smalsesniems skaitytojams, 
norintiems sužinoti, kaip
buvo sugalvota šį plotą surasti, rekomenduojama grįžti į antrą skyrelį, kur yra
pateiktas teoremos įrodymas, o mes keliaujame toliau, prie uždavinių pavyzdžių.


\begin{pavnr}
(Litmo 2008)Apie smailųjį trikampį ABC apibrėžtas apskritimas. Atkarpa BD yra to
apskritimo skersmuo. Iš viršūnės A nubrėžta aukštinė kerta apskritimą
taške E. Įrodykite, kad keturkampio BECD plotas yra lygus trikampio
ABC plotui.
\end{pavnr}
\begin{sprendimas}
\\   $\phantom{a}$Šiame uždavinyje mums labai pravers plotų formulė.
\\   $\phantom{a}$Pasiėmę vienetiniu apskritimu apie ABC apibrėžtą apskritimą, gauname, kad
$d = -b$ bei iš teoremos 3.1:
 $$ e= \frac {bc}{a}= {bc \bar a}.$$
 Lieka pagal teoremą  2.3 apsirašyti duotuosius plotus:
$$ S_{ABC} = \frac {i}{4}(a\bar b + b \bar c + c \bar a - \bar a b - \bar b c - \bar c a),$$
$\phantom{aaa}$ $ S_{BECD}=S_{BEC}+S_{BCD}= \\  \frac {i}{4}(b\bar e + e \bar c + c \bar b - \bar b e - \bar e c - \bar c b)+ 
\frac {i}{4}(b\bar c + c \bar d + d \bar b - \bar b c - \bar c d - \bar d b) = \\   \frac {i}{4}(b\bar {bc}a + {bc \bar a}
 \bar c + c \bar b - \bar b {bc \bar a} - \bar {bc} a c - \bar c b)+ 
\frac {i}{4}(b\bar c + c \bar( -b) + (-b) \bar b - \bar b c - \bar c (-b) - (\bar {-b}) b) =  
\frac {i}{4}(a\bar b + b \bar c + c \bar a - \bar a b - \bar b c - \bar c a) = S_{ABC}.$
\\
\\ $\phantom{a}$Štai ir uždavinys nugriautas!
\end{sprendimas}

\begin{pavnr}

 IMO 2007

Trikampio ABC kampo BCA pusiaukampinė kerta apibrėžtą apie ABC apskritimą kitame taške R. Tarkime, kad K yra atkarpos BC vidurio taškas, o L yra atkarpos AC vidurio taškas. Tiesė, kuri eina per tašką K ir yra statmena atkarpai BC, kerta tiesę CR taške P, o tiesė, kuri eina per tašką L ir yra statmena atkarpai AC, kerta tiesę CR taške Q. Įrodykite, kad trikampių RPK ir RQL plotai yra lygūs.

\end{pavnr}
\begin{sprendimas}


Naudosimės 4.1 teorema: tarsime, kad $a=u^2$, $b=v^2$ ir $c=w^2$. Tada $r=-uv$, o $l=\tfrac{u^2+w^2}{2}$. Pasinaudoję tuo, kad q priklauso stygai rc ir tuo, kad ql statmena ac, randame $q=\tfrac{a(c^2 - ab)}{(a-b)}$. Pažymime: S - trikampio rql plotas, o $t=(r\overline{q} +q\overline{l}+l\overline{r})$, tada $S=\tfrac{i}{4}(t-\overline{t})$. Randame:
\begin{equation*}
t=\tfrac{c^4(b-2a) +c^2(3a^2b-2ab^2+b^3) +a^2b^3 - 2a^3b^2}{2abc^2(a-b)}
\end{equation*}
\begin{equation*}
-\overline{t}=\tfrac{c^4(a-2b) +c^2(3b^2a-2ba^2+a^3) +b^2a^3 - 2b^3a^2}{2abc^2(a-b)},
\end{equation*}gauname:

\begin{equation*}
S=\tfrac{i}{4}(t-\overline{t})=\tfrac{i}{4}\tfrac{-(a+b)c^4+(a^3+b^3+a^2b+b^2a)c^2-(a^2b^3+a^3b^2)}{2abc^2(a-b)}
\end{equation*}

Matome, kad gauta ploto išraiška yra simetriška a ir b atžvilgiu (t.y. išraiškoje sukeitę a su b vietomis gausime tokį pat plotą), todėl $\triangle$pkr plotas bus lygus S.
\end{sprendimas}

\subsection{Įvairiausių uždavinių sprendimas}

Šiame skyriuje bus taikomos visos pirmojo skyrelio teoremos. Uždavinių sprendimai taps
itin įmantrūs, nes bus išnaudojamos pačios sudėtingiausios savybės - sprendžiama
su konkrečiais kampais, tokiais kaip $\frac \pi 4$ ar $\frac \pi 3$, uždaviniuose
dings pradinis vienetinis apskritimas ir teks verstis be jo, arba, dar įdomiau, 
išdygs keli apskritimai su kuriais teks skaitytis!

Prižadėję visą gausybę naujovių skubame prie pavyzdžių, parodysiančių, kad
kompleksinių skaičių metodas reikalauja iš matematiko nemažai išradingumo.


\begin{pavnr} 
 MEMO 2010 
\\
Duotas keturkampis ABCD, apie kurį galima apibrėžti apskritimą. E yra toks įstrižainės AC
taškas, kad AD = AE ir CB = CE. M yra apskritimo k, apibrėžto apie trikampį BDE,
centras. Apskritimas k kerta tiesę AC taškuose E ir F . Įrodykite, kad tiesės FM, AD ir BC
kertasi viename taške.
\end{pavnr}
\begin{sprendimas}


Tegu apie $b, d, e$ apibrėžtas apskritimas yra vienetinis apskritimas kompleksinėje plokštumoje, tada $f$ irgi priklauso šiam apskritimui, o $m=0$. Kadangi $a\in ef$ iš 3.2 teoremos gauname: $\overline{a}=\frac{e+f-a}{ef}$. $ae = ad$, tai $|a-e|=|a-d| \Leftrightarrow (a-e)\overline{(a-e)}=(a-d)\overline{(a-d)}$. Iš šių lygčių išsireiškiame a:

\begin{equation*}
a=\frac{d(e+f)}{d+f}.
\end{equation*}

Analogiškai randame $c$:
\begin{equation*}
c=\frac{b(e+f)}{b+f}.
\end{equation*}

Kadangi $a,b,c,d$ priklauso vienam apskritimui, tai:
\begin{equation*}
\frac{a-c}{b-c}:\frac{a-d}{b-d}\in \mathbb{R} \Leftrightarrow \frac{a-c}{b-c}:\frac{a-d}{b-d}= \overline{\bigg(\frac{a-c}{b-c}:\frac{a-d}{b-d}\bigg)} \Leftrightarrow bed=f^3.
\end{equation*}

Tarkime, kad $ fm$ ir $ad$ kertasi taške $x_1$, o $fm$ ir $bc$ taške $x_2$. Kadangi $x_1, x_2 \in fm$, tai iš 1.1 teoremos gauname $\frac{x_i-m}{(\overline{x_i}-\overline{m})}=\frac{f-m}{(\overline{f}-\overline{m})} \Leftrightarrow          \overline{x_i}=\frac{x_i}{f^2}$ ($i \in \{1, 2\})$.
Kadangi $x_1 \in ad$, tai iš 1.1 teoremos gauname $\frac{x_1-d}{(\overline{x_1}-\overline{d})}=\frac{a-d}{(\overline{a}-\overline{d})} \Rightarrow x_1=\frac{df^2(e+f)}{(f^3+d^2e)}$,
analogiškai $x_2=\frac{bf^2(e+f)}{(f^3+b^2e)}$.
Iš čia $x_1=x_2 \Rightarrow bed=f^3$
\end{sprendimas}
\begin{pavnr}

IMO 2009

Tegul ABC yra lygiašonis trikampis, kuriame AB =
AC. Kampo CAB pusiaukampinė kerta kraštinę BC taške D, o kampo
ABC pusiaukampinė kerta kraštine CA taške E. Taškas K yra įbrėžto
į trikampi ADC apskritimo centras, o $\angle$BEK = $45^0$. Raskite visas
įmanomas $\angle$CAB reikšmes.
\end{pavnr}
\begin{sprendimas}

Spresdami geometrinius uždavinius kompleksiniais skaičiais kartais turime performuluoti sąlygą taip, kad sprendžiant gautume paprastesnius reiškinius ir lygybės netaptų pernelyg komplikuotos. Pasižymime tiesių ad ir ck susikirtimo tašką g, o $\angle acd = 2\alpha$. Nesunku įsitikinti susižymėjus kampus, kad $AB=A$C tada ir tik tada, kai $\angle kec=3(45^0-\alpha)$. Todėl spręsdami uždavinį tašką b galime pamiršti, o suradę $\alpha$ reikšmes, nesunkiai surasime ir ieškomojo kampo reikšmes.

Tegu įbrėžtinis į adc apskritimas yra vienetinis ir liečia kraštines da, ac, cd  taškuose p, q ir r. Pasižymime $\phi=e^{i45^0}$ ir $\omega=e^{i\alpha}$. Tada $r=p\phi^2$, nes $\angle pkr=90^0$.

Kadangi $g\in ad$, tai $\overline{g}=\tfrac{2p-g}{p^2}$, ir $g\in ck$, $\tfrac{c}{\overline{c}}=\tfrac{g}{\overline{g}}$, tai
\begin{equation*}
g=\tfrac{2pq\phi ^2}{p+q\phi^2}.
\end{equation*}

Kadangi $e\in ac$, tai $\overline{e}=\tfrac{2q-e}{q^2}$ ir $\angle kec=45^0 \Rightarrow  \tfrac{e}{\overline{e}}\tfrac{\phi^6}{\omega^6}= \tfrac{e-q}{\overline{e}-\overline{q}}$ ir $\phi^4=-1$, tai
\begin{equation*}
e=\tfrac{2q\omega^6}{\omega^6 +\phi ^2}.
\end{equation*}

\begin{equation*}
\angle gek = 45^0 \Rightarrow \tfrac{e-g}{\overline{e}-\overline{g}}\phi^2=\tfrac{e}{\overline{e}}.\tag{*}
\end{equation*}

Apskaičiuojame:
\begin{equation*}
\tfrac{e}{\overline{e}}= \tfrac{\omega^6q^2}{\phi^2} 
\end{equation*}
\begin{equation*}
\tfrac{1}{2}(e-g)=\tfrac{q(q\omega^6\phi^2 +p\omega^6 -p\omega^6\phi^2 +p)}{(\omega^6+\phi^2)(p +q\phi^2)}
\end{equation*}
\begin{equation*}
\tfrac{1}{2}(\overline{e}-\overline{g})=\tfrac{p\phi^2 -q\omega^6 -q\phi^2 - q}{q(\omega^6+\phi^2)(p +q\phi^2)}
\end{equation*}

Įsistatome į (*) ir suprastinę gauname:
\begin{equation*}
(q\omega^6 -p)(\omega^6+1)=0. \tag{**}
\end{equation*}

Išsprendę $(\omega^6+1)=0$ lygtį ir atsižvelgę į tai, kad $ \alpha < 45^0$, gauname vienintelę reikšmę $\alpha = 30$. Tada mūsų ieškomas kampas yra $60^0$.

Antras atvejis, kai $q\omega^6 =p$. Nesunkiai paskaičiuojame (tiesiog susižymėję kampus), kad $\angle pkq=90^0 +2\alpha$, todėl gauname:
\begin{equation*}
\omega^6=e^{i6\alpha}=e^{i(90^0 +2\alpha)} \Leftrightarrow 6\alpha=90^0 +2\alpha +2k\pi \text{ (kažkokiam } k\in \mathbb{Z}),
\end{equation*}
Matome, kad tinka tik viena $\alpha$ reikšmė, kai $4\alpha=90^0$, tai ieškomas kampas irgi lygus $90^0$.

Kadangi iš pradinių sąlygų gavome (**), tai dar nereiškia, kad abu sprendiniai tenkina pradinę sąlygą, todėl turime juos abu įsistatyti ir patikrinti ar tokie trikampiai egzistuoja, tenkinantys visas sąlygas. Galima tarti, kad $\angle cab= 60^0,90^0$ ir parodyti, kad $\angle bek =45^0$.
\end{sprendimas}
\begin{pavnr}
 APMO 2005

Tegu ABC smailusis trikampis, kurio $\angle BAC = 60^0$ ir $AB>AC$. Tegu I įbrėžtinio apskritimo centras, o H - aukštinių susikirtimo taškas. Įrodykite, kad $2\angle AHI=3\angle ABC$.
\end{pavnr}
\begin{sprendimas}


Tegu įbrėžtinis į trikampį abc apskritimas yra vienetinis, kuris liečia kraštines ab, bc, ca taškuose r, p, q. Pažymime: $\angle ahi = \alpha$, $\angle abc = \beta$ ir $\omega = e^{i\alpha}$, $\phi = e^{i\beta}$.

Tada $i=0$, $a=\tfrac{2rq}{(r+q)}$, $b=\tfrac{2rp}{(r+p)}$, o $\angle riq=120^0$ iš keturkampio $riqa$ kampų. Gauname, kad $e^{i120^0}=r/q$, todėl $\Big(\tfrac{r}{q}\Big)^3=1$. Iš čia gauname:
\begin{equation*}
r^2 + rq + q^2 = 0 \tag{*}
\end{equation*}

Pasinaudoję teoremą 6.2 ir (*) gauname:
\begin{equation*}
h=\tfrac{2qr(pq +qr+rp)}{(p+q)(q+r)(r+p)}.
\end{equation*}

$\omega$ tenkina:
\begin{equation*}
\tfrac{a-h}{\overline{a}-\overline{h}}\omega^2=\tfrac{i-h}{\overline{i}-\overline{h}} \Leftrightarrow \omega^2=\tfrac{qr}{p^3}\tfrac{(pq +qr+rp)}{(p+q+r)}
\end{equation*}

$\phi$ tenkina:
\begin{equation*}
\tfrac{p-b}{|p-b|}\phi=\tfrac{r-b}{|r-b|} \Rightarrow \phi^3=-\tfrac{r^3}{p^3}
\end{equation*}

Pasinaudoję (*) įrodome, kad 
\begin{equation*}
\phi^3=\omega^2 \Rightarrow 3\beta=2\alpha + 360^0k \text{, kažkokiam } k\in\mathbb{Z}.
\end{equation*}

Jei $k>0$, tai $b>120^0$, kas neįmanoma. Jei $k<0$, tai $\alpha > 180^0 \Rightarrow \angle bah < \angle bai \Rightarrow ab<ac$, kas nėra teisinga. Todėl $k=0$ ir $3\beta=2\alpha$.
\end{sprendimas}


\begin{pavnr}
$\phantom{a}$ Duota trapecija ABCD (su pagrindais BC ir AD), o jos įstrižainių susikirtimo taškas yra lygus P.
  Taškas M yra BC vidurio taškas bei yra žinoma, kad $ \angle ABD + \angle ACD = \pi. $ Ant kraštinės AD paimtas
taškas X, toks, kad $ ( \frac {PD}{PA})^2 = \frac {DX}{XA} $.
 Įrodykite, kad MX yra statmena trapecios pagrindams.
\end{pavnr}
\begin{sprendimas}
 Šį kartą spręsime kompleksinių koordinačių centru imdami tašką $p$. Iš Euklidinės geometrijos
nesunku pastebėti, kad $\Delta APD$ yra panašus su $ \Delta CDP$. Šis faktas yra labai naudingas tvarkantis su trapecijos kraštinėmis, nes gauname lygybes:
$$ c = ka,  b =  kd, $$
 kur k yra racionalusis skaičius.
 Kadangi yra žinoma, jog $ ( \frac {PD}{PA})^2 = \frac {DX}{XA} $, tai pagal teoremą XXX
gauname, kad 
$$ x = \frac { da \bar a + ad \bar d }{ a \bar a + d \bar d},$$
 nes $ ( \frac {PD}{PA})^2 = \frac {DX}{XA} = \frac {d \bar d} {a \bar a} $.
\\$ \phantom{a}$ Koeficientą $k$ mums padės surasti žinojimas, kad $ \angle ABD + \angle ACD = \pi $, 
nes tai duoda lygybę:
 $$ \frac {b-a}{\bar b - \bar a} : \frac { b-d}{\bar b - \bar d} =  \frac {d-c}{\bar d - \bar c} : \frac { c-a}{\bar c- \bar a} $$
$ \phantom{a}$ Sutvarkius lygybę, gauname, kad 
$$ k =  \frac { \bar d a + \bar a d }{a \bar a + d \bar d}. $$
\\ $ \phantom{a}$Galiausiai, MX yra statmena trapecijos pagrindams tada ir tik tada, kai
pagal teoremą 1.2 :
 $$  \frac {m-x}{ \bar m - \bar x} = - \frac {a - d}{ \bar a - \bar d},$$
\\kur žinome, jog $ m = k \cdot \frac { a+d} { 2} $, nes $M$ yra atkarpos $BC$ vidurio taškas:
$$ \frac {m-x}{ \bar m - \bar x}= 
\frac {(a-b)( \bar a b - \bar b a) }{(\bar a- \bar b)( \bar b a - \bar a b) } = - \frac {a - d}{ \bar a - \bar d}.$$
 $ \phantom{a}$ Tad, uždavinys išspęstas!

\end{sprendimas}
\begin{pavnr}

Tegul F yra taškas ant trapecijos ABCD pagrindo AB, toks kad $DF=CF$ . 
Bei pavadinkime raide E trapecijos įstrižainių susikirtimo tašk1, o taškais $O_1$ ir $O_2$
apibrėžtų apie $ \Delta ADF$ ir $\Delta FBC$ apskritimų centrus. 
Įrodykite, kad $ FE \bot O_1 O_2 $.
\end{pavnr}
\begin{sprendimas}
\\ $ \phantom{a}$ Raktas į šio uždavinio sprendimą yra teisingas koordinačių centro parinkimas. 
Šiuo atveju, ypač gražiai tvarkosi reiškiniai, jei pasirenkame centru tašką F. Kodėl gražiai? Ogi 
todėl, kad tuomet $ d = \bar c $, (nes $FC=FD$ ir mes galime tarti, jog Ox, realiųjų skaičių ašis 
yra statmena trapecijos pagrindams). Toliau darosi dar gražiau, nes mes pastebime, jog taškai 
a ir b priklauso menamųjų skaičių Oy ašiai, todėl galime nesunkiai (mintyse įsivaizduodami 
kompleksinę plokštumą ir prisiminę, jog jungtinis tai taškas simetriškas pradiniam realiųjų skaičių 
Ox ašies atžvilgiu) pastebėti jog $ \bar a = - a $ bei $ \bar b = - b$.
\\ $ \phantom{a}$ Toliau eina kiek griozdiškesni reiškiniai - reikia surasti $o_1$ bei $o_2$ taškus. Šie
taškai yra ne kas kita, kaip kraštinių vidurio statmenų susikirtimai, todėl pagal teoremos 1 formules gauname, kad:
$$ o_1 = \frac { ad( \bar a - \bar d) } { \bar a d - a \bar d } = \frac {\bar c ( a + c) } { \bar c + c }  $$ 
bei 
$$  o_2 = \frac { bc( \bar b - \bar c) } { \bar b c - b \bar c } = \frac { c( b + \bar c) } { \bar c + c }.$$
\\$ \phantom{a}$ Dabar dorosime tašką $E$: šį tašką apsibrėšime per dvi lygtis: naudojantis teorema $ 1.3$ taškai $a, c, e$ bei $d, b, e$ priklauso vienai tiesei, gauname:
$$ \frac { a- c} {\bar a-\bar c} = \frac {e-a}{ \bar e - \bar a}$$
$$ \frac { b-d} {\bar b-\bar d} = \frac {e-b}{ \bar e - \bar b}.$$
\\ Išsireiškę iš šių dviejų lygčių  $ \bar e$ ir juos sulyginę, gauname, kad
\\ $$ e = \frac { a \bar c - bc}{ a+ \bar c -b-c}.$$
\\$ \phantom{a}$ Galiausiai mums lieka įsistatyti visas gautąsias grožybes tam, kad išspręstume uždavinį - įrodytume, kad $ FE \bot O_1 O_2 $. Pastaroji sąlyga pagal teoremą 1.2 yra ekvivalenti sąlygai :
 $$  \frac {o_1 - o_2} { \bar o_1 - \bar o_2}= - \frac { f-e}{\bar f - \bar e},$$
o tai išeina elementariausiai prastinantinant $ o_1 - o_2 = \frac {a \bar c - cb} {c + \bar c}$.
\end{sprendimas}


\begin{pavnr}
(,,Baltic Way" 2002) Tegu ABC yra smailusis trikampis $ \angle BAC > \angle BCA$,
ir tegu D yra toks taškas ant kraštinės AC, kad $AB = BD$. Dar daugiau, F
yra toks ypatingas taškas ant apibrėžto apie ABC apskritimo, kad
tiesė FD yra statmena BC bei taškai
F, B yra ant priešingų pusių kraštinės AC. Įrodykite, kad tiesė FB yra statmena
kraštinei AC.
\end{pavnr}
\begin{sprendimas}
\\   $\phantom{a}$Tegu apie trikampį $ABC$ apibrėžtas apskritimas yra vienetinis. 
\\   $\phantom{a}$ Paimsime tokį tašką $d'$, kad $fd'$ būtų statmena $bc$ bei $d'$ priklausytų
 stygai ac ir įrodysime, kad tuomet $d=d'$. 
Pagal teoremą 3.2, taško $d'$, priklausančio stygai $ac$ jungtinis yra
\\ $$\bar d' = \frac {c+a-d}{ca}$$
\\Pagal teoremas
1.2 ir 3.1 bei faktą, kad bf statmena ca, turime, kad:
\\ $$ f'= - \frac {ca}{b}.$$
\\ Dar daugiau, d'f yra statmena bc, tad pagal tas pačias teoremas gauname:
$$ d= b+ \frac {c}{b}(b-a).$$
\\ $\phantom{aaaa}$ Lieka įrodyti, kad d'b=ab ir tuomet gausime, kad 
d'=d. Atstumui apskaičiuoti prisiminsime kompleksinių skaičių modulių sąvybę:
$ |x|^2=\bar x \cdot x$\\. Tuomet sąlyga, kad d'b=ab užsirašo štai taip:
\\
\\  $$ (d-b)( \bar d - \bar b)=(a-b)( \bar a - \bar b).$$
\\
\\ Įstačius žinomą d' reikšmę gauname:
\\
\\ $$ (d-b)( \bar d - \bar b)=( b+ \frac {c}{b}(b-a)-b)(( \bar b+ \frac {\bar c}{\bar b}(\bar b-\bar a)-\bar b)=(a-b)( \bar a - \bar b).$$
Gavome, kad $d=d'$, todėl ir šis uždavinys buvo nugriautas!
\end{sprendimas}

