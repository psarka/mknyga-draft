\section{Funkcinės lygtys}
Dauguma suprantame, ką reiškia išspręsti lygtį. Tai yra rasti visus užrašytos
lygybės sprendinius ir įrodyti, kad daugiau jų nėra. Išspręsti funkcinę
lygtį reiškia beveik tą patį - rasti visas funkcijas, tenkinančias lygybę ir
įrodyti, kad daugiau tokių nėra. Daugumos funkcinių lygčių sprendimai turi
panašų pobūdį - manipuliuojama duota lygtimi siekiant gauti kuo daugiau
apribojimų tikėtiniems sprendiniams. Sėkmės atveju, apribojimų pakanka ir
galima nusakyti sprendinių aibę (kartais ji būna tuščia) bei patikrinti,
kad išties visos rastos funkcijos yra sprendiniai. Pirmąjame skyrelyje
supažindinsime su pačia pagrindine sprendimo idėja - fiksuotų reikšmių
įstatymu vietoje kintamųjų funkcinėje lygtyje. Antrąjame parodysime, kaip
iš duotos lygties gauti informacijos apie funkcijos tipą (pvz. lyginumą,
monotoniškumą, injektyvumą), bei kaip ją pritaikyti. Trečiąjame išspręsime
žymiąją Cauchy funkcinę lygtį ir panaudosime ją spręsdami sudėtingesnius
uždavinius.

\subsection{Įsistatykime $x=0$}

Nieko nelaukdami užsirašykime pirmąją funkcinę lygtį:

\begin{pavnr}Raskite visas funkcijas $f:\R \to \R$, tenkinančias lygtį
  $$f(x+y)=f(x)$$ su visais realiaisiais $x$ ir $y$, bei lygybę $f(0)=0$.
\end{pavnr}

Šios funkcinės lygties sąlyga susideda iš keturių dalių. Apžvelkime jas:
\begin{itemize}
  \item[-] Ieškomų funkcijų apibrėžimo ir reikšmių sritys. Šiuo atveju
    duota $f:\R \to \R$, t.y. sprendinių reikia ieškoti tarp visų funkcijų
    apibrėžtų realiuosiuose skaičiuose ir su realiomis reikšmėmis.
  \item[-] Lygtis, kurią turi tenkinti ieškomos funkcijos.
  \item[-] Lygtyje dalyvaujančių kintamųjų kitimo sritys. Šiuo atveju
    duota, kad lygtį funkcijos turi tenkinti su visomis realiomis $x$ ir
    $y$ reikšmėmis
  \item[-] Papildomos sąlygos. Šiuo atveju duota, kad reikia ieškoti tik tų
    lygties sprendinių, kurie papildomai tenkina $f(0) = 0$.
\end{itemize}

\begin{sprendimas} Įsistatykime $x=0$, gausime $f(y)=f(0)=0$, t.y. $f(y)=0$
  su visais $y \in \R$. Patikrinę gauname, kad sprendinys tinka.
\end{sprendimas}

Sprendimas trumpesnis už sąlygą, tad su nespėjusiais pastebėti, kaip jis
pralekė, pasižiūrėkime sulėtintą kartojimą. Sąlygoje duota, kad ieškomos
funkcijos turi tenkinti lygtį $f(x+y) = f(x)$ su visomis realiosiomis $x$
ir $y$ reikšmėmis. Vadinasi, turės tenkinti lygtį ir kai vienam iš
kintamųjų parinksime konkrečią reikšmę, ką paprastai įvardijome kaip
,,įstatykime $x=0$''. Toliau, žinodami, kad ieškomos funkcijos turi su
visomis realiomis $y$ reikšmėmis tenkinti lygtį $f(y) = f(0)$, bei kad
ieškomos funkcijos turi tenkinti papildomą sąlygą $f(0)=0$, darome išvadą,
kad ieškomos funkcijos turi su visomis realiosiomis $y$ reikšmėmis tenkinti
$f(y)=0$. Tačiau ši sąlyga yra tokia stipri, kad ji nurodo vieną vienintelę
funkciją! Lieka patikrinti, ar ji yra sprendinys. Kadangi visuose realiuose
taškuose ji įgyja reikšmę $0$, tai įstatę ją į lygtį gausime akivaizdžiai
teisingą lygybę $0=0$. Lygtis išspręsta.

Įsistatyti vietoje vieno ar kelių kintamojų nulį yra dažniausiai
pasitaikanti funkcinių lygčių sprendimo idėja, nuo kurios neretai verta
pradėti spręsti nematytą lygtį. Tačiau reikia turėti omenyje, kad retai
kada vien šio triuko užteks, tad svarbu turėti ir kitų ginklų.
Pavyzdžiui:

\begin{pavnr}Raskite visas funkcijas $f:\R \to \R$, kurios tenkina lygtį
  $$f(x+y)=f(x^2+y^2)$$ su visais realiaisiais $x$ ir $y$.
\end{pavnr}

\begin{sprendimas}
  Įsistatykime $x=y$, gausime $f(2y)=f(2y^2)$. Įsistatykime $x=-y$, gausime
  $f(0) = f(2y^2)$. Abi lygybės turi galioti su visomis realiosiomis $y$
  reikšmėmis, tad galime jas sujungti: $f(2y)=f(0)$. Lieka įsižiūrėjus
  konstatuoti, kad ieškomos funkcijos visuose realiuose taškuose įgis tą
  pačią reikšmę kaip ir taške $0$. Tokių funkcijų be galo daug, ir jos
  įprastai užrašomos $f(x)=c$, kur $c$ - bet kuris iš realiųjų skaičių (dar
  vadinamas konstanta). Lieka patikrinti, ar visos tokios funkcijos tinka.
  Įstatę gausime $c=c$, vadinasi tinka.
\end{sprendimas}

Naudodami šias paprastas nulio ir $x=y$ įsistatymo idėjas išspręskime dar
keletą lygčių. Atkreipsime dėmesį į tai, kad labai svarbi dalis yra
teisingai interpretuoti gautą po įsistatymo lygybę. Kartais ji būna
bevertė, o kartais sujungus su kažkuo papildomu galima gauti ką nors
naudingo.  Sunkesniuose uždaviniuose tai ne visuomet pavyksta, tad verta
apsišarvuoti kantrybe ir bandyti įsistatyti įvairias kintamųjų reikšmių
kombinacijas.

\begin{pavnr} \emph{[LitKo 2008]} Raskite visas tokias realiąsias funkcijas
  $f$, kad $f(x)f(y) - f(xy) = x+y$ su visomis realiųjų skaičių $x$ ir $y$
  poromis.
\end{pavnr}

\begin{sprendimas} Įsistatykime $x=0$ ir $y=0$. Gausime $f(0)^2=f(0)$, t.y.
  $f(0)=0$ arba $f(0)=1$. Panagrinėkime abu atvejus:
  \begin{description}
    \item[$f(0)=0$ -] Įsistatykime į pradinę lygtį $x=0$, gausime $0=y$. Ši
      lygybė jokiai funkcijai negalioja su visomis realiomis $y$
      reikšmėmis, todėl šį atvejį atmetame.
    \item[$f(0)=1$ -] Įsistatykime į pradinę lygti $x=0$, gausime $f(y) =
      y+1$. Patikrinę matome, kad ši funkcija tinka: $(x+1)(y+1) - (xy + 1)
      = x+y$.
  \end{description}
  Gavome, kad funkcija $f(x)=x+1$ bus vienintėlis sprendinys.
\end{sprendimas}

\begin{pavnr}
  Raskite visas funkcijas $f: \R\to\R_{\geq 0}$ \footnote{$\R_{\geq0 }$
  žymėsime visus neneigiamus realiuosius, o $\R^+$ visus teigiamus
  realiuosius skaičius.} tenkinančias lygybę $$f(2u)=f(u+v)f(v-u)+f(u-v)f(-u-v)$$ su
  visomis realiomis $u$ ir $v$ reikšmėmis.
\end{pavnr}

\begin{sprendimas} Įsistatykime $u=0$ ir $v=0$. Gausime $f(0)=2f(0)^2$,
  t.y. $f(0)=0$ arba $f(0) = \frac{1}{2}$. Panagrinėkime abu atvejus:
  \begin{description}
    \item[$f(0)=0$ -] Įsistatykime $u = 0$, gausime $0 = f(v)^2 +
      f(-v)^2$. Šią lygtį tenkina vienintėlė funkcija - $f(v)=0$.
    \item[$f(0)=\frac{1}{2}$ -] Vieną kartą įsistatykime $u = 0$, kitą $u =
      v$, gausime dvi lygtis: $\frac{1}{2} = f(v)^2 + f(-v)^2$ ir $f(2v) =
      f(-2v)$. Iš jų seka, kad $\frac{1}{2} = f(v)^2 + f(v)^2 = 2f(v)^2$
      ir, kadangi ieškome funkcijų įgyjančių tik neneigiamas reikšmes,
      $f(v)= \frac{1}{2}$.
  \end{description}
Patikrinę matome, kad abi rastos funkcijos $f(v)=0$ ir $f(v)=\frac{1}{2}$
tinka.
\end{sprendimas}

\begin{pavnr}
  Raskite visas funkcijas $f:\R \to \R$, kurios su visais realiaisiais $x$
  ir $y$ tenkina $f(x+f(y))=x+f(f(y))$ ir $f(2004)=2005$.
\end{pavnr}

\begin{sprendimas} Įstatykime $y=0$, gausime $f(x+f(0)) = x + f(f(0))$.
  Įsižiūrėjus į gautą lygybę tampa aišku, kad ją tenkina tiktai funkcijos
  $f(x)=x+c$, kur $c$ - konstanta.  Pridėjus papildomą sąlyga lieka
  vienintelė funkcija $f(x)=x+1$, kuri ir yra sprendinys.
\end{sprendimas}

Pasiaiškinkime kiek išsamiau, kaip iš lygybės $f(x + f(0)) = x + f(f(0))$
gauti $f(x) = x+c$. Paimkime bet kurią funkciją, kuri tenkina pirmąją
lygtį. Kad ir kokia ji būtų, $f(0)$ ir $f(f(0))$ bus konretūs skaičiai,
nepriklausantys nuo $x$. Patogumo dėlei pakeiskime $t=x+f(0)$, tuomet
gausime, kad su visomis realiosiomis $t$ reikšmėmis $f(t) = t + f(f(0)) -
f(0)$. Lieka tik konkretų skaičių $f(f(0)) - f(0)$ pažymėti $c$. Ši
nekintančių reiškinių pažymėjimo idėja yra gana dažna, tad verta ją
įsidėmėti.

\subsubsection{Uždaviniai}

\begin{enumerate}
  \item Raskite visas funkcijas $f:\R\to\R$, kurios su
    visais realiaisiais $x$, $y$ tenkina $$f(x+y)+f(x-y)=2x^2+2y^2.$$
    %Įsistatykime $y=0$, gausime $f(x)=x^2$. Patikrinę matome, kad ši
    %funkcija tinka.
  \item Raskite visas funkcijas $f:\R\to\R$, kurios su visais realiaisiais
    $x$, $y$ tenkina $$f(x)+f(x+y)=y+2.$$
    %Įsistatykime $y=0.$ Gausime, kad su visais $x$ turi būti $f(x)=1,$
    %tačiau ši funkcija lygties netenkina. Sprendinių nėra.
  \item Raskite visas funkcijas $f:\R\to\R$, kurios su visais realiaisiais
    $x$, $y$ tenkina $$f(x)=(x-y)f((x-y+1)x) + f(y).$$
    %Įsistatę $x=0$ gauname $f(y) = (y+1)f(0)$, t.y. vienintelės funkcijos
    %kurios galėtų tikti yra $f(x)=c(x+1)$. Patikrinę gauname, kad tinka tik
    %$c=0$, t.y. $f(x)=0$.
  \item Raskite visas funkcijas $f:\R\to\R$, kurios su visais realiaisiais
    $x$ ir $y$ tenkina $$yf(x) = xf(y).$$
    %Įsistatykime vietoje $y$ bet kokį nelygų nuliui skaičių, pavyzdžiui
    %$1$. Gausime $f(x)=f(1)x$, vadinasi, ieškomos funkcijos bus pavidalo
    %$f(x)=cx$, kur $c$ reali konstanta. Patikrinę gauname, kad visos
    %tokios funkcijos tinka.
  \item Raskite visas funkcijas $f:\R\to\R$ su visais realiaisiais
    $x$ ir $y$ tenkinančias lygtį $$f(x +f(y)) = f(x) + yf(x).$$
    %Įsistatykime $y=-1$, gausime $f(x+f(-1))=0$. Kadangi $f(-1)$ yra
    %konkretus skaičius, tai $x=f(-1)$ įgyja visas realias reikšmes, iš kur
    %gauname, kad funkcija turi tenkinti $f(x)=0$. Patikrinę matome, kad
    %šis sprendinys tinka.
  \item Raskite visas funkcijas $f:\R\to\R$ su visais $x\in \R$ tenkinančias
    lygtį $$xf(x) + f(-x) + 1 = 0.$$
    %Įsistatykime $x=-x.$ Gausime $-xf(-x) + f(x) + 1 = 0.$ Iš pradinės
    %lygties išsireiškę $f(-x)$ ir įsistatę gausime
    %$f(x)=-\frac{1+x}{1+x^2},$ kas ir yra sprendinys.
  \item Raskite visas funkcijas $f:\R\to\R$ su visais $x\in\R, x\neq 0,1$
    tenkinančias lygtį $$f(x) + f(\frac{x-1}{x})= 1-x.$$
    %Įsistatę $x=\frac{x-1}{x}$ ir $x=\frac{1}{1-x}$ kartu su pradine turime
    %tris lygtis, iš kurių paplušėję išsireiškiame $f(x)$. Gauname
    %$f(x)=\frac{x^3-3x^2+2x-1}{2x-2x^2}$.
  \item Raskite visas funkcijas $f:\R\to\R$, kurios su visais realiaisiais
    $x$, $t$, $z$ tenkina $$(x+t)f(z) = f(xz) + f(tz).$$
    %Įsistatę $x=1$ ir $z=1$ gauname $f(t) = tf(1)$, t.y. funkcija gali
    %būti tiktai pavidalo $f(x)=ax, a\in \R$. Patikrinę matome, kad visos
    %tokios funkcijos tinka.
  \item \text{[LitMo 2000, Pan African 2003]} Raskite funkcijas $f:\R\to\R$ su visomis realiosiomis
    $x$,$y$ reikšmėmis tenkinančias lygtį
    $$(x+y)(f(x)-f(y))=f(x^2)-f(y^2).$$
    %Įsistatykime $y=0$ ir $y=1$. Iš gautų lygybių gauname, kad $f(x)$ -
    %tiesinė funkcija ($f(x) = (f(1)-f(0))x +f(0)$). Patikrinę matome, kad
    %funkcijos $f(x)=ax+b$ tinka su visais $a,b\in \R.$
  \item Raskite visas funkcijas $f:\R\to\R$, kurios su visais realiaisiais
    $x$, $t$ tenkina $$f(x)f(t) = f(x) + f(t) + xt -1.$$
    %Įstatę vietoje $t$ bet kokią reikšmę, su kuria $f(t)\neq 0$, gauname, kad
    %$f(x)$ - tiesinė funkcija. Patikrinę gauname, kad tinka tik $f(x)
    %= 1-x$ ir $f(x) = 1+x$.
  \item \text{[LitMo 1994]} Ar egzistuoja bent viena funkcija $f:\R \to \R$
    tenkinanti lygtį $$f(f(x))=x^3 \text{ ?}$$
    %Taip: $$f(x) = \begin{cases}-x,& x<0, \\ -x^3,& x\geq 0.\end{cases}$$
  \item Raskite visas funkcijas $f:\R\to\R$, kurios su visais realiaisiais
    $x$, $y$ tenkina $$f(f(x-y)) = f(x) - f(y) - f(x)f(y) - xy.$$
    %Įsistatę $x=y$ gauname $f(f(0)) = -x^2 - f(x)^2$. Įsistatę $x=y=0$
    %gauname $f(f(0)) = -f(0)^2$. Iš šių dviejų lygybių gauname $f(0)^2 -
    %x^2 = f(x)^2 \geq 0$, kas negalioja su visais $x \in \R$. Vadinasi
    %funkcijų tenkinančių lygtį nėra.
  \item Raskite visas funkcijas $f:\R\to\R$, kurios su visais realiaisiais
    $x$, $y$ tenkina $$(x-y)^2f(x+y)=(x+y)^2f(x-y).$$
    %Kadangi visiems realiesiems $a$, $b$ egzistuoja tokie $x$,$y$, kad
    %$x+y=a$ ir $x-y=b$, tai lygtį galime užrašyti $b^2f(a)=a^2f(b)$. Jei
    %egzistuoja toks $b_0$, kad $f(b_0) \neq 0$, tai jį įstatę vietoje $b$
    %gauname $f(a)=ca^2$, kur $c$ - konstanta (jei neegzistuoja, tai
    %$f(x)=0$). Įsistatę gauname, kad tinka visos $c$ reikšmės, vadinasi,
    %sprendiniai yra $f(x) = cx^2$.

  \item Raskite visas funkcijas $f:\R\to\R$, kurios su visais realiaisiais
    $x$, $y$ tenkina $$f(x+f(y))=f(f(x))+y.$$
    %Įstatę $x=0$ ir įstatę $y=0$ gauname $f(f(x))=f(f(0)) +x$ ir
    %$f(x+f(0))=f(f(x))$, iš kur $f(x+f(0))=f(f(0))+x\implies
    %f(x)=x+c$.
  \item Raskite visas funkcijas $f:\R\to\R$, kurios tenkina $f(1)=1$ ir su
    visais realiaisiais $x$, $y$ $$f(x+y)=3^y\cdot f(x)+2^x\cdot f(y)$$
    %Įsistatykime $x=y$ ir $y=x$ (t.y sukeiskime kintamuosius vietomis).
    %Gausime $f(x+y)=3^x\cdot f(y)+2^y\cdot f(x)$. Atėmę iš šios lygybės
    %pradinę ir įsistatę $y=1$ gausime $f(x) = 3^x-2^x$. Patikriname -
    %tinka.
  \item \text{[LitMo 2008]} Funkcija $f(x)$ apibrėžta teigiamiems
    skaičiams, įgyja teigiamąsias reikšmes ir su visais teigiamais $x$,
    $y$ tenkina lygybę $$f(x)f(y) = f(xy) + f(\frac{x}{y}).$$
    a.) Nurodykite bent tris tokias funkcijas.  \\[0.5cm]
    b.) Įrodykite, kad $f(x)\geq 2$, $f(1)=2$.  \\[0.5cm]
    c.) Įrodykite, kad jei $f(x)$ tenkina sąlygą, tai ją tenkina ir
    funkcija $f^2(x)-2$.
    %Rasti bent vieną funkciją nėra visai paprasta, tačiau kiek pamąstę
    %matome, kad $f(x)=x+\frac{1}{x}$. Taip pat tiks ir
    %$f(x)=x^2+\frac{1}{x^2}$ ir $f(x)=x^3+\frac{1}{x^3}$.
    %
    %Įstatykime $x=1$ ir $y=1$. Gausime $f(1)^2=2f(1)$. Kadangi funkcija
    %įgyja tik teigiamas reikšmes, tai $f(1)=2$.
    %
    %Įstatykime $y=x$. Gausime, kad $f(x)^2 = f(x^2) + 2$. Kadangi su
    %visais $x\in \R$ $f(x^2)\geq 0$, tai su visais $x \in \R$ $f(x)\geq
    %\sqrt{2}$. Dabar, kadangi su visais $x\in \R$ $f(x^2)\geq \sqrt{2}$,
    %tai su visais $x \in \R$ $f(x)\geq \sqrt{2+\sqrt{2}}$. Taip tęsdami,
    %gauname, kad su kiekvienu $x$ ir kiekvienu $n\in \N$ $$f(x)\geq
    %\underbrace{\sqrt{2+\sqrt{2+\cdots +\sqrt{2+\sqrt{2}}}}}_{n}.$$
    %
    %Kadangi, kai $n$ artėja į begalybę, $\underbrace{\sqrt{2+\sqrt{2+\cdots
    %+\sqrt{2+\sqrt{2}}}}}_{n}$ artėja į $2$ (seka akivaizdžiai didėjanti,
    %mažesnė už $2$ $\Rightarrow$ turi ribą. Ją randame išsprendę
    %$\sqrt{2+x}=x \Rightarrow x=2$), tai $f(x) \geq 2$.  \\
    %c.) dalyje užtenka pasinaudoti įsistačius $f(x)^2 = f(x^2) + 2$,
    %ir norint įsitikinti, kad $f^2(x)-2\geq 0$ - b.) dalyje gauta nelygybe.
  \item Raskite visas funkcijas $f:\R ^{+}\to \R ^{+}$ su visais teigiamais
    $x$ ir $y$ tenkinančias lygtį $$f(xy)=f(x+y).$$
    %Atkreipsime dėmesį, kad jei nebūtų lygties apribojimo vien teigiamiems
    %skaičiams, tai įsistatę $y = 0$ iš karto gautume, kad $f(x)=c$. Tačiau
    %apribojimas yra, todėl suktis teks kiek kitaip.
    %
    %Fiksuokime sumą ir žiūrėkime, kaip kinta sandauga. T.y. įsistatykime
    %pvz., $y=2-x$. Gausime $f(x(2-x))=f(2)$. Kadangi galime statyti tik
    %teigiamas reikšmes, tai ši lygybė yra teisinga tik, kai $x\in (0,2)$.
    %Šiame intervale, kintant $x$ reikšmei, reiškinio $x(2-x)$ reikšmė kinta
    %nuo $0$ iki $1$, t.y. $x(2-x)\in(0,1]$. Tad gauname, kad $f(x)$ yra
    %pastovi intervale $(0,1]$. Lieka pastebėti, kad ji periodinė: įstatę
    %$y=1$ gausime $f(x)=f(x+1)$, todėl pastovi ir visur.
  \item Raskite visas funkcijas $f:\R\to\R$, kurios su visais realiaisiais
    $x$ ir $y$, nelygiais nuliui, tenkina $$f(x+y)=f(1/x+1/y).$$
    %Fiksuokime sumą. Tegu $x=2-y$. Tuomet $f(2)=f(\frac{2}{x(2-x)})$.
    %Kai $x$ kinta nuo $-\infty$ iki $\infty$ reiškinys
    %$\frac{2}{x(2-x)}$ kinta intervaluose
    %$(-\infty,0)\cup[\frac{1}{2},\infty)$, kartu ir funkcija tuose
    %intervaluose pastovi. Likusią dalį $(0,\frac{1}{2})$ galime
    %prijungti naudodami $f(x+\frac{1}{2}) = f(\frac{1}{x}+2)$.
    %Pastebėsime, kad funkcijos reikšmė taške 0 taip ir lieka
    %neapibrėžta. Atsakymas $$f(x) = \begin{cases}a,& x\neq0 \\ b,& x= 0
    %\text{      \ \ \ \ \   } a,b\in\R\end{cases}$$
  \item \text{[Brazil 1993]} Raskite bent vieną funkciją $f:\R\to\R$ su kiekvienu $x \in \R$
    tenkinančią $$f(0)=0 \text{ ir } f(2x+1)=3f(x)+5.$$
    %Atsikratykime penketo: įsistatykime $f(x) = g(x) -\frac{5}{2}$.
    %Gausime, kad $g$ tenkina lygtį $g(2x+1)=3g(x)$. Pabandę pirmas keletą
    %reikšmių gausime, kad $$g(1)=3g(0), g(3)=3g(0), g(7)=3^2g(0), g(15) =
    %3^3g(0).$$ Įsižiūrėję pamatysime, kad taip tęsdami gausime
    %$$g(2^n-1)=3^n g(0).$$ Pažymėję $2^n-1 = x$ gauname $n =
    %\log_2(x+1)$ arba $g(x)=3^{\log_2 x+1}\cdot g(0).$ Iš
    %$f(0)=0$ seka, kad $g(0)=\frac{5}{2}$ ir susitvarkę su neigiamų
    %skaičių keliamais nepatogumais gauname, kad  $$f(x) =
    %3^{\log_2 |x+1|}\cdot \frac{5}{2} - \frac{5}{2}$$ tenkina
    %lygtį.
  \item Raskite visas funkcijas $f:\R\to\R$, kurios su visais realiaisiais
    $x$, $y$ tenkina $$f(x^{3})-f(y^{3})=(x^{2}+xy+y^{2})(f(x)-f(y))$$
    %Įsistatykime $y=0$ ir $y=1$:
    %\begin{eqnarray*}
    %  f(x^3) &=& (x^2 + x + 1)(f(x)-f(1))+f(1),\\
    %  f(x^3) &=& x^2 (f(x)-f(0)) + f(0).
    %\end{eqnarray*}
    %Lygybės teisingos su visomis $x$ reikšmėmis, tad sulyginę dešiniąsias
    %puses gausime $$f(x)=xf(1) + (1-x)f(0),$$ t.y. funkcija tiesinė.
    %Patikrinę matome, kad lygtį tenkina visos funkcijos $f(x)=ax+b$, kur
    %$a,b\in \R.$
  \item Raskite visas funkcijas $f:\R\to\R$ tenkinančias lygybę $f(x)^2=1$
    su kiekvienu $x\in\R$ .
    %Šis uždavinys, nors ir paprastas, yra gerai žinomi spąstai. Iš pirmo
    %žvilgsnio padaryta išvada, kad sprendinai yra tik $f(x)=1$ ir
    %$f(x)=-1$ nėra teisinga. Atidžiau pažvelgus tampa aišku, kad viskas, ką
    %galima pasakyti apie funkciją, yra tai, kad bet kuriame taške ji įgyja
    %reikšmę $1$ arba $-1$. Užrašius tą matematiškiau, sprendiniai atrodo
    %kaip $$f(x) = \begin{cases}1,& x\in A, \\ -1,& x\not\in A,\end{cases}$$
    %kur $A$ bet koks $\R$ poaibis.

\end{enumerate}

\newpage
\subsection{Funkcijų tipai}

Šioje užduotyje panagrinėsime įvarius funkcijų tipus, sutinkamus sprendžiant
funkcines lygtis. Greičiausiai jau yra tekę girdėti, kas yra lyginė,
nelyginė, monotoninė ar periodinė funkcija, tad per daug nesiplėsdami
prisiminkime tikslius apibrėžimus.

\begin{api}
Funkciją $f:A\to B$, kur aibė $A$ simetrinė nulio atžvilgiu,
vadinsime lygine, jei $\forall x \in A$ teisinga $f(-x)=f(x) $, ir nelygine,
jei $\forall x \in A$ teisinga $f(-x) = -f(x)$.
\end{api}

\begin{api}
Funkciją $f:A\to B$ vadinsime periodine, jei egzistuoja toks $a\in A$, kad
$f(a+x)=f(x)$ $\forall x\in A$.
\end{api}

\begin{api}
Funkciją $f:A\to B$ vadinsime monotonine, jei ji yra arba nedidėjanti, arba
nemažėjanti, t.y. arba $f(x)\leq f(y)$ su visais $x>y$ $(x,y\in A)$, arba
$f(x)\geq f(y)$ su visais $x>y$ $(x,y \in A)$.
\end{api}

Atkreipsime dėmesį, kad didėjanti funkcija dažniausiai reiškia nemažėjanti
(ir atvirkščiai mažėjanti - nedidėjanti), todėl yra vartojami terminai
\emph{griežtai didėjanti} ir \emph{griežtai mažėjanti}, norint pabrėžti,
jog funkcija negali būti pastovi.

Vos prisiminę, lyginumą, nelyginumą, periodiškumą ir monotoniškumą iš karto
paliksime nuošalyje ir pereisime prie \emph{injektyvių} ir
\emph{surjektyvių} funkcijų nagrinėjimo. Vargu ar suklysime teigdami, kad
šios dvi sąvokos yra centrinės sprendžiant kiek sudėtingesnes olimpiadose
sutinkamas funkcines lygtis, tad joms skirsime labai daug dėmesio.

\subsubsection{Injektyvumas ir surjektyvumas}

Funkciją vadinsime injektyvia, jei ji kiekvieną reikšmę įgyja tik vieną
kartą. Dažnai sutinkamos injektyvios funkcijos yra tiesės $f(x)=ax +
b$, $a\neq 0$ (ypač $f(x)=x$ ir $f(x)=-x$), bet nesunku rasti ir daugiau pavyzdžių:
$f(x)=x^3$, $f(x)=\frac{1}{x}$, $f(x)=e^{x}$. Elementariausias
neinjektyvios funkcijos pavyzdys - $f(x)=x^2$. Iš ties - ji, pavyzdžiui,
reikšmę $1$ įgyja du kartus: $f(1)=f(-1)=1$.

Atkreipsime dėmesį, kad nagrinėjant injektyvumą yra svarbi apibrėžimo
sritis. Pavyzdžiui, nors $f(x)=x^2$ ir nėra injektyvi visoje realiųjų
tiesėje, ji tokia tampa apribojus apibrėžimo sritį iki neneigiamų skaičių.

Pateiksime formalų apibrėžimą, kurį, kaip pamatysime, labai patogu
tiesiogiai taikyti sprendžiant funkcines lygtis:

\begin{api}
Funkciją $f:A\to B$ vadinsime injektyvia, jei visiems $a,b \in A$ teisinga $$f(a)=f(b) \Rightarrow a=b.$$
\end{api}

Antroji sąvoka - surjektyvumas - apibūdina funkcijas, kurios įgyja visas
savo reikšmių srities reikšmes. Jei nagrinėsime funkcijas $f:\R \to \R$,
tai surjektyviomis bus tos pačios tiesės $f(x)=ax +b$, $a\neq 0$, arba,
pavyzdžiui, visi nelyginio laipsnio daugianariai. Nesurjektyvios bus
pavyzdžiui $f(x)=x^2$ ir $f(x)=e^x$, nes neįgyja neigiamų reikšmių.

Vėlgi, apribojus reikšmių sritį nesurjektyvi funkcija gali tapti
surjektyvia, tad visada reikia aiškiai suprasti, kas tiksliai yra
apibrėžimo ir kas yra reikšmių sritis kiekvienu atveju ir po kiekvieno
pertvarkymo.

Formalus surjektyvumo apibrėžimas:

\begin{api}
Funkciją $f:A\to B$ vadinsime surjektyvia, jei kiekvienam $b\in B$
egzistuoja toks $a\in A$, kad $f(a)=b$.
\end{api}

Funkciją, kuri yra ir injektyvi, ir surjektyvi, vadinsime
\emph{bijektyvia}.  Bijektyvi funkcija, arba tiesiog bijekcija, kiekvienam
apibrėžimo srities elementui priskiria unikalų reikšmių srities elementą,
ir kiekvienas reikšmių srities elementas yra priskirtas. Kaip jau žinote
(arba jei ne, tai nesunku suvokti), bijektyvi funkcija turi atvirkštinę.

\subsubsection{Panaudojimas}

Panagrinėkime keletą situacijų darydami prielaidą, kad ieškoma funkcija yra
injektyvi arba surjektyvi.

\begin{pav}
Raskite visas funkcijas $f:\R \to \R$, tenkinančias lygtį $$f(f(x))=f(x).$$
\end{pav}

Ši lygtis turi be galo daug sprendinių, kurių struktūra kiek komplikuota.
Galite pabandyti juos rasti.

Kas pasikeistų, jei žinotume, kad ieškoma funkcija yra injektyvi?
Pažiūrėkime - jei funkcija injektyvi, tai iš $f(a)=f(b)$ seka, kad $a=b$ su
visais $a$, $b$. Šiuo atveju vietoje $a$ stovi $f(x)$, o vietoje $b$ stovi
$x$, todėl iš $f(f(x))=f(x)$ sektų $f(x)=x$ su visais $x$ - lygtis
išspręsta!

Kas atsitiktų, jei žinotume, kad mūsų ieškoma funkcija yra surjektyvi?
Surjektyvi funkcija įgyja visas reiškmių srities reikšmes, šiuo atveju
visus realiuosius skaičius. Jei $f(x)$ galėtų būti bet koks realus skaičius,
tai tuomet pažymėję $f(x)=y$ gautume $f(y)=y$ su visais realiaisiais $y$ -
lygtis išspręsta!

\begin{pav}
Raskite visas funkcijas $f:\R \to \R$, tenkinančias lygtį $$f(x+f(y)) =
f(f(x)+y).$$
\end{pav}

Jei žinotume, kad funkcija yra injektyvi, iš karto gautume $f(x+f(y)) =
f(f(x)+y) \Rightarrow x+f(y) = f(x)+y$, o tokią lygtį jau spręsti mokame.
Užtenka įsistatyti, pavyzdžiui, $y=0$ ir gauti $f(x)=x+c$, kur $c$ bet koks
realus skaičius.

\begin{pav}
Raskite visas funkcijas $f:\R \to \R$, tenkinančias lygtį $$f(x+f(y)) =
f(y^2 + x^2f(y)) + xf(x).$$
\end{pav}

Jei žinotume, kad ieškoma funkcija injektyvi, užtektų įsistatyti $x = 0$ ir
iš lygties $f(f(y))=f(y^2)$ gauti, kad $f(y)=y^2$ su visais $y \in \R$.
Patikrinę pastebėtume, kad ši funkcija netinka, vadinasi sprendinių nebūtų.

\begin{pav}
Funkcija $f:\R \to \R$ tenkina lygtį $f(f(x)+x) = x$. Raskite $f(0)$.
\end{pav}

Jei žinotume, kad $f$ yra surjektyvi funkcija, tai reikštų, kad egzistuoja
toks $a$, kad $f(a)=0$ (kitaip sakant - nulis yra įgyjamas). Įstatę $x
= a$, gautume $f(f(a)+a)= a \Rightarrow f(0+a)=a \Rightarrow 0 =a$,
vadinasi $a = 0$, t.y. $f(0)=0$.

Šį uždavinį galima buvo išspręsti ir kitaip - įsistačius $x=0$ bei
$x=f(0)$, tačiau idėja, kuria pasinaudojome, yra daug bendresnė ir neretai
labai naudinga.

\subsubsection{Gavimas}
\bigskip

Pamačius, kad kartais injektyvumas ir surjektyvumas tikrai yra naudingi,
kyla klausimas, kaip gauti, jog ieškomos funkcijos pasižymėtų šitomis
savybėmis. Pabandykime pasiaiškinti.

\begin{pav}
Funkcija $f:\R \to \R$ tenkina lygtį $f(f(x)) = x$. Įrodykite, kad ji yra
injektyvi ir surjektyvi.
\end{pav}

Injektyvumas. Mums reikia įrodyti, kad jei $f(a)=f(b)$, tai $a=b$. Pasirodo,
tai visai nesudėtinga. Jei $f(a)=f(b),$ tai ir $f(f(a))=f(f(b))$ (funkcija
nuo vienodų argumentų tikrai duoda vienodas reikšmes), bet kadangi
$f(f(a))=a$ ir $f(f(b))=b,$ tai aišku, kad $a=b.$


Surjektyvumas. Mums reikia įrodyti, kad kiekvienam $a$ egzistuoja toks $b$,
kad $f(b)=a$. Bet pažiūrėkime į lygtį dar kartą - jei įstatysime $x=a$
gausime $f(f(a))=a$, t.y. $f$ nuo kažko lygu $a$, vadinasi reikšmė $a$ yra
įgyjama. Šiuo atvėju, žinoma, ieškomas $b$ bus lygus $f(a)$, bet
dažniausiai mums jis nelabai įdomus - pakanka žinoti, kad egzistuoja.

\begin{pav}
Funkcija $f:\R \to \R$ tenkina lygtį $f(x +f(y)) = f(x) + y$. Įrodykite,
kad ji yra injektyvi ir surjektyvi.
\end{pav}

Injektyvumas. Mums reikia įrodyti, kad jei $f(a)=f(b)$, tai $a=b$.
Pasinaudosime laisvu kintamuoju $y$: perrašę lygtį $y=f(x+f(y))-f(x)$
ir vietoje $y$ paeiliui įstatę $a$ ir $b$ gauname, kad dešiniosios pusės
bus vienodos (nes $f(a)=f(b)$), todėl vienodomis turės būti ir kairiosios.

Surjektyvumas. Mums rekia įrodyti, kad funkcija įgyja visas reikšmes.
Laisvas kintamasis $y$ čia taip pat pravers, nes jis gali įgyti bet kokią
reikšmę. Iš pažiūros lyg ir trukdo $f(x)$, bet lengvai galime jį apeiti -
įstatę $x=0$ gausime $f(f(y))=f(0) + y$. Kadangi $f(0)$ yra skaičius, o $y$
įgyja visas reikšmes, tai ir $f(0)+y$ įgyja visas reikšmes. Iš čia jau
aišku, kad ir funkcija jas visas įgis.

\begin{pav}
Funkcijos $f,g:\R \to \R$ tenkina lygtį $f(g(x)) = x$. Įrodykite, kad $g$ yra injektyvi, o $f$ surjektyvi.
\end{pav}

Jei dvi funkcijos vienoje lygtyje neišgąsdina, tai sprendimas akivaizdus.
Injektyvumas - jei $g(a)=g(b)$, tai ir $f(g(a))=f(g(b)) \Rightarrow a=b$. Surjektyvumas
dar paprastesnis, mat kiekvienam $a$ teisinga $f(g(a))=a$, taigi $f$
reikšmę $a$ įgyja.

Taigi, bendru atveju, strategija paprasta. Norėdami įrodyti ieškomos
funkcijos injektyvumą tariame, kad $f(a)=f(b)$, ir statomės $a$ ir $b$ į
lygtį, tikėdamiesi kokiu nors būdu gauti $a=b$. Norėdami įrodyti
surjektyvumą bandome gauti $f$ nuo bet kokio argumento lygią reiškiniui,
kuris gali įgyti visas reikšmes. Abi strategijos yra gana bendros ir
atskiru atveju jas pritaikyti gali būti gana sudėtinga, tad nusiteikite
pakovoti dėl šių naudingų funkcijos savybių.


\subsubsection{Pavyzdžiai}

\begin{pavnr} Raskite visas funkcijas $f:\R \to \R$ su visomis $x$ ir $y$
  reikšmėmis tenkinančias $$f(f(x)+y) = 2x +f(f(y)-x).$$
\end{pavnr}

\begin{sprendimas}
  Įstatykime $y = -f(x)$, gausime, kad su visais $x$ teisinga $f(0) - 2x =
  f(f(y)-x)$, vadinasi, funkcija surjektyvi.  Įrodysime, kad funkcija
  yra ir injektyvi. Jei ji tokia nėra, tai egzistuoja tokie $a$, $b$,
  kad $f(a)=f(b)$ ir $a\neq b$. Įstatykime $y=a$ ir $y=b$, gausime
  $$f(f(x)+a) = 2x +f(f(a)-x),$$ $$f(f(x)+b) = 2x +f(f(b)-x)$$ ir iš čia
  $$f(f(x)+a)=f(f(x)+b).$$ Kandangi funkcija surjektyvi, tai gauname
  $$f(x+a)=f(x+b) \Rightarrow f(x)=f(x+(b-a)).$$ Pažymėję $b-a=r$
  gauname, kad funkcija periodinė su periodu $r \neq 0$. Tačiau įstatę
  $x=y=r$ į pradinę lygtį, gauname $f(f(r))=2r + f(f(r)) \Rightarrow r =
  0$, prieštara.  Gavome, kad funkcija turi būti injektyvi, ir įstatę $x
  = 0$ gauname $f(y) = y + c$.
\end{sprendimas}

\begin{pavnr} Raskite visas funkcijas $f: \R \to \R$ su visais
  realiaisiais $x$, $y$ tenkinančias lygtį $$f(xy+f(x))=xf(y)+f(x).$$
\end{pavnr}

\begin{sprendimas}
  Pabandykime įrodyti, kad funkcija injektyvi. Tam pasinaudosime labai
  elegantiška idėja - sukeisime vietomis kintamuosius:
  $$f(xy+f(y)) = yf(x) + f(y).$$
  Jei tarsime, kad $f(x)=f(y)$, tai gautos ir pradinės lygčių kairiosios
  pusės bus lygios, vadinasi, turės būti lygios ir dešiniosios: 
  $$xf(y) + f(x) = yf(x) + f(y) \implies f(y)(x-1)=f(x)(y-1).$$ 
  Iš čia gauname, kad funkcija visas reikšmes įgyja po vieną kartą,
  išskyrus, galbūt, nulį (nes jei $f(x)\neq 0$, tai $f(x)=f(y)\implies
  x=y$). 
  
  Natūralus sprendimo tęsinys, patyrinėti, kas atsitinka, kai funkcija
  įgyja nulį keliuose taškuose, tad tarkime, kad $f(x_0) = 0$ ir $x_0 \neq
  0$.  Įsistatykime $x=x_0, y=1$, gausime $f(1)=0$. Įstatę $y=1$, gausime
  $f(x+f(x))=f(x)$. Jei kokiam nors taške $f(x)\neq 0$, tai, kaip jau
  žinome, tame taške funkcija yra injektyvi, bet tada $x + f(x) = x \implies f(x)=0$ -
  prieštara. Vadinasi, jei funkcija įgyja reikšmę $0$ ne tik nulyje, tai ji
  tapačiai lygi nuliui.  
  
  Liko išnagrinėti atvejį, kai funkcija nulį įgyja tik nulyje. Tuomet
  žinome, kad funkcija injektyvi. Įstatę $x=0$ gauname $f(f(0))=f(0)
  \implies f(0)=0$, įstatę $y=0$ gauname $f(f(x))=f(x) \implies f(x)=x$.
\end{sprendimas}
 
\subsubsection{Uždaviniai}

\begin{enumerate}
  \item Įrodykite, kad griežtai didėjanti funkcija yra injektyvi. Ar
    būtinai bijektyvi funkcija turi būti monotoniška?
    %Jei funkcija griežtai didėjanti, tai visiems skirtingiems $a>b$
    %turėsime $f(a)>f(b)$, todėl funkcija neįgis vienodų reikšmių. 
    %
    %Bijektyvi funkcija nebūtinai turi būti monotoniška. Pavyzdžiui,
    %$f(x)=\frac{1}{x}$, kai $x\neq 0$, ir $f(0)=0$.
  \item Ar funkcija $f:\R \to \R$ tenkinanti lygtį $f(x+y) = f(x^2) +
    f(y^2)$ su visais $x,y \in \R$ gali būti injektyvi?
    %Negali, nes, pavyzdžiui, įstatę $x=0$ matome, kad $f(y)=f(-y)$ su visais $y$.
  \item Įrodykite, kad funkcija $f:\R\to\R$ tenkinanti lygtį $f(x+y) =
    xf(y^2) + yf(x^2)$ yra nelyginė.
    %Įstatę $x=-x$ ir $y=-y$ gauname $f(x+y) = -f(-x-y)\Rightarrow
    %f(t)=-f(-t)$ $\forall t \in \R.$
  \item Raskite visas lygines monotonines ir lygines injektyvias funkcijas.
    Raskite bent vieną lyginę surjektyvią funkciją.
    %Lyginės monotoninės yra tik $f(x)=c$, $f:A\to \R$, lyginė injektyvi
    %tik $f(0)=c$, $f:\{0\}\to \R$ (jei dar bent viena taškų pora
    %priklausytų apibrėžimo sričiai iš karto gautume neinjektyvią). Lyginės
    %surjektyvios pavyzdys gali būti $f(x)=ln|x|$, kai $x\neq 0$, $f(0)=0$.
  \item Žinome, kad $f:\R^+ \to \R^+$ tenkina $f(x)\leq 1$ su visais $x\in
    \R^+$ ir $f(x+y)f^2(y)=f(x)$. Įrodykite, kad $f$ didėjanti.  ($\R^+$
    čia ir toliau žymi teigiamus realiuosius)
    %Tegu $a>b$. Įstatę $x =b$, $y = a-b$ gausime $f(a)f^2(a-b)=f(b)$.
    %Kadangi $f(a-b)^2\leq 1$, tai $f(a)\geq f(b)$.
  \item Funkcija $f:\R \to \R$ tenkina lygtį $f(xf(x)) = x$ ir yra
    surjektyvi. Raskite $f(1)$.
    %Jei žinome, kad funkcija yra surjektyvi, tai egzistuoja toks $a$, kad
    %$f(a)=1$. Įstatę gauname $f(a\cdot 1) =a \Rightarrow 1 = a$, vadinasi
    %$f(1)=1$.
  \item Raskite visas didėjančias funkcijas $f:\R\to\R$ tenkinančias lygtį
    $f(f(x))=x$.
    %Įrodysime, kad $f(x)=x$. Tarkime priešingai - tegu egzistuoja toks
    %$a$, kad $f(a)>a$. Tuomet, kad kadangi $f$ yra didėjanti, tai $$f(a)>a
    %\Rightarrow f(f(a))>f(a) \Rightarrow f(f(a))>f(a)>a \Rightarrow
    %f(f(a))\neq a \text { - prieštara}.$$ Tardami, kad $f(a)<a$, prieštarą
    %gauname analogiškai.
  \item Tegu $f:\R \to \R$ yra injektyvi ir su visais $x$ tenkina
    $$f(x)f(1-x)=f(ax+b).$$ Įrodykite, kad $a=0$, $f(1-b)=1$ ir kad $f$
    nėra surjektyvi.
    %Įstatę $x=0$ ir $x=1$ gauname, kad $f(a+b)=f(b)$, todėl pasinaudoję injektyvumu gauname $a = 0$.
    %Įstatę $x=b$ gauname $f(b)f(1-b)=f(b)$, tad arba $f(1-b)=1$, arba
    %$f(b)=0$. Tačiau $f(b)$ negali būti lygus nuliui, nes gautume
    %$f(x)f(1-x)=0$, o iš čia be galo daug reikšmių, su kuriomis funkcija
    %lygi nuliui, kas prieštarauja injektyvumui.
    %Galiausiai pastebėkime, kad funkcija nulio iš vis neįgyja, nes jei,
    %tarkime, $f(c)=0$, tai įstatę gauname $f(b)=0$, ko negali būti. Taigi
    %ji nėra surjektyvi.
  \item Raskite visas griežtai didėjančias funkcijas $f:\R\to\R$ su visais
    $x,y \in \R$ tenkinančias lygybę $$f(x+f(y))=f(x+y)+2005.$$
    %Įsistatykime $x=y$, $y=x$, gausime $f(x+f(y))=f(y+f(x))$. Kadangi $f$
    %yra griežtai didėjanti, tai ji injektyvi, tai $x+f(y)=y+f(x)
    %\Rightarrow f(x)=x+c$. Įstatę randame $c=2005$.
  \item Raskite visas funkcijas $f:\R^+\to\R^+$ su visais $x, y \in \R^+$
    tenkinančias lygybę $$(x+y)f(f(x)y)=x^2f(f(x)+f(y)).$$
    %Įsistatykime $x=y$, gausime $(y+y)(f(y)y)=y^2f(f(y)+f(y))$. Jei
    %$f(x)=f(y)$, tai iš abiejų lygybių gauname $\frac{x^2}{x+y} =
    %\frac{y^2}{y+y} \Rightarrow x=y$. Gavome kad funkcija injektyvi.
    %Įstatykime $y=1$ ir $x=\frac{1+\sqrt{5}}{2}=\lambda$, t.y. lygties
    %$x+1=x^2$ sprendinį. Tuomet gausime, kad
    %$f(f(\lambda))=f(f(1)+f(\lambda))\Rightarrow f(1)=0$, o taip būti
    %negali.
  \item Raskite visas funkcijas $f:\R^+\to\R^+$ su visais $x, y \in \R^+$
    tenkinančias lygybę $$(x+y)f(yf(x))=x^2(f(x)+f(y)).$$
    %Nesunku pastebėti, kad funkcija yra injektyvi. Įsistatę $x=1,y=1$
    %gauname $f(f(1))=f(1) \implies f(1)=1$. Įsistatę $x=1$ gauname
    %$(y+1)f(y)=f(y)+1 \implies f(y)=\frac{1}{y}.$
  \item Raskite visas funkcijas $f,g:\R\to\R$, kur $g$ yra bijekcija ir
    kurios su visais realiaisiais $x$, $y$ tenkina $$f(g(x)+y)=g(f(y)+x).$$
    %Kadangi $g$ yra surjektyvi ir $f(y)+x$ įgyja visas realiąsias reikšmes,
    %tai iš lygybės gauname, kad ir $f$ surjektyvi. Tegu $a$ toks, kad
    %$g(a) = 0$. Įstatę $x=a$ gauname $f(y)=g(f(y)+a)$. Kadangi $f$
    %surjektyvi, tai $g(x)=x-a$ su visas $x\in \R$. Įstatę $g$ išraišką į
    %pradinę lygybę gauname, kad $f(x+y-a)=f(y)+x-a$. Įstatę $y=a$ gauname
    %$f(x)=x+b$. Vadinasi, sprendiniai yra $g(x) =x+a$, $f(x)=x+b$, kur $a,b
    %\in \R$.
  \item Raskite visas funkcijas $f:\R\to\R$ su visais $x, y \in \R$
    tenkinančias lygybę $$f(x+y +f(xy)) = f(f(x+y)) + xy.$$
    %Įrodykime, kad $f$ injektyvi. Naudodami keitinį $x+y =a$, $xy =b$
    %gauname lygtį $f(a+f(b))=f(f(a)) + b$.  Tačiau ji galioja ne visiems
    %$a$ ir $b$, o tik tenkinantiems sąlygą $4b\leq a^2$, nes kitaip
    %sistema $x+y =a$, $xy =b$ neturi sprendinių.  Bet tai ne bėda -
    %kiekvieniem $b_1$ ir $b_2$ galime paimti $a$ tokį, kad $4b_1 \leq a^2$
    %ir $4b_2 \leq a^2$. Tuomet galime naudotis lygtimi ir iš
    %$f(b_1)=f(b_2)$ gauname $b_1 = b_2$ - injektyvumas įrodytas.
    %Įstatykime į pradinę lygtį $y=0$, gausime $f(x + f(0))=f(f(x))$, iš
    %injektyvumo $f(x)=x+c$.
  \item Įrodykite, kad nėra funkcijų $f,g:$ $\R \to \R$ su visais
    realiaisiais $x$ tenkinančių lygybes $$g(f(x)) = x^3 \text{ ir }
    f(g(x))=x^2.$$
    %Iš lygybės $g(f(x))=x^3$ seka, kad $f$ yra injektyvi ir kad
    %$f(g(f(x)))=f(x^3)\Rightarrow f^2(x)=f(x^3)$.  Įsistatę $x=-1,0,1$
    %gauname, kad $f(-1)$, $f(0)$ ir $f(1)$ gali įgyti tik reikšmes $0$
    %arba $1$, kas prieštarauja injektyvumui.
  \item Tegu funkcija $f: \R \to \R$ su visais realiaisiais $x$ tenkina
    lygtį $$4f(f(x))= 2f(x)+x$$ Įrodykite, kad $f(x)=0$ tada ir tik tada,
    kai $x=0$.
    %Pastebėkime, kad $f$ injektyvi. Įstatykime $x = 0$, gausime
    %$f(f(0))=\frac{f(0)}{2}$.  Įstatykime $x = f(0)$, gausime $4f(f(f(0)))
    %= 2f(f(0)) + f(0) = f(0) + f(0) = 2f(0)$, iš kur
    %$f(f(f(0)))=\frac{f(0)}{2} = f(f(0))$. Naudodamiesi injektyvumu
    %gauname $$f(f(f(0)))=f(f(0)) \Rightarrow f(f(0)) = f(0) \Rightarrow
    %f(0) = 0.$$ Kadangi funkcija injektyvi, tai išties $f(x)=0
    %\Leftrightarrow x=0$.
  \item Raskite visas funkcijas $f:\R^+\to\R^+$ su visais $x, y \in \R^+$
    tenkinančias lygybę $$f(x)f(yf(x))=f(x+y).$$
    %Patyrinėkime keitinį $y=\frac{x}{f(x)-1}$. Iš pradžių gali pasirodyti,
    %kad jis yra nuleistas iš dangaus, bet viskas daug paparasčiau - jis
    %tiesiog kyla iš natūralaus noro sulyginti $f(yf(x))$ ir $f(x+y)$
    %($yf(x)=x+y\implies y=\frac{x}{f(x)-1}$). Tačiau prisiminkime, kad
    %funkcijos apibrėžimo sritis yra teigiami skaičiai. Tuomet tenka
    %samprotauti taip: jei egzistuoja toks $x$, kad $f(x)>1$, tai galime
    %įsistatyti $y=\frac{x}{f(x)-1}$ ir gausime
    %$f(x)f(\frac{xf(x)}{f(x)-1})=f(\frac{xf(x)}{f(x)-1}) \Rightarrow
    %f(x)=1$ - prieštara!
    %
    %Vadinasi gavome, kad $f(x)\leq 1$ ir, iš pradinės lygybės, $f$ yra
    %nedidėjanti ($f(x+y)=f(x)f(yf(x))\leq f(x)$).
    %
    %Nagrinėkime injektyvumą: Jei egzistuoja tokie $a < b$, kad
    %$f(a)=f(b)$, tai gauname, kad $f(a+y)=f(b+y)$ su visais $y$, todėl
    %$f(y)=f(b-a+y)$ su visais $y>a$, vadinasi, funkcija yra monotoniška ir
    %periodinė $\Rightarrow f(x)= c$ su visais $x >a$. Įsistatę į pradinę
    %lygtį pakankamai didelius $x$ ir $y$ gauname $c = 1$, o įsistatę $x$
    %pakankamai didelį gauname $f(y)=1$ su visais $y$.
    %Lieka atvejis, kai funkcija yra injektyvi. Pakeitę $y=\frac{z}{f(x)}$
    %gausime $f(x)f(z)=f(x+\frac{z}{f(x)})$ su visais $z,x >0$. Sukeitę $x$
    %ir $z$ vietomis bei pasinaudoję injektyvumu gauname $x +
    %\frac{z}{f(x)} = z + \frac{x}{f(z)}$, iš kur lengvai randame
    %$f(x)=\frac{1}{1+cx}$, kur $c\in \R^+$.
   \item Raskite visas funkcijas $f:\R^+ \to \R^+$ tenkinančias lygybę
    $$f(x+yf(x))=f(x)f(y), \forall x,y \in \R^+,$$ jei žinome, jog
    egzistuoja tik baigtinis skaičius tokių $x \in \ \R^+$, kad $f(x)=1$.
    %Tegu $f(x_0)=1$, tada įsistatę $x=x_0$ gauname $f(x_0+y)=f(y)$,
    %vadinasi, funkcija yra periodinė ir vienetą įgis be galo daug kartų, o
    %to būti negali, vadinasi, $f(x)\neq 1, \forall x\in \R^+$.
    %
    %Tegu $f(a)=f(a+b)$, tuomet įsistatykime $x=a, y=\frac{b}{f(a)}$,
    %gausime $1=f(\frac{b}{f(a)})$, prieštara, vadinasi funkcija injektyvi.
    %
    %Pradinėje lygtyje įstatykime $x=y$, $y=x$, gausime
    %$f(x+yf(x))=f(x)f(y) =f(y+xf(y))$. Kadangi $f$ injektyvi, tai
    %$x+yf(x)=y+xf(y) \rightarrow f(x)=kx+1$. Patikrinę matome, kad tinka.
  \item Raskite visas funkcijas $f: \R \to \R$ su visais realiaisiais $x$,
    $y$ tenkinančias lygtį $$f(y)+f(x+f(y))= y + f(f(x)+f(f(y))).$$
    %Pastebėkime, kad $f$ injektyvi.  Įstatę $x=0$, $y=0$ ir pažymėję
    %$f(0)+f(f(0))=u$ gauname $f(u)=u$.  Įstatę $y=u$ gauname
    %$f(x+u)=f(f(x)+u) \Rightarrow f(x)+u=x+u \Rightarrow f(x)=x$.
  \item Raskite visas funkcijas $f: \R \to \R$ su visais realiaisiais $x$,
    $y$ tenkinančias lygtį $$f(f^2(x)+f(y))=xf(x)+y.$$
    %Funkcija akivaizdžiai bijektyvi, todėl egzistuoja toks $x_0$, kad
    %$f(x_0)=0$. Įsistatę $x=x_0$ gauname $f(f(y))=y$.  Įsistatę $x=f(x)$
    %gauname $f(x^2+f(y))=xf(x)+y=f(f^2(x)+f(y))$. Kadangi $f$ injektyvi,
    %tai $f^2(x)=f(x)^2$. Vadinasi, kiekviename taške $x$ funkcija lygi arba
    %$x$, arba $-x$. Tegu egzistuoja du nenuliniai taškai, kuriuose $f(x)=x$
    %ir $f(y)=-y$. Tuomet gauname $f(x^2+y)=x^2-y$, kas yra neįmanoma
    %($x^2+y = x^2-y\implies y=0$, $-x^2-y = x^2-y\implies x=0$ ).
    %Vadinasi, tinka tik $f(x)=x$ ir $f(x)=-x$.
  \item Raskite visas funkcijas $f: \R \to \R$ su visais realiaisiais $x$,
    $y$ tenkinančias lygtį $$f(xf(x)+f(y))=f^2(x)+y.$$
    %Funkcija bijektyvi. Įstatykime $x=0$ ir $x=a$, kur $a$ toks, kad
    %$f(a)=0$. Gausime lygybes $f(f(y))=f^2(0) + y$ ir $f(f(y))=y$, iš kur
    %$f(0) = 0$ ir $a=0$.
    %
    %Įstatykime $x = f(x)$ ir pasinaudokime lygybe $f(f(x))=x$. Gausime
    %$f(f(x)x + f(y))=x^2 + y$, vadinasi $f^2(x)=x^2$ su visais $x$.
    %
    %Tegu $x$ ir $y$ tokie, kad $f(x)=x$ ir $f(y)=y$ ir $x$,$y$ $\neq 0$.
    %Tada iš pradinės lygties gauname $f(x^2 - y) = x^2 + y$. Kadangi
    %$f(x^2 - y)$ gali būti lygus tik $x^2 -y$ arba $y-x^2$, tai gauname,
    %kad arba $y=0$ arba $x=0$ - prieštara. Vadinasi, sprendiniai yra
    %$f(x)=x$ ir $f(x)=-x$.
  \item Raskite visas funkcijas $f,g,h:\R\to\R$, kurios su visais
    realiaisiais $x$, $y$, $z$ tenkina $$f(h(g(x))
    +y)+g(z+f(y))=h(y)+g(y+f(z))+x.$$
    %Įstatę $y=z=0$ gauname $f(h(g(x)))=x+h(0)$. Įstatę $y=0$ gauname
    %$g(z+f(0))=g(f(z))$. Kadangi $g$ injektyvi (atkreipkite dėmesį į
    %kintamąjį $x$ pradinėje lygtyje), tai $f(x)=x+a$.
    %
    %Įsistatę gauname lygtį $h(g(x)) + y + a = h(y) + x$, iš kurios
    %akivaizdžiai $h(x)=x+b$, ir $g(x)=x-a$.
 \item Raskite visas funkcijas $f: \R \to \R$ su visais realiaisiais $x$,
    $y$ tenkinančias lygtį $$f(x^2 + xy +f(y))=f^2(x)+xf(y)+y.$$
    %Funkcija injektyvi. Raskime $f(0)$: $x=0 \Rightarrow f(f(y))=y +
    %f^2(0) \Rightarrow f(f(0))=f^2(0)$. Pažymėkime $f(0)=a$, tuomet
    %paskutinioji lygybė pavirsta į $f(a)=a^2$. Įstatykime $x=0$, $y=a$ ir
    %$x=a$, $y=0$. Gausime $f(a^2)=a^2+a$ ir $f(a^2+a)=a^4+a^2$. Iš čia
    %$f(f(a^2))=f(a^2+a)\implies 2a^2 = a^4 + a^2 \implies a=-1$,$0$
    %arba $1$.
    %
    %Jei $f(0)=1$, tai tuomet iš $f(f(y))=y + f^2(0)$ gauname $f(1)=1$ -
    %prieštara injektyvumui.
    %
    %Jei $f(0)=-1$, tai iš $f(f(y))=y + f^2(0)$ gauname $f(-1)=1
    %\Rightarrow f(1)=0 \Rightarrow f(0)=2$ - prieštara.
    %
    %Vadinasi, $f(0)=0$. Tuomet $f(f(x))=x$ ir $f(x^2)=f^2(x)$. Įstatę
    %$x=-y$ gauname $f(f(y))=f^2(-y) -yf(y) + y \Rightarrow y = f((-y)^2) -
    %yf(y) + y\Rightarrow f^(y)=yf(y)$. Kadangi funkcija injektyvi, tai
    %$f(y)=0$ tik kai $y=0 \Rightarrow f(y)=y$.
  \item Raskite visas funkcijas $f: \R \to \R$ su visais realiaisiais $x$,
    $y$ tenkinančias lygtį $$f(f(x)-f(y))=(x-y)^2f(x+y).$$
    %$f(x)=0$ yra sprendinys, ieškosime likusių. Įrodykime, kad $f$ turi
    %būti lyginė. Pastebėkime, kad $f(f(x)-f(y))=f(f(y)-f(x))$, todėl
    %užtenka įrodyti, kad $f(x)-f(y)$ įgyja visas reikšmes. Išties, tegu
    %$a$ toks, kad $f(a)\neq 0$. Įstatykime $y= a-x \implies
    %f(f(x)-f(a-x))=(2x-a)^2f(a)$. Iš čia matome, kad $f$ įgyja visas
    %teigiamas arba visas neigiamas reikšmes (priklausomai nuo $f(a)$),
    %vadinasi, $f(x)-f(y)$ tikrai įgyja visas realiąsias reikšmes.
    %
    %Įstatykime $y=-y$. Gausime $f(f(x)-f(y))=(x+y)^2f(x-y) \implies
    %(x-y)^2f(x+y)=(x+y)^2f(x-y)$. Kadangi visiems realiesiems $a$, $b$
    %egzistuoja tokie $x$,$y$, kad $x+y=a$ ir $x-y=b$, tai lygtį galime
    %užrašyti $b^2f(a)=a^2f(b) \implies f(x)=cx^2$. Patikrinę gauname
    %$c=1$, vadinasi, sprendiniai yra $f(x)=x$ ir $f(x)=0$.
  \item Raskite visas funkcijas $f,g:\R\to\R$, kurios su visais
    realiaisiais $x$, $y$ tenkina $$f(xg(y+1))+y=xf(y) + f(x+g(y))$$ ir
    $$f(0)+g(0)=0.$$
    %Įstatykime $x = 0$, gausime $f(0)+y=f(g(y))$, vadinasi $f$ surjektyvi,
    %$g$ injektyvi.
    %
    %Įrodykime, kad $g(1)=1$. Įstatykime $y = 0$, gausime $f(xg(1)) = xf(0)
    %+ f(x + g(0))$. Jei $g(1)\neq 1$, tai galime sulyginti $xg(1)=x+g(0)$
    %paėmę $x = \frac{g(0)}{g(1)-1}$. Tuomet gauname
    %$\frac{g(0)f(0)}{g(1)-1}=0 \Rightarrow f(0)=g(0)=0$ (pasinaudojus
    %antrąja sąlyga). Įsistatę $y = -1$ gauname $f(x)=ax$ ir
    %$g(x)=\frac{x}{a}$. Patikrinę gauname, kad $a =1$, taigi $f(x)=g(x)=x$
    %- prieštara prielaidai $g(1)\neq 1$.
    %
    %Iš $f$ surjektyvumo žinome, kad egzistuoja toks $u$, kad $g(u)=0$.
    %Įrodykime, kad $u = 0$.  Tegu $u \neq 0$, tada $g(u+1)\neq g(1)=1$ (iš
    %$g$ injektyvumo).  Įstatykime $x = \frac{g(u)}{g(u+1)-1}$ ir $y = u$,
    %gausime $u = 0$ - prieštara.
    %Taigi gavome, kad $f(0)=0$ ir $g(0)=0$, ir iš čia jau žinome, kad
    %gaunasi $f(x)=g(x)=x$.
   \item Raskite visas funkcijas $f:\R\to\R$, kurios su visais realiaisiais
    $x$, $y$ tenkina $$f(x^2 + f(y)) = y + xf(x).$$
    %Įstatykime $x = 0$. Gausime $f(f(y))=y$. Įstatykime $x=f(x)$, gausime
    %$f(f^2(x) + f(y)) = y + f(x)x = f(x^2 + f(y))$. Kadangi funkcija
    %tenkinanti lygtį yra akivaizdžiai bijektyvi, tai gauname
    %$f^2(x)=x^2 \Rightarrow f(x)=\pm x$.
    %
    %Tegu $x$ ir $y$ tokie, kad $f(x)=x$ ir $f(y)=-y$ bei $x$,$y$ $\neq 0$.
    %Tada pradinė lygtis tampa $f(x^2 - y)=y+x^2$. Kadangi $f(x^2-y)$ =
    %$x^2 - y$ arba $f(x^2-y) = y -x^2$, tai gauname $y=0$ arba $x = 0$ -
    %prieštara. Vadinasi, sprendiniai yra tik $f(x)=x$ ir $f(x)=-x$.
  \item \text{[IMO 1992]} Raskite visas funkcijas $f:\R\to\R$, kurios su
    visais realiaisiais $x$, $y$ tenkina $$f(x^2 + f(y)) = y + f^2(x).$$
    %Funkcija bijektyvi, todėl egzistuoja toks $a$, kad $f(a)=0$. Įsistatę
    %$x=y=a$ gauname $f(a^2)=a \Rightarrow f(f(a^2))=0$. Tačiau kadangi
    %$f(f(y))=y + f^2(0)$, tai $a^2+f^2(0)=0 \Rightarrow a=0$ ir $f(0)=0$.
    %
    %Tuomet iš pradinės lygties gauname, kad $f(x^2)=f^2(x)=f(-x)^2$. Dėl
    %injektyvumo $f(x)\neq f(-x)$, todėl $f(x)=-f(-x)$. Iš čia ir iš
    %$f(x^2)=f^2(x)$ gauname, kad $f(x)>0$, kai $x>0$ ir $f(x)<0$, kai
    %$x<0$.
    %
    %Galiausiai įstatę $y=-x^2$ gauname, kad $f(x^2-f^2(x))=-(x^2-f^2(x))
    %\Rightarrow f^2(x)=x^2 \Rightarrow f(x)=x$.
  \item Raskite visas funkcijas $f:\R \to \R$ su visomis $x$ ir $y$
    reikšmėmis tenkinančias $$f(x +f(xy))=f(x+f(x)f(y))=f(x)+xf(y).$$
    %$f(x)=0$ yra sprendinys, nagrinėkime galimus likusius. Tegu $f(a)=0$,
    %tuomet įstatę $x=a$ gausime $0=af(y) \implies a=0$, vadinasi, jei
    %$0$ yra įgyjamas, tai tik taške $0$. Įstatę $x=y=-1$ gausime
    %$f(f(1)-1)=0 \implies f(0)=0, f(1)=1$.
    %
    %Įstatę $x=1$ gauname $f(f(y)+1)=f(y)+1(*)$ iš kur $f(n)=n$ visiems
    %$n\in \N$. Įrodysime, kad $f$-injektyvi. Pažymėję $xy=a$ gauname
    %$f(x+f(a))=f(x)+xf(\frac{a}{x})$. Jei $f(a)=f(b)$, tai visiems $x$
    %teisinga $f(\frac{a}{x}) = f(\frac{b}{x})$. Pakeitę $x=\frac{b}{y}$ ir pažymėję
    %$\frac{a}{b}=m$, gausime, kad su visomis $y$ reikšmėmis $f(ym)=f(y)$. Iš čia
    %randame $f(m)=1$. Įstatę $x=m$ gauname $f(m + f(y))=1+mf(y)$, iš kur
    %$f(m+1)=m+1$ $(y=1)$ ir $f(m+2)=1+2m$ $(y=2)$. Tačiau pagal (*)
    %$f(m+2)=m+2$ $(y=m+1)$, todėl $1+2m = m+2 \implies m=1 \implies
    %a=b \implies f$ injektyvi.
    %
    %Įstatykime $x=y=-2 \implies f(-2)=-2 \implies f(-1)=-1$.
    %Įstatykime $x=-1 \implies f(-1+f(-y))=-1-f(y)$.
    %
    %Naudodamiesi $f(-1+f(-y))=-1-f(y)$ ir $f(f(y)+1)=f(y)+1$ gausime, kad
    %kiekvienam $x$ egzistuoja toks $y$, kad $f(x)=f(y)+1$. Išties: jei
    %$a$ priklauso $f$ vaizdui $\implies -1 -a$ priklauso vaizdui
    %$\implies -a$ priklauso vaizdui $\implies -1 + a$ priklauso
    %vaizdui. Kadangi kiekvienam $x$ $f(x)$ priklauso vaizdui, tai $f(x)-1$
    %priklauso vaizdui, todėl egzistuoja toks $y$, kad $f(y)=f(x)-1$.
    %Įstatę į $f(f(y)+1)=f(y)+1$ gauname, kad kiekvienam $x$
    %$f(f(x))=f(x)$. Kadangi $f$ injektyvi, tai $f(x)=x$.
  \item Raskite visas funkcijas $f:\R\to\R$, kurios su visais realiaisiais
    $x$, $y$ tenkina $$f(xf(y))+f(yf(x))=2xy.$$
    %Pastebėkime, kad $f(0)=0$ ir $f(xf(x))=x^2$(*). Įstatę $x=1$ gauname
    %$f(f(1))=1$, įstatę $x=f(1)$ gauname $1=f(1)^2$. Jei $f(1)=1$, tai
    %$f(x)+f(f(x))=2x$ - $f$ injektyvi. Jei $f(1)=-1$, tai įstatę $x=y=1$
    %gauname $f(-1)=1$ ir įstatę $y=-1$ gauname $f(x)+f(-f(x))=-2x$ - $f$
    %injektyvi.
    %
    %Įrodysime, kad $f(\frac{1}{x})= \frac{1}{f(x)}$. Įstatykime
    %$y=f(\frac{1}{x})\frac{1}{x}$: $$f(xf(f(\frac{1}{x})\frac{1}{x}))+
    %f(f(\frac{1}{x})\frac{1}{x}f(x))=2f(\frac{1}{x}).$$
    %Iš (*) gauname, kad $$f(f(\frac{1}{x})\frac{1}{x}) = \frac{1}{x^2},$$
    %todėl lygybę galime perrašyti į
    %$$f(f(\frac{1}{x})\frac{1}{x}f(x))=f(\frac{1}{x}).$$ Lieka pasinaudoti
    %injektyvumu ir gauname $f(x)=\frac{1}{x}$.
    %
    %Jei $f(1)=1$, tai įstatę $x=\frac{1}{x}$ į $f(x)+f(f(x))=2x$ gauname
    %$$\frac{1}{f(x)} + \frac{1}{f(f(x))}= \frac{2}{x} \Rightarrow \\
    %\frac{1}{f(x)} + \frac{1}{2x - f(x)}= \frac{2}{x} \\ \Rightarrow
    %(f(x)-x)^2=0 \Rightarrow f(x)=x.$$
    %Jei $f(-1)=-1$, tai $x = \frac{1}{x}$ statome į $f(x)+f(-f(x))=-2x$ ir
    %analogiškai gauname $f(x) = -x$.
%%noparse
  \item *Raskite visas funkcijas $f:\R\to\R$, kurios su visais realiaisiais
    $x$, $y$ tenkina $$f(xf(y))=(1-y)f(xy)+x^2y^2f(y).$$
  \item *Raskite visas funkcijas $f:\R^+\to\R^+$, kurios su visais
    realiaisiais $x$, $y$ tenkina $$f(x+f(y))=f(x+y)+f(y).$$
  \item *[Japan 2008] Raskite visas funkcijas $f: \R \to \R$ su visais
    realiaisiais $x$, $y$ tenkinančias lygtį
    $$f(x+y)f(f(x)-y)=xf(x)-yf(y).$$
  \item *Raskite visas funkcijas $f: \R \to \R$ su visais realiaisiais $x$,
    $y$ tenkinančias lygtį $$f(x+y+f(xy))=xy+f(x+y).$$
  \item *[Brazil 2006] Raskite visas funkcijas $f:\R\to\R$, kurios su
    visais realiaisiais $x$, $y$ tenkina $$f(xf(y)+f(x))=2f(x)+xy.$$
    %%Long Solution in "Nice funct.equation" + "Yet another (but nice!) functional equation" + "f.e.9"
  \item *[Dan Barbilian 2005] Tegu $f:\R^+ \to \R^+$ yra nelygi konstantai
    funkcija su visais $x,y,z \in \R^+$ tenkinanti
    $$f(x)f(yf(x))f(zf(x+y))=f(x+y+z)$$ Įrodykite, kad $f$ yra injektyvi ir
    raskite visas tokias funkcijas.
    %%Very long solution in "funcional eq('Barbilian' contest)"
  \item *Raskite visas funkcijas $f:\R^+\to\R^+$, kurios su visais
    teigiamais $x$, $y$ tenkina $$f\left(\frac{f(x)}{yf(x) + 1}\right) =
    \frac{x}{xf(y)+1}.$$
    %%Sudėtingas sprendimas "Nested fraction function equation"
%%parse
\end{enumerate}

\newpage
\subsection{Cauchy funkcinė lygtis}

Sprendžiant sudėtingas funkcines lygtis dažnai susiduriama su lygtimi:
$$f(x+y)=f(x)+f(y).$$
Ši lygtis vadinama Cauchy funkcine lygtimi. Ją nesudėtinga išspręsti jei
ieškosime funkcijų, kurių apibrėžimo sritis racionalieji skaičiai. Tą ir
padarykime:

\begin{thm}
  Jei $f:\Q \to \R$ tenkina lygtį $f(x+y)=f(x)+f(y)$ su visais
  racionaliaisiais $x$ ir $y$, tai $f$ - tiesinė, t.y. $f(q)=kq$ visiems $q
  \in Q$, kur $k\in \R$ - konstanta.
\end{thm}

\begin{proof}[Įrodymas]
  Įstatę $y=x$, gausime $f(2x)=2f(x)$. Įstatę $y=2x$, gausime $f(3x)=3f(x)$.
  Taip tęsdami toliau, po nesudėtingos indukcijos turėsime
  $$f(nx)=nf(x).$$
  Į šią lygybę įstatę $x=\frac{1}{n}$ gausime
  $\frac{f(1)}{n}=f(\frac{1}{n})$. Tada, pradinėje lygtyje imdami
  $x=\frac{1}{n}$, o $y=\frac{1}{n},\frac{2}{n},\frac{3}{n}$ ir t.t., vėl po
  paprastos indukcijos išreikšime:
  $$f(\frac{m}{n})=mf(\frac{1}{n})=f(1)\frac{m}{n},$$ kur $m$ ir $n$ -
  bet kokie natūralieji skaičiai, vadinasi, $\frac{m}{n}$ - bet koks teigiamas
  racionalusis. Tada, pažymėję $f(1)=k$, gausime $$f(q)=kq,$$ kur $k$ -
  realioji konstanta, o $q$ - bet koks teigiamas racionalusis. Kita vertus,
  pradinėje lygtyje paėmę $y=0$ ir $y=-x$, gausime $f(0)=0$ ir $f(-x)=-f(x)$
  $\forall x\in\R$, taigi, $f(q)=kq$ bus lygties sprendinys ir neigiamiems
  racionaliesiems.
\end{proof}

Deja, jei pradinę salygą $f:\Q\rightarrow\R$ pakeisime į
$f:\R\rightarrow\R$, tai Cauchy funkcinę lygtį išspręsti pasidarys labai
sudėtinga. Racionaliesiems skaičiams ir toliau galios $f(q)=kq$, tad būtų
visai natūralu manyti, kad  $f(x)=kx$ visiem realiesiems $x$, tačiau
įrodyta, kad egzistuoja begalybė labai neelementarių, netiesinių
sprendinių. Jų egzistenciją priimsime be įrodymo ir žvilgtelsime į labai
svarbią šių sprendinių savybę:

\begin{thm}
  Tarkime, turime funkciją $f:\R\to\R$, kuri tenkina Cauchy funkcinę lygtį ir
  $f(q)=q$ visiems $q\in\Q$, o kažkokiam $\alpha\in\R: f(\alpha) \neq
  \alpha$. Duoti trys skaičiai $x, y, r \in \Q$, $r>0$, $x\neq y$.  Jei $(x,
  y)$ pažymėsime apskritimo centro kordinates, o $r$ - jo spindulį, tai
  nesvarbu, kokius $x, y, r$ parinksime, tame apskritime visados galėsime
  rasti funkcijos $f$ grafiko tašką.
\end{thm}

\begin{proof}[Įrodymas]
  Tarkime, kad $f(\alpha)=\alpha+\delta,$ $\delta\not=0$. Pažymėkime
  $\beta=\frac{y-x}{\delta}$. Aišku, kad įmanoma pasirinkti tokį racionalų
  skaičių $b\not=0$, kad: $|\beta-b|<\frac{r}{2|\delta|}$, ir tokį racionalų
  skaičių $a$, kad: $|\alpha-a|<\frac{r}{2|b|}$. Pažymėkime $X=x+b(\alpha-a)$. Tada
  \begin{align*}
    f(X) & =f(x+b(\alpha-a))\\
    & = x+bf(\alpha)-bf(a)\\
    & = y-\delta\beta+b(\alpha+\delta)-ba\\
    & = y+b(\alpha-a)-\delta(\beta-b).
  \end{align*}
  Aišku, kad $x-r<X<x+r$ ir $y-r<f(X)<y+r$, todėl taškas $(X,f(X))$ bus mūsų
  apskritimo viduje.
\end{proof}

\begin{pastaba}
  Nors teoremą įrodėme tik atveju, kai $f(q)=q$ visiems $q \in \Q$, nesunku
  įsitikinti, kad teorema galios ir bendru atveju, kai $f(q)=kq$.
\end{pastaba}

Jei sugebėtume nupiešti Cauchy lygties netiesinio sprendinio grafiką, tokio
grafiko taškų galėtume rasti, kur tik sugalvotume, visoje begalinėje
plokštumoje - išties labai žavu ir gražu, bet taip pat aišku, kad rimtai
sprendžiant uždavinius, geriau su šiais sprendiniais neprasidėti. Jei
turime funkciją iš realiųjų į realiuosius ir lygtis susiveda į Cauchy
lygtį, reikia ieškoti kažkokių papildomų sąlygų, kurios leistų atmesti
,,žaviuosius'' Cauchy lygties sprendinius.

\subsubsection{Papildomos sąlygos}
Naudodamiesi paskutiniąja teorema nesunkiai galime sugalvoti keletą sąlygų,
leisiančių atmesti imantriuosius netiesinius sprendinius. Tarkime,
$f:\R\rightarrow\R$ - funkcija visiems realiesiems tenkinanti Cauchy
funkcinę lygtį. Tada:

\begin{thm}
  Jei egzistuoja intervalas $(a,b)$, kuriame funkcija $f$ aprėžta (t.y.
  $f(x)>m$ arba $f(x)<M$ su visomis $x\in(a,b)$ reikšmėmis, $m$, $M$ -
  konstantos), tai $f$ - tiesinė.
\end{thm}

\begin{proof}[Įrodymas]
  Iš tikrųjų, iš antrosios teoremos seka, kad jei $f$ - netiesinė, tai ji
  gali bet kuriame intervale įgyti reikšmę iš bet kokio mūsų norimo
  intervalo, vadinasi, jei $f$ yra apribota kažkokiame intervale, tai ji gali
  būti tik tiesinė.
\end{proof}

\begin{thm}
  Jei egzistuoja intervalas, kuriame $f$ yra monotoninė, tai $f$ - tiesinė.
\end{thm}

\begin{proof}[Įrodymas]
  Jei $f$ - monotoninė kažkokiame intervale (jei intervalas neuždaras, tai
  galime paimti kokią nors jo uždarą dalį), tai tame intervale ji bus ir
  aprėžta - egzistuos jos maksimumas arba minimumas, taigi, ji gali
  būti tik tiesinė.
\end{proof}

\begin{thm}
  Jei egzistuoja intervalas, kuriame $f$ yra tolydi, tai $f$ - tiesinė.
\end{thm}

\begin{proof}[Įrodymas]
  Jei $f$ - tolydi kažkokiame intervale (jei intervalas neuždaras, tai
  galime paimti kokią nors jo uždarą dalį), tai tame intervale ji ir aprėžta,
  taigi, ji gali būti tik tiesinė.
\end{proof}

Trys pastarosios teoremos - klasikiniai, gerai žinomi faktai. Naudojant jas
kokioje nors rimtoje olimpiadoje įrodyti jų nebūtina.

\subsubsection{Pavyzdžiai}

\begin{pavnr}
  Raskite visas funkcijas $f:\Q\to\R$, kurios su visomis racionaliųjų
  skaičių poromis $x$ ir $y$ tenkina $f(x+y)=f(x)+f(y)+xy(x+y)$.
\end{pavnr}

\begin{sprendimas}
  Pakeiskime - $f(x)=g(x)+\frac{x^{3}}{3}$. Įstatę į pradinę lygtį gausime
  $g(x+y)=g(x)+g(y)$, t.y. Cauchy funkcinę lygtį racionaliesiems skaičiams.
  Gauname $g(x)=kx$, kur $k$ - kažkokia realioji konstanta, o tada
  $f(x)=kx+\frac{x^{3}}{3}$.
\end{sprendimas}


\begin{pavnr}
  Raskite visas funkcijas $f:\R\to\R$, kurios su visomis realiųjų
  skaičių poromis $x$ ir $y$ tenkina $f(x+y)=f(x)+f(y)$, ir su visais
  $x\not=0$ tenkina $f(x)f(\frac{1}{x})=1$.
\end{pavnr}

\begin{sprendimas}
  Turime Cauchy funkcinę lygtį realiesiems skaičiams, taigi, iš duotosios
  sąlygos $f(x)f(\frac{1}{x})=1$ reikia išpešti ką nors naudingo. Iš šios
  sąlygos išplaukia, kad $f(x)$ ir $f(\frac{1}{x})$ yra vienodo ženklo, t.y.
  abu neigiami arba teigiami. Įstatę į Cauchy lygti $y=\frac{1}{x}$ gausime:
  $$|f(x+\frac{1}{x})|=|f(x)|+|f(\frac{1}{x})|\geq2\sqrt{|f(x)|*|f(\frac{1}{x})|}=2.$$
  Reiškinys $x+\frac{1}{x}$, keičiant $x$, įgauna bet kokią reikšmę iš
  intervalo $[2,+\infty)$, vadinasi intervale $[2,+\infty)$ $f(x)\geq2$, arba
  $f(x)\leq-2$. Gavome, kad funkcija šiame intervale yra savotiškai aprėžta
  (neįgauna reikšmių iš intervalo $(-2,2)$), tad galime atmesti netiesinius
  Cauchy lygties sprendinius. Belieka į antrą sąlygą įstatyti $f(x)=kx$.
  Gausime $k=1$ arba $k=-1$, ir, nesunku patikrinti, kad sprendiniai $f(x)=x$ ir
  $f(x)=-x$ tiks.
\end{sprendimas}

\begin{pavnr}
  Raskite visas funkcijas $f:\R\to\R$, kurios su visomis realiųjų skaičių
  poromis $x$ ir $y$ tenkina $f(xy)=f(x)f(y)$ ir $f(x+y)=f(x)+f(y).$
\end{pavnr}

\begin{sprendimas}
  Pirmoje lygtyje pakeitę $y=x$ gausime, kad $f(x^{2})=f(x)^{2}$, vadinasi,
  visiems neneigiamiems $x$, $f(x)\geq0$ ir intervale $[0,+\infty)$ funkcija
  yra aprėžta. Tada $f(x)=kx$. Patikrinę randame, kad tiks tik $k=1$.
\end{sprendimas}

\begin{pavnr}
  Raskite visas funkcijas $f:\R\to\R$, kurios su visomis realiųjų
  skaičių poromis $x$ ir $y$ tenkina $f(xy)=f(x)f(y)$ ir intervale
  $(0,+\infty)$ yra monotoniškos.
\end{pavnr}

\begin{sprendimas}
  Statykime $x=y=0$, gausime $f(0)=0$, arba $f(0)=1$. Jei $f(0)=1$, tai
  įsistatę $x=0$ gausime $f(x)=1$ visiems $x$, tad nagrinėkime atvejį, kai
  $f(0)=0$.

  Tarkime, kad egzistuoja $z\not=0$, toks, kad $f(z)=0$. Tada pradinėje
  lygtyje paėmę $x=\frac{x}{z}$ ir $y=z$ gausime $f(y)=0$ visiems $y$.

  Belieka išnagrinėti atvejį, kai $f(0)=0$ ir su jokia kita reikšme funkcija
  nelygi nuliui. Pradinėje lygtyje įstatę $y=x$ gausime, kad
  $f(x^{2})=f(x)^{2}$, arba $f(x)>0$, kai $x>0$. Vadinasi teigiamiems $x,y$
  galios:
  $$\ln f(xy)=\ln f(x)f(y)=\ln f(x) + \ln f(y).$$
  Pažymėję $\ln f(x)=g(x)$, gausime $g(xy)=g(x)+g(y)$. Aišku, kad ir funkcija
  $g$ yra monotoniška. Kintamieji $x$ ir $y$ teigiami, taigi galime pakeisti
  $x=e^{x}$, $y=e^{y}$. Gausime $$g(e^{x+y})=g(e^{x})+g(e^{y}).$$ Pažymėję dar
  kartą $h(x)=g(e^{x})$, gausime, kad $h$ - monotoninė ir jai galioja
  $$h(x+y)=h(x)+h(y),$$ taigi $h(x)=kx$. Lieka grįžti atgal - $g(e^{x})=kx$,
  kur pakeitę $x=\ln x$, gausime $g(x)=k \ln x=\ln x^{k}$. Vadinasi, $\ln
  f(x)=\ln x^{k}$, arba $f(x)=x^{k}$, kur  $k$ - kažkoks realusis, o $x$ -
  teigiamas.

  Lieka rasti tik reikšmes neigiamiems skaičiams. Statykime į pagrindinę lygtį
  $x=y=-1$, gausime $f(-1)=-1$, arba $f(-1)=1$. Tuomet įsistatę į lygtį $y=-1$
  gausime $f(x)=-f(-x)$, arba $f(x)=-f(x)$. Pirmu atveju neigiamiems $x$
  gausime $f(x)=x|x|^{k-1}$, antruoju: $f(x)=|x|^{k}$. Taigi, visus sprendinius
  galime užrašyti taip:
  $$f(x)=0,  \hspace{13 mm}
  f(x)=
  \left\{\begin{array}{cc}
    x^{k}, & x>0,\\
    0, & x=0,\\
    |x|^{k}, & x<0;\\
  \end{array}  \right.\hspace{10 mm}
  f(x)=
  \left\{\begin{array}{cc}
    x^{k}, & x>0,\\
    0, & x=0,\\
    x|x|^{k-1}, & x<0.\\
  \end{array}  \right. $$
  \mbox{}

\end{sprendimas}


Prie šio pavyzdžio galėtume paminėti dar dvi dažnai pasitaikančias
paprastesnes, vadinamąsias "Cauchy tipo" lygtis - $t(x+y)=t(x)t(y)$ ir
$z(xy)=z(x)+z(y)$. Turint atitinkamus apribojimus (tolydumas,
monotoniškumas (aprėžtumas netiks, nes darant ketinius jis dingsta)) jų
sprendiniai yra atitinkamai $t(x)=a^x$ ir $z(x)=log_a x$, ir sprendžiamos
jos analogiškais keitiniais.

\subsubsection{Uždaviniai}

\begin{enumerate}
  \item Raskite visas funkcijas $f:\Q\to\R$, kurios su visomis
    racionaliųjų skaičių poromis $x$ ir $y$ tenkina $f(x+y)=f(x)+f(y)+2xy$.
    %Pasižymėkime $f(x)=g(x)+x^2$. Gausime $g(x+y)=g(x)+g(y)$. Tada
    %$g(x)=kx$, ir $f(x)=kx+x^2$. Nesunku patikrinti, kad sprendinys tiks.
  \item Raskite visas funkcijas $f:\Q\to\R$, kurios su visomis
    racionaliųjų skaičių poromis $x$ ir $y$ tenkina $f(x+f(y))=f(x+1)+y$.
    %Pasižymėkime $f(x)=g(x)+1$. Gausime $g(x+1+g(y))=g(x+1)+y$, arba
    %$g(t+g(y))=g(t)+y$. Iš čia nesunku įsitikinti, kad funkcija bijektyvi.
    %Įstatę $t=y=0$, gausime $g(0)=0$, o paskui įstatę $t=0$ - $g(g(y))=y$.
    %Tada prieš tai gautoje lygtyje pakeitę $y=g(y)$, gausime Cauchy
    %funkcinę lygtį, iš kur $g(x)=kx$. Nesunku patikrinti, kad tiks tik
    %$k=1$ arba -1. Randame sprendinius $f(x)=x+1$ arba $f(x)=1-x$.
  \item Raskite visus tolydžių funkcijų  $f,g,h:\R\to\R$ trejetus,
    kurie su visomis realiųjų skaičių poromis $x$ ir $y$ tenkina
    $f(x+y)=g(x)+h(y)$.
    %Statykime $x=y=0$. Gausime $h(0)=f(0)-g(0)$. Paimkime pradinėje
    %lygtyje $y=0$. Tada turėsime $g(x)=f(x)-h(0)=f(x)+g(0)-f(0)$. Paimkime
    %pradinėje lygtyje $x=0$. Gausime $h(y)=f(y)-g(0)$. Įstatę gautas
    %$g(x)$ ir $h(y)$ išraiškas į pradinę lygtį gausime:
    %$f(x+y)=f(x)+f(y)-f(0)$. Įsivedę keitinį $i(x)=f(x)-f(0)$, gausime,
    %kad $i$ tenkina Cauchy funkcinę lygtį ir yra tolydi vadinasi
    %$i(x)=kx$. Tada, jei pažymėsime $f(0)=a$ ir $g(0)=b$, gausime
    %$f(x)=kx+a$, $g(x)=kx+b-a$, $h(x)=kx-b$
    %
    %Įstatę į pradinę lygtį, gausime, kad $a=0$, o $k$ ir $b$ - bet kokios
    %realiosios konstantos.
  \item Raskite visas funkcijas $f:\R\to\R$, kurios su visomis realiųjų
    skaičių poromis $x$ ir $y$ tenkina $f(xy)=f(x)f(y)-f(x+y)+1$.
    %Pasižymėkime $f(x)=g(x)+1$. Tada pradinė lygtis virs
    %$g(xy)+g(x+y)=g(x)g(y)+g(x)+g(y)$. Įsistatę $x=y=0$ gausime $g(0)=0$.
    %Tada, pažymėję $g(1)=k$, po nesudėtingos indukcijos gausime
    %$$g(n)=k^{n}+k^{n-1}+...+k, \forall \, n\in\N.$$ Jei $g(1)=1$, tai gausime
    %$g(n)=n$, kitu atveju $g(n)=\frac{k^{n+1}-1}{k-1}-1$. Įstatę į
    %prieš tai turėtą lygtį ir išprastinę gausime
    %$$k^{xy+2}-k^{xy+1}-k^{x+y+1}=k^{2}-k^{x+1}-k^{y+1}.$$ Čia galime
    %statyti bet kokius naturaliuosius $x$ ir $y$. Tą darydami, nesunkiai
    %gausime $k=1,0,-1$. Kai $k=0$ gausime sprendinį $f(x)=1$. Kai $k=-1$,
    %nesunkiai gausime prieštarą. Kai $k=1$, pradinėje lygtyje įstatę $x=1,
    %y=-1$ gausime $g(-1)=-1$. Tada pradinėje lygtyje paėmę $y=-1$, o
    %paskui $x=-x, y=1$ ir sudėję abi gautas lygybes gausime $-g(x)=g(-x)$
    %visiems $x\in\R$. Tada pradinėje lygybėje paėmę $x=-x,y=-y$ ir
    %pritaikę paskutiniąją lygybę gausime:
    %$g(xy)-g(x+y)=g(x)g(y)-g(x)-g(y)$. Sudėję su pradine lygybę
    %gausime $g(x+y)=g(x)+g(y)$ ir $g(xy)=g(x)g(y)$, iš ko, kaip jau matėme
    %pavyzdyje, gausime $g(x)=x$. Taigi, šios lygties sprendiniai yra
    %$f(x)=x+1$ ir $f(x)=1$.
  \item Raskite visas funkcijas $f:\R\to[0,+\infty)$, kurios su visomis
    realiųjų skaičių poromis $x$ ir $y$ tenkina
    $f(x^{2}+y^{2})=f(x^{2}-y^{2})+f(2yx)$.
    %Nesunku atspėti, kad $f(x)=x^2$ yra lygties sprendinys. Iš čia kyla
    %idėja įsivesti keitinį $f(x)=g(x^2)$, visiems $x\geq 0$. Gausime
    %$g((x^2-y^2)^2+(2xy)^2)=g((x^2-y^2)^2)+g((2xy)^2)$. Kita vertus, jei
    %pažymėsime $a=x^2-y^2$ ir $b=2xy$, tai nesunku įsitikinti, kad lygčių
    %sistema
    %$$\left\{\begin{array}{cc}
    %a&=x^2-y^2\\
    %b&=2xy\\
    %\end{array}  \right.$$ visados turės sprendinių, kad ir kokius $a$ ir
    %$b$ pasirinktume (tiesiog išsireikštume iš antros lygties $x$,
    %įstatytume į pirmą ir gautume kvadratinę lygtį $y^2$ atžvilgiu, kurios
    %diskriminantas tikrai teigiamas). Tada gautą funkcinę lygtį galime
    %pasikeisti į $g(a^2+b^2)=g(a^2)+g(b^2)$ , arba į $g(z+t)=g(z)+g(t)$,
    %kur $z$ ir $t$ bet kokie neneigiami realieji. Kadangi turime
    %$f:\R\rightarrow[0,+\infty)$, funkcija $g$ bus aprėžta iš apačios ir
    %galime teigti, kad $g(x)=kx$ visiems neneigiamiems $x$ (nors funkcija
    %Cauchy lygtį tenkina tik neneigiamiems skaičiams, nesunku įsitikinti,
    %kad aprėžtumo vistiek užteks). Tada $f(x)=kx^2$ visiems teigiamiems
    %$x$, bet pradinėje lygtyje paėmę $y=0$, gausim $f(0)=0$, o tada vėl
    %pradinėje lygtyje paėmę $x=0$ gautume $f(y)=f(-y)$ visiems $y$, taigi
    %$f(x)=kx^2$ ir neigiamiems $x$.
  \item Raskite visas funkcijas $f:\R\to\R$, kurios su visomis realiųjų
    skaičių poromis $x$ ir $y$ ir $2\leq n\in\N$ tenkina
    $f(x^{n}+f(y))=f^{n}(x)+y$.
    %Nesunku įsitikinti, kad funkcija bijektyvi. Įstačius $x=0$, $y=x^n$,
    %gausime: $f(f(x^n))=x^n+f(0)^n$. Iš bijektyvumo aišku, kad egzistuoja
    %toks $t$, kad  $f(t)=0$ ir $z$, kad  $f(z)=t$. Tada įsistatę pradinėje
    %lygtyje $y=t$ gausime: $f(x^n)=f(x)^n+t$. Panaudoję tai
    %ankščiau gautoje lygtyje gauname $f(f(x)^n+t)=x^n+f(0)^n$. Dabar pradinėje
    %lygtyje pakeitę $x$ į $f(x)$ ir $y$ į $z$, gausime, kad
    %$f(f(x)^n+t)=f(f(x))^n+z$ ir, sulyginę tai su prieš tai gauta lygtimi,
    %gausime $f(f(x))^n+z=x^n+f(0)^n$. Galiausiai pagrindinėje lygtyje
    %paėmę $x=0$ ir $y=x$, gausime $f(f(x))=x+f(0)^n$. Šią $f(f(x))$
    %išraišką įstatę į prieš tai gautą lygtį gauname:
    %$(x+f(0)^n)^n+z=x^n+f(0)^n$ visiems $x$, iš kur lengvai gauname
    %$f(0)=t=z=0$. Tai įstatę į prieš tai turėtas lygtis gausime $f(f(x))=x$
    %ir $f(x^n)=f(x)^n$. Tada pradinėje lygtyje pakeitę $y$ į $f(y)$
    %turėsime $f(x^n+y)=f(x^n)+f(y)$, kas jau labai panašu į Cauchy
    %funkcinę lygtį.  Lyginiams $n$ $f(x+y)=f(x)+f(y)$, kur $x$ teigiamas, o
    %$y$ - betkoks. Tada paėmę $y=-x$ gauname, kad $f(-x)=-f(x)$ $ \forall
    %x\geq 0$ ir taip $f(x+y)=f(x)+f(y)$ bet kokiems realiesiems $x,y$.
    %Tačiau ankščiau turėjome $f(x^n)=f(x)^n$, taigi $f(x)\geq 0$ visiems
    %$x\geq 0$ ir funkcija yra aprėžta intervale, vadinasi, - pavidalo $kx$.
    %Patikrinę pradinėje lygtyje gauname, kad tiks tik $k=1$, taigi kai $n$
    %- lyginis gauname sprendinį $f(x)=x$.
    %
    %Kai $n$ - nelyginis, tai iškart gauname, kad $f(x+y)=f(x)+f(y)$ bet
    %kokiems realiesiems $x,y$. Be to, turėjome, kad $f(x^n)=f(x)^n$, tada
    %$f(1)=f(1)^n$ ir $f(1)=1,-1$ ($0$ netiks, nes $f$ - injektyvi). Tada
    %gauname du atvejus: $f(p)=p$ arba $-p$ $\forall p\in\Z$ ir abiem
    %atvejais galios $f(px)=pf(x)$. Pažymėkime $b_{k}=f(x^k)$,
    %$k=2,3,...,n$ ir $q=f(x)$. Iš ankščiau gauto rezultato galios
    %$f((x+p)^n)=(f(x+p))^n=(f(x)+f(p))^n$. Čia galime statyti bet kokį
    %sveiką $p$ ir tai yra tiesinė lygtis bet kurio  $b_{k}$ atžvilgiu. Tada
    %keisdami įvairias $p$ reikšmes galime gauti $n-1$ neekvivalenčių
    %lygčių su $n-1$ kintamųjų $b_{k}$. Tada aišku, kad tokia tiesinių
    %lygčių sistema turės daugiausiai tik vieną sprendinį. Nesunku
    %patikrinti, kad pirmam atvejui tiks sprendinys $b_{k}=q^k$, o antram
    %$b_{k}=-q^k$, lyginiams $k$ ir $b_{k}=q^k$ nelyginiams $k$. Tada pirmu
    %atveju gausime  $f(x^2)=f(x)^2$, o antru $f(x^2)=-f(x)^2$. Iš čia
    %funkcija ir vėl aprėžta ir gausime, kad kai $n$ - nelyginis, tiks tik
    %tiesiniai sprendiniai $f(x)=x$ ir $f(x)=-x$.
  \item Raskite visus funkcijų  $f,g:\R\rightarrow\R$ dvejetus tokius,
    kad:\\
    a) Jei $x<y$, $f(x)<f(y)$.\\
    b) Visoms realiųjų poroms $x$ ir $y$ galioja $f(xy)=g(y)f(x)+f(y)$.
    %Įstatę duotojoje lygtyje $x=0$, gausime $f(0)\not=0$, nes kitaip
    %$f(y)=0$ visiems $y$, bet $f$ -  nekonstanta. Taigi
    %$g(y)=1-\frac{f(y)}{f(0)}$. Įstatę pradinėję lygtyje $x=y=1$ ir
    %panaudoję a), gausime $f(1)=0$ ir tada galime pažymėti $f(0)=-k$, kur
    %$k$ - kažkoks teigiamas skaičius. Tada įstatę $g$ išraišką į pradinę
    %lygtį gausime $f(xy)=f(x)+f(y)+\frac{f(x)f(y)}{k}$, arba:
    %$k+f(xy)=(\sqrt{k}+\frac{f(x)}{\sqrt{k}})(\sqrt{k}+\frac{f(y)}{\sqrt{k}})$.
    %Pakeitę $h(x)=\sqrt{k}+\frac{f(x)}{\sqrt{k}}$, gausime
    %$\sqrt{k}h(xy)=h(x)h(y)$, o tada pakeitę $h(x)=\sqrt{k}i(x)$:
    %$i(xy)=i(x)i(y)$. Monotoniškumas niekur nedingo ir šią lygtį jau esamę
    %sprendę, tad nesunku gauti atsakymą:
    %
    %$$f(x)=-k+k\cdot sgn(x)\cdot |x|^a \text{ ir } g(y)=sgn(y)\cdot |y|^a,$$
    %kur $sgn(x)$ - $x$ ženklo funkcija.
  \item Raskite visas funkcijas  $u:\R\to\R$, kurioms egzistuoja tokia
    griežtai monotoninė funkcija $f:\R\to\R$, kad visoms realiųjų poroms
    $x$ ir $y$ yra teisinga lygybė $f(x+y)=u(y)f(x)+f(y)$.
    %Įstatę į pradinę lygtį $x=y=0$, gausime, kad $f(0)=0$ (jei $u(0)=0$,
    %$f$ - konstanta). $f$ - griežtai monotoninė, taigi $0$ ji neįgys su
    %jokia kita argumento reikšme. Iš pradinės lygties
    %$u(y)f(x)+f(y)=f(x+y)=u(x)f(y)+f(x)$. Čia fiksavę $y$, gausime:
    %$u(x)=\frac{u(y)-1}{f(y)}f(x)+1=Af(x)+1$. Jei $u(z)=1$, visiems
    %realiesiems $z$, tai egzistuos $f(x)=x$, tenkinanti pradines sąlygas.
    %Kitu atveju: $f(x+y)=Af(x)f(y)+f(x)+f(y)$. Tada pakeitę $h(x)=Af(x)+1$
    %gausime $$h(x+y)=h(x)h(y).$$ Tai viena iš Cauchy tipo lygčių, kurias
    %sutikome ankščiau. Kadangi $f$ monotoninė, $h$ irgi monotoninė ir
    %$h(x)=b^x$, kur $b>0$. Tada $f(x)=A^{-1}(b^x-1)$ ir $u(y)=b^y$ bus
    %sprendiniai. Viską apibendrinus, $u(x)=b^x$, kur $b>0$ (įskaitant ir
    %$b=1$), bus vienintelės sąlygas tenkinančios funkcijos.
  \item Raskite visas funkcijas $f:(0,1)\rightarrow\R$, kurios su visomis
    realiųjų skaičių poromis $x$ ir $y$ tenkina
    $f(\frac{x+y}{1+xy})=\frac{f(x)f(y)}{|1+xy|}$ ir yra tolydžios.
    %Pirmiausiai darykime keitinį $f(x)=g(x)|1+x|$. Pradinė lygtis taps:
    %$g(\frac{x+y}{1+xy})=g(x)g(y)$. Pastebėkime, kad reiškinys
    %$\frac{x+y}{1+xy}$ primena tangentų sumos formulę, tačiau tangentas
    %nepaprastas, o - hiperbolinis. Hiperbolinis tangentas - tai funkcija
    %$\tanh (x)=\frac{e^{2x}-1}{e^{2x}+1}$. Nesunku įsitikinti, kad irgi
    %galios panaši į tangentų sumos formulė, t.y. $\tanh(x+y)=\frac{\tanh
    %(x)+\tanh (y)}{1+\tanh (x) \tanh (y)}$. Taigi keičiame lygtyje
    %$x=\tanh(x)$, $y=\tanh(y)$. Gausime $g(\tanh(x+y))=g(\tanh(x))g(\tanh(y))$.
    %Įsiveskime keitinį $h(x)=g(\tanh(x))$. Gausime lygtį $h(x+y)=h(x)h(y)$.
    %Galime nesunkiai įsitikinti, kad tolydumas niekur nedingo, tai viena
    %iš Cauchy tipo lygčių, kurios sprendiniai bus $h(x)=a^{x}$, kur $a\geq
    %0$. Tada $a^{x}=g(\tanh(x))\implies g(x)=a^{\arctanh(x)}$,
    %$f(x)=a^{\arctanh(x)}|1+x|$.
\end{enumerate}



