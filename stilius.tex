\section{Knygos stiliaus pavyzdžio failas}

Matematikos knygoje yra keturi skyriai \chapter - skaičių teorija, algebra,
kombinatorika ir geometrija. Skyriai skirstomi į sekcijas \section, o
pastarosios, jei reikia (pavyzdžiui nelygybės ar funkcinės lygtys), į
subsekcijas \subsection. Šie trys struktūriniai vienetai yra vaizduojami
turinyje. Sekcijos su tekstu arba, jei be teksto, jų subsekcijos yra
pagrindiniai "teksto vienetai", vadinami skyreliais.

Skyreliai yra galiausiai skirstomi į subsubsekcijas \subsubsection, kurios
turinyje nevaizduojamos. Didelės sekcijos \section, t.y. tos, kurios turi
subsekcijas \subsection, yra iškeliamos į atskirus failus, kurie nurodomi
komanda \input (žr. pavyzdžiui failą algebra.tex).

\subsubsection{aplinkos}

Subsubsekcijos nėra vienintėlis struktūrinis elementas, padedantis
organizuoti rašomą skyrelį. Taip pat yra naudojamos aplinkos teiginiams,
teoremoms, išvadoms, pavyzdžiams, sprendimams ir įrodymams, bei jų
numeruotiems atitikmenims. Jas visas galite rasti aprašytas faile
knyga.sty, ir, jei ko trūksta, galite pridėti savo. Štai kaip jos atrodo
tekste:

\begin{thm}[Matematiko teorema]
  Labai įdomios teoremos vadinamos matematiko vardu formuluotė. Ši teorema
  bus nenumeruota. Numeruotos santrumpa yra thmnr.
\end{thm}

\begin{proof}[Įrodymas]
  Teoremos įrodymas. (Dėl babel sulietuvinimo klaidos tenka pridėti
  [Įrodymas], arba tą klaidą pataisyti. Tam, Ubuntu naudotojai gali faile
  /usr/share/texmf-texlive/tex/latex/lithuanian/lithuanian.ldf pridėti
  eilutę \def\proofname{\k{I}rodymas}.)
\end{proof}

\begin{pav}
  Pavyzdys. Numeruotų pavyzdžių santrumpa pavnr.
\end{pav}

\begin{sprendimas}
  Pavyzdžio sprendimas. Pavyzdžiai ir uždaviniai yra sprendžiami, o
  teoremos įrodinėjamos.
\end{sprendimas}

\begin{teig}
  Labai įdomus teiginys. Numeruoti teignr.
\end{teig}

\begin{isv}
  Kokio nors teiginio išvada. Numeruota išvada isvnr.
\end{isv}

\subsubsection{santrumpos}

Faile knyga.sty taip pat yra aprašytos kai kurios šiek tiek gyvenimą
palengvinančios santrumpos:

\begin{itemize}
  \item Sveikųjų, natūraliųjų, kompleksinių ir t.t. aibės rašomos kaip \Z,
    \N, \C ir pan, užuot rinkus {\mathbb X}.
  \item Didžiausio bendro daliklio, mažiausio bendro kartotinio santrumpos
    rašomos \dbd ir \mbk. Taip pat yra santrumpa ir hiperboliniam
    arktangentui \arctanh, kuris kažkokiu būdu įsliūkino į knygą.
  \item Komanda \lez{a}{b} pagamins Ležandro simbolį.
  \item Komanda \m{p} parašys "(mod p)" (daug maž tą patį daro \pmod{p}).
\end{itemize}

\subsubsection{kodo stilius}

Panaršę Mknygos kodą, pastebėsite, kad visi failai prie kurių kai kas
prikiša nagus yra daugiau mažiau vieno (standartinio vim-latex įskiepio
užduoto) stilius. Aplinkose esantis tekstas atitrauktas per du tarpus, \item
esantis tekstas dar per du. Jūsų naudojamas teksto radaktorius greičiausiai
nemokės automatiškai lygiuoti teksto tokiu būdu, tad rašydami naują tekstą
per daug nevarkite ir rašykite kaip greičiau ir patogiau. Kita vertus,
darydami smulkius pakeitimus jau egzistuojančiame sutvarkytame tekste,
stenkitės lygiavimo nesugadinti.

Kitas, kiek labiau komplikuotas aspektas, yra teksto eilutės. Jos rašomos
neilgesnės nei 75 simboliai (ar 80, tai nelabai svarbu). Dėja, naudojantis
daugeliu teksto redaktorių taip rašyti yra sudėtinga, nes pasiekus 80
simbolių jie užuot automatiškai padarę naują eilutę (hard wrap), tęsia
senąją, tik vaizduoja ekrane lyg būtų pradėję naują (soft wrap) ir tokiu
būdu visas paragrafas tampa viena eilutė.  

Įprastai būtų galima nekreipti dėmesio į šį niuansą, bet jis tampa labai
svarbus naudojant versijų kontrolės sistemas, kurios pakeitimus kaip tik
vaizduoja pagal eilutes. Jei jūs, pavyzdžiui, redaguodami tekstą įrašysite
praleistą raidę, versijų sistema tai užfiksuos kaip senos eilutės (be
raidės) pakeitimą nauja eilute (su raide). Jei eilutė buvo trumpa, šį
pakeitimą pamatę lengvai suprasime, kad buvo įrašyta raidė. Jei eilutė buvo
visas paragrafas, atsekti, kas buvo ištaisyta, bus sudėtinga.

Texmaker ir TeXnicCenter naujų eilučių daryti (hard wrap), deja, nemoka.
Galite susirasti redaktorių, kuris mokėtų, arba galite nekreipti dėmesio ir
rašyti taip, kaip išeina. Tik turėkite omenyje, kad tą patį failą redaguoti
dviem skirtingai eilutes ``wrapinančiais'' redaktoriais yra labai
skausminga, tad pasirinkite vieną stilių ir jo laikykitės, o atlikdami
smulkius pataisymus išlaikykite taisomo failo stilių.  

(Jei įdomu:
http://stackoverflow.com/questions/899114/when-you-write-tex-source-how-do-you-use-your-editors-word-wrap)

\subsubsection{Uždaviniai}

Paskutinė subsubsekcija kiekviename skyrelyje yra uždavinių. Ją visuomet
reikia formatuoti taip, kaip čia, nes ją automatiškai apdoroja programa
parse.py. Ji surenka po uždaviniais esančius sprendimus, sudeda į sprendimų
failą ir padaro nuorodas, kurias galite matyti ir spaudyti spalvotoje
versijoje. Taigi, po \subsubsection{Uždaviniai} visuomet eina:

\begin{enumerate}
  \item Pirmo uždavinio sąlyga. Sąlyga labai trumpa, matyt uždavinys
    nesudėtingas.
    %Pirmo uždavinio sprendimas komentaruose. Iš ties buvo paprastas.
  \item {[IMO 2012]} Antrasis uždavinys su žinoma kilme. Jei žinote iš
    kurios olimpiados buvo uždavinys, visuomet nurodykite. Atkreipkite
    dėmesį į riestinius skliaustus - jie būtini.
    %Jo sprendimas komentaruose.
  \item Trečias uždavinys, kuris prieš tai buvo ketvirtas, bet kadangi
    nereikia rašyti uždavinių numerių, tai sukeisti buvo nesudėtinga. Taip
    pat nereikia ieškoti ir atitinkamame faile sukeisti sprendimų, nes juos
    perkėlėme kartu su sąlyga.
    %Sprendimas trečiojo uždavinio.
  \item Ketvirtas uždavinys, minėtas anksčiau.
    %Ir jo sprendimas.
\end{enumerate}
