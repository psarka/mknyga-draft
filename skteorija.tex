\section{Vieta-Jumping (Vieto šokinėjimas)}
Vieta-Jumping, tai gana lengvai atskiriamo tipo, bet dažniausiai sudėtingų skaičių teorijos uždavinių sprendimo technika, taip vadinama dėl to, jog sprendime yra naudojamos Vieto kvadratinio trinario šaknų formulės bei Ferma išvystytas begalinio nusileidimo metodas. Ji labiausiai pritaikoma, kada žinome, kad duoti sveikieji skaičiai tenkina kokį nors sąryšį (dažniausiai tam tikrą dalumą) ir reikia įrodyti kokią nors savybę tarp šių sveikujų skaičių (pavyzdžiui, kad tam tikras jų santykis yra natūralaus skaičiaus kvadratas ar panašiai). 

\begin{pavnr}
(IMO 1988 Nr.6) $a$ ir $b$ yra tokie teigiami sveikieji skaičiai, kad $ab + 1|a^2 + b^2$. Įrodykite, kad $\frac{a^2 + b^2}{ab + 1}$ yra sveikojo skaičiaus kvadratas.
\end{pavnr}

Toliau išskirsime porą skirtingų šios technikos atšakų ir pateiksime jas iliustruojančius pavyzdžius.

\subsection{Standartinis Vieta-Jumping}
Standartinis Vieta-Jumping, tai įrodymas prieštara susidedantis iš trijų žingsnių:
\begin{enumerate}
\item Tariame, kad egzistuoja sprendinys duotajam sąryšiui toks, kad negalioja mūsų norima įrodyti savybė.
\item Visoms tokių ,,blogųjų'' sprendinių poroms $(a,b)$ priskiriama tam tikra funkcija (dažniausiai $y=a+b$) ir naudojantis naturaliųjų ar sveikųjų skaičių savybėmis leidžiame sau tarti, kad tarp visų šių porų egzistuoja pora (A,B), tokia, kad $Y=A+B$ yra mažiausia. Tada paliekame $B$ fiksuotą ir perstatę gauname kvadratinę lygtį $a$ atžvilgiu. Žinodami, kad viena iš jos šaknų yra $A$ naudodami Vieto formulėmis randame antrą šaknį $x$. 
\item Įrodome, kad $(x,B)$ irgi yra tinkamas sprendinys ir su šia antrąja šaknimi $x$, $x+B<A+B$, taip gaudami prieštarą (sakėme, kad $(A,B)$ yra pora su minimalia suma $A+B$)
\end{enumerate}

Grįžtame prie mūsų pirmojo pavyzdžio ir iliuostruojame visus tris žingsnius:
\begin{sprendimas}
\begin{enumerate}
\item Tegu $\frac{a^2+b^2}{ab+1}=k$. Tarkime, kad egzistuoja vienas ar daugiau sprendinių $(a,b)$, tenkinančių šį sąryšį, tokių, kad $k$ nėra sveikojo skaičiaus kvadratas.
\item Tegu $(A,B)$ būna toji (ar viena iš tų) sprendinių porų, tenkinančių sąryšį šiam fiksuotam $k$, tokia, kad šiai porai $a+b$ yra minimali. Neprarasdami bendrumo galime tarti, kad $A\geq B$. Fiksuodami $B$ ir perstatatydami, gauname kvadratinę lygtį $a$ atžvilgiu, kurios viena iš šaknų yra $A$: $$a^2-(kB)a+(B^2-k)=0$$ Naudojantis Vieto šaknų formulėm antroji šaknis $x$ yra lygi: $$x=kB-A=\frac{B^2-k}{A}$$
\item Iš pirmos lygties matome, kad $x$ sveikasis. Jei $x=0$, iš antros lygties $k=B^2$, o mes jau tarėme priešingai. Galiausiai $x$ negali būti neigiamas, nes tada: $$1+Bx\leq 0 \rightarrow -kBx\geq k \rightarrow x^2-kBx+B^2-k\geq x^2+k+B^2-k$$ ir turime:$$x^2-kBx+B^2-k>0$$ Taigi $x$ yra teigiamas sveikasis ir turime, kad pora $(x,B)$ yra tinkamas mūsų sąryšiui sprendinys. Pabaigai:$$A\geq B \rightarrow x=\frac{B^2-k}{A}<A \rightarrow x+B<A+B$$ Taigi (A,B) nėra minimalus sprendinys, kaip mes tarėme ankščiau, prieštara. Koks elegantiškas ir elementarus sprendimas, atkreipus dėmesį į tai, jog tai IMO uždavinys ,,žudikas'', numeris 6!
\end{enumerate}
\end{sprendimas}
\subsection{Pastoviai besileidžiantis Vieta-Jumping}
Šis metodas yra naudojamas, kada norime įrodyti, kad tam tikra konstanta $k$ yra susijusi su duotuoju sąryšiu tarp $a$ ir $b$ ir kitaip nei standartinis Vieta-Jumping tai nėra įrodymas prieštara, nors jo esmė ta pati, tik formuluotė kitokia. Jis susideda iš 4 dalių:
\begin{enumerate}
\item Patikrinamas atvejis, kai $a$ ir $b$ yra lygus, kad galėtumėme tarti, kad $a>b$.
\item Užfiksuojame $b$ ir $k$ ir pertvarkome turimą sąryšį į kvadratinę lygtį, kurios viena iš šaknų yra $a$. Naudodamiesi Vieto formulėmis žiūrime, kokia galėtų būti antroji šaknis.
\item Parodome, kad visoms poroms $(a,b)$ didesnėms už tam tikrą ,,bazinę" porą ar poras galioja $0<x<b<a$ . Vadinasi pradinę sprendinio porą $(a,b)$ galime pakeisti į $(b,x)$ ir kartoti procesą iš naujo.
\item Kartojant procesą ir mažėjant skaičiams ankščiau ar vėliau turime priartėti iki ,,bazinių'' atvejų. Kadangi atliekant šį ,,nusileidimo'' procesą konstanta $k$ buvo fiksuota, užtenka norimą įrodyti teiginį įrodyti tik šiems ,,baziniams'' atvejams ir tada galėsime teigti, kad norimas teiginys galioja visoms galimoms sprendinių poroms $(a,b)$.
\end{enumerate}
\begin{pavnr}
Tegu $a$ ir $b$ yra tokie teigiami sveikieji, kad $ab|a^2+b^2+1$. Įrodykite, kad $3ab=a^2+b^2+1$.
\end{pavnr}
\begin{sprendimas}
\begin{enumerate}
\item Jei $a=b$, tada $a^2|2a^2+1$, vadinasi $a|1$, taigi $a=b=1$ ir $3*1*1=1^2+1^2+1$, ką ir norėjome parodyti. Viskas simetriška, neprarasdami bendrumo, galime tarti, kad $a>b$.
\item Tarkime $\frac{a^2+b^2+1}{ab}=k$. Perstatę gauname $x^2-(kb)x+(b^2+1)=0$, kuri yra tenkinama šaknies $a$. Tada naudojantis Vijeto formulėmis antrajai šakniai $x$ gauname: $$x=kb-a=\frac{b^2+1}{a}$$
\item Iš pirmos lygties matome, kad antroji šaknis $x$ yra sveikas skaičius, o iš antrosios turime, kad jis teigiamas. Kadangi $a>b$, $x=\frac{b^2+1}{a}<b$, jei $b>1$.
\item Taigi galime ,,mažinti'' iki bazinio atvejo, kai $b=1$. Tada turime $a|a^2+2$ taigi $a=1$ arba $a=2$. $a=1$ netinka, o antruoju atveju turime $k=\frac{6}{2}=3$. Kadangi $k$ buvo fiksuotas viso proceso metu, įrodėme, kad $k=3$ bet kokiu atveju.
\end{enumerate}
\end{sprendimas}

Alternatyvus sprendimas:
Tarkime $\frac{a^2+b^2+1}{ab}=k$. Fiksuokime $k$ ir nagrinėkime visas galimas šio sąryšio sprendinių poras. Tegu pora $(A,B)$ būna toji pora (ar viena iš jų), kuriai suma $a+b$ mažiausia. Įrodysime, kad $A=B$. Tarkime priešingai, t.y., kad $A>B$. Fiksavę $b$ kaip $B$, gauname kvadratinę lygtį $a$ atžvilgiu: $$a^2-(kB)a+(B^2+1)=0$$ Viena iš jos šaknų yra $A$, taigi iš Vijeto formulių antroji šaknis $x$ yra: $$x=kB-A=\frac{B^2+1}{A}$$ Taigi $x$ yra teigiamas sveikasis, be to, iš $A>B\geq 1$ gauname, kad: $$x=\frac{B^2+1}{A}<A$$ Vadinasi pora $(x,B)$ taip pat yra tinkama pora-sprendinys, bet $x+B<A+B$, prieštara. Kadangi viskas simetriška, dėt tų pačių priežasčių $A\not< B$, taigi $A=B$. Galiausiai turime $A^2|2A^2+1\rightarrow A=B=1\rightarrow k=\frac{1^2+1^2+1}{1*1}=3$.
\subsection{Uždaviniai}
\begin{enumerate}
\item $a$ ir $b$ yra tokie teigiami sveikieji skaičiai, kad $ab-1|a^2+b^2$. Įrodykite, kad $\frac{a^2+b^2}{ab-1}=5$

\begin{sprendimas}
Tarkime $\frac{a^2+b^2}{ab-1}=k$. Fiksuokime $k$, tarkime, kad $k\not =5$ ir nagrinėkime visas galimas šio sąryšio sprendinių poras. Tegu pora $(A,B)$ būna toji pora (ar viena iš jų), kuriai suma $a+b$ yra mažiausia. Neprarandant bendrumo $A\geq B$. Jei $A=B$, tada $\frac{A^2+B^2}{AB-1}=\frac{2A^2}{A^2-1}=2+\frac{2}{A^2-1}$, kas nėra sveikasis skaičius. Jei $A=B+1$, $\frac{A^2+B^2}{AB-1}=\frac{2B^2+2B+1}{B^2+B-1}=2+\frac{3}{B^2+B-1}$, taigi $B^2+B-1=3$ arba $1\rightarrow B=1 \rightarrow k=5$, prieštara. Taigi $A\geq B+2$. Perstatome turimą sąryšį į kvadratinę lygtį $a$ atžvilgiu fiksavę $b$ kaip $B$:$$a^2-(kB)a+B^+k=0$$ Naudojantis Vijeto formulėmis, antroji šaknis $x$ yra lygi:$$x=kB-A=\frac{B^2+k}{A}$$ ir dėl to yra teigiama ir sveikoji. Vadinasi pora $(x,B)$ taip pat yra sprendinys pradiniam sąryšiui. Kadangi tarėme, kad pora $(A,B)$ yra minimalioji, turime turėti: $x=\frac{B^2+k}{A}\geq A$, $B^2+\frac{A^2+B^2}{AB-1}\geq A^2$, $AB^2+A^2\geq BA^3-A^2$, $2A\geq BA^2-B^3=B(A^2-B^2)\geq B(A^2-(A-2)^2)\geq 4A-4$. Vadinasi $A \leq2$, o tai yra prieštara, nes $A\geq B+2$. Vadinasi $k=5$.
\end{sprendimas}

\item (IMO 2007, Nr.5) $a$ ir $b$ yra teigiami sveikieji skaičiai ir 
\mbox{$4ab-1|(4a^2-1)^2$}. Įrodykite, kad $a=b$.

\begin{sprendimas}
$4ab-1|(4a^2-1)^2 \rightarrow 4ab-1|b^2(4a^2-1)^2-(4ab-1)(4a^3b-2ab+a^2)=a^2-2ab+b^2=(a-b)^2$. Tegu $\frac{(a-b)^2}{4ab-1}=k$ ir tarkime, kad egzistuoja skirtingi $a$ ir $b$, kuriems galioja šis sąryšis. Fiksuokime $k$ ir iš visų galimų sprendinių porų imkime porą $(A,B)$, kuriai suma $a+b$ yra mažiausia. Neprarandant bendrumo $A>B$. Pertvarkome turimą sąryšį į kvadratinę lygtį $a$ atžvilgiu fiksuodami $b$ kaip $B$, kurios viena iš šaknų yra $A$, o antrają šaknį $x$ randame iš Vijeto formulių:$$a^2-(2B+4kB)a+(B^2+k)=0$$ $$x=2B+4kB-A=\frac{B^2+k}{A}$$ Iš čia gauname, kad $x$ yra teigiamas sveikasis, vadinasi pora $(x,B)$ taip pat yra sprendinys sąryšiui. Vadinasi $x>A$, $\frac{B^2+k}{A} \geq A$, $k\geq A^2-B^2$. Tada: $$\frac{(A-B)^2}{4AB-1}=k\geq A^2-B^2$$
$$A-B\geq (A+B)(4AB-1).$$ Paskutinė nelygybė yra tikrai neįmanoma, nes $A,B$ teigiami sveikieji, taigi gavome prieštarą.
\end{sprendimas}
\item $a$ ir $b$ yra tokie sveikieji teigiami skaičiai, kad $ab|a^2+b^2+3$. Raskite visas galimas $\frac{a^2+b^2+3}{ab}$ reikšmes.

\begin{sprendimas}
Tarkime, kad $\frac{a^2+b^2+3}{ab}=k$. Fiksavę $k$ iš visų galimų šio sąryšio sprendinių porų išsirenkame porą $(A,B)$, tokią, kad suma $a+b$ būtų mažiausia. Fiksuojame $b$ kaip $B$ ir gauname kvadratinę lygtį $a$ atžvilgiu: $$a^2-(kB)a+(B^2+3)=0,$$ kurios viena iš šaknų yra $A$, taigi iš Vijeto formulių antrajai šakniai $x$ galioja: $$x=kB-A=\frac{B^2+3}{A}.$$ Aišku, kad $x$ yra teigiamas sveikasis, vadinasi pora $(x,B)$ taip pat yra sprendinys pradiniam sąryšiui. Tarkime $A>B\geq 2$. Tada $$A^2>B^2+3=Ax\rightarrow A>x\rightarrow x+B>A+B,$$ pieštara. Vadinasi $A=B$ arba $B=1$. Jei $A=B$, turėsime $(k-2)A^2=3 \rightarrow A=1,k=5$. Jei $B=1$, $A^2+4=kA$, vadinasi $A|4$, taigi $A=1,2,4 \rightarrow k=5,4,5$. Apibendrinus $k=4$ arba $5$.
\end{sprendimas}
\item $x,y,z$ yra teigiami sveikieji skaičiai. Įrodykite, kad $(xy+1)(yz+1)(zx+1)$ yra sveikojo skaičiaus kvadratas tik tada, kai $xy+1,yz+1,zx+1$ visi yra sveikųjų skaičių kvadratai (būtų pravartu panagrinėti $t=x+y+z+2xyz\pm 2\sqrt{(xy+1)(yz+1)(zx+1)}$ ir pasižiūrėti, kokią kvadratinę lygtį šios šaknys tenkina).

\begin{sprendimas}
Duotosios $t$ reikšmės yra lygties $$x^2+y^2+z^2+t^2-2(xy+yz+zx+xt+yt+zt)-4xyzt-4=0 (*)$$ šaknys. Pastebėkime, kad ši lygtis gali būti perrašyta į $$(x+y-z-t)^2=4(xy+1)(zt+1)$$ $$(x+z-y-t)^2=4(xz+1)(yt+1)$$ $$(x+t-y-z)^2=4(yz+1)(xt+1).$$ Tarkime egzistuoja trejetai $x,y,z$ tokie, kad $(xy+1)(yz+1)(zx+1)$ yra sveikojo skaičiaus kvadratas, bet kažkuris iš $xy+1,yz+1,zx+1$ nėra kvadratas. Iš šių trejetų išrinkime tokį, kad suma $x+y+z$ būtų mažiausia, neprarasdami bendrumo tarikime, kad $x\leq y \leq z$ ir paimkime $t=x+y+z+2xyz-2\sqrt{(xy+1)(yz+1)(zx+1)}$. Parodysime, kad $0<t<z$, bet $(xy+1)(yt+1)(tx+1)$ yra sveikojo skaičiaus kvadratas, nors kažkuris iš $xy+1,yt+1,tx+1$ nėra ir taip gausime prieštara, nes $x+y+t<x+y+z$. Iš išraiškų viršuje turime, kad: $$16(xy+1)^2(xz+1)(yt+1)(yz+1)(xt+1)=(xy+1)^2(x+y-z-t)^2(x+t-y-z)^2.$$ Taigi $(xy+1)(yt+1)(tx+1)$ yra kvadratas, nes $(xy+1)(yz+1)(zx+1)$ yra kvadratas, o dešinė lygybės pusė taip pat kvadratas. Taip pat iš trijų lygties $(*)$ išraiškų viršuje turime, kad $xt+1$ yra kvadratas tada ir tik tada, kad $yz+1$ yra kvadratas ir $xz+1$ yra kvadratas tada ir tik tada, kai $yt+1$ yra kvadratas, taigi kažkuris iš $xy+1,yt+1,tx+1$ nėra kvadratas.
Iš pertvarkytos lygties išraiškos: $$zt+1=\frac{(x+y-z-t)^2}{4(xy+1)}\geq 0,$$ vadinasi $t\geq -1/z$. $z=1\rightarrow x=y=z=1$, bet tada $(xy+1)(yz+1)(zx+1)$ nėra kvadratas - prieštara. Taigi $z>1$ ir $t\geq 0$. Jei $t=0$, tada turime: $$4(xy+1)=(x+y-z)^2, 4(yz+1)=(-x+y+z)^2, 4(zx+1)=(x-y+z)^2,$$ prieštara, nes kažkuris iš  $xy+1,yt+1,tx+1$ nėra kvadratas. Taigi $t>0$.
Galiausiai paimkime $t'=x+y+z+2xyz+2\sqrt{(xy+1)(yz+1)(zx+1)}$, t.y. antrąją, didesnę lygties šaknį. Tada: $$tt'=x^2+y^2+z^2-2xy-2yz-2zx-4<z^2.$$ Kadangi $t<t'$, $t<z$, taigi $(x,y,t)$ yra tinkamas sąryšiui sprendinys kurio suma mažesnė už jau turėtą minimalią - prieštara.
\end{sprendimas}
\item $x,y,z$ yra teigiami sveikieji skaičiai, kuriems galioja $0<x^2+y^2-xyz\leq z$. Įrodykite, kad $x^2+y^2-xyz$ yra sveikojo skaičiaus kvadratas.

\begin{sprendimas}
Tarkime egzistuoja tokie $x,y,z,t>0$ tokie, kad $z\geq t$ ir $x^2+y^2-xyz=t$, bet $t$ nėra sveikojo skaičiaus kvadratas. Paimkime tokį šių skaičių ketvertą, kad suma $x+y$ būtų minimali ir neprarasdami bendrumo tarkime, kad $x\geq y$. Sprendžiame kvadratinę lygtį $x$ atžvilgiu ir naudodamiesi Vijeto formulėmis išreiškiame antrą šaknį $x'$: $$x'=yz-x=\frac{y^2-t}{x}.$$ Iš pirmos lygties $x'$ yra sveikasis, taip pat $$x'=\frac{y^2-t}{x}\leq \frac{y^2}{x}\leq x.$$ Lygybių atveju turime $t=0$ - prieštara. Belieka įrodyti, kad $x'>0$, tada turėsime, kad pora $(x',y)$ yra tinkama sprendinių pora, bet $x'+y<x+y$ - prieštara.
Tarkime, kad $x>yz$. Tada $x-yz\geq 1$. Vadinasi: $$yz(x-yz)+y^2-t\geq yz+y^2-t\geq y^2+(y-1)z\geq 0=x(x-yz)+y^2-t,$$ $yz\geq x$ - prieštara, vadinasi $yz\geq x$. Tada turime $y^2=t-x(x-yz)>t$ ($t\not =y^2$), vadinasi $x'=\frac{y^2-t}{x}>0$. To ir reikėjo mūsų norimai prieštarai gauti.
\end{sprendimas}
\item Nelyginiai sveikieji $a$ ir $b$ tenkina $a^2-b^2+1|b^2-1$. Įrodykite, kad $a^2-b^2+1$ yra sveikojo skaičiaus kvadratas.

\begin{sprendimas}
$a^2-b^2+1|b^2-1\rightarrow a^2-b^2+1|a^2$. Tarkime, kad $\frac{a^2}{a^2-b^2+1}=k$. Pažymėkime $u=\frac{a+b}{2}$, o $v=\frac{a-b}{2}$. Tada lygtis tampa: $$u^2-(4k-2)uv+v^2-k=0.$$
Jei $k$ yra kvadratas iš turimo sąryšio $a^2-b^2+1$ taip pat yra kvadratas. Turimai kvadratinei lygčiai iš visų sprendinių porų $(u,v)$, paimkime porą $(U,V)$, tokią, kad suma $U+V$ būtų minimali. Fiksuokime V 
\end{sprendimas}

\end{enumerate}
