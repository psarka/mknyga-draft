\section{Daugianariai}

Daugianariu vadinsime funkciją $P:\R \rightarrow \R$, turinčią pavidalą
$$P(x) = a_nx^n + a_{n-1}x^{n-1} + \cdots +a_1x + a_0, \text{ kur } a_i\in
\R \text{ ir } a_n \neq 0.$$
Natūralusis skaičius $n$ yra vadinamas daugianario \emph{laipsniu},
realieji skaičiai $a_i$ - \emph{koeficientais}, koeficientas $a_n$ -
\emph{vyriausiuoju koeficientu}, o $a_0$ - \emph{laisvuoju nariu}. 

Sakysime, kad daugianaris $P(x)$ dalijasi iš daugianario $S(x)$, jei
egzistuoja toks daugianaris $Q(x)$, kad $$P(x)=S(x)Q(x).$$ Jei daugianaris
$P(x)$ nesidalija iš daugianario $S(x)$, tai kaip ir su sveikaisiais
skaičiais vis vien galime dalinti ir gauti liekaną, t.y. egzistuos tokie
daugianariai $S(x)$ ir $R(x)$, kad $$P(x) = S(x)Q(x) + R(x).$$ Dalindami
sveikuosius skaičius gauname, kad liekana yra mažesnė už daliklį,
analogiškai, dalindami daugianarius gauname, kad liekanos $R(x)$ laipsnis
yra mažesnis už daliklio $S(x)$ laipsnį. Pavyzdžiui:

Daugianaris $x^4 - 3x^2 + 2$ dalijasi iš daugianario $x^2 - 1$, nes $x^4 -
3x^2 + 2 = (x^2 - 1)(x^2 - 2)$. Tačiau daugianaris $x^4 - 3x^2 + 2$
nesidalija iš daugianario $x^3$. Dalindami gausime $x^4 - 3x^2 + 2 =
x^3\cdot x - 3x^2 + 2.$ Matome, kad liekanos - $3x^2 + 2$ laipsnis iš ties
mažesnis už daliklio $x^3$ laipsnį.      

Lygties $P(x) = 0$ sprendinius vadinsime daugianario $P(x)$
\emph{šaknimis}. Ši sąvoka yra esminė, tad daugianario šaknims skirsime
visą likusią užduotį. Pradėkime nuo labai svarbios teoremos.

\begin{thm}
$n$-tojo laipsnio daugianaris turi ne daugiau nei $n$ šaknų.
\end{thm}

Prieš įrodydami teoremą panagrinėkime pavyzdžiui daugianarį $x^3 - x$.
Kadangi daugianaris kubinis (trečio laipsnio), tai jis turi turėti ne
daugiau kaip tris šaknis. Kaip tuo įsitikinti? Išskaidykime jį
dauginamaisiais: $x^3-x =x(x-1)(x+1).$ Spręsdami lygtį $x(x-1)(x+1) = 0$
samprotaujame, kad sandauga lygi nuliui tik tada, kai bent vienas iš
dauginamųjų lygus nuliui. Tad lengvai gauname, kad šiuo atveju šaknys kaip
tik yra trys, ir ne daugiau. Remdamiesi šia skaidymo idėja įrodysime ir
teoremą. Tam mums reiks pagalbinio teiginio:

\begin{teig}
Jei daugianaris $P(x)$ turi šaknį $s$, tai jis dalijasi iš $x-s$. 
\end{teig}

\emph{Įrodymas} Tarkime, kad $s$ yra $P(x)$ šaknis, bet $P(x)$ nesidalija
iš $x-s$. Tuomet dalindami gausime liekaną, kurios laipsnis mažesnis už $1$
- t.y. realųjį skaičių: $$P(x) = (x-s)Q(x) + b.$$ Ši lygybė teisinga
visiems $x$, tad įstatykime $x=s$. Gausime $$0=P(s) = 0\cdot R(x) + b =
b,$$ t.y. $b=0$. Vadinasi liekana lygi nuliui, ir $P(x)$ iš ties dalijasi
iš $x-s$.

Šis teiginys yra labai svarbus, todėl pateiksime ir antrą įrodymą:

Užrašykime $$P(x) = a_nx^n + \cdots + a_1x +a_0.$$ Kadangi $s$ yra $P(x)$
šaknis, tai $$a_ns^n + \cdots + a_1s +a_0 = 0$$ ir tuomet $$P(x) = a_nx^n +
\cdots + a_1x +a_0 - 0 = a_nx^n + \cdots + a_1x +a_0 - a_ns^n - \cdots -
a_1s -a_0 = a_n(x^n -s^n) + \cdots + a_1(x-s).$$ Kiekviename dėmenyje
galime iškelti $(x-s)$ prieš skliaustus, todėl $P(x)$ tikrai dalijasi iš
$x-s$. $\square$


\emph{Teoremos įrodymas}

Tegu daugianaris $P(x)$ turi šaknį $s_1$. Tuomet jis dalijasi iš $a-s_1$ ir
galime užrašyti $P(x)=(x-s_1)Q_1(x)$, kur $Q_1$ - $n-1$-ojo laipsnio
daugianaris. Jei $P(x)$ turi dar vieną šaknį $s_2$, tai ir ją galime
atskelti: $P(x) = (x-s_1)(x-s_2)Q_2(x)$, kur $Q_2$ - $n-2$-ojo laipsnio
daugianaris. Kadangi su kiekviena $P(x)$ šaknimi $s_i$ gauname vis naują
dauginamąjį $(x-s_i)$, tai šaknų tikrai negali būti daugiau nei $n$.


Pastebėkime, kad šį teiginį galime griežčiau įrodyti naudodami matematinę
indukciją. Pirmo laipsnio daugianariai iš ties turi ne daugiau nei vieną
šaknį. Tarkime, kad $n-1$ laipsnio daugianariai turi ne daugiau kaip $n-1$
šaknį. Nagrinėkime $n$-tojo laipsnio daugianarį $P(x)$. Jei jis neturi
šaknų, tai jo šaknų skaičius neviršija $n$. Jei turi bent vieną, pažymėkime
$s$, tai $P(x)=(x-s)Q(x)$, kur $Q(x)$ - $n-1$-ojo laipsnio daugianaris.
Pagal indukcinę prielaidą jis turi ne daugiau kaip $n-1$ šaknį, todėl
$P(x)$ turi ne daugiau nei $n$ šaknų.
 
$\square$
\bigskip

\subsection{Šaknų paieška} 

Žinome, kad kvadratinis daugianaris $ax^2 + bx + c$ turi dvi realiąsias
šaknis $$\frac{-b \pm \sqrt{b^2 - 4ac}}{2a},$$ kai $b^2-4ac\geq 0$. 
Naudodamiesi teiginiu, kad jei daugianaris $P(x)$ turi šaknį $s$, tai jis
dalijasi iš $x-s$, kartais galime rasti didelio laipsnio daugianarių
šaknis. Tam reikia atspėti pradinio daugianario $P(x)$ šaknį $s$, padalinti
iš $x-s$ ir liks mažesnio laipsnio daugianaris. Taip spėliojame ir
daliname, kol lieka kvadratinis daugianaris, kurio šaknis rasti mokame. 


Pabandykime šiuo būdu rasti daugianario $x^4 + x^3 - x - 1$ šaknis. Labai
greitai galime pamatyti, kad $x=1$ yra viena iš šaknų. Padalinkime: $x^4 +
x^3 - x - 1 = (x-1)(x^3 + 2x^2 + 2x + 1)$. Paspėlioję randame ir
daugianario $x^3 + 2x^2 + 2x + 1$ šaknį - $x=-1$. Vėl dalijame: $x^3 + 2x^2
+ 2x + 1 = (x+1)(x^2+x+1)$. Likęs daugianaris realiųjų šaknų neturi, tad
daugianario $x^4 + x^3 - x - 1$ realiosios šaknys yra $\pm 1$. Nors
daugianarį $x^3 + 2x^2 + 2x + 1$ padalinti iš (x+1) galima ir mintinai, bet
kartais daug paprasčiau yra naudoti daugianarių dalybą kampu. Ji labai
panaši į sveikųjų skaičių dalybą, bet jos išdėstymas užima labai daug
vietos, tad paklauskite savo matematikos mokytojo(-os) - tikrai žinos ir
jums paaiškins. 

\bigskip

Kartais šaknis atspėti gali būti gana sudėtinga, o kartais to visai
nepavyks padaryti. Spėliojimo procesą gerokai palengvina šis teiginys:

\begin{teig}
Jei daugianaris $P(x)$ su sveikaisiais koeficientais turi racionaliąją
šaknį, tai tos šaknies skaitiklis dalo laisvąjį narį, o vardiklis dalo
vyriausiąjį koeficientą. 
\end{teig}  

\emph{Įrodymas}

Tegu $P(x) = a_nx^n + \cdots +a_1x + a_0$ ir $\frac{s}{v}$ (tarkime
trupmena suprastinta) jo šaknis. Tuomet $$a_n\frac{s^n}{v^n} + \cdots +
a_1\frac{s}{v} + a_0 = 0.$$ Padauginkime lygybę iš $v^n$: $$a_n{s^n} +
\cdots + a_1sv^{n-1} + a_0v^n = 0.$$ Matome, kad visi lygybės nariai
išskyrus $a_0v^n$ dalijasi iš $s$, todėl ir $a_0$ dalijasi iš $s$. Taip
pat, visi lygybės nariai išskyrus $a_n{s^n}$ dalijasi iš $v$, todėl ir
$a_n$ dalijasi iš $v$.$\square$

\begin{isv}
Jei daugianaris $P(x)$ su sveikaisiais koeficientais turi sveikąją šaknį,
tai ji dalo daugianario laisvąjį narį.
\end{isv} 

Pasinaudoję tuo raskime daugianario $x^4 + 3x^3 - 3x^2 -15x + 2$ sveikąsias
šaknis. Kadangi laisvasis narys yra $2$, tai turime tik keturis kandidatus:
$\pm 1$ ir $\pm 2$. Patikrinę matome, kad tik $x=2$ yra šaknis, o daugiau
sveikųjų (ir racionaliųjų, nes vyriausiasis koeficientas lygus $1$) šaknų
daugianaris neturi.

\subsection{Daugianarių skaidymas}

Daugianariams skaidyti dažnai naudojamos lygybės, kurias privalu mokėti
mintinai. Visiems $n$:
$$a^n - b^n = (a-b)(a^{n-1} + a^{n-2}b + a^{n-3}b^2 + \cdots +
a^2b^{n-3}+ab^{n-2} + b^{n-1} ),$$
nelyginiams n:
$$a^n + b^n = (a+b)(a^{n-1} - a^{n-2}b + a^{n-3}b^2 - \cdots -
a^2b^{n-3}+ab^{n-2} - b^{n-1} ).$$

Atskirais atvejais jos pavirsta į labai dažnai sutinkamas
$$a^2 - b^2 = (a+b)(a-b) \text{ ir } a^3+b^3 = (a+b)(a^2-ab+b^2).$$  

Pavyzdžiui išskaidykime daugianarį $x^4 - x^2 - 2x - 1$ dauginamaisiais.
Užtenka pastebėti pilną kvadratą: $$x^4 - x^2 - 2x - 1 = x^4 - (x+1)^2 =
(x^2- x -1)(x^2+ x+ 1) = (x - \frac{-1 + \sqrt{5}}{2})(x - \frac{-1 -
\sqrt{5}}{2})(x^2+ x+ 1).$$

Atkreipkime dėmesį į tai, kad pradinio daugianario koeficientai buvo
sveikieji skaičiai, o dviejų gautų daugiklių nebe. Toks skaidymas kartais
gali neturėti prasmės (pavyzdžiui sprendžiant skaičių teorijos uždavinį),
todėl prasminga yra suvokti kokių koeficientų mes skaidydami tikimės -
sveikųjų ar realiųjų (ar kompleksinių). Tai ir nusako terminai skaidymas
\emph{virš realiųjų skaičių} ir \emph{virš sveikųjų skaičių}. Skaidymas
virš racionaliųjų gerokai skiriasi nuo skaidymo virš realiųjų. Pavyzdžiui
pastarajam galioja tokia teorema:

\begin{thm}
Kiekvieną daugianarį su realiaisiais koeficientais galima virš realiųjų
skaičių išskaidyti į kvadratinių ir pirmo laipsnio daugianarių sandaugą.
\end{thm}

Kaip bebūtų, tai dar nereiškia, kad tuos dauginamuosius mums pavyks rasti.
Pavyzdžiui daugianaris $x^3 + x + 1$ tikrai turi bent vieną realiąją šaknį
(ar žinote kodėl?), todėl jį galime tikėtis išskaidyti bent jau į du
dauginamuosius. Deja, tos šaknies taip lengvai nerasime (įdomumo dėlei ji
lygi $-\frac{(108+12\sqrt{93})^{1/3}}{6}+
\frac{2}{(108+12\sqrt{93})^{1/3}}$.)

Skaidymui virš racionaliųjų, kita vertus, galime naudoti neapibrėžtų
koeficientų metodą. Išskaidykime pavyzdžiui daugianarį $x^4 + x^2 + 1$.
Sveikųjų šaknų jis neturi, todėl nario $x-a$ atskelti nepavyks. Vadinasi,
jei ir išskaidysime virš racionaliųjų, tai į dviejų kvadratinių daugianarių
sandaugą: $$x^4 + x^2 + 1 = (A_1x^2 + B_1x + C_1)(A_2x^2 + B_2x + C_2).$$
Aišku, kad $A_1$ ir $A_2$ turi abu būti lygūs $1$ arba $-1$. Jei abu lygūs
$-1$, tai padauginę abu daugianarius iš $-1$ gausime skaidinį, kuriame abu
lygūs $1$. $C_1$ ir $C_2$ irgi gali būti lygūs $-1$ arba $1$. Pabandykime
pastarąjį variantą. Tuomet gauname: 
$$x^4 + x^2 + 1 = (x^2 + B_1x + 1)(x^2 + B_2x + 1).$$
Atskliaudę ir sulyginę koeficientus prie $x^3$, $x^2$ ir $x$ gausime
$B_1+B_2 = 0$, $2+B_1B_2 = 1$, $B_1+B_2=0.$ Išsprendę lengvai randame $B_1
= 1$, $B_2 = -1$. Vadinasi $$x^4 + x^2 + 1 = (x^2 + x + 1)(x^2 - x + 1).$$
Jei išbandę visus variantus gauname sistemas, kurios racionaliųjų
sprendinių neturi, tai vadinasi daugianaris virš racionaliųjų nesiskaido.

\subsection{Vijeto teorema}

Vijeto teorema apibrėžia santykį tarp daugianario koeficientų ir jo šaknų.
Žinome, kad norėdami nusakyti daugianarį turime nurodyti jo koeficientus,
bet taip pat galime nurodyti ir šaknis bei vyriausiąjį koeficientą:

\begin{thm}[Vijeto teorema]

Tarkime, kad daugianario $P(x)=a_nx^n + \cdots + a_1x + a_0$ šaknys yra
$s_1, \dots, s_n$. Tuomet
$$s_1 + \cdots + s_n = (-1)\frac{a_{n-1}}{a_n}$$
$$s_1s_2 + s_1s_3 + \cdots s_{n}s_{n-2} + s_{n}s_{n-1} =
(-1)^2\frac{a_{n-2}}{a_n}$$
$$\vdots$$
$$s_1\cdots s_n = (-1)^{n}\frac{a_{0}}{a_n}.$$

\end{thm}

\emph{Įrodymas}

Žinome, kad $n$-tojo laipsnio daugianaris, kurio šaknys yra $s_1, \dots,
s_n$ užsirašo kaip $a_n(x-s_1)\cdots(x-s_n).$ Atskliaudę ir gausime
ieškomas lygybes. Pabandykite. $\square$

Atskiru atveju kvadratiniam daugianariui $ax^2 + bx + c$ gauname $$s_1 +
s_2 = -\frac{b}{a}$$ ir $$s_1s_2 = \frac{c}{a}.$$ 

\begin{center} \textbf{Pavyzdžiai} \end{center}
1. [Baltic Way 2008] Raskite visus daugianarius $p(x)$ su realiaisiais
koeficientais, tokius, kad $$p((x+1)^3)= (p(x)+1)^3$$ ir $$p(0)=0.$$

\emph{Sprendimas}
Įstatę keletą pirmųjų reikšmių pastebime, kad $p(1) = 1$, $p(8)=8$, $p(9^3)
= 9^3$. Taip pat matome, kad jei kažkokiai reikšmei $c$ $p(c)=c$, tai ir
$p((c+1)^3) = (c+1)^3$. Todėl egzistuoja be galo daug skaičių, kuriuos
įstatę vietoje $x$ gauname teisingą lygybę $p(x) = x$. Arba, performulavę
gudriau galime teigti, kad daugianaris $p(x)-x$ turi be galo daug šaknų. Be
galo daug šaknų turi vienintelis daugianaris - 0. Taigi gavome, kad $p(x)-x
= 0$, arba $p(x)=x$.

Tuo pačiu išmokome įrodyti labai svarbų teiginį:
\begin{teig}
Tarkime turime du daugianarius, kurių laipsniai neviršija $n$. Jei
egzistuoja daugiau nei $n$ skaičių, kuriuose jų reikšmės vienodos, tai tie
daugianariai sutampa. 
\end{teig} 

Iš ties - jų skirtumas turės daugiau šaknų, nei bus jo laipsnis, todėl bus
lygus $0$.


2. Raskime liekaną kurią gauname, kai daugianarį $x^{100}$ dalijame iš $x^2
- 1$. Užsirašykime:
 
$$x^{100} = Q(x)(x^2-1) + R(x).$$

Įsistatę daugianario $x^2-1$ šaknis gausime $R(1) = 1$, $R(-1)=1$. Kadangi
$R(x)$ laipsnis turi būti mažesnis už $x^2 - 1$ laipsnį, tai tinka
vienintelis daugianaris $R(x) = 1$.  

Šis metodas yra gana bendras, tačiau ne visad daugianaris iš kurio dalijame
turės visas realiąsias šaknis. Kartais galima išsisukti naudojant
kompleksines šaknis, bet apie tai daugianariai II užduotyje. 
   
\subsection{Uždaviniai}

\begin{enumerate} \item Raskite daugianario $x^3 - 7x^2 +7x + 15$ šaknis.
  \item Išskaidykite daugianarį $x^8 + x^4 + 1$ dauginamaisiais virš
    racionaliųjų skaičių.  \item Įrodykite tapatybę $$(1+x)(1+x^2)(1+x^4)
    \cdots (1+x^{2^{k-1}}) = 1 + x + x^2 + \cdots + x^{2^{k}-1}.$$ \item
    Įrodykite \emph{Bezu} teoremą: dalindami daugianarį $P(x)$ iš $x-a$
    gausime liekaną $P(a)$.  \item Įrodykite, kad jei $n$-tojo laipsnio
    daugianario $S(x)$ vyriausiasis koeficientas lygus vienam, jis turi $n$
    realių šaknų ir visos jos yra ir daugianario $P(x)$ šaknys, tai $P(x)$
    dalijasi iš $S(x)$.  \item Tegu $a$, $b$, $c \in \R$. Jei $a + b + c >
    0$, $ab + bc + ca > 0$ ir $abc > 0$, tai $a$, $b$, $c > 0.$ \item
    Įrodykite, kad $(a+b+c)^{333} - a^{333} - b^{333} - c^{333}$ dalijasi
    iš $(a+b+c)^3 - a^3 - b^3 - c^3$.  \item Kokią liekaną gausime
    dalindami $x^{100} -2x^{51} + 1$ iš $x^2 - 1$?  \item Dalindami
    daugianarį $P(x)$ iš $x-1$ gavome liekaną $2$, o dalindami iš $x-2$
    gavome liekaną $1$. Kokią liekaną gausime jį dalindami iš $(x-1)(x-2)$?
  \item Raskite visus daugianarius $P(x)$, kurie su kiekvienu daugianariu
    $Q(x)$ tenkina $P(Q(x)) = Q(P(x)).$ \item Tegu $x_1$ ir $x_2$
    daugianario $x^2 + ax + bc$ šaknys, o $x_2$ ir $x_3$ - daugianario $x^2
    + bx + ac$ šaknys ir $ac\neq bc$. Įrodykite, kad $x_1$ ir $x_3$ yra
    daugianario $x^2 + cx + ab$ šaknys.  \item Įrodykite, kad jei
    daugianario su sveikaisiais koeficientas reikšmės taškuose $0$ ir $1$
    yra nelyginės, tai daugianaris neturi sveikųjų šaknų.  \item
    Išspręskite lygtį $x^8 + (x+2)^8 = 2$.  \item Daugianario $x^2 + px +
    q$ šaknų santykis lygus daugianario $x^2 + 3px + q_1$ šaknų santykiui.
    Nė vienas daugianarių koeficientas nelygus nuliui. Raskite $q$ ir $q_1$
    santykį.  \item Išspręskite lygtį $x^6 + x^5 + x^4 + x^3 + x^2 + x + 1
    = 0.$ \item Išskaidykite dauginamaisiais reiškinį $x^4 + a^4 +
    (x+a)^4.$ \item Su kuriomis $q$ reikšmėmis daugianario $x^4 - 2x^2 + q
    = 0$ šaknys sudaro aritmetinę progresiją?  \item Daugianaris $x^3 + px
    + q$ turi tris skirtingas realiąsias šaknis. Įrodykite, kad $p<0$.
  \item Tegu $P$ - daugianaris su sveikaisiais koeficientais ir $P(x) = 5$
    su penkiomis skirtingomis sveikosiomis $x$ reikšmėmis. Įrodykite, kad
    nėra tokio sveikojo skaičiaus $x$, kad $-6\leq P(x)\leq 4$ arba $6 \leq
    P(x) \leq 16.$ \item Raskite visus daugianarius $P(x)$ su sveikaisiais
    koeficientais, kurie tenkina $P(x)^2|P(P(x)+x)$.  \item Tegu $P(x) =
    x^2 + 2007x + 1.$ Įrodykite, kad su kiekviena $n\in \N$ reikšme
    daugianaris $\underbrace{P(P(...(P}_{\text{n kartų}}(x))...))$ turi
    bent vieną realią šaknį.  \item Tarkime daugianaris $P(x) = x^3 + ax^2
    + bx +c$ turi tris realiais šaknis ir tegu $Q(x) = 5x^2 - 16x + 2004$.
    Žinome, kad $P(Q(x))=0$ neturi realiųjų šaknų. Įrodykite, kad $P(2004)
    > 2004$.  \end{enumerate}

  
\section{Daugianariai II} \bigskip \noindent

Šioje dalyje truputį giliau susipažinsime su daugianarių šaknimis bei
skaidymu dauginamaisiais. Iš pradžių panagrinėsime skaidymą virš realiųjų
ir kompleksinių skaičių, o po to virš sveikųjų ir racionaliųjų.

Prisiminkime vieną iš pagrindinių teoremų:

\begin{thm}[Fundamentalioji algebros teorema] Kiekvienas $n$-tojo laipsnio
  daugianaris turi $n$ kompleksinių šaknų.  \end{thm}

Šios teoremos įrodymas yra labai sudėtingas, tad jo įrodymo nepateiksime.
Pažiūrėkime, kaip praktiškai galime pritaikyti šią teoremą:  

Įrodysime, kad $x^{9999} + x^{8888} + \cdots + x^{1111} + 1$ dalijasi iš
$x^{9} + x^{8} + \cdots + x^{1} + 1.$

Tam užteks parodyti, kad visos mažojo daugianario šaknys yra ir didžiojo
šaknys. Tačiau mažojo daugianario šaknų mes kol kas nemokame rasti!
Išsisukime - tiesiog imkime bet kurią šaknį $\omega$. Žinome, kad
$\omega^{9} + \omega^{8} + \cdots + \omega^{1} + 1 = 0$. Maža to, padauginę
abi lygybės puses iš $\omega - 1$ ($\omega \neq 1$, nes $1$ nėra šaknis, o
jei net ir būtų tai panagrinėtume atskirai) gauname, kad $\omega^{10} = 1$.
Dabar visiškai nesunkiai galime įsitikinti, kad $\omega$ yra ir didžiojo
daugianario šaknis, mat $\omega^{9999} = (\omega^{10})^{999}\cdot \omega^9$
ir t.t., todėl  $$ \omega^{9999} + \omega^{8888} + \cdots + \omega^{1111} +
1 = \omega^{9} + \omega^{8} + \cdots + \omega^{1} + 1 = 0.$$

Prisiminkime pirmosios daugianarių dalies užduotį: 

Įrodykite, kad jei $n$-tojo laipsnio daugianario $S(x)$ vyriausiasis
koeficientas lygus vienam, jis turi $n$ \textbf{realių} šaknų ir visos jos
yra ir daugianario $P(x)$ šaknys, tai $P(x)$ dalijasi iš $S(x)$. 

Nežinodami apie kompleksinius skaičius ir norėdami daugianarį išskaidyti
mes privalėjome rasti jo realiąsias šaknis. Jei daugianaris jų neturėjo
(pvz. $x^2+1$), tai išskaidyti negalėdavome. Todėl sąlyga, kad visos
daugianario šaknys yra realios buvo būtina.

Žinodami apie kompleksinius skaičius mes visuomet galime pilnai išskaidyti
daugianarį virš kompleksinių skaičių, todėl tai ar šaknys realios ar ne
nebeturi įtakos. Pabandykime apibendrinti:

Bet koks $n$-tojo laipsnio daugianaris $S(x)$ turi lygiai $n$ kompleksinių
šaknų, kai kurios iš jų gali būti realios ($\R \subset \C$!). Jei
daugianario vyriausias koeficientas lygus vienetui, tai jis užsirašo kaip
$(x-x_1)\cdots(x-x_n)$. Jei jis dalo kokį nors daugianarį $P(x)$, tai $x_1,
\dots, x_n$ yra ir $P(x)$ šaknys. 

Gali kilti klausimas, ar daugianariui, kurio šaknys kompleksinės galioja
Vijeto teorema. Atsakymas žinoma! Įrodymas visiškai nesiskiria - tiesiog
atskliaudžiame išraišką $(x-x_1)\cdots(x-x_n)$. Nepaisant to, panagrinėkime
ir konkretų atvejį:

$x^2 + 1$ šaknys yra dvi: $i$ ir $-i$. Jų suma lygi nuliui, o sandauga
$-i^2 = 1$. Veikia.

Remdamiesi fundamentaliąja algebros teorema įrodykime praeito skyrelio
teiginį, sakantį, kad kiekvieną daugianarį virš realiųjų skaičių galima
išskaidyti į pirmo ir antro laipsnio daugianarius. Tam prireiks teiginio,
kuris ir pats savaime yra svarbus:

\begin{teig} Daugianario su realiaisiais koeficientais šakniai
  kompleksiškai jungtinis skaičius taip pat yra daugianario šaknis.
\end{teig}

\emph{Įrodymas}

Tegu $s$ daugianario šaknis. Užrašykime tai kaip $a_ns^n + a_{n-1}s^{n-1} +
\cdots + a_1s + a_0 = 0.$ Kadangi sumos jungtinis lygus jungtinių sumai,
kaip ir sandaugos jungtinis lygus jungtinių sandaugai, tai gauname, kad
$$\overline{a_ns^n + a_{n-1}s^{n-1} + \cdots + a_1s + a_0} = \overline{0}
\Rightarrow$$ $$\overline{a}_n\overline{s}^n +
\overline{a}_{n-1}\overline{s}^{n-1} + \cdots + \overline{a}_1\overline{s}
+ \overline{a_0} = 0 \Rightarrow$$ $$a_n\overline{s}^n +
a_{n-1}\overline{s}^{n-1} + \cdots + a_1\overline{s} + a_0 = 0,$$ vadinasi
ir $\overline{s}$ yra to paties daugianario šaknis. $\square$


Tad jei daugianaris turi realią šaknį $a$, tai galime atskelti dauginamąjį
$(x-a)$, o jei turi kompleksinę $s$, tai kartu turi ir jai jungtinę
$\overline{s}$. Sudauginę $(x-s)(x-\overline{s})$ gauname kvadratinį
dauginamąjį su realiaisiais koeficientais.

\subsection{Daugianariai su sveikaisiais koeficientais}

Daugianarius, kurių negalime nė kiek išskaidyti vadinsime neredukuojamais,
nurodydami kokių koeficientų tikimės skaidydami. Pavyzdžiui $x^2 + 1$ yra
neredukuojamas virš sveikųjų, racionaliųjų ir realiųjų skaičių, bet yra
redukuojamas virš kompleksinių. Įrodykime tris naudingas teoremas.

\begin{thm}[Gauso lema] Jei daugianaris yra neredukuojamas virš sveikųjų
  skaičių, tai jis yra neredukuojamas ir virš racionaliųjų.  \end{thm}

\emph{Įrodymas}

Daugianarį su sveikaisiais koeficientais, kurių didžiausias bendras
daliklis lygus vienetui vadinkime primityviuoju. Neprarasdami bendrumo
laikykime, kad nagrinėjamas daugianaris $p(x)$ yra primityvus ir tarkime
priešingai - tegu jis yra neredukuojamas virš sveikųjų, bet redukuojamas
virš racionaliųjų. Tada $p(x)=s(x)t(x)$, kur $s$, $t$ - daugianariai su
racionaliaisiais koeficientais. Padauginkime abu juos atitinkamai iš $a$,
$b \in \Z$, kad gautume du primityviuosius daugianarius: $$abp(x) =
(as(x))(bt(x)).$$ Taigi kairėje pusėje turime neprimityvų daugianarį (jo
visi koeficientai dalijasi iš $ab\neq 1$), o dešinėje dviejų primityvių
daugianarių sandaugą. Parodysime, kad tokia lygybė yra negalima, nes dviejų
primityvių daugianarių sandauga yra primityvus daugianaris, ir gausime
prieštarą.  Tegu $a_nx^n + \cdots + a_1x + a_0$ ir $b_mx^m + \cdots + b_1x
+ b_0$ primityvūs daugianariai. Tegu jų sandauga nėra primityvusis
daugianaris, t.y. visi koeficientai dalinasi iš kažkokio pirminio skaičiaus
$p$ : $$p\cdot c_{n+m}x^{m+n} + \cdots + p\cdot c_1x + p\cdot c_0 = (a_nx^n
+ \cdots + a_1x + a_0)(b_mx^m + \cdots + b_1x + b_0).$$

Abiejų daugianarių visi koeficientai negali dalintis iš $p$, kitaip jie
nebūtų primityvūs. Imkime pirmus koeficientus (prie mažiausių laipsnių)
nesidalijančius iš $p$ $a_i$ ir $b_j$. Tada $$p\cdot c_{i+j} =
a_{1}b_{i+j-1} + \cdots + a_{i}b_{j} + \cdots a_{i+j-1}b_{1}.$$ Matome, kad
visi sumos nariai dalijasi iš $p$, išskyrus $a_{i}b_{j}$ - prieštara. (Suma
nebūnai turi prasidėti nuo $a_{1}b_{i+j-1}$, nes gali būti taip, kad $i+j-1
> m$, bet tai nieko nekeičia) $\square$

\begin{thm}[Eizenšteino neredukuojamumo kriterijus] Tegu $p(x) = a_nx^n +
  a_{n-1}x^{n-1} + \cdots + a_1x+a_0$ - daugianaris su sveikaisiais
  koeficientais. Jei egzistuoja toks pirminis skaičius $p$, kad $p \not |
  a_n$, $p|a_i, i=0\dots n-1$ ir $p^2 \not | a_0$, tai daugianaris
  neredukuojamas virš sveikųjų skaičių (ir virš racionalių pagal Gauso
  lemą).  \end{thm}

Prieš įrodymą panagrinėkime teoremos sąlygos. Reikalaujama, kad iš $p$
dalintųsi visi koeficientai išskyrus pirmąjį, bei paskutinis nesidalintų iš
$p^2$. Pavyzdžiui $x^5 + 2x + 2$. Jei nors vienos iš sąlygų atsisakysime,
tai lengvai rasime kontrapavyzdžių. Pavyzdžiui jei leisime pirmam
koeficientui dalintis iš $p$, tai redukuojamas bus $3x^2 - 3$, jei leisime
paskutiniam dalintis iš $p^2$, tai redukuojamas bus $x^2 + 10x + 25$.

\emph{Įrodymas} Tegu $p(x)$ tenkina teoremos sąlygas. Tarkime, kad jis
redukuojamas, tada galime užrašyti jį kaip dviejų daugianarių sandaugą:
$$a_nx^n + \cdots + a_1x+a_0 = (b^kx^k + \cdots + b_1x +
b_0)(c^{n-k}x^{n-k} + \cdots + c_1x + c_0)$$

Kadangi $a_0$ dalijasi iš $p$ ir $a_0 = b_0c_0$, tai $b_0$ arba $c_0$
dalijasi iš $p$. Tegu tai bus $b_0$. Kadangi $a_0$ nesidalija iš $p^2$, tai
$c_0$ nesidalija iš $p$.\\ Kadangi $a_1$ dalijasi iš $p$ ir $a_1 = b_0c_1 +
b_1c_0$, kur $b_0$ dalijasi iš $p$, o $c_0$ ne, tai $b_1$ irgi dalijasi iš
$p$.\\ Kadangi $a_2$ dalijasi iš $p$ ir $a_2 = b_0c_2 + b_1c_1 + b_2c_0$,
kur $b_0$, $b_1$ dalijasi iš $p$, o $c_0$ ne, tai ir $b_2$ dalijasi iš
$p$.\\ Taip tęsdami gauname, kad visi koeficientai $b_0, b_1, \cdots, b_k$
dalijasi iš $p$, o tai reiškia, kad ir $a_n$ dalijasi iš $p$.
Prieštara.$\square$

Iš pirmo žvilgsnio gali pasirodyti, kad ši teorema yra gana ribota dėl
daugybės kriterijų, kuriuos turi atitikti daugianaris, jei mes norime ją
pritaikyti. Tačiau pasirodo kad naudojant ją kūrybiškai galima gerokai
praplėsti jos taikymo ratą.

Išspręskime tokį uždavinį - Įrodykite, kad daugianaris $p(x) = x^{p-1} +
x^{p-2} + \cdots + x + 1$ yra neredukuojamas virš sveikųjų skaičių. Nė
vienas koeficientas nesidalija iš nė vieno sveikojo skaičiaus, kaip čia
pasinaudoti Eizenšteino kriterijumi? Pagudraukime - jei daugianarį $p(x)$
galime išskaidyti, tai galėsime išskaidyti ir $p(x+1)$ - tereiks $p(x)$
skaidinyje $x$ pakeisti į $x+1$. Tad jei įrodysime, kad $p(x+1)$
neredukuojamas, tai neredukuojamas bus ir $p(x)$. Užsirašykime $$p(x) =
x^{p-1} + x^{p-2} + \cdots + x + 1 = \frac{x^p - 1}{x-1},$$ tada $$p(x+1) =
\frac{(x+1)^p - 1}{x} = \frac{x^p + px^{p-1} + \cdots + \frac{p(p-1)}{2}
x^2 + px}{p} = x^{p-1} + \cdots + \frac{p(p-1)}{2} x + p.$$

Kadangi visi koeficientai dalinsis iš $p$ (prisiminkite šią binominio
koeficiento savybę), paskutinis koeficientas nesidalija iš $p^2$, o
pirmasis iš $p$, tai šis daugianaris yra neredukuojamas. Įrodėme.

Paskutinei teoremai reiks prisiminti dalybos liekanas. Kadangi jos irgi
sudaro kūną, kaip ir $\Z, \Q, \R, \C,$ t.y. liekanas galima įprastai
dauginti, dalinti(išskyrus iš nulio), sudėti ir atimti, tai galime
nagrinėti daugianarius, kurių koeficientai yra dalybos liekanos moduliu
kažkokio pirminio skaičiaus $p$. Kad atskirtume nuo sveikųjų skaičių, juos
žymėsime su brūkšniu. Pavyzdžiui $\overline{2}x^2 + \overline{3}$.

Pastebėkime, kad tokių daugianarių skaidymas gerokai skiriasi nuo skaidymo
virš sveikųjų. Pavyzdžiui daugianaris $x^2$ + $\overline{1}$, kur
koeficientai yra moduliu $2$ išsiskaido į $(x+\overline{1})^2$, nes $2
\equiv 0$.  

Iš kitos pusės, patikrinti ar daugianaris skaidosi tampa lengviau, nes
liekanų yra nedaug (ypač jei pasirenkame nedidelį pirminį $p$) ir galime
tiesiog perrinkti visus variantus. Tai ir motyvuoja tolesnę teoremą:

    
\begin{thm} Jei daugianaris su sveikaisiais koeficientais ir vyriausiuoju
  koeficientu lygiu vienetui yra neredukuojamas moduliu kažkokio pirminio
  skaičiaus $p$, tai jis yra neredukuojamas ir virš sveikųjų.  \end{thm}



\emph{Įrodymas} Jei jis būtų redukuojamas virš sveikųjų, tai jo skaidinį
paėmę moduliu $p$ gautume, jog jis skaidosi ir moduliu $p$.  $\square$

Panaudokime šią teoremą klausimui ar daugianaris $x^3+1001x^2 + 1111^2$ yra
redukuojamas virš sveikųjų atsakyti. Jei nagrinėsime jį moduliu $2$,
gausime $x^3 + x^2 + \overline{1}$. Pabandę visas kombinacijas
(kvadratiniai dauginamieji gali būti $x^2 + x + \overline{1}$, $x^2 + x$,
$x^2 + \overline{1}$, $x^2$, tiesiniai $x+\overline{1}$ ir $x$, taigi $8$
variantai) gauname, kad šis daugianaris moduliu $2$ yra neredukuojamas.
Vadinasi jis yra neredukuojamas ir virš sveikųjų.
  
Atkreipsime dėmesį, kad jei gauname, jog daugianaris yra redukuojamas
moduliu pirminio skaičiaus, tai dar nieko negalime pasakyti apie jo
redukuojamumą virš sveikųjų. Atsiminkime $x^2 + 1$. 


\newpage

\subsection{Uždaviniai}

\begin{enumerate}

\item Tegu $\alpha$, $\beta$, ir $\gamma$ daugianario $x^3 - x - 1$ šaknys.
  Raskite $\frac{1+\alpha}{1-\alpha} + \frac{1+\beta}{1-\beta} +
  \frac{1+\gamma}{1-\gamma}.$ \item Tegul
  $F(x)=a_0+a_1x+a_2x^2+\ldots+a_nx^n\in\mathbb{Z}[X]$. Parodykite, kad
  $F(x)$ yra neredukuojamas virš $\mathbb{Q}[X]$, jei $a_0$ yra pirminis
  skaičius, kuris tenkina $|a_0|>|a_1|+|a_2|+\ldots+|a_n|$.  \item Duoti du
  daugianariai $F, G\in\mathbb{N}[X]$. Tegul $m$ yra didžiausias $F$
  koeficientas. Be to egzistuoja du natūralieji skaičiai $a<b$, tokie kad
  $F(a)=G(a)$ ir $F(b)=G(b)$. Įrodykite, kad $F=G$, jei $m<b$.  \item
  Įrodykite, kad kiekvienam natūraliajam $n$ egzistuoja toks daugianaris
  $P\in\mathbb{Z}[X]$, $P(x)=x^n+a_1x^{n-1}+\ldots+a_{n-1}x+a_n$, kad i.)
  $a_1a_2\cdot\ldots\cdot a_n\neq 0$, \\ ii.)  $P$ nėra dvieju daugianarių
  su teigiamais laipsniais iš $\mathbb{Z}[X]$ sandauga\\ iii.) $|P(x)|$
  nėra pirminis su bet kuriuo sveikuoju $x$.  \item Įrodykite, kad su
  kiekvienu natūraliuoju $n\ge5$ ir skirtingais $a_1$, $a_2$, $\ldots$,
  $a_n$ daugianaris $P(x)=(x-a_1)(x-a_2)\ldots(x-a_n)+1$ yra neredukuojamas
  virš $\mathbb{Z}[X]$.  \item Raskite dauginarius tenkinančius sąlygą
  $P(x^2) + P(x)P(x+1) = 0$ su visais $x \in \R$.
%Perrašykime lygybę patogiau: $$P(x^2) = -P(x)P(x+1).$$ Pirmiausia
%įrodysime, kad visos daugianario tenkinančio šią lygtį šaknys yra
%išsidėsčiusios ant vienetinio apskritimo arba nuliai. Tarkime priešingai -
%tegu bus šaknų, kurių moduliai didesni už vienetą. Imkime vieną iš jų $x$,
%su didžiausiu moduliu. Jos kvadratas taip pat bus šaknis, bet |x^2|> |x| -
%prieštara. Analogiškai ir su mažesniais už vienetą moduliais.  Taip pat
%pastebėkime, kad jei x+1 yra daugianario šaknis, tai x^2 irgi yra šaknis.
%Iš čia gauname tris atvejus: |x+1| = 1 |x^2| = 1, |x+1| = 0 |x^2| = 1,
%|x+1| = 1 |x^2| = 0, iš kur x = e^{\frac{2 \pi}{3}}, x = e^{\frac{- 2
%\pi}{3}} x = -1, x = 0, o x+1 tuomet lygus \omega, \omega^5, 0, 1, kur
%\omega = e^{\frac{2 \pi}{6}}.  Gauname, kad ieškomas daugianaris yra
%pavidalo P(x)=ax^n(x-1)^m(x-\omega)^p(x-\omega^5)^q. Įstatę į lygtį
%gauname a=-1, p=q=0, m=n \Rightarrow P(x)=0 arba P(x)=-x^n(x-1)^n, n\in\N. 
%http://www.mathlinks.ro/resources.php?c=60&cid=82&year=2005&p=243253
%http://www.mathlinks.ro/viewtopic.php?p=342476#342476
\end{enumerate}
